\chapter{Conclusion}

\section{Summary}

\begin{table}
    \centering
	\caption{Overview of the themes covered by each chapter of the thesis.}
    \def\yes{\checkmark}
    \footnotesize
	\begin{tabular}{lccccc}
                          & TD   & Tournaments & Modal & Belief change & Truth-tracking \\
\toprule
Aggregating reports       & \yes &             &       & \yes          & \yes \\
Assessing trustworthiness & \yes & \yes        &       & \yes          & \yes \\
Axiomatic analysis        & \yes & \yes        &       & \yes          & \yes \\
Defining new methods      & \yes & \yes        &       & \yes          &      \\
Reasoning with expertise  &      &             & \yes  & \yes          &      \\
Learning the truth        &      &             &       &               & \yes \\
\bottomrule

	\end{tabular}
    \label{conc_tab_themes}
\end{table}

This thesis has studied topics surrounding trustworthiness and expertise.
\cref{conc_tab_themes} shows a rough overview of the different themes tackled
in each chapter. We now reflect chapter-by-chapter on these themes in more
detail, and discuss the connections between chapters.

{
    % symbols
\newcommand{\Nat}{\mathbb{N}}
\renewcommand{\S}{\mathbb{S}}
\renewcommand{\O}{\mathbb{O}}
\newcommand{\V}{\mathbb{V}}
\newcommand{\R}{\mathbb{R}}
\newcommand{\T}{\mathcal{T}}
\newcommand{\D}{\mathcal{D}}
\newcommand{\F}{\mathcal{F}}

% network notation
\newcommand{\src}{\operatorname{src}}
\newcommand{\antisrc}{\operatorname{antisrc}}
\newcommand{\claims}{\operatorname{cl}}
\newcommand{\obj}{\operatorname{obj}}
\newcommand{\conflict}{\operatorname{conflict}}
\newcommand{\addclaim}[3]{{#1} + ({#2}, {#3})}

% rankings
\newcommand{\sle}{\sqsubseteq}
\newcommand{\slt}{\sqsubset}
\newcommand{\seq}{\simeq}
\newcommand{\cle}{\preceq}
\newcommand{\clt}{\prec}
\newcommand{\ceq}{\approx}
\newcommand{\rankequiv}{\sim}
\newcommand{\witprec}[4]{{#1}:{#2} \xrightarrow{#4} {#3}}
\newcommand{\mrelex}{M_\exists}
\newcommand{\mrelfa}{M_\forall}

% TD operators
\newcommand{\voting}{T^{\operatorname{vote}}}
\newcommand{\wagree}{w^{\operatorname{agg}}}
\newcommand{\wvoting}[1]{T^{#1}}
\newcommand{\sums}{T^{\operatorname{sums}}}
\newcommand{\truthfinder}{T^{\operatorname{tf}}}
\newcommand{\crh}{T^{\operatorname{crh}}}

% axioms
\newcommand{\claimcoherence}{\axiomref{Claim-coherence}}
\newcommand{\sourcecoherence}{\axiomref{Source-coherence}}
\newcommand{\symmetry}{\axiomref{Symmetry}}
\newcommand{\classicalindependence}{\axiomref{Classical-independence}}
\newcommand{\disjointindependence}{\axiomref{Disjoint-independence}}
\newcommand{\freshposresp}{\axiomref{Fresh-pos-resp}}
\newcommand{\sourceposresp}{\axiomref{Source-pos-resp}}
\newcommand{\anticoherence}{\axiomref{Anti-coherence}}
\newcommand{\conflictcoherence}{\axiomref{Conflict-coherence}}
\newcommand{\flatsources}{\axiomref{Flat-sources}}
\newcommand{\objectirrelevance}{\axiomref{Object-irrelevance}}
\newcommand{\marginaltrustworthiness}{\axiomref{Marginal-trustworthiness}}
\newcommand{\trustbasedmon}{\axiomref{Trust-based-monotonicity}}

% misc
\newcommand\restrict[2]{#1|_{#2}}
\newcommand\identity{\operatorname{id}}
\newcommand\indicator[1]{\mathds{1}[{#1}]}
\newcommand\norm[1]{\operatorname{norm}({#1})}

% tikz styles and macros
\tikzset{
    objectgrouping/.style={
        decorate,decoration={brace,raise=10pt,amplitude=6pt}
    }
}
\tikzset{
    objectlabel/.style={
        pos=0.5,right=15pt
    }
}
\newcommand{\networkinit}[2]{
    \readlist \sources {#1}
    \readlist \claims {#2}
    \def \height {max(\sourceslen, \claimslen)}
    \foreachitem \s \in \sources{
        \def \offset {(\height -  \sourceslen) / 2}
        \def \y {\offset + \sourceslen * ((\scnt - 1) / (\sourceslen - 1))}
        \node (\s) at (0, -{(\y)}) {$\s$};
    }
    \foreachitem \c \in \claims{
        \def \offset {(\height -  \claimslen) / 2}
        \def \y {\offset + \claimslen * ((\ccnt - 1) / (\claimslen - 1))}
        \node (\c) at (3, -{(\y)}) {$\c$};
    }
}
\newcommand{\report}[3][]{% optional arg is extra params to \draw (e.g. dashed)
    \draw[-,#1] (#2.east) -- (#3.west);
}
\newcommand{\object}[3]{
    \draw[objectgrouping] (#2.center) -- (#3.center) node[objectlabel] {\large $#1$};
}
\newcommand{\objectsingleclaim}[2]{
    \draw[objectgrouping] ($ (#2.center) + (0, 0.3) $)
        -- ($ (#2.center) + (0, -0.3) $)
        node[objectlabel] {\large $#1$};
}


In \cref{chapter_td} we took a social choice perspective on truth discovery.
Truth discovery concerns aggregating reports from unreliable sources, and
assessing their trustworthiness in the process. The representation of
information was simple in this chapter: we considered a number of ``objects''
of interest, each of which takes categorical values.  Trustworthiness was
expressed via a ranking of the sources, allowing us to say when one source
should be considered \emph{more trustworthy} than another. We would later take
a different approach in \cref{chapter_belief_change}, with information
represented by propositional formulas and with expertise-based semantics for
trustworthiness. However, the open-ended nature of ranking-based
trustworthiness without precise semantics allowed us to define several truth
discovery methods from the literature in our framework. Most importantly,
\cref{chapter_td} introduced several \emph{axioms} for truth discovery. We
proved the first impossibility results for truth discovery; this showed there
are fundamental constraints on which properties one can hope for when
constructing new truth discovery methods. The axioms highlighted a possible
problem with Sums -- a well known method introduced by
\textcite{pasternack2010} -- in its failure of \disjointindependence{}. This
led us to define a new method, called UnboundedSums, which modifies Sums to
resolve this axiomatic failure.

}

{
    \newcommand{\ch}{\mathcal{C}}
\newcommand{\minch}[1]{\operatorname{\mathcal{M}}\left({#1}\right)}
\newcommand{\minchmon}[1]{\operatorname{\mathcal{M}}_{\mathsf{mon}}\left({#1}\right)}
\newcommand{\minchw}[2]{\operatorname{\mathcal{M}_{#1}}({#2})}
\newcommand{\mindist}[1]{m({#1})}
\newcommand{\T}{T}
\newcommand{\K}{\mathcal{K}}
\newcommand{\N}{\mathbb{N}}
\newcommand{\R}{\mathbb{R}}
\newcommand{\ale}{\preceq}
\newcommand{\alt}{\prec}
\newcommand{\aeq}{\approx}
\newcommand{\asymb}{\mathcal{A}}
\newcommand{\bsymb}{\mathcal{B}}
\newcommand{\anle}{\leqslant^{\asymb}}
\newcommand{\aneq}{\approx^{\asymb}}
\newcommand{\anlt}{<^{\asymb}}
\newcommand{\bnle}{\leqslant^{\bsymb}}
\newcommand{\ble}{\sqsubseteq}
\newcommand{\blt}{\sqsubset}
\newcommand{\beq}{\approx}
\newcommand{\tr}{\top}
\newcommand{\dual}[1]{{\overline{#1}}}
\newcommand{\vect}{\operatorname{vec}}
\newcommand{\argmin}{\operatorname*{arg\ min}}
\newcommand{\argmax}{\operatorname*{arg\ max}}
\newcommand{\rs}{\upharpoonright}
\newcommand{\dotprod}{\bullet}
\newcommand{\tpos}[1]{\operatorname{\mathcal{L}}({#1})}
\newcommand{\expectation}[2]{\operatorname*{\mathbb{E}}_{#1}{#2}}
\newcommand{\orderings}[1]{\operatorname*{\mathcal{L}}({#1})}
\newcommand{\cherr}{\varepsilon}
\newcommand{\avgmin}{\text{avg}}
\newcommand{\symdiff}{\mathrel{\triangle}}
\newcommand{\Tcount}{{\T_{\mathsf{count}}}}
\newcommand{\Tcardint}{{\T_{\mathsf{CI}}}}
\newcommand{\Tlex}{{\T_{\mathsf{lex}}}}
\newcommand{\tuple}[1]{\langle{#1}\rangle}
\newcommand{\ranks}[1]{\mathsf{ranks}({#1})}
\renewcommand{\intop}[1]{{\T_{#1}^\mathsf{int}}}
\newcommand{\swap}[3]{\mathsf{swap}({#1}; {#2}, {#3})}

% axioms
\newcommand{\chainmin}{\axiomref{Chain-min}}
\newcommand{\anon}{\axiomref{Anon}}
\newcommand{\dualaxiom}{\axiomref{Dual}}
\newcommand{\iim}{\axiomref{IIM}}
\newcommand{\mon}{\axiomref{Mon}}
\newcommand{\posresp}{\axiomref{Pos-resp}}
\newcommand{\chaindef}{\axiomref{Chain-def}}
\newcommand{\smi}{\axiomref{SMI}}
\newcommand{\rankremoval}{\axiomref{Rank-removal}}
\newcommand{\argmaxaxiom}{\axiomref{Argmax}}


\cref{chapter_bipartite_tournaments} stayed with the social choice methodology,
this time applied to bipartite tournament ranking. This problem was motivated
by truth discovery -- in particular, by semi-supervised truth discovery -- and
also relates to assessing trustworthiness of sources. Concretely, bipartite
tournaments naturally model situations where the performance of sources on a
number of ``ground truth'' objects is known in advance. One can then use
bipartite tournament ranking methods to rank the sources by trustworthiness
(with respect to the ground truth) and the objects by difficulty. But in fact,
we gave several other example domains where bipartite tournaments provide a
natural model, such as education and solo sports. We therefore took a more
general view throughout the chapter and considered \emph{abstract} bipartite
tournaments, where no fixed interpretation was assigned to the players involved
or the nature of the matches between them. The overall approach was again
axiomatic.  We introduced several axioms; some adaptations of standard social
choice ideas and some specific to bipartite tournaments (e.g. \dualaxiom). We
also paid special attention to the class of \emph{chain-minimal} operators, the
ranking operators associated with the graph modification problem of \emph{chain
editing}. Such operators were rationalised by a result showing them to be
maximum likelihood estimators in a particular probabilistic model. In a sense
this is a truth-tracking result: chain-minimal operators find the ``true''
strengths of the tournament participants, if one accepts the assumptions of the
probabilistic model. However, this is a different sense of truth-tracking as
studied in \cref{chapter_truthtracking}, which refers to finding true
\emph{information} when given conflicting reports.

}

The remainder of the thesis employed logic-based formalisms. The modal logic
framework of \cref{chapter_expertise} studied expertise and ``soundness'' of
information. Here we took a deep dive into conceptual and mathematical
properties of expertise. For instance, we established precise connections
between expertise and \emph{knowledge} via one-to-one mappings between classes
of expertise models and relational models of epistemic logic. Roughly speaking,
we saw that expertise on $\phi$ is equivalent to the statement that the source
knows $\neg\phi$ in all states where $\phi$ is false. The properties of
knowledge involved in this translation depend on the properties one takes for
expertise; if expertise collections are closed under intersections and unions
we get S4 knowledge, and closure under complements strengthens this to S5.
Technical results included axiomatisation results for various classes of
expertise models, with completeness for the class of all models requiring a
novel and non-trivial proof technique.

Importantly, \cref{chapter_expertise} laid the groundwork for subsequent
chapters. \cref{chapter_belief_change} took up the expertise framework in
pursuit of roughly the same goals as truth discovery: to handle conflicting
reports from multiple sources, and to both form beliefs about the world and
assess the trustworthiness of sources. The \emph{objects} of truth discovery
played a similar role to the \emph{cases} in the belief change problem; in both
problems the patterns of reports across objects/cases allows operators to
reason about the trustworthiness or expertise of sources. The representations
in use differed, though: in \cref{chapter_td} we used rankings of sources and
claims, whereas \cref{chapter_belief_change} used sets of logical formulas.
While rankings also played a major role in the construction of our belief and
expertise operators, it was through rankings of \emph{worlds} rather than
sources. We argued on \cpageref{chapter_interlude} that neither choice is
better than the other; they merely give different perspectives on the broader
problem. The result was that our axioms/postulates and operators share high
level intuitions in places, but vary in their technical manifestation.

{
    % tikz styles
\tikzset{drcells/.style={
    rounded corners=5mm,draw=blue,line width=0.6mm
}}
\tikzset{techcells/.style={
    draw=red,line width=0.6mm
}}
\tikzset{vals/.style={
    draw=black!60!green,line width=0.5mm
}}


% postulates
\newcommand{\equivalence}{\axiomref{Equivalence}}
\newcommand{\repetition}{\axiomref{Repetition}}
\newcommand{\soundness}{\axiomref{Soundness}}
\newcommand{\credulity}{\axiomref{Credulity}}
\newcommand{\refinement}{\axiomref{Refinement}}

% logical framework symbols
\renewcommand{\phi}{\varphi}
\newcommand{\srcs}{\mathcal{S}}
\newcommand{\css}{\mathcal{C}}
\newcommand{\propvars}{\mathsf{Prop}}
\newcommand{\vals}{\mathcal{V}}
\newcommand{\lang}{\mathcal{L}}
\newcommand{\Lprop}{\lang_0}
\renewcommand{\L}{\lang}
\newcommand{\E}{\mathsf{E}}
\renewcommand{\S}{\mathsf{S}}
\newcommand{\B}{\mathsf{B}}
\renewcommand{\b}{\mathcal{B}}
\newcommand{\limplies}{\rightarrow}
\newcommand{\liff}{\leftrightarrow}
\newcommand{\worlds}{\mathcal{W}}
\newcommand{\propmodels}{\Vdash}
\newcommand{\propmods}[1]{\|{#1}\|}
\newcommand{\falsum}{\bot}
\newcommand{\cnprop}[1]{\operatorname{Cn}_0\left({#1}\right)}
\newcommand{\parts}{\mathcal{P}}
\newcommand{\repr}[1]{\widehat{#1}}
\newcommand{\bel}[3]{\operatorname{Bel}_{{#1}}({#2}, {#3})}

% methods
\newcommand{\methodname}[1]{\mathsf{#1}}
\newcommand{\vbc}{\methodname{vbc}}
\newcommand{\pbc}{\methodname{pbc}}
\newcommand{\exm}{\methodname{exm}}

% orders
\newcommand{\refines}{\preceq}
\newcommand{\specleq}{\sqsubseteq}
\newcommand{\speclt}{\sqsubset}
\newcommand{\speceq}{\approx}
\newcommand{\seqss}{\preceq}

% misc
\newcommand{\tuple}[1]{\langle{#1}\rangle}
\newcommand{\N}{\mathbb{N}}
\newcommand{\X}{\mathcal{X}}
\newcommand{\snd}{\mathsf{snd}}
\newcommand{\Xsnd}{\X^{\snd}}
\newcommand{\Xbel}[1]{\X^{#1}}
\newcommand{\seqappend}{\circ}
\newcommand{\argmin}{\operatornamewithlimits{argmin}}
\newcommand{\plurality}{\mathsf{pl}}

% example characters
\newcommand{\dr}{\mathsf{D}}
\newcommand{\tech}{\mathsf{T}}
\newcommand{\patientA}{A}
\newcommand{\patientB}{B}


Finally, \cref{chapter_truthtracking} took the same framework of the belief
change problem to consider not just how to form \emph{rational} beliefs on the
basis of non-expert information, but how to find the \emph{truth}. This
concerned both finding the facts about the world in each case -- i.e. reliably
\emph{aggregating} reports -- and finding the true expertise of each source.
We saw that when sources make all possible sounds reports, an operator must be
somewhat \emph{trusting}; this was expressed formally by the \credulity{}
axiom. Consequently truth-tracking is inherently \emph{non-monotonic}. We also
found precise conditions under which the facts of the world can be found by
truth-tracking methods, which gives an insight into the extent to which
truth-tracking is possible at all with non-expert sources. Indeed, the running
example of \cref{chapter_truthtracking} served as an example where the truth
cannot be found in some cases due to lack of relevant expertise.

}

\section{Future Work}
