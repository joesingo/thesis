\chapter{Introduction}

\begin{notes}
    \begin{itemize}
        \item Overall theme: how should we deal with unreliable information?
        \item We want to:
        \begin{itemize}
            \item Aggregate conflicting reports (crowdsourcing, news)
            \item Assess the trustworthiness of information sources
            \item Understand what reliability, trustworthiness and expertise
                  \emph{mean}
            \item Find the truth without imperfect information
        \end{itemize}
        \item This thesis offers two main perspectives on these general themes
        \begin{itemize}
            \item \textbf{Social choice theory.}
            \begin{itemize}
                \item By posing the aggregation problem as one of social
                      choice, we can apply the axiomatic method to investigate
                      desirable properties of aggregation methods. We can then
                      analyse and evaluate such methods in a formal and
                      principled way.
                \item Related ranking problems can be addressed through the
                      lens of social choice.
            \end{itemize}
            \item \textbf{Logic and knowledge representation.}
            \begin{itemize}
                \item We develop a logical system to formalise notions of
                      expertise, and explore connections with knowledge and
                      information.
                \item We use these formal notions to express the aggregation
                      problem in logical terms, taking an alternative look at
                      the problems of the first part of the thesis. We use what
                      is essentially still an axiomatic approach, but now in
                      the tradition of knowledge representation and rational
                      belief change.
                \item This logical model is well-suited to investigate
                      \emph{truth-tracking}: the question of when we can find
                      the truth given that not all sources are experts.
            \end{itemize}
        \end{itemize}

        \item Note that while there are many links between the two major parts,
              they are not tightly connected and may be read independently.

    \end{itemize}
\end{notes}

\chapter{Thesis Outline}
