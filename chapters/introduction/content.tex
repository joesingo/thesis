\chapter{Introduction}

% \begin{notes}
%     \begin{itemize}
%         \item Overall theme: how should we deal with unreliable information?
%         \item We want to:
%         \begin{itemize}
%             \item Aggregate conflicting reports (crowdsourcing, news)
%             \item Assess the trustworthiness of information sources
%             \item Understand what reliability, trustworthiness and expertise
%                   \emph{mean}
%             \item Find the truth with imperfect information
%         \end{itemize}
%         \item This thesis offers two main perspectives on these general themes
%         \begin{itemize}
%             \item \textbf{Social choice theory.}
%             \begin{itemize}
%                 \item By posing the aggregation problem as one of social
%                       choice, we can apply the axiomatic method to investigate
%                       desirable properties of aggregation methods. We can then
%                       analyse and evaluate such methods in a formal and
%                       principled way.
%                 \item Related ranking problems can be addressed through the
%                       lens of social choice.
%             \end{itemize}
%             \item \textbf{Logic and knowledge representation.}
%             \begin{itemize}
%                 \item We develop a logical system to formalise notions of
%                       expertise, and explore connections with knowledge and
%                       information.
%                 \item We use these formal notions to express the aggregation
%                       problem in logical terms, taking an alternative look at
%                       the problems of the first part of the thesis. We use what
%                       is essentially still an axiomatic approach, but now in
%                       the tradition of knowledge representation and rational
%                       belief change.
%                 \item This logical model is well-suited to investigate
%                       \emph{truth-tracking}: the question of when we can find
%                       the truth given that not all sources are experts.
%             \end{itemize}
%         \end{itemize}
%         \item Note that while there are many links between the two major parts,
%               they are not tightly connected and may be read independently.
%     \end{itemize}
% \end{notes}

The overall theme of this thesis is \emph{unreliable information}. How should
unreliable information be aggregated? Who should be trusted when conflicts
arise between unreliable sources? And what, if anything, can be \emph{learned}
from non-expert information? These are the central issues the thesis aims to
address.

Indeed, methods for understanding and reasoning with unreliable information are
becoming ever more relevant in today's world, as the volume of data produced
and consumed grows year-on-year \todo{c}. With the growth of user-generated
content on the internet, most prominently on social media platforms, false
information can spread rapidly -- sometimes with dramatic
consequences \todo{c}. As such, much research effort has gone into identifying
false information, estimating source reliability, and understanding the nature
of how people may come to believe and share false information \todo{cs}.

The gap this thesis aims to fill concerns \emph{formal models} of problems
surrounding unreliable information. We take a mathematical approach, putting
forward formal frameworks in which the relevant problems can be formulated
precisely. In doing so we obtain conceptual results which aim to shine light on
the core, abstract features of such problems. It is hoped that the thesis
complements the various empirical, practical and philosophical approaches
\todo{c} to our topic, and contributes to the broader understanding of
trustworthiness, expertise and unreliable information from a mathematical point
of view.

The thesis is split into two parts along methodological lines. In
\cref{chapter_td,chapter_bipartite_tournaments} we use the tools and ideas of
\emph{social choice theory}~\cite{zwicker2016voting} to explore the problems of
aggregating unreliable information and ranking sources by trustworthiness.
%
% We employ the \emph{axiomatic method}, in which intuitively desirable
% properties called \emph{axioms} are introduced and studied. Aggregation and
% ranking methods can then be evaluated and compared with respect to these
% axioms, often leading to fresh insights on both individual methods and the
% nature of the problem itself.
%
\cref{chapter_expertise,chapter_belief_change,chapter_truthtracking} take a
\emph{logic-based} approach. We develop a modal logic framework to give precise
semantics for expertise, and explore the connection between expertise,
knowledge, and the truthfulness of information. This framework is used as the
foundation for a belief change problem in the tradition of knowledge
representation and rational belief change \todo{c}. Finally, we combine ideas
from \emph{formal learning theory} \todo{c} (and in particular, the
intersection of formal learning theory and belief revision \todo{c}) to
investigate the extent to which one can \emph{learn} from unreliable
information.
%
In the remainder of this chapter we briefly survey the literature for both
halves, and outline the main contributions of the thesis.

\section{Social Choice Perspectives}

Broadly speaking, social choice theory is the study of \emph{aggregating
preferences}. The prototypical example is \emph{voting}. In an election, each
member of the electorate submits a vote in the form of their preferences over
the candidates. A voting rule then aggregates these preferences into a
\emph{collective decision} by declaring the winning candidate, the runner-up,
and so on. There are, of course, many different voting methods which can be
used to aggregate votes, and several distinct methods are in use in different
contexts across the world \todo{c}.

In analysing and comparing such methods, the \emph{axiomatic approach} has been
a crucial methodological tool since the seminal work of \textcite{arrow1952},
who initiated the age of so-called \emph{classical social choice theory}. In
taking this approach, one formalises intuitively desirable properties called
\emph{axioms}, which are expected to hold for ``reasonable'' aggregation
methods. The benefits of this approach were already shown by Arrow, who proved
it is mathematically impossible for any voting method to simultaneously satisfy
a short list of seemingly desirable axioms. This type of result -- known as an
\emph{impossibility theorem} -- highlights a fundamental and inescapable
property of voting.\footnotemark{} This has practical consequences: if one
needs to actually choose a rule for use in a vote, which axiom will be
sacrificed? The axioms also provide a normative basis on which to judge and
compare different voting rules.

The axiomatic approach has since been adapted to various domains besides
voting, including tournaments~\cite{brandt2016a}, judgement
aggregation~\cite{endriss2016ja}, the ranking of web
pages~\cite{altman2005ranking}, reputation systems~\cite{tennenholtz2004} and
collective annotation~\cite{kruger2014}.

\footnotetext{
    At least, a property of voting in the sense of Arrow's formal framework.
}

In the last 20 years, \emph{computational social choice}~\cite{moulin_2016} has
combined social choice theory with ideas from theoretical computer science. For
example, complexity theory has been used to show certain voting rules are
resistant to strategic manipulation, approximation algorithms have been
developed for computationally difficult rules, and SAT solvers have been used
to automatically discover new impossibility theorems \todo{cs}.

In the spirit of combining the axiomatic approach with computer science, we
proceed in \cref{chapter_td,chapter_bipartite_tournaments} by introducing a
social-choice-style framework and several axioms for the problems of
\emph{truth discovery} and \emph{bipartite tournament ranking}. \todo{explain
what these are.}

As it turns out, many existing axioms from different problems in the literature
can be applied in our settings; if not directly, then the underlying idea can
be adapted. By looking at the consequences of such axioms we obtain insights
into the similarities and differences between our problems and other social
choice problems. \todo{explicit contributions?}

\section{Logic-based Perspectives}

\paragraph{Modal logic.}

{

\renewcommand{\phi}{\varphi}

\newcommand{\Prop}{\mathsf{Prop}}

\newcommand{\R}{\mathbb{R}}
\newcommand{\N}{\mathbb{N}}

\newcommand{\limplies}{\to}
\newcommand{\limpliedby}{\leftarrow}
\newcommand{\liff}{\leftrightarrow}
\newcommand{\bigland}{\bigwedge}
\newcommand{\biglor}{\bigvee}

\newcommand{\modalnec}{\Box}
\newcommand{\modalposs}{\Diamond}

\newcommand{\E}{\mathsf{E}}
\renewcommand{\S}{\mathsf{S}}
\newcommand{\A}{\mathsf{A}}
\newcommand{\K}{\mathsf{K}}

\newcommand{\J}{\mathcal{J}}
\newcommand{\dist}{\mathsf{dist}}
\newcommand{\shared}{\mathsf{sh}}
\newcommand{\common}{\mathsf{com}}

\newcommand{\Kdist}{\mathsf{K}^\dist}
\newcommand{\Kshared}{\mathsf{K}^\shared}
\newcommand{\Kcommon}{\mathsf{K}^\common}

\newcommand{\Pdist}{P^\dist}
\newcommand{\Pcommon}{P^\common}
\newcommand{\Rdist}{R^\dist}
\newcommand{\Rcommon}{R^\common}

\newcommand{\cL}{\mathcal{L}}
\newcommand{\cLSA}{\mathcal{L}_{\S\A}}
\newcommand{\cLJ}{\cL^{\J}}
\newcommand{\cLKA}{\mathcal{L}_{\K\A}}
\newcommand{\cLKAJ}{\cLKA^{\J}}
\newcommand{\cLzero}{\cL_0}

\newcommand{\sL}{\mathsf{L}}
\newcommand{\sLint}{\mathsf{L}_{\mathsf{int}}}
\newcommand{\sLtop}{\mathsf{L}_{\mathsf{top}}}
\newcommand{\sLintcompl}{\mathsf{L}_{\mathsf{int-compl}}}
\newcommand{\sLsfoura}{\mathsf{L}_{\mathsf{S4A}}}
\newcommand{\sLsfivea}{\mathsf{L}_{\mathsf{S5A}}}

\newcommand{\axiomm}[1]{(#1)}  % double m to not conflict with global axiom env
\newcommand{\axiomstyle}{\text}
\newcommand{\EA}{\axiomm{\axiomstyle{EA}}}
\newcommand{\weakeningE}{\axiomm{\axiomstyle{W}_{\E}}}
\newcommand{\redE}{\axiomm{\axiomstyle{Red}_{\E}}}
\newcommand{\weakeningS}{\axiomm{\axiomstyle{W}_{\S}}}
\newcommand{\reE}{\axiomm{\axiomstyle{RE}_\E}}
\newcommand{\Kuniv}{\axiomm{\axiomstyle{K}_{\A}}}
\newcommand{\Kk}{\axiomm{\axiomstyle{K}_{\K}}}
\newcommand{\Tk}{\axiomm{\axiomstyle{T}_{\K}}}
\newcommand{\fourk}{\axiomm{\axiomstyle{4}_{\K}}}
\newcommand{\fivek}{\axiomm{\axiomstyle{5}_{\K}}}
\newcommand{\inck}{\axiomm{\axiomstyle{Inc}_{\K}}}
\newcommand{\Tuniv}{\axiomm{\axiomstyle{T}_{\A}}}
\newcommand{\fiveuniv}{\axiomm{\axiomstyle{5}_{\A}}}
\newcommand{\necuniv}{\axiomm{\axiomstyle{Nec}_{\A}}}
\newcommand{\Ksoundness}{\axiomm{\axiomstyle{K}_{\S}}}
\newcommand{\Tsoundness}{\axiomm{\axiomstyle{T}_{\S}}}
\newcommand{\foursoundness}{\axiomm{\axiomstyle{4}_{\S}}}
\newcommand{\fivesoundness}{\axiomm{\axiomstyle{5}_{\S}}}
\newcommand{\modpon}{\axiomm{\axiomstyle{MP}}}
\newcommand{\inc}{\axiomm{\axiomstyle{Inc}}}

\newcommand{\thm}{\vdash}
\newcommand{\entails}{\thm}

\newcommand{\M}{\mathbb{M}}
\newcommand{\Mint}{\M_{\mathsf{int}}}
\newcommand{\Munions}{\M_{\mathsf{unions}}}
\newcommand{\Mfunions}{\M_{\mathsf{finite-unions}}}
\newcommand{\Mtop}{\M_{\mathsf{top}}}
\newcommand{\Mcompl}{\M_{\mathsf{compl}}}
\newcommand{\Mrel}{\M^*}
\newcommand{\Msfour}{\M^*_{\mathsf{S4}}}
\newcommand{\Msfive}{\M^*_{\mathsf{S5}}}
\newcommand{\MJ}{\M^\J}
\newcommand{\MintJ}{\Mint^\J}
\newcommand{\MunionsJ}{\Munions^\J}
\newcommand{\McomplJ}{\Mcompl^\J}
\newcommand{\MsfourJ}{\M^{\J}_{\mathsf{S4}}}
\newcommand{\MsfiveJ}{\M^{\J}_{\mathsf{S5}}}

\newcommand{\expinc}{+}
\newcommand{\sndann}{?}

\newcommand{\econ}{\mathsf{econ}}
\newcommand{\dr}{\mathsf{dr}}
\newcommand{\analyst}{\mathsf{analyst}}


In the logic-based part of the thesis, we start with a \emph{modal logic}
framework for reasoning about expertise. A modal language augments a
propositional logical language with one or more \emph{modalities} which qualify
the truth value of a proposition~\cite{seplogicmodal}. A typical interpretation
is \emph{necessity}: $\modalnec\phi$ means the proposition $\phi$ is
\emph{necessarily true}, as opposed to merely being true. The dual notion of
\emph{possibility}, denoted $\modalposs\phi$, is defined in terms of necessity
by $\modalposs\phi \equiv \neg\modalnec\neg\phi$. That is, $\phi$ is
\emph{possibly true} if it is not necessarily false.

Various senses of ``necessity'' give rise to a rich landscape of logical
systems, useful for modelling various domains. For example, in temporal logics
$\modalnec\phi$ means $\phi$ will necessarily always hold in the future. In
deontic logics, $\modalnec\phi$ means $\phi$ is a moral necessity; it is
obligatory for $\phi$ to hold. In epistemic and doxastic logics,
$\modalnec\phi$ means $\phi$ is necessary from the point of view of an agent's
knowledge or beliefs about the world; one typically writes $\K\phi$ or
$\mathsf{B}\phi$ instead of $\modalnec\phi$ in these cases, to express that the
agent ``knows'' or ``believes'' $\phi$ \todo{cs}.

The most prominent semantic interpretation of modal formulas is based on
\emph{relational models} (also known as \emph{Kripke models}). The key
ingredients here are a set of \emph{states} and a binary relation called the
\emph{accessibility relation}. The modal formula $\modalnec\phi$ holds at a
state $x$ if $\phi$ holds at all states $y$ accessible from $x$.\footnotemark{}
In this way the accessibility relation directly reflects which states are
``possible'' from others.
%
Later, \emph{neighbourhood semantics} were developed by \textcite{Scott1970}
and \textcite{montague1970universal}, which generalise relational semantics.
The idea is to replace the accessibility relation with a so-called
\emph{neighbourhood function}, which assigns to each state a collection of sets
of states (called its \emph{neighbourhood}). This neighbourhood explicitly
lists the ``necessary'' propositions at each state: $\modalnec\phi$ holds at
$x$ if the set of states where $\phi$ holds is a member of the neighbourhood of
$x$~\cite{pacuit2017neighborhood}.
%
Further still, \emph{topological semantics} (also called the \emph{interior
semantics}) were first studied mathematically by \textcite{mckinsey41} and
\textcite{mckinseytarski44}, and later reinterpreted in epistemic terms (see
\textcite[Chapter 1]{ozgun_evidence} for a historical overview). Here one
equips the set states with a topology, and $\modalnec\phi$ holds at all points
in the \emph{interior} of the set where $\phi$ holds. That is, $\phi$ is
necessary at a state $x$ when there is an open neighbourhood of $x$ in which
$\phi$ holds at all points.

\footnotetext{
    A formal definition will be given in \cref{exp_sec_connection_with_ep_logic};
    for now we only wish to sketch the underlying ideas.
}

Relational, neighbourhood and topological semantics have each proven to be
useful for modelling notions related to our topic, such as information, trust,
belief, and evidence.

On trust, \textcite{Liau_2003} considered modalities $B_i\phi$ (agent $i$
believes $\phi$), $I_{ij}\phi$ ($i$ acquires information $\phi$ from $j$) and
$T_{ij}\phi$ ($i$ trusts $j$ on $\phi$), where trust has a neighbourhood
interpretation. \textcite{dastani2004inferring} extended this framework to
consider how trust may be inferred, using notions of topics and
\emph{questions}. \textcite{herzig2010logic} introduced notions of trust and
\emph{reputation}, in a framework where trust is not primitive but built from
beliefs, goals and actions. Further logical developments of trust were set out
by \textcite{rodenhauser2014matter} and \textcite{tagliaferri2019logical}; see
the references therein for a more thorough review of the literature on trust.

Interactions between knowledge, belief and \emph{evidence} have been studied in
epistemic logic. \textcite{moss1992topological} introduced the so-called subset
space semantics to model knowledge and \emph{epistemic effort}, which
represents a kind of evidence-gathering performed by an epistemic agent, and
has topological roots. \textcite{ozgun_evidence} further developed notions of
evidence in epistemic logic from a topological perspective. In a series of
papers, \textcite{van2011dynamic,van2012evidence,vanbenthem2014106} made
extensive use of neighbourhood structures to model evidence, and in particular
how inconsistent evidence can be combined to form beliefs.

The final strand of the modal literature we mention is \emph{dynamic epistemic
logic}~\cite{van_Ditmarsch_2008,sep_del}. Here the modal operators describe
\emph{actions}; in general one has formulas $[a]\phi$ to express that $\phi$ is
true after the action $a$ is performed~\cite{sep_del}. Examples include public
announcements~\cite{plaza2007logics}, testimony~\cite{holliday2009dynamic} --
both particularly interesting in the case of multiple agents -- belief
revision~\cite{baltag2008qualitative} and
learning~\cite{gierasimczuk2009bridging,gierasimczuk2010knowing}.

Our work in \cref{chapter_expertise} combines elements from each of the
above-surveyed areas of the literature to model \emph{expertise} and its
relation to relation to \emph{truthfulness of information}. Specifically, we
introduce a logical system with two new modalities: $\E\phi$, meaning the
source in question is an expert on $\phi$, and $\S\phi$, meaning the
information $\phi$ is ``sound'' for the source to report. Informally, the
latter notion means $\phi$ is true \emph{up to lack of expertise}, i.e. if the
report becomes true when discarding parts of the statement on which the
reporting source lacks expertise.

For example, suppose an economist reports that energy policies proposed by the
government will stimulate economic growth and help tackle climate change. Since
this goes beyond the expertise of the economist (who we assume is not a climate
expert), we should only take their comments on the economy into account. Our
notion of soundness models this kind of filtering: whereas the statement in its
entirety may be false (e.g. if the proposed policies are not in fact
climate-friendly), the report is \emph{sound} whenever the economist is correct
on its economic content.

On the technical side, we use (a special case of) neighbourhood semantics for
expertise, and topological semantics for soundness. We show in detail how such
notions relate to knowledge in epistemic logic under relational semantics.
Dynamic operators are also considered; we define an analogue of public
announcements (called ``sound announcements'') and consider ``expertise
increase'', which models the effects of a source increasing their domain of
expertise by learning.

\begin{notes}
    \begin{itemize}
        \item Other logical approaches to unreliable information: e.g.
              many-valued logics, possibilistic logic, rough sets (?).
        \item We introduce a modal logic of expertise, with a modality
              $\mathsf{E}$ representing expertise. To the best of our knowledge
              it is the first treatment of expertise explicitly in terms of
              modal logic
        \item Prime reader for the results: conceptual links with epistemic
              logic, collection notions of expertise, technical axiomatisation
              results, dynamics.
    \end{itemize}
\end{notes}

% different acccounts of trust.
% primitive:
%     liau
% non-primitive:
%     herzig (a logic of trust and reputation): built from beliefs, goals,
%       actions

}

\paragraph{Belief change.}

\begin{notes}
    \begin{itemize}
        \item AGM introduced postulates for rational belief change
        \item Uses propositional theories to represent beliefs. Such theories
              are to be revised by incoming reliable information
        \item List postulates in full
        \item Selective revision: drop Success.
        \item Representation of expertise (or credibility, etc) is still at the
              meta-level; go through constructions of Ferme and Hansson, Booth
              and Hunter
        \item Like AGM, we use logically closed theories to represent beliefs
        \item But we bring expertise to the object level by adapting the modal
              framework for expertise
        \item Trust is represented by belief in expertise: $i$ is trusted on
            $\phi$ if $\mathsf{E}_i\phi$ is in the belief set
        \item Our work is also close to belief merging. Give some background
              here, but refer reader to related work section of the actual
              chapter.
        \item Mention Yasser and Ismail's work at some point
    \end{itemize}
\end{notes}

\paragraph{Learning.}

\begin{notes}
    \begin{itemize}
        \item Inductive reasoning and learning
        \item e.g. Kevin Kelly's work, Nina
    \end{itemize}
\end{notes}

\section{Contributions}

\todo{Needed? Already mention contributions at the start of the introduction
and in each chapter.}

\section{Overview}

\begin{notes}
    Chapter-by-chapter breakdown of the thesis.
\end{notes}
