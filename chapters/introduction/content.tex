\chapter{Introduction}

% \begin{notes}
%     \begin{itemize}
%         \item Overall theme: how should we deal with unreliable information?
%         \item We want to:
%         \begin{itemize}
%             \item Aggregate conflicting reports (crowdsourcing, news)
%             \item Assess the trustworthiness of information sources
%             \item Understand what reliability, trustworthiness and expertise
%                   \emph{mean}
%             \item Find the truth with imperfect information
%         \end{itemize}
%         \item This thesis offers two main perspectives on these general themes
%         \begin{itemize}
%             \item \textbf{Social choice theory.}
%             \begin{itemize}
%                 \item By posing the aggregation problem as one of social
%                       choice, we can apply the axiomatic method to investigate
%                       desirable properties of aggregation methods. We can then
%                       analyse and evaluate such methods in a formal and
%                       principled way.
%                 \item Related ranking problems can be addressed through the
%                       lens of social choice.
%             \end{itemize}
%             \item \textbf{Logic and knowledge representation.}
%             \begin{itemize}
%                 \item We develop a logical system to formalise notions of
%                       expertise, and explore connections with knowledge and
%                       information.
%                 \item We use these formal notions to express the aggregation
%                       problem in logical terms, taking an alternative look at
%                       the problems of the first part of the thesis. We use what
%                       is essentially still an axiomatic approach, but now in
%                       the tradition of knowledge representation and rational
%                       belief change.
%                 \item This logical model is well-suited to investigate
%                       \emph{truth-tracking}: the question of when we can find
%                       the truth given that not all sources are experts.
%             \end{itemize}
%         \end{itemize}
%         \item Note that while there are many links between the two major parts,
%               they are not tightly connected and may be read independently.
%     \end{itemize}
% \end{notes}

The overall theme of this thesis is unreliable information. How should
unreliable information be aggregated? Who should be trusted when conflicts
arise between information sources? And what, if anything, can be learned from
non-expert information? These are the central issues the thesis to address.

Indeed, methods for understanding and reasoning with unreliable information are
becoming ever more relevant in today's world, as the volume of data produced
and consumed grows year-on-year \todo{c}. With the growth of user-generated
content on the internet, most prominently on social media platforms, false
information can spread rapidly if unchecked -- sometimes with dramatic
consequences \todo{c}. As such, much research effort has gone into identifying
false information, estimating source reliability, and understanding the nature
of how people may come to believe and share false information \todo{cs}.

The gap this thesis aims to fill concerns \emph{formal models} of problems
surrounding unreliable information. We take a mathematical approach, putting
forward formal frameworks in which the relevant problems can be formulated
precisely. In doing so we obtain conceptual results shining light on their core
properties and limitations. It is hoped that the thesis complements the various
empirical, practical and philosophical approaches \todo{c} to our topic, and
contributes to the broad understanding of trustworthiness, expertise and
unreliable information from a mathematical point of view.

The thesis is split into two parts along methodological lines. In
\cref{chapter_td,chapter_bipartite_tournaments} we use the tools and ideas of
\emph{social choice theory}~\cite{zwicker2016voting} to explore the problems of
aggregating unreliable information and ranking sources by trustworthiness. We
employ the \emph{axiomatic method}, in which desirable properties called
\emph{axioms} are introduced and studied. Aggregation and ranking methods can
then be evaluated and compared with respect to these axioms, often leading to
fresh insights on both individual methods and the nature of the problem itself.
%
\cref{chapter_expertise,chapter_belief_change,chapter_truthtracking} take a
\emph{logic-based} approach. We develop a modal logic framework to give precise
semantics of expertise, and explore the connection between expertise,
knowledge, and the truthfulness of information. This framework is used as the
foundation for a belief change problem in the tradition of knowledge
representation and rational belief change \todo{c}. Finally, we combine ideas
from \emph{formal learning theory} \todo{c} (and in particular, the
intersection of formal learning theory and belief revision \todo{c}) to
investigate the extent to which one can \emph{learn} from unreliable
information.

We now briefly survey existing work both from social choice theory and from
logic.

\section{Social Choice Perspectives}

Loosely speaking, social choice theory is the study of \emph{aggregating
preferences}. The prototypical example is \emph{voting}. In an election, each
member of the electorate submits a vote in the form of their preferences over
the candidates. A voting rule then aggregates these preferences into a
\emph{collective decision} by declaring the winning candidate, the runner-up,
and so on. There are, of course, many different voting methods which can be
used to aggregate votes, and several distinct methods are in use in different
contexts across the world \todo{c}.

In analysing and comparing such methods, the \emph{axiomatic approach} has been
a crucial methodological tool since the seminal work of \textcite{arrow1952},
who initiated the age of so-called \emph{classical social choice theory}. In
taking this approach, one formulates formal, desirable properties called
\emph{axioms}, which are expected to hold for ``reasonable'' aggregation
methods. The benefits of this approach were already shown by Arrow, who proved
it is mathematically impossible for any voting method to simultaneously satisfy
a short list of seemingly desirable axioms. This type of result -- known as an
\emph{impossibility theorem} -- highlights a fundamental and inescapable
property of voting.\footnotemark{} There are also practical consequences. If
one needs to actually choose a rule for use in a vote, which axiom will be
sacrificed?

\footnotetext{
    At least, a property of voting in the sense of Arrow's formal framework.
}

The axiomatic approach has since been adapted to various domains besides
voting; e.g. tournaments~\cite{brandt2016a}, judgement
aggregation~\cite{endriss2016ja}, the ranking of web
pages~\cite{altman2005ranking}, reputation systems~\cite{tennenholtz2004} and
collective annotation~\cite{kruger2014}. It is in this spirit we proceed in
\cref{chapter_td,chapter_bipartite_tournaments}, introducing a
social-choice-style framework and several axioms for the problems of
\emph{truth discovery} and \emph{bipartite tournament ranking}. As it turns
out, many existing axioms from different problems in the literature can be
applied in our settings; if not directly, then the underlying idea can be
adapted to fit our problem. By looking at the consequences of such axioms we
obtain insights into the similarities and differences between our problems and
other social choice problems.

\section{Logic-based Perspectives}

\begin{notes}
    \begin{itemize}
        \item Write a bit about AGM revision (revising propositional theories
              by reliable information) and explicitly list the postulates.
    \end{itemize}
\end{notes}

\section{Overview}

\begin{notes}
    Chapter-by-chapter breakdown of the thesis.
\end{notes}
