\chapter{Belief Change with Non-Expert Sources}
\label{chapter_belief_change}

In the previous chapter we introduced a logical framework to reason about the
expertise of sources and soundness of information. We now build on this
framework to study a belief change problem in which expertise is not fixed
upfront, but is to be estimated on the basis of reports from multiple sources.
In this way we develop a logic-based analogue of the truth discovery
aggregation problem, which complements the social-choice-style framework of
\cref{chapter_td}. Specifically, we identify trustworthiness with \emph{belief
in expertise}: an agent deems source $i$ trustworthy on $\phi$ if its belief
set contains $\E_i\phi$. By also including
propositional formulas in belief sets, we are able to express beliefs about the
actual world, i.e. the aggregation of the reports from sources.

Our point of departure is logic-based belief change in the AGM
paradigm (named after \textcite{alchourron1985logic}).
%
\todo{
    Refer back to introduction for an overview of AGM.
}
%
To illustrate the problem -- and how it
differs from existing forms of belief change -- consider the following scenario
in a hospital. Suppose we observe the results of a blood test of patient $\patientA$,
confirming condition $p$. Assuming the test is reliable, AGM revision tells us
how to revise our beliefs in light of the new information. Dr.\ $\drX$ then claims
that patient $\patientB$ suffers from the same condition, but Dr.\ $\drY$ disagrees. Given
that doctors specialise in different areas and may make mistakes, who should we
trust?
%
Since the \axiomref{Success} postulate ($\alpha \in K \ast \alpha$) assumes
information is reliable, we are outside the realm of AGM revision, and must
instead apply some form of \emph{non-prioritised}
revision~\cite{hansson1999survey}.

Suppose it now emerges that $\drX$ had earlier claimed $\patientA$ did \emph{not}
suffer from condition p, contrary to the test results. We now have reason to
suspect $\drX$ may \emph{lack expertise} on diagnosing p, and may subsequently
revise beliefs about $\drX$'s domain of expertise and the status of patient
$\patientB$ (e.g. by opting to trust $\drY$ instead).

While simple, this example illustrates the key features of the belief change
problem we study: we consider multiple sources, whose expertise is \emph{a
priori} unknown, providing reports on various instances of a problem domain. On
the basis of these reports we form beliefs both about the expertise of the
sources and the state of the world in each instance.

By including a distinguished \emph{completely reliable} source (the test
results in the example) we extend AGM revision. This is also analogous to
\emph{semi-supervised truth
discovery}~\cite{yin_supervised_2011,rekatsinas2017slimfast}, in which some
ground truths are known ahead of time. In some respects we also extend
approaches to non-prioritised revision (e.g. selective
revision~\cite{ferme1999selective}, credibility-limited
revision~\cite{hansson_2001}, and trust-based
revision~\cite{booth_trust_2018}), which assume information about the
reliability of sources is known up front. The problem is also related to
\emph{belief merging}~\cite{konieczny2002merging} which deals with combining
belief bases from multiple sources; a detailed comparison will be given in
\cref{kr_sec_relatedwork}.

Our work is also connected to trust and belief revision. As
\textcite{yasser_21} note in recent work, trust and belief are inexorably
linked: we should accept reports from sources we believe are trustworthy, and
we should trust sources whose reports turn out to be reliable. Trust and belief
should also be revised in tandem, so that we may increase or decrease trust in
a source as more reports are received, and revoke or reinstate previous reports
from a source as its perceived trustworthiness changes.

To unify the trust and belief aspects, we work in (a fragment of) the language
of expertise and soundness from \cref{chapter_expertise}, including formulas of
the form $\E_i\phi$, read as ``source $i$ has expertise on $\phi$'', and
$\S_i\phi$, read as ``$\phi$ is sound for source $i$ to report''. The output of
our belief change problem is then a collection of belief and knowledge sets in
the language, describing what we \emph{know} and \emph{believe} about the
expertise of the sources and the state of the world in each instance. For
example, we should \emph{know} reports from the reliable source are true,
whereas reports from ordinary sources may only be believed.

We do, however, make some simplifying assumptions compared to the modal
framework in \cref{chapter_expertise}. Firstly, we only consider only a finite
set of propositional variables, and identify states of the world with
propositional valuations. Secondly, we assume expertise is closed under both
intersections and complements, so that -- by \cref{exp_thm_s5_semantic_link} --
expertise of a source is fully captured by an equivalence relation; or
equivalently, a \emph{partition} of the propositional valuations.  In other
words, each source has a \emph{indistinguishibility relation} over valuations,
whereby any two valuations in the same partition cell are indistinguishable.

The semantics of expertise and soundness can be expressed directly in terms of
partitions; we have that a source is an expert on a proposition $\phi$ exactly
when they can distinguish every $\phi$ valuation from every $\neg\phi$
valuation, and $\phi$ is sound for $i$ if the ``actual'' state of the world is
indistinguishable from a $\phi$ valuation.

We then make the assumption that \emph{sources only report sound propositions}.
That is, reports are only false due to sources overstepping the bounds of their
expertise. In particular, we assume sources are honest in their reports, and
that experts are always right.

Note that in our introductory example, the fact that we had a report from Dr.\ $\drX$
on patient $\patientA$ (together with reliable information on patient $\patientA$) was essential
for determining the expertise of $\drX$, and subsequently the status of patient
$\patientB$. While the patients are independent, reports on one can cause beliefs about
the other to change, as we update our beliefs about the expertise of the
sources.

In general we consider an arbitrary number of \emph{cases}, which are
seen as labels for instances of the domain. For example, a crowdsourcing worker
may label multiple images, or a weather forecaster may give predictions for
different locations. Each report in the input to the
problem then refers to a specific case. Via these cases and the presence of the
completely reliable source, we are able to model scenarios where some ``ground
truth'' is available, listing how often sources have been correct/incorrect on a
proposition (e.g. the \emph{report histories} of
\textcite{hunter_building_21}). We can also generalise this scenario, e.g. by
having only partial information about ``previous'' cases.

Throughout the chapter we make the assumption that \emph{expertise is fixed
across cases}: the expertise of a source does not depend on the particular
instance of the domain we look at. For instance, the expertise of Dr.\ $\drX$ is the
same for patient $\patientA$ as for patient $\patientB$. This is a simplifying assumption, and may
rule out certain interpretations of the cases (e.g. if cases represent
different points in time, it would be natural to let expertise evolve over
time as per the dynamics of \cref{exp_sec_dynamic_extension}).

\paragraph{Contributions.} The main contribution of this chapter is the
formulation of a belief change problem in the setting of the logic of expertise
developed in the previous chapter. This allows us to explore how belief and
expertise-based trust should interact and evolve as reports are received from
the various sources.  We put forward several postulates and two concrete
classes of operators -- with a representation result for one class -- and
analyse these operators with respect to the postulates.

\begin{chapteroutline}
    In \cref{kr_sec_framework} we set out the formal framework.
    \Cref{kr_sec_the_problem} introduces the problem and lists some core
    postulates. We give two constructions and specific example operators in
    \cref{kr_sec_constructions}. \Cref{kr_sec_one_step_postulates} introduces
    some further postulates concerning belief change on the basis of one new
    report. An analogue of selective revision~\cite{ferme1999selective} is
    presented \cref{kr_sec_selective_change}. \Cref{kr_sec_relatedwork}
    discusses related work, and we conclude in \cref{kr_sec_conclusion}.
\end{chapteroutline}

\section{The Framework}
\label{kr_sec_framework}

Let $\srcs$ be a finite set of information sources. For convenience, we assume
there is a \emph{completely reliable} source in $\srcs$, which we denote by
$\ast$. For example, we can treat our first-hand observations as if they are
reported by $\ast$. Other sources besides $\ast$ will be termed \emph{ordinary
sources}. Let $\C$ be a finite set of \emph{cases}, which we interpret as
labels for different instances of the problem domain.

\paragraph{Syntax.}

In this chapter we work with the fragment of the language from
\cref{chapter_expertise}, in which expertise and soundness formulas are
restricted to propositional formulas only\footnotemark{} and the universal
modality $\mathsf{A}$ is excluded. Concretely, we assume
a fixed \emph{finite} set $\propvars$ of propositional variables, and let
$\lprop$ denote the set of propositional formulas generated from $\propvars$
using the usual propositional connectives. Formulas in $\lprop$ are used to
describe properties of the world in each case $c \in \C$. We use lower case
Greek letters ($\phi$, $\psi$ etc) for formulas in $\lprop$. The classical
logical consequence operator will be denoted by $\cnprop$, and $\equiv$ denotes
equivalence of propositional formulas.

\footnotetext{
    While this assumption is made for simplicity's sake, we do not lose much by
    excluding iterated applications of $\E$ and $\S$, at least for expertise
    models closed under intersections and complements. Indeed, we have that
    $\E\phi$ either holds globally in a model or holds nowhere, so
    $\E\E\phi$ always holds. One can show that $\E\S\psi$ also always holds in
    such models, by taking $\phi = \S\phi$ in \cref{exp_prop_frame_conditions}
    \cref{exp_item_frame_condition_intersections} and recalling that $\S\S\psi
    \limplies \S\psi$ is valid. Similarly, one can show that $\S\S\phi \liff
    \S\phi$ and $\S\E\phi \liff \E\phi$ in such models.
}

The extended language of expertise $\lext$ additionally describes the
expertise of the sources, and is defined by the following grammar:
\[
    \Phi ::= p \mid
             \Phi \land \Phi \mid
             \neg \Phi \mid
             \E_i\phi \mid
             \S_i\phi
             \\
\]
where $i\in \srcs$, $p \in \propvars$ and $\phi \in \lprop$. We introduce
Boolean connectives $\lor$, $\limplies$, $\liff$ and $\falsum$ as
abbreviations. We use upper case Greek letters ($\Phi$, $\Psi$ etc) for
formulas in $\lext$.
%
For $\Gamma \subseteq \lext$, we write $\proppart{\Gamma} = \Gamma \cap
\lprop$ for the propositional formulas in $\Gamma$.

As before, the intuitive reading of $\E_i\phi$ is ``source $i$ has expertise on
$\phi$''. The intuitive reading of $\S_i\phi$ is that ``$\phi$ sound for $i$ to
report'': i.e. that $\phi$ is true up to the expertise of $i$. In other words,
the parts of $\phi$ on which $i$ has expertise are true. Since both operators
are restricted to propositional formulas, we will not consider iterated
formulas such as $\E_i\S_j\phi$.

\paragraph{Semantics.}

The semantics in this chapter are, in essence, a special case of
\cref{chapter_expertise}. Instead of considering arbitrary sets of possible
states, as in \cref{exp_def_expertise_model}, we fix states as propositional
valuations over $\propvars$. Expertise is also assumed to be closed under both
intersections and complements for all sources.
%
At the same time, we generalise slightly by considering the distinguished
source $\ast$ and multiple cases $c \in \C$. For convenience we also offer a
different presentation, using \emph{partitions} instead of expertise
collections to represent expertise.

Formally, let $\vals$ denote the set of propositional valuations over
$\propvars$. For each $\phi \in \lprop$, the set of valuations making $\phi$
true is denoted by $\propmods{\phi}$. A \emph{world} $W = \tuple{\{v_c\}_{c \in
\C}, \{\Pi_i\}_{i \in \srcs}}$ is a possible complete specification of the
environment we find ourselves in:
\begin{itemize}
    \item $v_c \in \vals$ is the ``true'' valuation at case $c \in \C$;
    \item $\Pi_i$ is a partition of $\vals$ for each $i \in \srcs$, representing
          the ``true'' expertise of source $i$; and
    \item $\Pi_\ast$ is the unit partition $\{\{v\} \mid v \in \vals\}$.
\end{itemize}
Let $\W$ denote the set of all worlds. Note that the partition corresponding
to the distinguished source $\ast$ is fixed in all worlds as the finest
possible partition, reflecting the fact that $\ast$ is completely reliable.

For any partition $\Pi$ and valuation $v$, write $\Pi[v]$ for the unique cell
in $\Pi$ containing $v$. For a set of valuations $U$, write $\Pi[U] =
\bigcup_{v \in U}{\Pi[v]}$. For brevity, we write $\Pi[\phi]$ for
$\Pi[\propmods{\phi}]$. Then $\Pi[\phi]$ is the set of valuations
indistinguishable from a $\phi$ valuation.

For our belief change problem we will be interested in maintaining a
collection of several belief sets, describing beliefs about each case $c \in
\C$. Towards determining when a world $W$ models such a collection, we define
semantics for $\lext$ formulas with respect to a world and a case:
\[
    \begin{array}{lll}
        &W, c \models p &\iff v_c \in \propmods{p} \\
        &W, c \models \E_i\phi &\iff \Pi_i[\phi] = \propmods{\phi} \\
        &W, c \models \S_i\phi &\iff v_c \in \Pi_i[\phi]
    \end{array}
\]
where $i \in \srcs$, $\phi \in \lprop$, and the clauses for conjunction and
negation are the expected ones. Since $\propmods{\phi} \subseteq \Pi_i[\phi]$
always holds, we have that $\E_i\phi$ holds iff there is no
$\neg\phi$ valuation which is indistinguishable from a $\phi$ valuation (c.f.
\textcite{booth_trust_2018}). Note
that since each source $i$ has only a single partition $\Pi_i$ used to
interpret the expertise formulas, the truth value of $\E_i\phi$ does not
depend on the case $c$. On the other hand, $\S_i\phi$ holds in case $c$ iff
the $c$-valuation of $W$ is indistinguishable from some model of $\phi$. That
is, it is consistent with $i$'s expertise that $\phi$ is true.

Note that the mapping $2^\vals \to 2^\vals$ given by $U \mapsto \Pi[U]$
satisfies the \emph{Kuratowski closure axioms},\footnotemark{} so can be
considered a closure operator of the set of propositional valuations. Then $W,
c \models \E_i\phi$ iff $\propmods{\phi}$ is closed in $\vals$, and $W, c \models
\S_i\phi$ iff $v_c$ lies in the closure of $\propmods{\phi}$, i.e. $\phi$ is
true after closing $\propmods{\phi}$ along the lines of the expertise of source
$i$. Also note that $\Pi[U] = U$ iff $U$ can be expressed as a union of the
partition cells in $\Pi$, so that $W, c \models \E_i\phi$ can alternatively be
interpreted as saying $\phi$ is a disjunction of stronger formulas on which $i$
also has expertise.
%
\footnotetext{ That is, \begin{inlinelist}
        \item $\Pi[\emptyset] = \emptyset$,
        \item $U \subseteq \Pi[U]$,
        \item $\Pi[\Pi[U]] = \Pi[U]$, and
        \item $\Pi[U_1 \cup U_2] = \Pi[U_1] \cup \Pi[U_2]$ \end{inlinelist}.  }

Also note that if $\phi$ is a propositional tautology, $\E_i\phi$ holds for
every source $i$. Thus, all sources are experts on \emph{something}, even if
just the tautologies.

The semantics so defined are indeed the same as those of
\cref{chapter_expertise}, as the following result shows.

\begin{proposition}
    \label{kr_prop_modal_semantics_link}
    Let $W$ be a world. Then there is a multi-source expertise model $M = (X,
    \{P_i\}_{i \in \srcs}, V)$ and $\{x_c\}_{c \in \C} \subseteq X$ such that
    for all $\Phi \in \lext$ and $c \in \C$,
    \begin{equation}
        \label{kr_eqn_modal_semantics_link}
        W, c \models \Phi \iff M, x_c \models \Phi.
    \end{equation}
    Moreover,
    \begin{inlinelist}
        \item $X$ is finite;
        \item each $P_i$ is closed under intersections and complements; and
        \item using the notation from \cref{exp_def_rp}, $u R_{P_i} v$ iff
              $\Pi_i[u] = \Pi_i[v]$, i.e. $R_{P_i}$ is the equivalence relation
              associated with the partition $\Pi_i$.
    \end{inlinelist}
\end{proposition}

\begin{proof}
    Take $X = \vals$ and set $V(p) = \propmods{p}$. For each $i \in \srcs$, set
    $P_i = \{A \subseteq \vals \mid \Pi_i[A] = A\}.$ For each $c \in \C$,
    simply let $x_c = v_c$. Then one can easily show that for all $U \subseteq
    \vals$ and $i \in \srcs$,
    \begin{equation}
        \label{kr_eqn_modal_semantics_link_expansion}
        \Pi_i[U] = \bigcap\{A \in P_i \mid U \subseteq A\}.
    \end{equation}
    A simple induction on $\lext$ formulas then shows
    \cref{kr_eqn_modal_semantics_link}.

    Since $\propvars$ is finite there are only finitely many propositional
    valuations, and thus $X = \vals$ is finite. It is easily checked that
    each $P_i$ is closed under intersections and complements using properties
    of partitions. Finally, we have by
    \cref{kr_eqn_modal_semantics_link_expansion} and the definition of
    $R_{P_i}$ that
    \begin{align*}
        u R_{P_i} v
        &\iff \forall A \in P_i: (v \in A \implies u \in A)  \\
        &\iff u \in \bigcap\{A \in P_i \mid v \in A\} \\
        &\iff u \in \bigcap\{A \in P_i \mid \{v\} \subseteq A\} \\
        &\iff u \in \Pi_i[\{v\}] \\
        &\iff \Pi_i[u] = \Pi_i[v]
    \end{align*}
    as required.
\end{proof}

In other words, a world corresponds to a particular kind of expertise model
together with a state $x_c$ for each case $c \in \C$. Having shown this
equivalence, we henceforth deal exclusively with worlds and models instead of
expertise models and collections. We come to an example.

\begin{example}
    \label{kr_ex_hospital_world}
    Let us extend the hospital example from the introduction. Let $\srcs =
    \{\ast, \drX, \drY\}$ denote the reliable source, Dr.\ $\drX$ and Dr.\
    $\drY$, and let $\C = \{\patientA, \patientB\}$ denote patients $\patientA$
    and $\patientB$. Consider propositional variables $\propvars = \{p, q\}$,
    standing for conditions $p$ and $q$ respectively. Suppose that $\drX$ has
    expertise on diagnosing condition $q$ only, whereas $\drY$ only has
    expertise on $p$. For the sake of the example, suppose that patient
    $\patientA$ suffers from both conditions, and patient $\patientB$ suffers
    only from condition $q$. This situation is modelled by the following world
    $W = \tuple{\{v_c\}_{c \in \{\patientA, \patientB\}}, \{\Pi_i\}_{i \in
    \{\ast, \drX, \drY\}}}$:
    \[
        \begin{array}{cc}
            v_{\patientA} = pq;
            &
            v_{\patientB} = \bar{p}q;
            \\
            \Pi_{\drX} = pq, \bar{p}q \mid p\bar{q}, \bar{p}\bar{q};
            &
            \Pi_{\drY} = pq, p\bar{q} \mid \bar{p}q, \bar{p}\bar{q},
        \end{array}
    \]
    where $\Pi_\ast$ is the unit partition. This world is also depicted
    graphically in \cref{kr_fig_example_world}.
    %
    We have $W, c \models \E_{\drX} q \land \E_{\drY} p$ for each $c \in
    \{\patientA, \patientB\}$. Also note that $W, \patientA \models p$
    ($\patientA$ suffers from $p$), $W, \patientA \models \S_{\drX}{\neg p}$
    (it is sound for $\drX$ to report otherwise; this holds since
    $\Pi_{\drX}[\neg p] = \{pq, \bar{p}q\} \cup \{p\bar{q}, \bar{p}\bar{q}\}
    \ni pq = v_{\patientA}$), but $W, \patientA \models \neg \S_{\drY}{\neg p}$
    (the same formula is \emph{not} sound for $\drY$; we have $\Pi_{\drY}[\neg
    p] = \{\bar{p}q, \bar{p}\bar{q}\} = \propmods{\neg p} \not\ni pq =
    v_{\patientA}$).
\end{example}

\begin{figure}
    \centering
    \begin{tikzpicture}[scale=2]
    \def\p{0.3}
    \def\q{0.45}

    \node at (0, 0) {\Large ${pq}$};
    \node at (1, 0) {\Large ${p\bar{q}}$};
    \node at (0, 1) {\Large ${\bar{p}q}$};
    \node at (1, 1) {\Large ${\bar{p}\bar{q}}$};

    \draw[drycells]
        (-\p, -\p) --
        (1 + \p, -\p) --
        (1 + \p, \p) --
        (-\p, \p) --
        cycle;
    \draw[drycells]
        (-\p, 1 + \p) --
        (1 + \p, 1 + \p) --
        (1 + \p, 1 - \p) --
        (-\p, 1 - \p) --
        cycle;

    \draw[drxcells]
        (-\q, -\q) --
        (\q, -\q) --
        (\q, 1 + \q) --
        (-\q, 1 + \q) --
        cycle;
    \draw[drxcells]
        (1 - \q, -\q) --
        (1 + \q, -\q) --
        (1 + \q, 1 + \q) --
        (1 - \q, 1 + \q) --
        cycle;

    \node[color=red] at (1.75, 0) {\Large ${\Pi_{\drY}}$};
    \node[color=blue] at (1.75, 1) {\Large ${\Pi_{\drX}}$};

    \node[color=black!60!green] (va) at (-0.75, 0) {\Large ${v_{\patientA{}}}$};
    \node[color=black!60!green] (vb) at (-0.75, 1) {\Large ${v_{\patientB{}}}$};
    \draw[vals,->] (va) -- (-0.175, 0);
    \draw[vals,->] (vb) -- (-0.175, 1);
\end{tikzpicture}

    \caption{
        Depiction of the world $W$ defined in \cref{kr_ex_hospital_world}.
    }
    \label{kr_fig_example_world}
\end{figure}

Say $\Phi$ is \emph{valid} if $W, c \models \Phi$ for all $W \in \W$ and $c \in \C$.
For future reference we collect a list of validities.\footnotemark{}

\footnotetext{
    Note that some of these validities follow from
    \cref{kr_prop_modal_semantics_link} and the validities in
    \cref{chapter_expertise}.
}

\begin{proposition}
\label{kr_prop_validities}
For any $i \in \srcs$, $c \in \C$ and $\phi, \psi \in \lprop$, the following
formulas are valid
\begin{enumerate}
    \item \label{kr_item_replacement_equivalents_e_s}
          $\S_i\phi \liff \S_i\psi$ and $\E_i\phi \liff
          \E_i\psi$, whenever $\phi \equiv \psi$
    \item \label{kr_item_e_symmetric}
          $\E_i\phi \liff \E_i\neg\phi$ and $\E_i\phi \land
          \E_i\psi \limplies \E_i(\phi \land \psi)$
    \item \label{kr_item_exp_on_all_variables} $\E_i{p_1} \land \cdots \land
          \E_i{p_k} \limplies \E_i\phi$, where $p_1, \ldots, p_k$ are the
          propositional variables appearing in $\phi$
    \item \label{kr_item_e_and_s_implies_phi}
          $\E_i\phi \land \S_i\phi \limplies \phi$, and
          $\S_i\phi \land \neg \phi \limplies \neg \E_i\phi$
    \item \label{kr_item_sound_neg_pair}
          $\S_i\phi \land \S_i{\neg \phi} \limplies \neg \E_i\phi$
    \item \label{kr_item_star_exp}
          $\S_\ast{\phi} \liff \phi$ and $\E_\ast\phi$
    % \item \label{kr_item_soundness_weakening}
    %       $\S_i\phi \limplies \S_i\psi$ whenever $\phi \limplies \psi$ is
    %       a propositional tautology
\end{enumerate}
\end{proposition}

We comment on each property before giving the proof.
\cref{kr_item_replacement_equivalents_e_s} states syntax-irrelevance
properties.
%
\cref{kr_item_e_symmetric} says that expertise is symmetric with respect to
negation, and closed under conjunctions. Intuitively, symmetry means that $i$
is an expert on $\phi$ if they know \emph{whether or not} $\phi$ holds.
%
\cref{kr_item_exp_on_all_variables} says that expertise on each
propositional variable in $\phi$ is sufficient for expertise on $\phi$ itself.
%
\cref{kr_item_e_and_s_implies_phi} says that, in the presence of expertise,
soundness of $\phi$ is sufficient for $\phi$ to in fact be true.
%
\cref{kr_item_sound_neg_pair} says that if both $\phi$ and $\neg\phi$ are
true up to the expertise of $i$, then $i$ cannot have expertise on $\phi$.
%
Finally, \cref{kr_item_star_exp} says that the reliable source $\ast$ has
expertise on \emph{all} formulas, and thus $\phi$ is sound for $\ast$ iff it is
true.

\begin{proof}\leavevmode
\begin{enumerate}
    \item
        If $\phi \equiv \psi$ then $\propmods{\phi} = \propmods{\psi}$; since
        the semantics for $\S_i\phi$ and $\E_i\phi$ only refer to
        $\propmods{\phi}$ (and likewise for $\psi$), we have that $\S_i\phi
        \liff \S_i\psi$ and $\E_i\phi \liff \E_i\psi$
        are valid.

    \item
        For the first validity, suppose $W, c \models \E_i\phi$. Then
        $\propmods{\phi} = \Pi_i[\phi]$. We show $W, c \models \E_i\neg\phi$.
        Indeed, take $v \in \Pi_i[\neg\phi]$. Then there is $v' \in
        \propmods{\neg\phi}$ such that $v \in \Pi_i[v']$. Thus $v' \in
        \Pi_i[v]$ also. Supposing for contradiction that $v \in
        \propmods{\phi}$, we get
        \[
            v' \in \Pi_i[v] \subseteq \Pi_i[\phi] = \propmods{\phi}.
        \]
        But then $v' \in \propmods{\neg\phi} \cap \propmods{\phi} = \emptyset$;
        contradiction. Hence $v \notin \propmods{\phi}$, i.e. $v \in
        \propmods{\neg\phi}$. This shows that $\Pi_i[\neg\phi] \subseteq
        \propmods{\neg\phi}$, so $W, c \models \E_i\neg\phi$.

        We have shown that $\E_i\phi \limplies \E_i\neg\phi$ is valid. For
        the converse note that, by symmetry, $\E_i\neg\phi \limplies
        \E_i\neg\neg\phi$ is valid; since $\E_i\neg\neg\phi$ is equivalent to
        $\E_i\phi$ by \cref{kr_item_replacement_equivalents_e_s} we get
        $\E_i\phi \liff \E_i\neg\phi$.

        For the second validity, suppose $W, c \models \E_i\phi \land
        \E_i\psi$. Note that
        \[
            \Pi_i[\phi \land \psi]
            \subseteq \Pi_i[\phi]
            = \propmods{\phi}
        \]
        and, similarly, $\Pi_i[\phi \land \psi] \subseteq \propmods{\psi}$.
        Hence
        \[
            \Pi_i[\phi \land \psi]
            \subseteq \propmods{\phi} \cap \propmods{\psi}
            = \propmods{\phi \land \psi},
        \]
        which shows $W, c \models \E_i(\phi \land \psi)$.

    \item
        %
        Let $\phi$ be a propositional formula, and let $p_1, \ldots, p_k$ be
        the variables appearing in $\phi$. Let $\widehat{\lprop} \subseteq
        \lprop$ be the propositional formulas over $p_1, \ldots p_k$ generated
        only using conjunction and negation. Then there is some $\psi \in
        \widehat{\lprop}$ with $\phi \equiv \psi$.

        Suppose $W, c \models \E_i{p_1} \land \cdots \land \E_i{p_k}$. By this
        assumption and the properties in \cref{kr_item_e_symmetric},
        one can show by induction that $W, c \models \E_i{\theta}$ for all
        $\theta \in \widehat{\lprop}$. In particular, $W, c \models \E_i\psi$.
        Since $\phi \equiv \psi$, we get $W, c \models \E_i\phi$.

    \item
        %
        Suppose $W, c \models \E_i\phi \land \S_i\phi$. Then $v_c \in
        \Pi_i[\phi] = \propmods{\phi}$, so $W, c \models \phi$. Hence
        $\E_i\phi \land \S_i\phi \limplies \phi$ is valid. Similarly,
        $\S_i\phi \land \neg\phi \limplies \neg \E_i\phi$ is valid.

    \item
        %
        Suppose $W, c \models \S_i\phi \land \S_i\neg\phi$, and, for
        contradiction, $W, c \models \E_i\phi$. On the one hand we have $W, c
        \models \E_i\phi \land \S_i\phi$, so
        \cref{kr_item_e_and_s_implies_phi} gives $W, c \models \phi$. On
        the other hand, $W, c \models \E_i\phi$ gives $W, c \models
        \E_i\neg\phi$ by \cref{kr_item_e_symmetric}, so $W, c \models
        \E_i\neg\phi \land \S_i\neg\phi$; by
        \cref{kr_item_e_and_s_implies_phi} again we have $W, c \models
        \neg\phi$. But then $W, c \models \phi \land \neg\phi$ --
        contradiction.

    \item
        %
        Since the distinguished source $\ast$ has the unit partition $\Pi_\ast$
        in any world $W$, we have $\Pi_\ast[\phi] = \propmods{\phi}$, so $W, c
        \models \E_\ast\phi$. Similarly, $W, c \models \S_i\phi$ iff $v_c \in
        \Pi_\ast[\phi] = \propmods{\phi}$ iff $W, c \models \phi$.
\end{enumerate}
\end{proof}

\paragraph{Case-Indexed Collections.}

In the remainder of this chapter we will be interested in forming beliefs about
each case $c \in \C$. To do so we use collections of belief
sets $G = \{\Gamma_c\}_{c \in \C}$, with $\Gamma_c \subseteq \lext$, indexed by
cases.
%
Say a world $W$ is a \emph{model} of $G$ iff
\[
    W, c \models \Phi \text{ for all } c \in \C \text{ and } \Phi \in \Gamma_c,
\]
i.e. iff $W$ satisfies all formulas in $G$ in the relevant case. Let
$\mods(G)$ denote the models of $G$, and say that $G$ is
\emph{consistent} if $\mods(G) \ne \emptyset$. For $c \in \C$, define the
\emph{$c$-consequences}
\[
    \cn_c(G) = \{\Phi \in \lext \mid \forall W \in \mods(G), W, c \models \Phi\}.
\]
We write $\cn(G)$
for the collection $\{\cn_c(G)\}_{c \in \C}$.

\begin{example}
    Suppose $\C = \{c_1, c_2, c_3\}$, and define $G$ by $\Gamma_{c_1} = \{\S_i(p
    \land q)\}$, $\Gamma_{c_2} = \{\E_i{p}\}$ and $\Gamma_{c_3} = \{\E_i{q}\}$.
    Then, since expertise holds independently of case, any model $W$ of $G$ has
    $W, c_1 \models \E_i{p} \land \E_i{q}$. By \cref{kr_prop_validities} part
    \cref{kr_item_exp_on_all_variables}, $W, c_1 \models \E_i(p \land q)$.
    Since $W$ satisfies $\Gamma_{c_1}$ in case $c_1$, \cref{kr_prop_validities}
    part \cref{kr_item_e_and_s_implies_phi} gives $W, c_1 \models p \land
    q$. Since $W$ was an arbitrary model of $G$, we have $p \land q \in
    \cn_{c_1}(G)$, i.e. $p \land q$ is a $c_1$-consequence of $G$.
    %
    This illustrates how information about distinct cases can be brought
    together to have consequences for other cases.

\end{example}

For two collections $G = \{\Gamma_c\}_{c \in \C}$, $D = \{\Delta_c\}_{c \in
\C}$, write $G \sqsubseteq D$ iff $\Gamma_c \subseteq \Delta_c$ for all $c$,
and let $G \sqcup D$ denote the collection $\{\Gamma_c \cup \Delta_c\}_{c \in
\C}$. With this notation, the case-indexed consequence operator satisfies
analogues of the Tarskian consequence properties.\footnotemark{}
%
\footnotetext{
    That is, \begin{inlinelist}
        \item $G \sqsubseteq \cn(G)$,
        \item $G \sqsubseteq D$ implies $\cn(G) \sqsubseteq \cn(D)$, and
        \item $\cn(\cn(G)) = \cn(G)$.
    \end{inlinelist}
}

Say a collection $G$ is \emph{closed} if $\cn(G) = G$. Closed collections
provide an idealised representation of beliefs, which will become useful later
on. For instance, when $G$ is closed we have
$
    \E_i\phi \in \Gamma_c \text{ iff } \E_i\phi \in \Gamma_d
$
for all $c, d \in \C$; i.e. expertise statements are either present for all
cases or for none. We also have
$
    \cnprop\proppart{\Gamma_c} = \proppart{\Gamma_c},
$
i.e. the propositional parts of $G$ are (classically) closed.

In propositional logic, $\propmods{\cdot}$ is a 1-to-1 correspondence between closed
sets of formulas and sets of valuations. This is not so in our setting, since
some subsets of $\W$ do not arise as the models of any collection. Instead, we
have a 1-to-1 correspondence into a restricted collection of sets of worlds.
%
Borrowing the terminology of
\textcite{delgrande2018general}, say a set of worlds $S \subseteq \W$ is
\emph{elementary} if ${S = \mod(G)}$ for some collection $G =
\{\Gamma_c\}_{c \in \C}$.\footnotemark{}

\footnotetext{
    Non-elementary sets can also exist for weaker logics
    (such as Horn logic~\cite{delgrande2018general}) which lack the syntactic
    expressivity to identify all sets of models. In our framework, $\C$-indexed
    collections are not expressive enough to specify \emph{combinations of
    valuations}, since each $\Gamma_c$ only says something about the valuation
    for $c$.
}

Elementariness is characterised by a certain closure condition. Say that two
worlds $W, W'$ are \emph{partition-equivalent} if $\Pi^W_i = \Pi^{W'}_i$ for
all sources $i$, and say $W$ is a \emph{valuation combination} from a set $S
\subseteq \W$ if for all cases $c$ there is $W_c \in S$ such that $v^W_c =
v^{W_c}_c$.  Then a set is elementary iff it is closed under valuation
combinations of partition-equivalent worlds.

\begin{proposition}
\label{kr_prop_elementary_characterisation}
    $S \subseteq \W$ is elementary if and only if the following condition
    holds: for all $W \in \W$ and $W_1, W_2 \in S$, if $W$ is
    partition-equivalent to both $W_1, W_2$ and $W$ is a valuation combination
    from $\{W_1, W_2\}$, then $W \in S$.
\end{proposition}

\begin{proof}
    ``if'': Suppose the stated condition holds for $S \subseteq \W$. Form a
    collection $G = \{\Gamma_c\}_{c \in \C}$ by setting $\Gamma_c = \{\Phi \in
    \lext \mid S \subseteq \mod_c(\Phi)\}$. Clearly $S \subseteq \mod(G)$. For the
    reverse inclusion, suppose $W \in \mod(G)$. For any set of valuations $U
    \subseteq \vals$, let $\phi_U$ be any propositional formula with
    $\propmods{\phi_U} = U$. For each $c \in \C$, consider the formula
    \[
        \Phi_c = \biglor_{W' \in S}\left(
            \phi_{\{v^{W'}_c\}}
            \land
            \bigland_{i \in \srcs}
                \bigland_{U \subseteq \vals}
                    R_{W', i, U}
        \right)
    \]
    where
    \[
        R_{W', i, U} = \begin{cases}
            \E_i{\phi_U},& W', c_0 \models \E_i{\phi_U} \\
            \neg \E_i{\phi_U},& \text{ otherwise }
        \end{cases}
    \]
    for some fixed case $c_0 \in \C$. It is straightforward to see that each
    $W' \in S$ satisfies its corresponding disjunct at case $c$, so $\Phi_c \in
    \Gamma_c$. Hence $W \in \mod(G)$ implies $W, c \models \Phi_c$ for each $c$.
    Consequently, for each $c$ there is some $W_c \in S$ such that
    \begin{inlinelist}
        \item \label{kr_item_vals_equal} $v^W_c = v^{W_c}_c$; and
        \item \label{kr_item_partition_equiv} for each $i \in \srcs$ and $U \subseteq
              \vals$, $W, c \models \E_i{\phi_U}$ iff $W_c, c \models \E_i{\phi_U}$.
    \end{inlinelist}
    From \ref{kr_item_vals_equal}, $W$ is a valuation combination from $\{W_c\}_{c
    \in \C}$. From \ref{kr_item_partition_equiv} it can be shown that in fact
    $\Pi^W_i = \Pi^{W_c}_i$ for each $c$ and $i$; that is, $W$ is
    partition-equivalent to each $W_c$. In particular, all the $W_c$ are
    partition-equivalent to each other.
    % Self-reminder if reading this in the future: \E_i{\phi_U} holds at W, c iff U
    % is a union of partition cells in Π^W_i. The equivalence above means that
    % the set of all unions formed from Π^W_i is the same as for Π^{W_c}_i.
    % This implies the partitions themselves are equal; see Proposition 1 in
    % online notes: /trust-contraction.html#trust-sets-and-partitions
    Repeatedly applying the closure condition assumed to hold for $S$, we see
    that $W \in S$ as required.

    ``only if'': Suppose $S$ is elementary, i.e. $S = \mod(G)$ for some
    collection $G = \{\Gamma_c\}_{c \in \C}$, and let $W, W_1, W_2$ be as in
    the statement of the \lcnamecref{kr_prop_elementary_characterisation}. Take $c
    \in \C$ and $\Phi \in \Gamma_c$. We will show $W, c \models \Phi$. By assumption,
    there is $n \in \{1, 2\}$ such that $v^W_c = v^{W_n}_c$. It can be shown by
    induction on $\lext$ formulas that, since $W$ and $W_n$ are
    partition-equivalent and have the same $c$ valuation, $W, c \models \Phi$ iff
    $W_n, c \models \Phi$. But $W_n \in S = \mod(G)$ implies $W_n, c \models
    \Phi$,
    so $W, c \models \Phi$ too.  Since $\Phi \in \Gamma_c$ was arbitrary, we have $W
    \in \mod(G) = S$ as required.
\end{proof}

\section{The Problem}
\label{kr_sec_the_problem}

With the framework set out, we can formally define the problem. We
seek an operator with the following behaviour:

\begin{itemize}

\item \textbf{Input:} A sequence of reports $\sigma$, where each report is a
      triple $\tuple{i, c, \phi} \in \srcs \times \C \times \lprop$ and $\phi
      \not\equiv \falsum$. Such a report represents that {\em source $i$
      reports $\phi$ to hold in case $c$}. Note that we only allow sources to
      make \emph{propositional} reports.

\item \textbf{Output:} A pair $\tuple{B^\sigma, K^\sigma}$, where $B^\sigma =
      \{B^\sigma_c\}_{c \in \C}$ is a collection of \emph{belief sets}
      $B^\sigma_c \subseteq \lext$ and $K^\sigma = \{K^\sigma_c\}_{c \in \C}$
      is a collection of \emph{knowledge sets} $K^\sigma_c \subseteq \lext$.

\end{itemize}

%\begin{example}
%    \label{kr_ex_most_basic_example}

%    \todo{Make sure this example links back to the informal one in the
%    introduction, and discuss knowledge before belief}
%    %
%    Consider ordinary sources $i$, $j$ and the sequence
%    $
%        \sigma = (\tuple{i, c, p}, \tuple{j, c, \neg p})
%    $.
%    Here $\sigma$ exhibits total symmetry between $i$, $j$ and $p$, $\neg p$,
%    so intuitively any operator should be uncertain about $p$ (i.e. $p \notin
%    B^\sigma_c$ and $\neg p \notin B^\sigma_c$) and uncertain about $i$ and $j$
%    (i.e.  $\E_i{p} \notin B^\sigma_c$ and $\E_j{p} \notin B^\sigma_c$).
%    %
%    An ``optimistic'' operator might have $\E_i{p} \lor \E_j{p} \in B^\sigma_c$
%    (i.e. we believe \emph{at least one} of the sources has expertise on $p$).
%    %
%    We would not want to commit this to the knowledge set
%    $K^\sigma_c$, though, since it is \emph{possible} that neither $i$ nor $j$ have
%    expertise on $p$. Under the assumption that sources are honest, we do
%    however know that both cannot be experts: $\neg(\E_i{p} \land \E_j{p}) \in
%    K^\sigma_c$.\footnote{This will in fact follow from the \closure{} and
%    \soundness{} postulates we introduce in \cref{kr_sec_basic_postulates}.}

%\end{example}

%\begin{example}
%    \label{kr_ex_most_basic_example_extended}
%    Consider an extension of \cref{kr_ex_most_basic_example}: set
%    $
%        \sigma = (
%            \tuple{i, c, p}, \tuple{j, c, \neg p}, \tuple{i, d, p},
%            \tuple{\ast, d, \neg p}
%        )
%    $ (also c.f. the introductory example).
%    Here the symmetry is broken by source $i$ reporting $p$ in another case
%    $d$, where we know $\neg p$ due to the completely reliable source $\ast$.
%    From this we should have $\neg \E_i{p} \in K^\sigma_c$, which could be
%    enough to tip the scales in favour of source $j$ in case $c$, so that
%    $\E_j{p} \land \neg p \in B^\sigma_c$.\footnote{Indeed, this is the case for
%    the example operators we define later.}
%    %
%    While simple, this important
%    example shows how trust affects belief (we disbelieve $p$ since we do not
%    trust $i$ on $p$) and how reports in one case can affect beliefs in
%    another.
%\end{example}

\subsection{Basic Postulates}
\label{kr_sec_basic_postulates}

We immediately narrow the scope of operators under consideration by introducing
some basic postulates which are expected to hold. In what follows, say a
sequence $\sigma$ is \emph{$\ast$-consistent} if for each $c \in \C$ the set
$\{\phi \mid \tuple{\ast, c, \phi} \in \sigma\} \subseteq \lprop$ is
classically consistent. Write $G^\sigma_\snd$ for the collection with
$(G^\sigma_\snd)_c = \{\S_i\phi \mid \tuple{i, c, \phi} \in \sigma\}$, i.e.
the collection of soundness statements corresponding to the reports in
$\sigma$.

\begin{axiomlist}
\begin{axiom}[\closure{}]
    $B^\sigma = \cn(B^\sigma)$ and  $K^\sigma = \cn(K^\sigma)$
\end{axiom}
\begin{axiom}[\containment{}]
    $K^\sigma \sqsubseteq B^\sigma$
\end{axiom}
\begin{axiom}[\consistency{}]
    If $\sigma$ is $\ast$-consistent, $B^\sigma$ and $K^\sigma$ are
    consistent
\end{axiom}
\begin{axiom}[\soundness{}]
    If $\tuple{i, c, \phi} \in \sigma$, then $\S_i\phi \in K^\sigma_c$
\end{axiom}
\begin{axiom}[\kbound{}]
    $K^\sigma \sqsubseteq \cn(G^\sigma_\snd \sqcup K^\emptyset)$
\end{axiom}
\begin{axiom}[\priorext{}]
    $K^\emptyset \sqsubseteq K^\sigma$
\end{axiom}
\begin{axiom}[\rearr{}]
    If $\sigma$ is a permutation of $\rho$, then $B^\sigma = B^\rho$ and
    $K^\sigma = K^\rho$
\end{axiom}
\begin{axiom}[\equivpost{}]
    If $\phi \equiv \psi$ then $B^{\sigma \concat \tuple{i, c, \phi}} = B^{\sigma
    \concat \tuple{i, c, \psi}}$ and $K^{\sigma \concat \tuple{i, c, \phi}} =
    K^{\sigma \concat \tuple{i, c, \psi}}$
\end{axiom}
\end{axiomlist}

\closure{} says that the belief and knowledge collections are closed under
logical consequence. In light of earlier remarks,
this implies that the propositional belief sets $\proppart{B^\sigma_c}$ are
closed under (propositional) consequence, and that $\E_i\phi \in B^\sigma_c$
iff $\E_i\phi \in B^\sigma_d$.
%
\containment{} says that everything which is known is also believed.
%
\consistency{} ensures the output is always consistent, provided
we are not in the degenerate case where $\ast$ gives inconsistent reports.
%
\soundness{} says we \emph{know} that all reports are sound in their respective
cases. This formalises our assumption that sources are \emph{honest}, i.e. that
false reports only arise due to lack of expertise. By \cref{kr_prop_validities}
part \cref{kr_item_e_and_s_implies_phi} it also implies \emph{experts are
always right}: if a source has expertise on their report then it must be true.
%
While \soundness{} places a lower bound on knowledge, \kbound{} places an upper
bound: knowledge cannot go beyond the soundness statements corresponding to the
reports in $\sigma$ together with the prior knowledge $K^\emptyset$. That is,
from the point view of knowledge, a new report of $\tuple{i, c, \phi}$ only
allows us to learn $\S_i\phi$ in case $c$ (and to combine this with other
reports and prior knowledge). Note that the analogous property for belief is
\emph{not} desirable: we want to be more liberal when it comes to beliefs, and
allow for \emph{defeasible inferences} going beyond the mere fact that reports
are sound.
%
\priorext{} says that knowledge after a sequence $\sigma$ extends the prior
knowledge on the empty sequence $\emptyset$.
%
\rearr{} says that the order in which reports are received is irrelevant. This
can be justified on the basis that we are reasoning about \emph{static worlds}
for each case $c$, so that there is no reason
to see more ``recent'' reports as any more or less important or truthful than
earlier ones.\footnote{This argument is from \cite{delgrande2006iterated}.}
Consequently, we can essentially view the input as a \emph{multi-set} of belief
sets -- one for each source -- bringing us close to the setting of belief
merging. This postulate also appears as the commutativity postulate
\textbf{(Com)} in the work of \textcite{schwind2020}.
%
Finally, \equivpost{} says that the syntactic form of reports is irrelevant.

Taking all the basic postulates together, the knowledge component $K^\sigma$ is
fully determined once $K^\emptyset$ is chosen.

\begin{proposition}
    \label{kr_prop_prior_knowledge}
    Suppose an operator satisfies the basic postulates. Then
    \begin{enumerate}
        \item $K^\sigma = \cn(G^\sigma_\snd \sqcup K^\emptyset)$
        \item $K^\emptyset = \cn(\emptyset)$ iff $K^\sigma =
              \cn(G^\sigma_\snd)$ for all $\sigma$.
    \end{enumerate}
\end{proposition}

\begin{proof}
    \leavevmode
    \begin{enumerate}
        \item The ``$\sqsubseteq$'' inclusion is just \kbound{}.
              For the ``$\sqsupseteq$'' inclusion, note that $G^\sigma_\snd
              \sqsubseteq K^\sigma$ by \soundness{}, and $K^\emptyset
              \sqsubseteq K^\sigma$ by \priorext{}. Hence
              \[
                  G^\sigma_\snd \sqcup K^\emptyset
                  \sqsubseteq
                  K^\sigma.
              \]
              By monotonicity of $\cn$,
              \[
                  \cn(G^\sigma_\snd \sqcup K^\emptyset)
                  \sqsubseteq
                  \cn(K^\sigma)
                  = K^\sigma
              \]
              where we use \closure{} in the final step.

          \item ``if'': Suppose $K^\sigma = \cn(G^\sigma_\snd)$ for all
                $\sigma$. Taking $\sigma = \emptyset$ we obtain
                \[
                    K^\sigma = \cn(G^\emptyset_\snd) = \cn(\emptyset).
                \]
                ``only if'': Suppose $K^\emptyset = \cn(\emptyset)$. Take any
                sequence $\sigma$. By \kbound{},
                \[
                    K^\sigma
                    \sqsubseteq \cn(G^\sigma_\snd \sqcup \cn(\emptyset))
                    = \cn(G^\sigma_\snd)
                \]
                On the other hand, \soundness{} and \closure{} give
                $\cn(G^\sigma_\snd) \sqsubseteq K^\sigma$. Hence $K^\sigma =
                \cn(G^\sigma_\snd)$.
    \end{enumerate}
\end{proof}

The choice of $K^\emptyset$ depends on the scenario one wishes to model.
While $\cn(\emptyset)$ is a sensible choice if the sequence $\sigma$ is all we
have to go on, we allow $K^\emptyset \ne \cn(\emptyset)$ in case \emph{prior
knowledge} is available (for example, the expertise of particular sources may
be known ahead of time).

Another important property of knowledge, which follows from the basic
postulates, says that \emph{knowledge is monotonic}: knowledge after receiving
$\sigma$ and $\rho$ together is just the case-wise union of $K^\sigma$ and
$K^\rho$.

\begin{axiom}[\kconj{}]
    $K^{\sigma \concat \rho} = \cn(K^\sigma \sqcup K^\rho)$
\end{axiom}

This postulate reflects the idea that one should be cautious when it comes to
knowledge, a formula should only be accepted as known if it won't be given up
in light of new information.

\begin{proposition}
    \label{kr_prop_kconj}
    Any operator satisfying the basic postulates satisfies \kconj{}.
\end{proposition}

\begin{proof}
    Suppose an operator satisfies the basic postulates, and take sequences
    $\sigma$ and $\rho$. By \cref{kr_prop_prior_knowledge},
    \[
        K^{\sigma \concat \rho}
        =
        \cn(G^{\sigma \concat \rho}_\snd \sqcup K^\emptyset)
    \]
    Note that $G^{\sigma \concat \rho}_\snd = G^\sigma_\snd \sqcup
    G^\rho_\snd$. Hence we may write
    \begin{align*}
        K^{\sigma \concat \rho}
        &= \cn(G^\sigma_\snd \sqcup G^\rho_\snd \sqcup K^\emptyset) \\
        &= \cn((G^\sigma_\snd \sqcup K^\emptyset) \sqcup (G^\rho_\snd \sqcup K^\emptyset))
    \end{align*}
    By \cref{kr_prop_prior_knowledge} again, we have $K^\sigma = \cn(G^\sigma_\snd
    \sqcup K^\emptyset)$ and $K^\rho = \cn(G^\rho_\snd \sqcup K^\emptyset)$. It
    is easily verified that for any collections $G, D$, we have
    \[
        \cn(G \sqcup D) = \cn(\cn(G) \sqcup \cn(D)).
    \]
    Consequently,
    \begin{align*}
        K^{\sigma \concat \rho}
        &= \cn(\cn(G^\sigma_\snd \sqcup K^\emptyset) \sqcup \cn(G^\rho_\snd \sqcup K^\emptyset)) \\
        &= \cn(K^\sigma \sqcup K^\rho)
    \end{align*}
    as required for \kconj{}.
\end{proof}

The postulates also imply some useful properties linking \emph{trust} (seen as
belief in expertise) and \emph{belief/knowledge}.

\begin{proposition}
    \label{kr_prop_basic_postulates_consequences}
    Suppose an operator satisfies the basic postulates. Then
    \begin{enumerate}
        \item \label{kr_item_knowledge_trust_link} If $\phi \in K^\sigma_c$ and
              $\neg\psi \in \cnprop(\phi)$ then $\neg \E_i\psi \in K^{\sigma
              \concat \tuple{i, c, \psi}}_c$.
        \item \label{kr_item_trust_belief_link} If $\tuple{i, c, \phi} \in
              \sigma$ and $\E_i\phi \in B^\sigma_c$ then $\phi \in
              B^\sigma_c$.
    \end{enumerate}
\end{proposition}

\begin{proof}\leavevmode
    \begin{enumerate}

        \item Suppose $\phi \in K^\sigma_c$ and $\neg\psi \in \cnprop(\phi)$.
              Write $\rho = \sigma \concat \tuple{i, c, \psi}$. By
              \soundness{}, $\S_i\psi \in K^\rho_c$. By \kconj{}, $\phi \in
              K^\sigma_c \subseteq (K^\sigma \sqcup K^{\tuple{i, c, \psi}})_c
              \subseteq \cn_c(K^\sigma \sqcup K^{\tuple{i, c, \psi}}) =
              K^\rho_c$. Since $\neg\psi \in \cnprop(\phi)$ and $\phi \in
              K^\rho_c$, \closure{} gives $\neg\psi \in K^\rho_c$. Recalling
              from \cref{kr_prop_validities} part
              \cref{kr_item_e_and_s_implies_phi} that $\S_i\psi \land \neg
              \psi \limplies \neg \E_i\psi$, \closure{} gives $\neg
              \E_i\psi \in K^\rho_c$, as desired.

          \item Suppose $\tuple{i, c, \phi} \in \sigma$ and $\E_i\phi \in
                B^\sigma_c$. By \soundness{} and \containment{}, $\S_i\phi \in
                B^\sigma_c$. From \cref{kr_prop_validities} part
                \cref{kr_item_e_and_s_implies_phi} again we have $\E_i\phi
                \land \S_i\phi \limplies \phi$. By \closure{}, $\phi \in
                B^\sigma_c$.

    \end{enumerate}
\end{proof}

\cref{kr_item_knowledge_trust_link} expresses how knowledge can negatively
affect trust: we should distrust sources who make reports we know to be false.
\cref{kr_item_trust_belief_link} expresses how trust affects belief: we
should believe reports from trusted sources.
It can also be seen as a form of \emph{success} for ordinary sources, and
implies AGM success when $i = \ast$ (by \cref{kr_prop_validities} part
\cref{kr_item_star_exp} and \closure{}). We illustrate the basic postulates
by formalising the introductory hospital example.

\begin{example}
    \label{kr_ex_hospital_ex_formalised}
    Set $\srcs, \C$ and $\propvars$ as in \cref{kr_ex_hospital_world}, and consider
    the sequence
    \[
        \sigma
        = (
            \tuple{\ast, \patientA, p},
            \tuple{\drX, \patientB, p},
            \tuple{\drY, \patientB, \neg p},
            \tuple{\drX, \patientA, \neg p}
        ).
    \]
    What do we know on the basis of this sequence, assuming the basic
    postulates? First note that by \soundness{}, \cref{kr_prop_validities}
    part \cref{kr_item_star_exp} and \closure{}, the report from $\ast$ gives $p
    \in K^\sigma_{\patientA}$, i.e. reliable reports are known. \soundness{} also
    gives $\S_{\drX}{p} \land \S_{\drY}{\neg p} \in K^\sigma_{\patientB}$. Combined with
    \cref{kr_prop_validities} parts \cref{kr_item_e_symmetric},
    \cref{kr_item_e_and_s_implies_phi} and \closure{}, this yields
    $\neg(\E_{\drX}{p} \land \E_{\drY}{p}) \in K^\sigma_c$ for all $c$, formalising the
    intuitive idea that $\drX$ and $\drY$ cannot both be experts on $p$, since they
    give conflicting reports.
    %
    Considering the final report from $\drX$, we get $p \land \S_{\drX}{\neg p} \in
    K^\sigma_{\patientA}$, and thus $\neg \E_{\drX}{p} \in K^\sigma_c$ by
    \closure{}. So in fact $\drX$ is known to be a non-expert on $p$.

    What about beliefs? The basic postulates do not require beliefs to go
    beyond knowledge, so we cannot say much in general. An ``optimistic''
    operator, however, may opt to believe that sources are experts unless we
    know otherwise, and thus maximise the information that can be (defeasibly)
    inferred from the sequence (in the next section we will introduce
    concrete operators obeying this principle). In this case we may believe
    that at least one source has expertise on $p$ (i.e. $\E_{\drX}{p} \lor \E_{\drY}{p} \in
    B^\sigma_c$).  Combined with $\neg \E_{\drX}{p} \in K^\sigma_c$, \closure{} and
    \containment{}, we get $\E_{\drY}{p} \in B^\sigma_{\patientB}$. Symmetry of expertise
    together with \cref{kr_prop_basic_postulates_consequences} part
    \cref{kr_item_trust_belief_link} then gives $\neg p \in
    B^\sigma_{\patientB}$, i.e. we trust $\drY$ in the example and believe $\patientB$
    does not suffer from condition $p$.

\end{example}

\subsection{Model-Based Operators}

While an operator is a purely syntactic object, it will be convenient to
specify $K^\sigma$ and $B^\sigma$ in semantic terms by selecting a set of
\emph{possible} and \emph{most plausible} worlds for each sequence $\sigma$.
We call such operators \emph{model-based}.

\begin{definition}
\label{kr_def_model_based}
An operator is \emph{model-based} if for every $\sigma$ there
are sets $\X_\sigma, \Y_\sigma \subseteq \W$ such that
\begin{inlinelist}
    \item $\X_\sigma \supseteq \Y_\sigma$;
    \item $\Phi \in K^\sigma_c$ iff $W, c \models \Phi$ for all $W \in
          \X_\sigma$; and
    \item $\Phi \in B^\sigma_c$ iff $W, c \models \Phi$ for all $W \in
          \Y_\sigma$.
\end{inlinelist}
\end{definition}

In other words, $K^\sigma_c$ (resp., $B^\sigma_c$) contains the formulas which
hold at case $c$ in \emph{all worlds} in $\X_\sigma$ (resp., $\Y_\sigma)$. It
follows from the relevant definitions that $\X_\sigma \subseteq
\mods(K^\sigma)$, and equality holds if and only if $\X_\sigma$ is elementary
(similarly for $\Y^\sigma$ and $B^\sigma$).
%
Model-based operators are characterised by our first two basic postulates.

\begin{theorem}
\label{kr_thm_model_based_characterisation}
An operator satisfies \closure{} and \containment{} if and only if it
is model-based.
\end{theorem}

\begin{proof}
    For ease of notation in what follows, write $\mods_c(\Phi) = \{W \in \W
    \mid W, c \models \Phi\}$.

``if'': Suppose an operator $\sigma \mapsto \tuple{B^\sigma, K^\sigma}$ is
model-based. For \closure{}, we need to show that $B^\sigma_c \supseteq
\cn_c(B^\sigma)$ and $K^\sigma_c \supseteq \cn_c(K^\sigma)$, for each $c$. Take
any $\Phi \in \cn_c(B^\sigma)$. Then $\mods(B^\sigma) \subseteq \mods_c(\Phi)$.
From the relevant definitions, one can easily check that $\Y_\sigma \subseteq
\mods(B^\sigma)$, so we have $\Y_\sigma \subseteq \mods_c(\Phi)$. That is, $W,
c \models \Phi$ for all $W \in \Y_\sigma$. By definition of model-based
operators, $\Phi \in B^\sigma_c$. The fact that $K^\sigma_c \supseteq
\cn_c(K^\sigma)$ follows by an identical argument upon noticing that $\X_\sigma
\subseteq \mods(K^\sigma)$.

\containment{} follows from $\X_\sigma \supseteq \Y_\sigma$: if $\Phi \in
K^\sigma_c$ then $W, c \models \Phi$ for all $W \in \X_\sigma$, and in
particular this holds for all $W \in \Y_\sigma$. Hence $\Phi \in B^\sigma_c$,
so $K^\sigma \sqsubseteq B^\sigma$.

``only if'': Suppose an operator satisfies \closure{} and
\containment{}. For any $\sigma$, set
\[
    \X_\sigma = \mods(K^\sigma)
\]
\[
    \Y_\sigma = \mods(B^\sigma)
\]
We show the three properties required in \cref{kr_def_model_based}.  $\X_\sigma
\supseteq \Y_\sigma$ follows from \containment{} and the definition of a
model of a collection. For the second property, note that $\Phi \in K^\sigma_c$
iff $\Phi \in \cn_c(K^\sigma)$ by \closure{}, i.e. iff $\mods(K^\sigma)
\subseteq \mods_c(\Phi)$.  By choice of $\X_\sigma$, this holds exactly when
$W, c \models \Phi$ for all $W \in \X_\sigma$, as required. The third property
is proved using an identical argument.
\end{proof}

Since we take \closure{} and \containment{} to be fundamental properties, all
operators we consider from now on will be model-based.
%
We introduce our first concrete operator.

\begin{definition}
    \label{kr_def_weakop}
    Define the model-based operator \weakop{} by
    \[
        \X_\sigma = \Y_\sigma = \{
            W \mid W, c \models \S_i\phi \text{ for all } \tuple{i, c, \phi}
            \in \sigma
        \}.
    \]
\end{definition}

That is, the possible worlds $\X_\sigma$ are
exactly those satisfying the soundness constraint for each report in
$\sigma$, i.e. false reports are only due to lack of expertise of the
corresponding source.
%
Syntactically, $K^\sigma = B^\sigma = \cn(G^\sigma_\snd)$.

Clearly \weakop{} satisfies \soundness{}, and one
can show that it satisfies all of the basic postulates of
\cref{kr_sec_basic_postulates}.\footnotemark In fact, it is the \emph{weakest}
operator satisfying \closure{}, \containment{} and \soundness{}, in that
for any other operator $\sigma \mapsto \tuple{\hat{B}^\sigma,
\hat{K}^\sigma}$ with these properties we have $B^\sigma \sqsubseteq
\hat{B}^\sigma$ and $K^\sigma \sqsubseteq \hat{K}^\sigma$ for any $\sigma$.
\footnotetext{
    For \consistency{}, note that for any $\ast$-consistent sequence
    $\sigma$ one can form a world $W$ such that $v_c$ is a model of all
    reports from $\ast$ at case $c$, and $\Pi_i = \{\vals\}$ for all $i \ne
    \ast$. This satisfies all the soundness constraints, so $W \in
    \X_\sigma = \Y_\sigma$.
}

\begin{example}
\label{kr_ex_model_based}
    Consider \weakop{} applied to the sequence
    $
        \sigma
        = (\tuple{\ast, c, p}, \tuple{i, c, \neg p \land q})
    $.
    By \soundness{}, \closure{} and the validities from \cref{kr_prop_validities},
    we have $p \in K^\sigma_c$ and $\neg \E_i{p} \in K^\sigma_c$. In fact, by
    \closure{}, we
    have $\neg \E_i{p} \in K^\sigma_d$ for all cases $d$.
    %
    However, we cannot say much about $q$: neither $q$, $\neg q$, $\E_i{q}$ nor
    $\neg \E_i{q}$ are in $B^\sigma_c = K^\sigma_c$.

\end{example}

\section{Constructions}
\label{kr_sec_constructions}

For model-based operators in \Cref{kr_def_model_based}, the sets $\X_\sigma$ and
$\Y_\sigma$ -- which determine knowledge and belief -- can
depend on $\sigma$ in a completely arbitrary manner. This lack of structure
leads to very wide class of operators, and one cannot say much about them in
general beyond the satisfaction of \closure{} and \containment{}. In this
section we specialise model-based operators by providing two constructions.

\subsection{Conditioning Operators}
\label{kr_sec_conditioning_operators}

Intuitively, $\Y_\sigma$ is supposed to represent the \emph{most
plausible} worlds among the possible worlds in $\X_\sigma$. This suggests the
presence of a \emph{plausibility ordering} on $\X_\sigma$, which is used to
select $\Y_\sigma$.
%
For our first construction we take this approach: we condition a fixed
plausibility total preorder\footnotemark{} on the knowledge $\X_\sigma$, and
obtain $\Y_\sigma$ by selecting the minimal (i.e. most plausible) worlds.
\footnotetext{
    A total preorder is a reflexive, transitive and total relation.
}

\begin{definition}
\label{kr_def_conditioning_operator}
An operator is a \emph{conditioning operator} if there is a total
preorder $\le$ on $\W$ and a mapping $\sigma \mapsto \tuple{\X_\sigma,
\Y_\sigma}$ as in \cref{kr_def_model_based} such that
$
    \Y_\sigma = \min_{\le}{\X_\sigma}
$
for all $\sigma$.
\end{definition}

Note that $\le$ is independent of $\sigma$: it is fixed before receiving any
reports. All conditioning operators are model-based by definition. Clearly
$\Y_\sigma$ is determined by $\X_\sigma$ and the plausibility order, so that to
define a conditioning operator it is enough to specify $\le$ and the mapping
$\sigma \mapsto \X_\sigma$. Write $W \simeq W'$ iff both $W \le W'$ and $W' \le
W$.

Conditioning in this manner is well-established in the belief change
literature. Our operators use a simplified case of conditionalisation as
introduced by \textcite{spohn1988ordinal}. \textcite{boutilier1998belief} use a
similar notion in their account of unreliable belief revision, wherein an
agent's plausibility ranking is successively conditioned by its observations.

We now present examples of how the plausibility ordering $\le$ may be defined.

\begin{definition}
    \label{kr_def_varbasedcond}
    Define the conditioning operator \varbasedcond{} by setting
    $\X_\sigma$ in the same way as \weakop{} in \cref{kr_def_weakop}, and $W \le
    W'$ iff $r(W) \le r(W')$, where
    \[
        r(W) = - \sum_{i \in \srcs}\left|\left\{
            p \in \propvars
            \mid
            \Pi^W_i[p] = \propmods{p}
        \right\}\right|.
    \]
\end{definition}

\varbasedcond{} aims to trust each source on \emph{as many propositional
variables} as possible. One can check that \varbasedcond{} satisfies the
basic postulates. \todo{Show?}

\begin{example}
    \label{kr_ex_conditioning_operator}
    Revisiting the sequence
    $
        \sigma
        = (\tuple{\ast, c, p}, \tuple{i, c, \neg p \land q})
    $
    from \cref{kr_ex_model_based} with \varbasedcond{}, the knowledge set
    $K^\sigma_c$ is the same as before, but we now have $q \land \E_i{q} \in
    B^\sigma_c$. This reflects the ``credulous'' behaviour of the ranking
    $\le$: while it is not possible to believe $i$ is an expert on $p$, we
    should believe they \emph{are} an expert on $q$ so long as this does not
    conflict with soundness. For the propositional beliefs generally, we have
    $\proppart{B^\sigma_c} = \cnprop(p \land q)$. That is, \varbasedcond{} takes
    the $q$ part of the report from $i$ (on which $i$ is credulously trusted)
    while ignoring the $\neg p$ part (which is false due to report from
    $\ast$).

\end{example}

\begin{definition}
    Define a conditioning operator \partbasedcond{} with $\X_\sigma$ as
    for \varbasedcond{}, and $\le$ defined by the ranking function
    \[
        r(W) = -\sum_{i \in \srcs}{|\Pi^W_i|}.
    \]
\end{definition}

\partbasedcond{} aims to maximise the \emph{number of cells} in the sources'
partitions, and thereby maximise the number of propositions on which they have
expertise. Unlike \varbasedcond{}, the propositional variables play no special
role. As expected, \partbasedcond{} satisfies the basic postulates.
\todo{Show?}

\begin{example}
    Applying \partbasedcond{} to $\sigma$ from
    \cref{kr_ex_model_based,kr_ex_conditioning_operator}, we no longer extract $q$
    from the report of $i$: $q \notin B^\sigma_c$ and $\E_i{q} \notin
    B^\sigma_c$. Instead, we have $\proppart{B^\sigma_c} = \cnprop(p)$, and
    $\E_i(p \lor q) \in B^\sigma_c$.
\end{example}

Note that both \varbasedcond{} and \partbasedcond{} are based on the general
principle of maximising the expertise of sources, subject to the constraint
that all reports are sound. This intuition is formalised by the following
postulate for conditioning operators. In what follows, write $W \preceq W'$ iff
$\Pi^W_i$ refines $\Pi^{W'}_i$ for all $i \in \srcs$, i.e. if all sources have
broadly more expertise in $W$ than in $W'$.\footnotemark{}

\footnotetext{
    $\Pi$ refines $\Pi'$ if $\forall A \in \Pi$, $\exists B \in \Pi'$ such that
    $A \subseteq B$.
}

\begin{axiom}[\refinement{}]
    If $W \preceq W'$ then $W \le W'$
\end{axiom}

Since $\preceq$ is only a partial order on $\W$ there are many possible total
extensions; \varbasedcond{} and \partbasedcond{} provide two specific examples.

We now turn to an axiomatic characterisation of conditioning operators.
Taken with the basic postulates from \cref{kr_sec_basic_postulates},
conditioning operators can be characterised using an approach similar to that of
\textcite{delgrande2018general} in their account of \emph{generalised AGM belief
revision}.\footnotemark{} This involves a technical property
\citeauthor{delgrande2018general}
call \axiomref{Acyc}, which finds its roots in the \emph{Loop}
property of \textcite{kraus1990nonmonotonic}.
%
\footnotetext{ Note that while the result is similar, our framework is not an
instance of theirs.  }

\begin{axiomlist}
\begin{axiom}[\incvac{}]
    $B^{\sigma \concat \rho} \sqsubseteq \cn(B^\sigma \sqcup K^\rho)$, with
    equality if $B^\sigma \sqcup K^\rho$ is consistent
\end{axiom}
\begin{axiom}[\acyc{}]
    If $\sigma_0, \ldots, \sigma_n$ are such that $K^{\sigma_j} \sqcup
    B^{\sigma_{j+1}}$ is consistent for all $0 \le j < n$ and $K^{\sigma_n}
    \sqcup B^{\sigma_0}$ is consistent, then $K^{\sigma_0} \sqcup B^{\sigma_n}$
    is consistent
\end{axiom}
\end{axiomlist}

\incvac{} is so-named since it is analogous to the combination of
\emph{Inclusion} and \emph{Vacuity} from AGM revision, if one informally views
$B^{\sigma \concat \rho}$ as the revision of $B^\sigma$ by $K^\rho$.
%
\acyc{} is the analogue of the postulate of \citeauthor{delgrande2018general},
which rules out cycles in the plausibility order
constructed in the representation result.

As with the result of \citeauthor{delgrande2018general}, a technical condition beyond
\cref{kr_def_conditioning_operator} is required to obtain the characterisation:
say that a conditioning operator is \emph{elementary} if for each $\sigma$ the
sets of worlds $\X_\sigma$ and $\Y_\sigma = \min_{\le}\X_\sigma$ are
elementary.\footnotemark{}
%
\footnotetext{
    Equivalently, there is a total preorder $\le$ such that $\mods(B^\sigma) =
    \min_{\le}\mods(K^\sigma)$ for all $\sigma$.
}

\begin{theorem}
    \label{kr_thm_conditioning_characterisation}
    Suppose an operator satisfies the basic postulates of
    \cref{kr_sec_basic_postulates}. Then it is an elementary
    conditioning operator if and only if it satisfies \incvac{} and \acyc{}.
\end{theorem}

The proof is roughly follows the lines of Theorem 4.9 in
\cite{delgrande2018general}, although some differences arise due to the form of
our input as finite sequences of reports. In fact, we will prove a result more
general than \cref{kr_thm_conditioning_characterisation}, which does not
require the full set of basic postulates to hold. For example, the general
result applies to conditioning operators which do not necessarily satisfy
\soundness{}. In order to state the result in full generality, we introduce two
auxiliary postulates -- each of which follows from the basic postulates and
\incvac{}.

\begin{axiomlist}
\begin{axiom}[\condcons{}]
    If $K^\sigma$ is consistent then so is $B^\sigma$
\end{axiom}
\begin{axiom}[\duprem{}]
    If $\tuple{i, c, \phi} \in \sigma$ then $B^{\sigma \concat \tuple{i, c,
    \phi}} = B^\sigma$ and $K^{\sigma \concat \tuple{i, c, \phi}} = K^\sigma$
\end{axiom}
\end{axiomlist}

\begin{lemma}
    \label{kr_lemma_condcons_duprem_follow}
    Any operator satisfying the basic postulates and \incvac{} also satisfies
    both \condcons{} and \duprem{}.
\end{lemma}

\begin{proof}
    For \condcons{} we show the contrapositive. Suppose $B^\sigma$ is
    inconsistent. By \consistency{}, $\sigma$ cannot be $\ast$-consistent.
    Thus, there is some case $c$ such that the reports from $\ast$ are
    inconsistent. But by \soundness{} and \closure{}, $\tuple{\ast, c, \phi}
    \in \sigma$ implies $\phi \in K^\sigma_c$. Hence $K^\sigma$ is
    inconsistent.

    For \duprem{}, suppose $\tuple{i, c, \phi} \in \sigma$. First note that if
    $\sigma$ is $\ast$-inconsistent, the argument above shows $K^\sigma$ is
    inconsistent. By \containment{}, so too is $B^\sigma$. By \closure{}, both
    $K^\sigma$ and $B^\sigma$ contain \emph{all} formulas of $\lext$. But
    $\sigma \concat \tuple{i, c, \phi}$ is also $\ast$-inconsistent, so
    $B^{\sigma \concat \tuple{i, c, \phi}} = B^\sigma$ and $K^{\sigma \concat
    \tuple{i, c, \phi}} = K^\sigma$.

    So, suppose $\sigma$ is $\ast$-consistent. By \consistency{}, $B^\sigma$ is
    consistent. By \rearr{} and \kconj{}, $\tuple{i, c, \phi} \in \sigma$
    implies $K^{\tuple{i, c, \phi}} \sqsubseteq K^\sigma$. Appealing to
    \containment{}, $K^{\tuple{i, c, \phi}} \sqsubseteq B^\sigma$. Thus
    \[
        \cn(B^\sigma \sqcup K^{\tuple{i, c, \phi}})
        = \cn(B^\sigma)
        = B^\sigma
    \]
    with \closure{} applied in the final step. Since this is a consistent
    collection, \incvac{} yields
    \[
        B^{\sigma \concat \tuple{i, c, \phi}}
        = \cn(B^\sigma \sqcup K^{\tuple{i, c, \phi}})
        = B^\sigma
    \]
    as required. For knowledge, pairing $K^{\tuple{i, c, \phi}} \sqsubseteq
    K^\sigma$ with \kconj{} gives
    \[
        K^{\sigma \concat \tuple{i, c, \phi}}
        = \cn(K^\sigma \sqcup K^{\tuple{i, c, \phi}})
        = \cn(K^\sigma)
        = K^\sigma,
    \]
    which completes the proof.
\end{proof}

\begin{lemma}
\label{kr_lemma_model_based_elementary}
    For any model-based operator and sequence $\sigma$, $\X_\sigma =
    \mod(K^\sigma)$ iff $\X_\sigma$ is elementary, and $\Y_\sigma =
    \mod(B^\sigma)$ iff $\Y_\sigma$ is elementary.
\end{lemma}

\begin{proof}
We prove the result for $\X_\sigma$ and $K^\sigma$ only. The ``only if''
direction is clear from the definition of an elementary set. For the ``if''
direction, suppose $\X_\sigma$ is elementary, i.e. $\X_\sigma = \mod(G)$ for
some collection $G$. Since $\Phi \in K^\sigma_c$ iff $\X_\sigma \subseteq
\mod_c(\Phi)$, we have $K^\sigma_c = \cn_c(G)$, i.e. $K^\sigma = \cn(G)$.
Consequently $\mod(K^\sigma) = \mod(\cn(G)) = \mod(G) = \X_\sigma$.
\end{proof}

We now come to the more general characterisation. Once proved, the main result
in \cref{kr_thm_conditioning_characterisation} easily follows in light of
\cref{kr_prop_kconj,kr_lemma_condcons_duprem_follow}.

\begin{proposition}
    \label{kr_prop_conditioning_pre_characterisation}
    Suppose an operator satisfies \closure{}, \containment{},
    \kconj{} and \equivpost{}. Then it is an elementary
    conditioning operator if and only if it satisfies \rearr{},
    \duprem{}, \condcons{},
    \incvac{} and \acyc{}.
\end{proposition}

\begin{proof}

Take some operator $\sigma \mapsto \tuple{B^\sigma, K^\sigma}$ satisfying
\closure{}, \containment{}, \kconj{} and
\equivpost{}.

``if'': Suppose the operator in question additionally satisfies
\rearr{}, \duprem{}, \condcons{},
\incvac{} and \acyc{}. For any $\sigma$, set
%
\[ \X_\sigma = \mods(K^\sigma) \] \[ \Y_\sigma = \mods(B^\sigma) \]
%
Then -- by \closure{} and \containment{} as shown in the proof of
\cref{kr_thm_model_based_characterisation} -- our operator is model based
corresponding to this choice of $\X_\sigma$ and $\Y_\sigma$. Clearly both are
elementary. We will construct a total preorder $\le$ over $\W$ such that
$\Y_\sigma = \min_{\le}{\X_\sigma}$; this will show the operator is an
elementary conditioning operator.

First, fix a function $c: \lprop / {\equiv} \to \lprop$ which chooses a fixed
representative of each equivalence class of logically equivalent propositional
formulas, i.e. any mapping such that $c([\phi]_{\equiv}) \equiv \phi$. To
simplify notation, write $\widehat{\phi}$ for $c([\phi]_{\equiv})$. Then $\phi
\equiv \widehat{\phi}$. Write $\widehat{\lprop} = \{\widehat{\phi} \mid \phi
\in \lprop\}$.  Note that $\widehat{\lprop}$ is finite (since we work with only
finitely many propositional variables) and every formula in $\lprop$ is
equivalent to exactly one formula in $\widehat{\lprop}$. For a sequence
$\sigma$, let $\widehat{\sigma}$ be the result of replacing each report
$\tuple{i, c, \phi}$ with $\tuple{i, c, \widehat{\phi}}$. Note that by
\rearr{} and \equivpost{}, $\X_{\widehat{\sigma}} =
\X_\sigma$ and $\Y_{\widehat{\sigma}} = \Y_\sigma$.

Now, for any world $W$, set
\newcommand{\reports}{\mathcal{R}}
\[
    \reports(W)
    = \{
        \tuple{i, c, \phi}
        \in \srcs \times \C \times \widehat{\lprop}
        \mid
        W \in \X_{\tuple{i, c, \phi}}
    \}
\]
Note that $\reports(W)$ is finite. For any pair of worlds $W_1$, $W_2$, let
$\rho(W_1, W_2)$ be some enumeration of $\reports(W_1) \cap \reports(W_2)$. We
establish some useful properties of $\rho(W_1, W_2)$.

    \begin{claim}
        \label{kr_claim_w_j_in_rho}
        If $\rho(W_1, W_2) \ne \emptyset$, $W_1, W_2 \in \X_{\rho(W_1, W_2)}$.
    \end{claim}
    \begin{claimproof}
        By \kconj{}, for any sequences $\sigma$, $\rho$ we have
        $K^{\sigma \concat \rho} = \cn(K^\sigma \sqcup K^\rho)$.  Taking the
        models of both sides, we have $\X_{\sigma \concat \rho} = \X_\sigma
        \cap \X_\rho$. It follows that for $\rho(W_1, W_2) \ne \emptyset$,
        \[
            \X_{\rho(W_1, W_2)}
            = \bigcap_{\tuple{i, c, \phi} \in \rho(W_1, W_2)}{\X_{\tuple{i, c,
            \phi}}}
        \]
        If $\tuple{i, c, \phi} \in \rho(W_1, W_2)$ then $W_1, W_2 \in
        \X_{\tuple{i, c, \phi}}$ by definition. Hence $W_1, W_2 \in
        \X_{\rho(W_1, W_2)}$.
    \end{claimproof}

    \begin{claim}
        \label{kr_claim_exists_delta}
        If a sequence $\sigma$ contains no equivalent reports (i.e. no distinct
        tuples $\tuple{i, c, \phi}$, $\tuple{i, c, \psi}$ with $\phi \equiv
        \psi$) and $W_1, W_2 \in \X_\sigma$, there is a sequence $\delta$ such
        that $W_1, W_2 \in \X_\delta$ and $\rho(W_1, W_2)$ is a permutation of
        $\widehat{\sigma} \concat \delta$.
    \end{claim}
    \begin{claimproof}
        If $\sigma = \emptyset$ then we can simply take $\delta = \rho(W_1,
        W_2)$. So suppose $\sigma \ne \emptyset$. By the same argument as in
        the proof of \cref{kr_claim_w_j_in_rho}, we have
        \[
            \X_\sigma = \bigcap_{\tuple{i, c, \phi} \in \sigma}{\X_{\tuple{i, c,
            \phi}}}
        \]
        Take any $\tuple{i, c, \phi} \in \widehat{\sigma}$. Then $\phi \in
        \widehat{\lprop}$, and there is $\psi \equiv \phi$ such that $\tuple{i,
        c, \psi} \in \sigma$. By \equivpost{}, we have
        \[
            W_1, W_2
            \in \X_\sigma
            \subseteq \X_{\tuple{i, c, \psi}}
            = \X_{\tuple{i, c, \phi}}
        \]
        i.e. $\tuple{i, c, \phi} \in \reports(W_1) \cap \reports(W_2)$. Hence
        $\tuple{i, c, \phi}$ appears in $\rho(W_1, W_2)$. By the assumption
        that $\sigma$ contains no equivalent reports, $\widehat{\sigma}$
        contains no duplicates. It follows that $\rho(W_1, W_2)$ can be
        permuted so that $\widehat{\sigma}$ appears as a prefix. Taking
        $\delta$ to be the sequence that remains after $\widehat{\sigma}$ in
        this permutation, we clearly have that $\rho(W_1, W_2)$ is a
        permutation of $\widehat{\sigma} \concat \delta$. Since $\sigma \ne
        \emptyset$ implies $\widehat{\sigma} \ne \emptyset$ and thus $\rho(W_1,
        W_2) \ne \emptyset$, by \rearr{}, \kconj{} and
        \cref{kr_claim_w_j_in_rho} we get
        \[
            W_1, W_2 \in \X_{\rho(W_1, W_2)}
            = \X_{\widehat{\sigma} \concat \delta}
            = \X_{\widehat{\sigma}} \cap \X_\delta
            \subseteq \X_\delta
        \]
        and we are done.
    \end{claimproof}

Now define a relation $R$ on $\W$ by
\[
    W R W' \iff W = W' \text{ or } W \in \Y_{\rho(W, W')}
\]
We have that any world in $\Y_\sigma$ $R$-precedes all worlds $\X_\sigma$.

    \begin{claim}
        \label{kr_claim_y_subset_minimal}
        If $W \in \Y_\sigma$, then for all $W' \in \X_\sigma$ we have $W R W'$
    \end{claim}
    \begin{claimproof}
        By \rearr{}, \equivpost{} and
        \duprem{}, we may assume without loss of generality that
        $\sigma$ contains no distinct equivalent reports.

        Let $W \in \Y_\sigma$ and $W' \in \X_\sigma$. Then $W \in \X_\sigma$
        too. By \cref{kr_claim_exists_delta} and \rearr{}, there is
        some sequence $\delta$ such that $\Y_{\rho(W, W')} =
        \Y_{\widehat{\sigma} \concat \delta}$ and $W, W' \in \X_\delta$.
        Consequently $W \in \Y_\sigma \cap \X_\delta = \Y_{\widehat{\sigma}}
        \cap \X_\delta$. Thus $B^{\widehat{\sigma}} \sqcup K^\delta$ is
        consistent. From \incvac{} we get
        \[
            \Y_{\widehat{\sigma} \concat \delta} = \Y_{\widehat{\sigma}} \cap
            \X_\delta
        \]
        Thus
        \[
            W \in
            \Y_{\widehat{\sigma}} \cap \X_\delta
            = \Y_{\widehat{\sigma} \concat \delta}
            = \Y_{\rho(W, W')}
        \]
        so $W R W'$ as required.
    \end{claimproof}

Now let $\le_0$ be the transitive closure of $R$. Then $\le_0$ is a (partial)
preorder. By \cref{kr_claim_y_subset_minimal}, every world in $\Y_\sigma$ is
$\le_0$-minimal in $\X_\sigma$. In fact, the converse is also true.

    \begin{claim}
        \label{kr_claim_minimal_subset_y}
        If $W \in \X_\sigma$ and there is no $W' \in \X_\sigma$ with $W' <_0
        W$, then $W \in \Y_\sigma$.
    \end{claim}
    \begin{claimproof}
        As before, assume without loss of generality that
        $\sigma$ contains no distinct equivalent reports.

        Take $W$ as in the statement of the claim. Then $\X_\sigma \ne
        \emptyset$, so $\Y_\sigma \ne \emptyset$ by
        \condcons{}. Let $W' \in \Y_\sigma$.  By
        \cref{kr_claim_y_subset_minimal}, $W' R W$ and thus $W' \le_0 W$. But by
        assumption, $W' \not<_0 W$. So we must have $W \le_0 W'$. By definition
        of $\le_0$ as the transitive closure of $R$, there are $W_0, \ldots
        W_n$ such that $W_0 = W$, $W_n = W'$ and
        \[
            W_j R W_{j+1} \qquad (0 \le j < n)
        \]
        Without loss of generality, $n > 0$ and each of the $W_j$ are distinct.
        From the definition of $R$, we therefore have that
        \[
            W_j \in \rho(W_j, W_{j+1}) \qquad (0 \le j < n)
        \]
        Now set
        \begin{align*}
            \rho_j &= \rho(W_j, W_{j+1}) \qquad (0 \le j < n) \\
            \rho_n &= \rho(W_0, W_n)
        \end{align*}
        Since $W' R W$, i.e. $W_n R W_0$, we in fact have $W_j \in \Y_{\rho_j}$
        for all $j$ (including $j = n$). For $j < n$, we also have $W_{j+1} \in
        \X_{\rho_j}$.\footnotemark{}
        %
        \footnotetext{
            If $\rho_j \ne \emptyset$ this follows from
            \cref{kr_claim_w_j_in_rho}. Otherwise, $W_{j+1} \in \Y_{\rho_{j+1}}
            \subseteq \X_{\rho_{j+1}} = \X_{\rho_{j+1} \concat \emptyset} =
            \X_{\rho_{j+1}} \cap \X_\emptyset \subseteq \X_\emptyset =
            \X_{\rho_j}$ by \kconj{}.
        }
        %
        Consequently, for $j < n$ we have
        \[
            W_{j+1} \in \X_{\rho_j} \cap \Y_{\rho_{j+1}}
        \]
        i.e. $K^{\rho_j} \sqcup B^{\rho_{j+1}}$ is consistent. Moreover, $W_0
        \in \X_{\rho_n} \cap \Y_{\rho_0}$, so $K^{\rho_n} \sqcup B^{\rho_0}$ is
        consistent. We can now apply \acyc{}: we get that $K^{\rho_0}
        \sqcup B^{\rho_n}$ is also consistent. On the one hand,
        \incvac{} and consistency of $K^{\rho_n} \sqcup
        B^{\rho_0}$ gives
        \[
            B^{\rho_0 \concat \rho_n}
            = \cn(B^{\rho_0} \sqcup K^{\rho_n})
        \]
        On the other, consistency of $B^{\rho_n} \sqcup K^{\rho_0}$ and
        \rearr{} gives
        \[
            B^{\rho_0 \concat \rho_n}
            = B^{\rho_n \concat \rho_0}
            = \cn(B^{\rho_n} \sqcup K^{\rho_0})
        \]
        Combining these and taking models, we find
        \[
            \Y_{\rho_0} \cap \X_{\rho_n}
            =
            \Y_{\rho_n} \cap \X_{\rho_0}
        \]
        In particular, since $W_0$ lies in the set on the left-hand side, we
        have $W_0 \in \Y_{\rho_n}$.

        Now, since $W_0, W_n \in \X_\sigma$ and $\rho_n = \rho(W_0, W_n)$,
        \cref{kr_claim_exists_delta} gives that there is $\delta$ with $W_0, W_n
        \in \X_{\delta}$ such that $\rho_n$ is a permutation of
        $\widehat{\sigma} \concat \delta$. Recalling that $W_n = W' \in
        \Y_\sigma = \Y_{\widehat{\sigma}}$ by assumption, we have $W_n \in
        \Y_{\widehat{\sigma}} \cap \X_\delta$, i.e. $B^{\widehat{\sigma}}
        \sqcup K^\delta$ is consistent. Applying \incvac{} once
        more, we get
        \[
            B^{\rho_n}
            = B^{\widehat{\sigma} \concat \delta}
            = \cn(B^{\widehat{\sigma}} \sqcup K^\delta)
            = \cn(B^{\sigma} \sqcup K^\delta)
        \]
        Taking models of both sides,
        \[
            \Y_{\rho_n} = \Y_{\sigma} \cap \X_\delta \subseteq \Y_\sigma
        \]
        But we already saw that $W_0 \in \Y_{\rho_n}$. Hence $W_0 \in
        \Y_\sigma$. Since $W_0 = W$, we are done.
    \end{claimproof}

To complete the proof we extend $\le_0$ to a \emph{total} preorder and show
that this does not affect the minimal elements of each $\X_\sigma$. Indeed, let
$\le$ be any total preorder extending $\le_0$ and preserving strict
inequalities, i.e. $\le$ such that
\begin{inlinelist}
    \item $W \le_0 W'$ implies $W \le W'$; and
    \item\label{kr_item_strict_ineq_preserved} $W <_0 W'$ implies $W < W'$.
\end{inlinelist}\footnote{
    Such $\le$ always exists. Indeed, note that $\le_0$ induces a partial order
    on the equivalence classes of $\W$ with respect to the symmetric part of
    $\le_0$ given by $W \simeq_0 W'$ iff $W \le_0 W'$ and $W' \le_0 W$. This
    partial order can be extended to a linear order $\le^*$ on the equivalence
    classes. Taking $W \le W'$ iff $[W]  \le^* [W']$, we obtain a total
    preorder on $\W$ with the desired properties.
}

    \begin{claim}
        For any sequence $\sigma$, $\Y_\sigma = \min_{\le}{\X_\sigma}$
    \end{claim}
    \begin{claimproof}
        Take any $\sigma$. For the left-to-right inclusion, take $W \in
        \Y_\sigma$. Then $W \in \X_\sigma$. Let $W' \in \X_\sigma$. By
        \cref{kr_claim_y_subset_minimal}, $W R W'$, so $W \le_0 W'$ and $W \le
        W'$. Hence $W$ is $\le$-minimal in $\X_\sigma$.

        For the right-to-left inclusion, take $W \in \min_{\le}{\X_\sigma}$.
        Then for any $W' \in \X_\sigma$ we have $W \le W'$. In particular, $W'
        \not< W$. By property \cref{kr_item_strict_ineq_preserved} of $\le$,
        we have $W' \not<_0 W$. Since $W'$ was an arbitrary member of
        $\X_\sigma$ and $W \in \X_\sigma$, the conditions of
        \cref{kr_claim_minimal_subset_y} are satisfied, and we get $W \in
        \Y_\sigma$.
    \end{claimproof}

This shows that our operator is an elementary conditioning operator as
required.

``only if'': Now  suppose the operator is an elementary conditioning operator.
i.e. there is a total preorder $\le$ on $\W$ and a mapping $\sigma \mapsto
\tuple{\X_\sigma, \Y_\sigma}$ such that for each $\sigma$, $\Y_\sigma =
\min_{\le}{X_\sigma}$, $\X_\sigma$ and $\Y_\sigma$ are elementary, and
$K^\sigma$, $B^\sigma$ are determined by $\X_\sigma$, $\Y_\sigma$ respectively
according to \cref{kr_def_model_based}. By elementariness and
\cref{kr_lemma_model_based_elementary}, $\X_\sigma = \mod(K^\sigma)$ and $\Y_\sigma
= \mod(B^\sigma)$.

The following claim will be useful at various points.

    \begin{claim}
        \label{kr_claim_x_equal_implies_output_equal}
        Suppose $\sigma$ and $\rho$ are such that $\X_\sigma = \X_\rho$. Then
        $K^\sigma = K^\rho$ and $B^\sigma = B^\rho$.
    \end{claim}
    \begin{claimproof}
        Since the total preorder $\le$ is fixed, we have
        \[
            Y_\sigma
            = \min\nolimits_{\le}{\X_\sigma}
            = \min\nolimits_{\le}{\X_\rho}
            = \Y_\rho
        \]
        Now, $\X_\sigma = \X_\rho$ means $\mod(K^\sigma) = \mod(K^\rho)$, so
        $\cn(K^\sigma) = \cn(K^\rho)$. By \closure{}, $K^\sigma = K^\rho$.
        Similarly, $\Y_\sigma = \Y_\rho$ gives $B^\sigma = B^\rho$.
    \end{claimproof}

We take the postulates to be shown in turn.

\begin{itemize}
    \item \rearr{}: Suppose $\sigma$ is a permutation of $\rho$.
          Without loss of generality, $\sigma, \rho \ne \emptyset$. Repeated
          application of \kconj{} gives
          \[
            \X_{\sigma} = \bigcap_{\tuple{i, c, \phi} \in \sigma}{\X_{\tuple{i,
            c, \phi}}}
          \]
          Since $\sigma$ and $\rho$ contain exactly the same reports -- just in
          a different order -- commutativity and associativity of intersection
          of sets gives $\X_\sigma = \X_\rho$. \rearr{} follows
          from \cref{kr_claim_x_equal_implies_output_equal}.

      \item \duprem{}: Suppose $\tuple{i, c, \phi} \in \sigma$. Then there is a
          (possibly empty) sequence $\rho$ such that $\sigma$ is a permutation
          of $\rho \concat \tuple{i, c, \phi}$. By \rearr{} just shown and
          \kconj{}, we have
          \begin{align*}
              \X_{\sigma \concat \tuple{i, c, \phi}}
              &= \X_{\rho \concat \tuple{i, c, \phi} \concat \tuple{i, c, \phi}} \\
              &= \X_\rho \cap \X_{\tuple{i, c, \phi}} \cap \X_{\tuple{i, c, \phi}} \\
              &= \X_\rho \cap \X_{\tuple{i, c, \phi}} \\
              &= \X_{\rho \concat \tuple{i, c, \phi}} \\
              &= \X_\sigma
          \end{align*}
          and we may conclude by \cref{kr_claim_x_equal_implies_output_equal}.
    \item \condcons{}: Suppose $K^\sigma$ is consistent,
          i.e. $\X_\sigma \ne \emptyset$. Since $\W$ is finite, $\X_\sigma$ is
          finite and thus some $\le$-minimal world must exist in $\X_\sigma$.
          Hence $\Y_\sigma \ne \emptyset$, so $B^\sigma$ is consistent.
    \item \incvac{}: Take any sequences $\sigma$, $\rho$. First
          we show $B^{\sigma \concat \rho} \sqsubseteq \cn(B^\sigma \sqcup
          K^\rho)$, or equivalently, $\Y_{\sigma \concat \rho} \supseteq
          \Y_\sigma \cap \X_\rho$. Suppose $W \in \Y_\sigma \cap \X_\rho$.
          Since $\Y_\sigma \subseteq \X_\sigma$, we have $W \in \X_\sigma \cap
          \X_\rho = \X_{\sigma \concat \rho}$ by \kconj{}. We need
          to show $W$ is minimal. Take any $W' \in \X_{\sigma \concat \rho}$.
          Then $W' \in \X_{\sigma}$, so $W \in \Y_\sigma =
          \min_{\le}{\X_\sigma}$ gives $W \le W'$. Hence $W \in
          \min_{\le}{X_{\sigma \concat \rho}} = \Y_{\sigma \concat \rho}$.

          Now suppose $B^\sigma \sqcup K^\rho$ is consistent, i.e. $\Y_\sigma
          \cap \X_\rho \ne \emptyset$. Take some $\widehat{W} \in \Y_\sigma
          \cap \X_\rho$. We need to show $B^{\sigma \concat \rho} \sqsupseteq
          \cn(B^\sigma \sqcup K^\rho)$, i.e. $\Y_{\sigma \concat \rho}
          \subseteq \Y_\sigma \cap \X_\rho$. To that end, let $W \in \Y_{\sigma
          \concat \rho}$. Then $W \in \X_{\sigma \concat \rho} = \X_\sigma \cap
          \X_\rho \subseteq \X_\rho$, so we only need to show $W \in
          \Y_\sigma$. Take any $W' \in \X_\sigma$. Then $\widehat{W} \in
          \Y_\sigma$ gives $\widehat{W} \le W'$. But $\widehat{W} \in \X_\sigma
          \cap \X_\rho = \X_{\sigma \concat \rho}$ and $W \in \Y_{\sigma
          \concat \rho}$ gives $W \le \widehat{W}$. By transitivity of $\le$,
          we have $W \le W'$.  Hence $W \in \min_{\le}{\X_\sigma} = \Y_\sigma$.

    \item \acyc{}: Let $\sigma_0, \ldots, \sigma_n$ be as in the statement
          of \acyc{}. Without loss of generality, $n > 0$. Then there are
          $W_0, \ldots, W_n$ such that
          \begin{align*}
            W_j &\in \X_{\sigma_j} \cap \Y_{\sigma_{j+1}} \qquad (0 \le j < n) \\
            W_n &\in \X_{\sigma_n} \cap \Y_{\sigma_0}
          \end{align*}
          Note that $W_j \in \X_{\sigma_j}$ for all $j$. For $j < n$, we also
          have $W_j \in \Y_{\sigma_{j+1}} = \min_{\le}{X_{\sigma_{j+1}}}$. It
          follows that $W_j \le W_{j+1}$ for such $j$, so
          \[
            W_0 \le \cdots \le W_n
          \]
          But we also have $W_n \in \Y_{\sigma_0} = \min_{\le}{\X_{\sigma_0}}$
          and $W_0 \in \X_{\sigma_0}$, so $W_n \le W_0$. By transitivity of
          $\le$, the chain flattens: we have
          \[
            W_0 \simeq \cdots \simeq W_n
          \]
          Now note that since $W_{n-1} \in \Y_{\sigma_n}$, $W_{n - 1}$ is
          minimal in $\X_{\sigma_n}$. But $W_n \in \X_{\sigma_n}$ and $W_{n -
          1} \simeq W_n$ by the above, so in fact $W_n \in \Y_{\sigma_n}$ too.
          Hence
          \begin{align*}
            W_n
            &\in \Y_{\sigma_0} \cap \Y_{\sigma_n} \\
            &\subseteq \X_{\sigma_0} \cap \Y_{\sigma_n} \\
            &= \mod(K^{\sigma_0} \sqcup B^{\sigma_n})
          \end{align*}
          i.e. $K^{\sigma_0} \sqcup B^{\sigma_n}$ is consistent, as required
          for \acyc{}.
\end{itemize}
\end{proof}

Note that while the requirement in \cref{kr_thm_conditioning_characterisation}
that $\X_{\sigma}$ and $\Y_{\sigma}$ are elementary is a technical
condition,\footnotemark{} the characterisation in
\cref{kr_prop_elementary_characterisation} implies a simple sufficient condition
for elementariness.
%
\footnotetext{
    \incvac{} may fail for non-elementary conditioning.
}

\begin{proposition}
    \label{kr_prop_partition_equiv_tpo_implies_elementariness}
    Suppose $\le$ is such that $W \simeq W'$ whenever $W$ and $W'$ are
    partition-equivalent. Then $\min_{\le}{S}$ is elementary for any
    elementary set $S \subseteq \W$.
\end{proposition}

\begin{proof}
    We use the characterisation of elementary sets from
    \cref{kr_prop_elementary_characterisation}. Take $S \subseteq \W$ elementary.
    Suppose $W \in \W$, $W_1, W_2 \in \min_{\le}{S}$ are such that $W$ is
    partition-equivalent to both $W_1, W_2$ and $W$ is a valuation combination
    from $\{W_1, W_2\}$. By hypothesis we have $W \simeq W_1 \simeq W_2$.

    Now since $\min_{\le}{S} \subseteq S$, we have $W_1, W_2 \in S$. Since $S$
    is elementary, $W \in S$. But now $W \simeq W_1$ and $W_1 \in
    \min_{\le}{S}$ gives $W \in \min_{\le}{S}$. This shows the required closure
    property for $\min_{\le}{S}$, and we are done.
\end{proof}

\Cref{kr_prop_partition_equiv_tpo_implies_elementariness} implies that
\varbasedcond{} and \partbasedcond{} are elementary. Indeed, for both operators
$\X_\sigma = \mods(G^\sigma_\snd)$ so is elementary by definition. Since the
ranking $\le$ for each operator only depends on the partitions of worlds,
$\Y_\sigma =
\min_{\le}{\X_\sigma}$ is elementary also.

\subsection{Score-Based Operators}
\label{kr_sec_score_based}

The fact that the plausibility order $\le$ of a conditioning operator is fixed
may be too limiting. For example, consider
\[
    \sigma
    =
    (\tuple{i, c, p},
    \tuple{j, c, \neg p},
    \tuple{i, d, p}).
\]
%
If one sets $\X_\sigma$ to satisfy the soundness constraints (i.e. as in
\weakop{}), there is a possible
world $W_1 \in \X_\sigma$ with $W_1, d \models \neg \E_i{p} \land \E_j{p} \land
\neg p$ (i.e. $W_1$ sides with source $j$ and $p$ is false at $d$) and another
world $W_2 \in \X_\sigma$ with $W_2, d \models \E_i{p} \land \neg \E_j{p} \land
p$ (i.e. $W_2$ sides with source $i$). Appealing to symmetry, one may argue
that neither world is \emph{a priori} more plausible than the other, so any
fixed plausibility order should have $W_1 \simeq W_2$. If these worlds
are maximally plausible (e.g. if taking the ``optimistic'' view outlined in
\cref{kr_ex_hospital_ex_formalised}), conditioning gives $p \notin B^\sigma_d$ and
$\neg p \notin B^\sigma_d$.
%
However, there is an argument that $W_2$ should be considered more plausible
than $W_1$ \emph{given the sequence $\sigma$}, since $W_2$ validates the final
report $\tuple{i, d, p}$ whereas $W_1$ does not. Consequently, there is an
argument that we should in fact have $p \in B^\sigma_d$.\footnote{At the very
least, the case $p \in B^\sigma_d$ should not be \emph{excluded}.} This shows
that we need the plausibility order to be responsive to the input sequence for
adequate belief change.\footnotemark{}

\footnotetext{
    In \cref{kr_sec_one_step_postulates} we make this argument more precise
    by providing an impossibility result which shows conditioning operators
    with some basic properties cannot accept $p$ in sequences such as this.
}

As a result of this discussion, we look for operators whose plausibility
ordering can depend on $\sigma$. One approach to achieve this in a controlled
way is to have a ranking for each
\emph{report} $\tuple{i, c, \phi}$, and combine these to construct a ranking
for each sequence $\sigma$. We represent these rankings by \emph{scoring
functions}, and call the resulting operators \emph{score-based}.

\todo{Also consider presenting the ``Cancellation'' postulate as a case against
simple conditioning.}

\begin{definition}
\label{kr_def_score_based}
    An operator is \emph{score-based} if there is a mapping $\sigma \mapsto
    \tuple{\X_\sigma, \Y_\sigma}$ as in \cref{kr_def_model_based} and functions
    $r_0: \W \to \Ninf$, $d: \W \times (\srcs \times \C \times \lprop) \to \Ninf$
    such that $\X_\sigma = \{W \mid r_\sigma(W) < \infty\}$ and $\Y_\sigma =
    \argmin_{W \in \X_\sigma}{r_\sigma(W)}$, where
    \[
        r_\sigma(W) = r_0(W) + \sum\nolimits_{\tuple{i, c, \phi} \in \sigma}{
            d(W, \tuple{i, c, \phi})
        }.
    \]

\end{definition}

Here $r_0(W)$ is the \emph{prior implausibility score} of $W$, and $d(W,
\tuple{i, c, \phi})$ is the \emph{disagreement score} for world $W$ and $\tuple{i,
c, \phi}$. The set of most plausible worlds $\Y_\sigma$ consists of those $W$
which minimise the sum of the prior implausibility and the total
disagreement with $\sigma$. Note that by summing the scores
of each report $\tuple{i, c, \phi}$ with equal weight, we treat each report independently.
This construction is informally inspired by the form of the so-called
\emph{Markovian observation systems} of \textcite[Eq.
(5)]{boutilier1998belief}.

Score-based operators generalise elementary conditioning operators with \kconj{}.

\begin{proposition}
\label{kr_prop_kconj_conditioning_implies_score_based}
    Any elementary conditioning operator satisfying \kconj{} is score-based.
\end{proposition}

\begin{proof}
    Take any elementary conditioning operator corresponding to some mapping
    $\sigma \mapsto \tuple{\X_\sigma, \Y_\sigma}$ and total preorder $\le$, and
    suppose \kconj{} holds. Write
    \[
        k(W) = |\{W' \in \W \mid W' \le W\}|
    \]
    Then we have $W \le W'$ iff $k(W) \le k(W')$. Set
    \[
        r_0(W) = \begin{cases}
            \infty,& W \notin \X_\emptyset \\
            k(W),& W \in \X_\emptyset
        \end{cases},
    \]
    \[
        d(W, \tuple{i, c, \phi}) = \begin{cases}
            \infty,& W \notin \X_{\tuple{i, c, \phi}} \\
            0,& W \in \X_{\tuple{i, c, \phi}} \\
        \end{cases}.
    \]
    For any sequence $\sigma$, repeated applications of \kconj{} (and the fact
    that $\X_\sigma$ is elementary) give $r_\sigma(W) < \infty$ iff $W \in
    \X_\sigma$. Similarly, the choice of $r_0$ gives $\argmin_{W \in
    \X_\sigma}{r_\sigma(W)} = \min_{\le}{\X_\sigma} = \Y_\sigma$. Hence the
    operator is score-based.
\end{proof}

We now give a concrete example.

\begin{definition}
    \label{kr_def_scorebasedop}
    Define a score-based operator \scorebasedop{} by setting
    $r_0(W) = 0$ and
    \[
        d(W, \tuple{i, c, \phi}) = \begin{cases}
            |\Pi^W_i[\phi] \setminus \propmods{\phi}|,& W, c \models \S_i\phi \\
            \infty,& \text{ otherwise. }
        \end{cases}
    \]
\end{definition}

The set of possible worlds $\X_\sigma$ is the same as for the earlier
operators.
All worlds are \emph{a priori} equiplausible according to
$r_0$. The disagreement score $d$ is defined as the number of propositional
valuations in the ``excess'' of $\Pi^W_i[\phi]$ which are not models of
$\phi$, i.e. the number of $\neg\phi$ valuations which are indistinguishable
from some $\phi$ valuation.
%
% The worlds which minimise $d(W, \tuple{i, c, \phi})$ are therefore those in
% which there are only a small number of $\neg \phi$ valuations which $i$
% cannot distinguish from some $\phi$ valuation.
The intuition here is that \emph{sources tend to only report formulas on which
they have expertise}. The minimum score 0 is attained exactly when $i$ has
expertise on $\phi$; other worlds are ordered by how much they deviate from
this ideal.

One can verify that \scorebasedop{} satisfies the basic postulates of
\cref{kr_sec_basic_postulates}. It can also be seen that $\X_\sigma$ and
$\Y_\sigma$ are elementary, and \scorebasedop{} fails \incvac{}.
%
\todo{
    Show? It is enough to show that \duprem{} fails, since \incvac{} plus the
    basic postulates implies \duprem{}.
}
%
It follows from
\cref{kr_thm_conditioning_characterisation} that \scorebasedop{} is \emph{not} a
conditioning operator.\footnotemark{}

\footnotetext{
    We will later give an alternative proof of this fact, via an impossibility
    result for conditioning operators
    (\cref{kr_prop_strongcondsucc_conditioning_impossibilitity}).
}

\begin{example}
\label{kr_ex_score_based}
     To illustrate the differences between \scorebasedop{} and conditioning,
     consider a more elaborate version of the
     example given at the start of this \lcnamecref{kr_sec_score_based}:
     \[
        \sigma = (
            \tuple{i, c, p \limplies q},
            \tuple{j, c, p \limplies \neg q},
            \tuple{\ast, c, p},
            \tuple{i, d, p},
            \tuple{i, d, q}
        ).
     \]
     Here the reports of $i$ and $j$ in case $c$ are consistent,
     but inconsistent when taken with
     the reliable information $p$ from $\ast$. Should we believe $q$ or $\neg
     q$? Both our conditioning operators \varbasedcond{} and \partbasedcond{}
     decline to decide, and have $\proppart{B^\sigma_c} = \cnprop(p)$. However,
     since \scorebasedop{} takes into account each report in the
     sequence, the fact that $i$ reports both $p$ and $q$ in case $d$ -- and is
     uncontested on these reports -- leads to
     $\E_i{p} \land \E_i{q} \in B^\sigma_c$. This gives $\E_i(p \limplies q)
     \in B^\sigma_c$ by \cref{kr_prop_validities} part
     \cref{kr_item_exp_on_all_variables}, so we can make use of the report
     from $i$ in case $c$: we have $\proppart{B^\sigma_c} = \cnprop(p \land
     q)$.
     %
     This example shows that score-based operators can be \emph{more credulous}
     than conditioning operators (e.g. we can believe $\E_i{p}$ when
     $i$ reports $p$), and can consequently hold stronger propositional
     beliefs. Indeed, the conditioning operators \varbasedcond{} and
     \partbasedcond{} are not able to make use of the reports from $i$ in case
     $d$ to form beliefs about case $c$, and in this sense do not make full use
     of the available information in $\sigma$.

\end{example}

\section{One-Step Revision}
\label{kr_sec_one_step_postulates}

The postulates of \cref{kr_sec_basic_postulates} only set out very basic
requirements for an operator. In this section we introduce some more demanding
postulates which address how beliefs should change when a sequence $\sigma$ is
extended by a new report $\tuple{i, c, \phi}$.  In view of \rearr{}, we do not
view this process as \emph{revision} of $B^\sigma$ by $\tuple{i, c, \phi}$, but
rather as \emph{reinterpretation} of $\sigma$ in light of a new report
$\tuple{i, c, \phi}$. The postulates we introduce can therefore be seen as
\emph{coherency} requirements, which place some constraints on this
reinterpretation.

First, we address how propositional beliefs should be affected by reliable
information.

\begin{axiom}[\agm{}]
    For any $\sigma$ and $c \in \C$ there is an AGM operator $\star$
    for $\proppart{B^\sigma_c}$ such that $\proppart{B^{\sigma \concat
    \tuple{\ast, c, \phi}}_c} = \proppart{B^\sigma_c} \star \phi$ whenever
    $\neg\phi \notin K^\sigma_c$
\end{axiom}

\agm{} says that receiving information from the reliable source $\ast$ acts in
accordance with the well-known AGM postulates~\cite{alchourron1985logic} for propositional belief
revision (provided we are not in the degenerate case where the new report
$\phi$ was already \emph{known} to be false). Since AGM revision operators are
characterised by total preorders over valuations
\cite{grove1988two,katsuno_1991}, it is no surprise that our order-based
constructions are consistent with \agm{}.

\begin{proposition}
    \label{kr_prop_examples_satisfy_agm}
    \varbasedcond{}, \partbasedcond{} and \scorebasedop{}
    satisfy \agm{}.
\end{proposition}

We require some preliminary results. For a case $c \in \C$ and valuation $v \in
\vals$, write $\W_{c\ :\ v} = \{W \in \W \mid v^W_c = v\}$ for the set of
worlds whose $c$ valuation is $v$.

\begin{lemma}
    \label{kr_lemma_model_based_models_of_proppart}
    For any model-based operator, sequence $\sigma$, case $c$, and valuation
    $v \in \vals$,
    \[
        v \in \propmods{\proppart{B^\sigma_c}}
        \iff
        \Y_\sigma \cap \W_{c\ :\  v} \ne \emptyset
    \]
\end{lemma}

\begin{proof}
    $\implies$: We show the contrapositive. Suppose $\Y_\sigma \cap \W_{c\
    :\ v} = \emptyset$. Let $\psi$ be any propositional formula such that
    $\propmods{\psi} = \vals \setminus \{v\}$. Now for any $W \in \Y_\sigma$,
    we have $W \notin \W_{c\ :\  v}$, i.e. $v^W_c \ne v$. Hence $v^W_c \in
    \propmods{\psi}$, so $W, c \models \psi$. By definition of the belief set
    of a model-based operator, we have $\psi \in B^\sigma_c$. But $\psi$ is a
    propositional formula, so $\psi \in \proppart{B^\sigma_c}$. Since $v \notin
    \propmods{\psi}$, we have $v \notin \propmods{\proppart{B^\sigma_c}}$.

    $\impliedby$: Suppose there is some $W \in \Y_\sigma \cap \W_{c\ :\ v}$.
    Let $\phi \in \proppart{B^\sigma_c}$. Then, in particular, $\phi \in
    B^\sigma_c$, so $W, c \models \phi$ by $W \in \Y_\sigma$ and the definition
    of the model-based belief set. That is, $v = v^W_c \in \propmods{\phi}$.
    Since $\phi \in \proppart{B^\sigma_c}$ was arbitrary, we have $v \in
    \propmods{\proppart{B^\sigma_c}}$.
\end{proof}

We have a sufficient condition for \agm{} for score-based operators.

\begin{lemma}
    \label{kr_lemma_score_based_agm_sufficient_conditions}
    Suppose a score-based operator is such that for each $c \in \C$ and $\phi
    \in \lprop$ there is a constant $M \in \mathbb{N}$ with
    \[
        d(W, \tuple{\ast, c,  \phi})
        = \begin{cases}
            M,& W, c \models \phi  \\
            \infty,& W, c \models \neg \phi
        \end{cases}
    \]
    for all $W$. Then \agm{} holds.
\end{lemma}

\begin{proof}
    Take a score-based operator with the stated property. Let $\sigma$ be a
    sequence and take $c \in \C$. Without loss of generality, there is some
    $\phi \in \lprop$ such that $\neg\phi \notin K^\sigma_c$ (otherwise \agm{}
    trivially holds). Since any score-based operator is model-based and
    therefore satisfies \closure{}, we have that $K^\sigma$ is inconsistent iff
    $K^\sigma_c = \lext$. But since $K^\sigma_c$ does not contain $\neg\phi$,
    it must be the case that $K^\sigma$ is consistent.

    Now, set
    \[
        k(v)
        = \min\{
            r_\sigma(W)
            \mid
            W \in \X_\sigma \cap \W_{c\ :\ v}
        \}
    \]
    where $\min\emptyset =  \infty$. Note that $k(v) = \infty$ if and only if
    $\X_\sigma \cap \W_{c\ :\  v} = \emptyset$. Then $k$ defines a total
    preorder $\preceq$ on valuations, where $v \preceq v'$ iff $k(v) \le
    k(v')$. Define a propositional revision operator $\star$ for
    $\proppart{B^\sigma_c}$ by
    \[
        \proppart{B^\sigma_c} \star \phi = \{
            \psi \in \lprop
            \mid
            \min\nolimits_{\preceq}{\propmods{\phi}} \subseteq \propmods{\psi}
        \}
    \]
    To show that $\star$ satisfies the AGM postulates (for
    $\proppart{B^\sigma_c}$) it is sufficient to show that the models of
    $\proppart{B^\sigma_c}$ are exactly the $\preceq$-minimal valuations.

        \begin{claim}
            $\propmods{\proppart{B^\sigma_c}} = \min_{\preceq}{\vals}$.
        \end{claim}
        \begin{claimproof}

            ``$\subseteq$'': let $v \in \propmods{\proppart{B^\sigma_c}}$. By
            \cref{kr_lemma_model_based_models_of_proppart}, there is some $W \in
            \Y_\sigma \cap \W_{c\ :\  v}$. Since $W \in \X_\sigma$ too, by
            definition of $k$ we have $k(v) \le r_\sigma(W) < \infty$. Now let
            $v' \in \vals$. Without loss of generality assume $k(v') < \infty$.
            Then there is some $W' \in \X_\sigma \cap \W_{c\ :\  v'}$ such that
            $k(v') = r_\sigma(W')$. But $W' \in \X_\sigma$ and $W \in
            \Y_\sigma$ gives $r_\sigma(W) \le r_\sigma(W')$, so
            \[
                k(v) \le r_\sigma(W) \le r_\sigma(W') = k(v')
            \]
            i.e. $v \preceq v'$. Hence $v$ is $\preceq$-minimal.

            ``$\supseteq$'': let $v \in \min_{\preceq}\vals$. Since $K^\sigma$
            is consistent, there is some $\hat{W} \in \X_\sigma$.  Writing
            $\hat{v} = v^{\hat{W}}_c$, we have $\hat{W} \in \X_\sigma \cap
            \W_{c\ :\ \hat{v}}$, so $v \preceq \hat{v}$ implies
            \[
                k(v)
                \le k(\hat{v})
                \le r_\sigma(\hat{W})
                < \infty
            \]
            Hence there must be some $W \in \X_\sigma \cap \W_{c\ :\ v}$ such
            that $k(v) = r_\sigma(W)$. We claim that, in fact, $W \in
            \Y_\sigma$. Indeed, for any $W' \in \X_\sigma$ we have $v \preceq
            v^{W'}_c$, so
            \[
                r_\sigma(W)
                = k(v)
                \le k(v^{W'}_c)
                \le r_\sigma(W')
            \]
            That is, $W \in \Y_\sigma \cap \W_{c\ :\  v}$. By
            \cref{kr_lemma_model_based_models_of_proppart}, $v \in
            \propmods{\proppart{B^\sigma_c}}$.
        \end{claimproof}

    So, $\star$ is indeed an AGM operator for $\proppart{B^\sigma_c}$. Now take
    $\phi \in \lprop$ such that $\neg\phi \notin K^\sigma_c$. Write $\rho =
    \sigma \concat \tuple{\ast, c, \phi}$. We claim the following.

        \begin{claim}
            \label{kr_claim_propmods_rho_min_models_of_phi}
            $\propmods{\proppart{B^\rho_c}} =
            \min\nolimits_{\preceq}{\propmods{\phi}}$.
        \end{claim}
        \begin{claimproof}
            ``$\subseteq$'': let $v \in \propmods{\proppart{B^\rho_c}}$.  By
            \cref{kr_lemma_model_based_models_of_proppart} again, there is some $W
            \in \Y_\rho \cap \W_{c\ :\  v}$. Since $\tuple{\ast, c, \phi} \in
            \rho$ and $d(W, \tuple{\ast, c, \phi}) \le r_\rho(W) < \infty$, we
            must have $W, c \models \phi$ by the assumed property of the score
            function $d$. Hence $v = v^W_c \in \propmods{\phi}$.

            Now since $\Y_\rho \subseteq \X_\rho$, we have $W \in \Y_\rho
            \subseteq \X_\rho \subseteq \X_\sigma$, so $W \in \X_\sigma \cap
            \W_{c\ :\  v}$. By definition of $k$, we have $k(v) \le
            r_\sigma(W)$. Take any $v' \in \propmods{\phi}$. Without loss of
            generality, assume $k(v') < \infty$, so that there is some $W' \in
            \X_\sigma \cap \W_{c\ :\  v'}$ with $k(v') = r_\sigma(W')$. Since
            $v^{W'}_c = v' \in \propmods{\phi}$, we have $W', c \models \phi$.
            Consequently, by the property of $d$ again, $d(W', \tuple{\ast, c,
            \phi}) = M$. Since $W' \in \X_\sigma$ gives $r_\sigma(W') <
            \infty$, it follows that
            \[
                r_\rho(W') = r_\sigma(W') + M < \infty
            \]
            so $W' \in \X_\rho$.
            %
            Recall that $W, c \models \phi$ too, so $d(W, \tuple{\ast, c,
            \phi}) = M$ also. From $W \in \Y_\rho$ and $W' \in \X_\rho$ we get
            \begin{align*}
                r_\sigma(W)
                &= r_\rho(W) - M \\
                &\le r_\rho(W') - M \\
                &= r_\rho(W') - d(W', \tuple{\ast, c, \phi}) \\
                &= r_\sigma(W')
            \end{align*}
            This yields
            \[
                k(v) \le r_\sigma(W) \le r_\sigma(W') = k(v')
            \]
            and $v \preceq v'$ as required.


            ``$\supseteq$'': let $v \in \min_{\preceq}{\propmods{\phi}}$. Since
            $\neg\phi \notin K^\sigma_c$, there is some $\hat{W} \in \X_\sigma$
            such that $\hat{W}, c \models \phi$. Writing $\hat{v} =
            v^{\hat{W}}_c$, we have $\hat{v} \in \propmods{\phi}$. Hence $v
            \preceq \hat{v}$. This implies
            \[
                k(v) \le k(\hat{v}) \le r_\sigma(\hat{W}) < \infty
            \]
            so there must be some $W \in \X_\sigma \cap \W_{c\ :\  v}$ with
            $k(v) = r_\sigma(W)$. Since $v^W_c = v \in \propmods{\phi}$, we
            have $W, c \models \phi$. By the assumed property of $d$, we get
            $d(W, \tuple{\ast, c, \phi}) = M$. Hence
            \[
                r_\rho(W)
                = r_\sigma(W) + d(W, \tuple{\ast, c, \phi})
                = r_\sigma(W) + M
                < \infty
            \]
            so $W \in \X_\rho$ too. We will show that $W \in \Y_\rho$. Let $W'
            \in \X_\rho$. Then we must have $d(W', \tuple{\ast, c, \phi}) = M$
            and $W', c \models \phi$. That is, $v^{W'}_c \in \propmods{\phi}$.
            By minimality of $v$, we have $v \preceq v^{W'}_c$. Noting that $W'
            \in \X_\rho \subseteq \X_\sigma$, we get
            \[
                r_\sigma(W)
                = k(v)
                \le k(v^{W'}_c)
                \le r_\sigma(W')
            \]
            Consequently,
            \[
                r_\rho(W)
                = r_\sigma(W) + M
                \le r_\sigma(W') + M
                = r_\rho(W')
            \]
            This shows $W \in \Y_\rho$, i.e. $\Y_\rho \cap \W_{c\ :\  v} \ne
            \emptyset$. By \cref{kr_lemma_model_based_models_of_proppart}, we are
            done.
        \end{claimproof}

    Noting that $\propmods{\proppart{B^\sigma_c}} \star \phi =
    \min_{\preceq}{\propmods{\phi}}$, it follows from
    \cref{kr_claim_propmods_rho_min_models_of_phi} that
    $\cnprop(\proppart{B^\rho_c}) = \cnprop(\proppart{B^\sigma_c} \star \phi)$.
    But $\proppart{B^\rho_c}$ is deductively closed by \closure{}, and
    $\proppart{B^\sigma_c}
    \star \phi$ is deductively closed by construction. Hence
    $\proppart{B^\rho_c} = \proppart{B^\sigma_c} \star \phi$, as required for
    \agm{}.
\end{proof}

As a consequence of \cref{kr_prop_kconj_conditioning_implies_score_based} (and the
construction of $d$ in its proof), one can apply
\cref{kr_lemma_score_based_agm_sufficient_conditions} with $M = 0$ for
conditioning operators with \kconj{} and a certain natural property.

\begin{corollary}
    \label{kr_cor_conditioning_agm}
    Suppose an elementary conditioning operator satisfying \kconj{} has the
    property that
    \[
        W \in \X_{\tuple{\ast, c, \phi}} \iff W, c \models \phi
    \]
    Then \agm{} holds.
\end{corollary}

We can now prove \cref{kr_prop_examples_satisfy_agm}.

\begin{proof}[Proof of \cref{kr_prop_examples_satisfy_agm}]
    For the conditioning operators \varbasedcond{} and \partbasedcond{}, it is
    easily verified that the condition in \cref{kr_cor_conditioning_agm} holds,
    and thus \agm{} does also.
    %
    For the score-based operator \scorebasedop{}, we may use
    \cref{kr_lemma_score_based_agm_sufficient_conditions} with $M = 0 $.
\end{proof}

Thus, we do indeed extend AGM revision in the case of reliable information.
What about non-reliable information? First note that the analogue of \agm{} for
ordinary sources $i \ne \ast$ is \emph{not} desirable. In particular, we
should not have the \axiomref{Success} postulate:
\[
    \phi \in B^{\sigma \concat \tuple{i, c, \phi}}_c.
\]
Indeed, the sequence in \cref{kr_ex_model_based} with $\phi = \neg p
\land q$ already shows that \axiomref{Success} would conflict with the
basic postulates.
%
However, there are weaker modifications of \axiomref{Success} which may be
more appropriate. We consider two such postulates.

\begin{axiomlist}
\begin{axiom}[\condsucc{}]
    If $\E_i\phi \in B^\sigma_c$ and $\neg\phi \notin B^\sigma_c$, then $\phi
    \in B^{\sigma \concat \tuple{i, c, \phi}}_c$
\end{axiom}
\begin{axiom}[\strongcondsucc{}]
    If $\neg(\E_i\phi \land \phi) \notin B^\sigma_c$, then $\phi \in B^{\sigma
    \concat \tuple{i, c, \phi}}_c$
\end{axiom}
\end{axiomlist}

\condsucc{} says that if $i$ is deemed an expert on $\phi$, which is consistent
with current beliefs, then $\phi$ is accepted after $i$ reports it. That is,
the acceptance of $\phi$ is \emph{conditional} on prior beliefs about the expertise of
$i$ (on $\phi$). \strongcondsucc{} weakens the antecedent by only requiring
that $\E_i\phi$ and $\phi$ are jointly consistent with current beliefs (i.e.
$i$ need not be considered an expert on $\phi$). In other words, we should
believe reports if there is no reason not to. It is easily shown that
\closure{} and \strongcondsucc{} implies \condsucc{}.
%
We once again revisit our examples.

\begin{proposition}
    \label{kr_prop_examples_satisfy_condsucc}
    \varbasedcond{}, \partbasedcond{} and \scorebasedop{}
    satisfy \condsucc{}, and \scorebasedop{} additionally
    satisfies \strongcondsucc{}.
\end{proposition}

As a first step in the proof, we present sufficient conditions for conditioning
operators to satisfy \condsucc{}. In fact, we do not need to impose any
condition on the total preorder $\le$: a natural constraint on the mapping
$\sigma \mapsto \X_\sigma$ (together with some basic postulates) is enough.

\begin{lemma}
    \label{kr_lemma_conditioning_condsucc_sufficient_condition}
    Suppose an elementary conditioning operator satisfies \kconj{},
    \soundness{} and
    \[
        W, c \models \phi \implies W \in \X_{\tuple{i, c, \phi}}
    \]
    Then \condsucc{} holds.
\end{lemma}

\begin{proof}
    Suppose an elementary conditioning operator corresponding to the mapping
    $\sigma \mapsto \tuple{\X_\sigma, \Y_\sigma}$ and total preorder $\le$
    satisfies \kconj{}, \soundness{} and has the stated property.

    Let $\sigma$ be a sequence and $c \in \C$. Suppose $\E_i\phi \in
    B^\sigma_c$ and $\neg\phi \notin B^\sigma_c$. Write $\rho = \sigma \concat
    \tuple{i, c, \phi}$. We need to show $\phi \in B^\rho_c$.

    By $\neg\phi \notin B^\sigma_c$, there is some $W \in \Y_\sigma$ such that
    $W, c \models \phi$. Hence $W \in \X_{\tuple{i, c, \phi}}$. By
    elementariness and \kconj{}, we have $\X_\rho = \X_\sigma \cap
    \X_{\tuple{i, c, \phi}}$. Since $W \in \Y_\sigma \subseteq \X_\sigma$, we
    get $W \in \X_\rho$.

    Now take any $W' \in \Y_\rho$. Then $W'$ is $\le$-minimal in $\X_\rho$, so
    $W' \le W$. But $W$ is $\le$-minimal in $\X_\sigma$, so $W' \in \Y_\rho
    \subseteq \X_\rho \subseteq \X_\sigma$ gives $W' \in \Y_\sigma$ also.
    Consequently, $\E_i\phi \in B^\sigma_c$ means $W', c \models \E_i\phi$.
    On the other hand, \soundness{} together with $\tuple{i, c, \phi} \in \rho$
    and $W' \in \X_\rho$ means $W', c \models \S_i\phi$. Hence $W', c \models
    \E_i\phi \land \S_i\phi$. From \cref{kr_prop_validities}
    part \cref{kr_item_e_and_s_implies_phi}, we get $W', c \models \phi$.

    We have shown that $\phi$ holds in case $c$ at an arbitrary world in
    $\Y_\rho$. Hence $\phi \in B^\rho_c$, as required.
\end{proof}

Similarly, we have a sufficient condition for score-based operators to satisfy
\strongcondsucc{}: the postulate follows if worlds in which $i$ makes an expert,
truthful report are strictly more plausible than worlds in which $i$ makes a
false report.

\begin{lemma}
    \label{kr_lemma_score_based_strongcondsucc_sufficient_conditions}
    Suppose a score-based operator is such that for any $i \in \srcs$, $c \in \C$,
    $\phi \in \lprop$ and $W, W' \in \W$,
    \begin{align*}
        W, c &\models \E_i\phi \land \phi
        \text{ and }
        W', c \models \neg\phi \\
        &\implies
        d(W, \tuple{i, c, \phi}) < d(W', \tuple{i, c, \phi})
    \end{align*}
    Then \strongcondsucc{} holds.
\end{lemma}

\begin{proof}
    Suppose a score-based operator has the stated property. Take $\sigma$ such
    that $\neg(\E_i\phi \land \phi) \notin B^\sigma_c$. Write $\rho = \sigma
    \concat \tuple{i, c, \phi}$. We need to show that $\phi \in B^\rho_c$.

    First note that by $\neg(\E_i\phi \land \phi) \notin B^\sigma_c$ and the
    definition of $B^\sigma$ for score-based operators, there is $W \in
    \Y_\sigma$ such that $W, c \models \E_i\phi \land \phi$.

    Take any $W' \in \Y_\rho$. Suppose, for the sake of contradiction, that
    $W', c \not\models \phi$. Then by the hypothesised property of the score
    function $d$, we have
    \[
        d(W, \tuple{i, c, \phi}) < d(W', \tuple{i, c, \phi})
    \]
    Now, $W \in \Y_\sigma$ and $W' \in \Y_\rho \subseteq \X_\rho \subseteq
    \X_\sigma$ gives $r_\sigma(W) \le r_\sigma(W')$. Thus
    \begin{align*}
        r_\rho(W)
        &= r_\sigma(W) + d(W, \tuple{i, c, \phi}) \\
        &\le r_\sigma(W') + d(W, \tuple{i, c, \phi}) \\
        &< r_\sigma(W') + d(W', \tuple{i, c, \phi}) \\
        &= r_\rho(W') < \infty
    \end{align*}
    i.e. $r_\rho(W) < r_\rho(W') < \infty$. But this means $W \in \X_\rho$ and
    $W'$ is not minimal in $\X_\rho$ under $r_\rho$, contradicting $W' \in
    \Y_\rho$. Hence $W', c \models \phi$.

    Since $W'$ was an arbitrary member of $\Y_\rho$, we have shown $\phi \in
    B^\rho_c$, and thus \strongcondsucc{} is shown.
\end{proof}

The main result now follows.

\begin{proof}[Proof of \cref{kr_prop_examples_satisfy_condsucc}]
    For the conditioning operators \varbasedcond{} and \partbasedcond{},
    \condsucc{} follows from
    \cref{kr_lemma_conditioning_condsucc_sufficient_condition} since $W, c \models
    \phi$ implies $W, c \models \S_i\phi$. For the score-based operator
    \scorebasedop{}, one can easily check that the condition in
    \cref{kr_lemma_score_based_strongcondsucc_sufficient_conditions} holds, and
    thus \strongcondsucc{} and \condsucc{} follow.
\end{proof}

By omission, the reader may suppose that the conditioning operators fail
\strongcondsucc{}. This is correct, and we can in fact say even more: \emph{no}
conditioning operator with a few basic properties -- all of which are satisfied
by \varbasedcond{} and \partbasedcond{} -- can satisfy \strongcondsucc{}.
%
In what follows, for a permutation $\pi: \srcs \to \srcs$ with $\pi(\ast) = \ast$,
write $\pi(W)$ for the world with $v^{\pi(W)}_c = v^W_c$ and $\Pi^{\pi(W)}_i =
\Pi^W_{\pi(i)}$. We have an impossibility result.

\begin{proposition}
    \label{kr_prop_strongcondsucc_conditioning_impossibilitity}
    No elementary conditioning operator satisfying the basic postulates can
    simultaneously satisfy the following properties:
    \begin{enumerate}
        \item \label{kr_item_conditioning_impossibility_first}
              $K^\emptyset = \cn(\emptyset)$

        \item \label{kr_item_anonymity}
              If $\pi$ is a permutation of $\srcs$ with $\pi(\ast) = \ast$, $W
              \simeq \pi(W)$

        \item \refinement{}

        \item \label{kr_item_conditioning_impossibility_last} \strongcondsucc{}
    \end{enumerate}
    However, any proper subset of
    \cref{kr_item_conditioning_impossibility_first} -
    \cref{kr_item_conditioning_impossibility_last} is satisfiable.
\end{proposition}

\cref{kr_item_conditioning_impossibility_first} says that before any
reports are received, we only know tautologies. As remarked earlier, this is
not an \emph{essential} property, but is reasonable when no prior knowledge is
available. \cref{kr_item_anonymity} is an anonymity postulate: it says that
permuting the ``names'' of sources does not affect the plausibility of a world,
and is a desirable property in light of
\cref{kr_item_conditioning_impossibility_first}. \refinement{}, introduced
in \cref{kr_sec_conditioning_operators}, says that worlds in which all sources
have more expertise are preferred.

\begin{proof}
    Take distinct sources $i_1, i_2 \in \srcs \setminus \{\ast\}$, distinct cases
    $c, d \in \C$, and distinct valuations $v_1, v_2 \in \vals$. Let $\phi_1,
    \phi_2 \in \lprop$ be propositional formulas with $\propmods{\phi_k} =
    \{v_k\}$
    ($k \in \{1, 2\}$). Suppose for contradiction that some elementary
    conditioning operator -- satisfying the basic postulates -- has the stated
    properties.

    Define a sequence
    \[
        \sigma
        = (
            \tuple{\ast, c, \phi_1 \lor \phi_2},
            \tuple{i_1, c, \phi_1},
            \tuple{i_2, c, \phi_2}
        ).
    \]
    Let $\Pi_\bot$ denote the unit partition $\{\{u\} \mid u \in \vals\}$, and
    let $\widehat{\Pi}$ denote the partition
    \[
         \{\{v_1, v_2\}\}
         \cup
         \{\{u\} \mid u \in \vals \setminus \{v_1, v_2\}\},
    \]
    i.e. the partition obtained from $\Pi_\bot$ by merging the cells of $v_1$
    and $v_2$.
    %
    Consider worlds $W_1$, $W_2$ given by
    \begin{align*}
         v^{W_k}_{c'} &= v_k \qquad (c' \in \C) \\
         \Pi^{W_k}_i &= \begin{cases}
            \widehat{\Pi},&
                (k = 1 \text{ and } i = i_2)
                \text { or }
                (k = 2 \text{ and } i = i_1) \\
            \Pi_\bot,& \text{ otherwise }
         \end{cases}
    \end{align*}
    That is, $W_1$ has $v_1$ as its valuation for all cases, $i_2$ has
    partition $\widehat{\Pi}$, and all other sources have the finest partition
    $\Pi_\bot$; similarly $W_2$ has $v_2$ for its valuations and all sources
    except $i_1$ have $\Pi_\bot$.

    Let $\le$ denote the total preorder associated with the conditioning
    operator.

        \begin{claim}
            \label{kr_claim_w1_simeq_w2}
            $W_1 \simeq W_2$.
        \end{claim}
        \begin{claimproof}
            Let $\pi$ be the permutation of $\srcs$ which swaps $i_1$ and $i_2$.
            It is easily observed that $\pi(W_1)$ is partition-equivalent to
            $W_2$. By reflexivity of partition refinement, $\pi(W_1) \preceq
            W_2$ and $W_2 \preceq \pi(W_1)$. By \refinement{}, we get $\pi(W_1)
            \simeq W_2$. By property \cref{kr_item_anonymity}, $W_1 \simeq
            \pi(W_1)$. By transitivity of ${\simeq}$ we get $W_1 \simeq W_2$ as
            desired.
        \end{claimproof}

    Now, from the basic postulates, property
    \cref{kr_item_conditioning_impossibility_first} and
    \cref{kr_prop_prior_knowledge} we have $K^\sigma = \cn(G^\sigma_\snd)$. By
    elementariness and \cref{kr_lemma_model_based_elementary}, we get
    $\X_\sigma = \mods(K^\sigma) = \mods(G^\sigma_\snd)$. It is easily checked
    that both $W_1$ and $W_2$ satisfy the soundness statements corresponding to
    $\sigma$, and thus $W_1, W_2 \in \mods(G^\sigma_\snd) = \X_\sigma$.

        \begin{claim}
            \label{kr_claim_w1_w2_in_ysigm}
            $W_1, W_2 \in \Y_\sigma$.
        \end{claim}
        \begin{claimproof}
            We show $W_1$ and $W_2$ are $\le$-minimal in $\X_\sigma$. Take any
            $W \in \X_\sigma$. Then $W \in \mods(G^\sigma_\snd)$, so $W, c
            \models \S_\ast(\phi_1 \lor \phi_2)$, i.e. $v^W_c \in \{v_1, v_2\}$.
            We consider two cases.
            \begin{itemize}
                \item \textbf{Case 1} ($v^W_c = v_1$). By $W \in
                    \mods(G^\sigma_\snd)$ again we have $W, c \models
                    \S_{i_2}{\phi_2}$, i.e
                    \[
                        v_1
                        = v^W_c
                        \in \Pi^W_{i_2}[\phi_2]
                        = \Pi^W_{i_2}[v_2].
                    \]
                    It follows that $\{v_1, v_2\} \subseteq \Pi^W_{i_2}[v_2]$,
                    and that $\widehat{\Pi}$ refines $\Pi^W_{i_2}$. Since
                    $\widehat{\Pi}$ is the partition of $i_2$ in $W_1$, and all
                    other sources have the finest partition $\Pi_\bot$, we get
                    $W_1 \preceq W$. By \refinement{}, $W_1 \le W$. Since $W_1
                    \simeq W_2$ we have $W_2 \le W$ also.

                \item \textbf{Case 2} ($v^W_c = v_2$). Applying a near-identical
                      argument to that used in case 1 with soundness of the
                      report $\tuple{i_1, c, \phi_1}$, we get $W_1, W_2 \le W$.
            \end{itemize}
            In either case, both $W_1 \le W$ and $W_2 \le W$, so $W_1, W_2 \in
            \Y_\sigma$.
        \end{claimproof}

    Now we consider case $d$. Since
    \[
        W_1, d \models \E_{i_1}{\phi_1} \land \phi_1
    \]
    and $W_1 \in \Y_\sigma$, $\neg(\E_{i_1}{\phi_1} \land \phi_1) \notin
    B^\sigma_d$. Writing $\rho = \sigma \concat \tuple{i_1, d, \phi_1}$, we get
    from \strongcondsucc{} that $\phi_1 \in B^\rho_d$.

    Note that $W_2, d \models \S_{i_1}{\phi_1}$, so $W_2 \in \mods(G^\rho_\snd)
    = \mods(K^\rho) = \X_\rho$. Since $W_2$ is $\le$-minimal in $\X_\sigma$ and
    \[
        X_\rho = \mods(G^\rho_\snd) \subseteq \mods(G^\sigma_\snd) = \X_\sigma,
    \]
    $W_2$ is also $\le$-minimal in $\X_\rho$, i.e. $W_2 \in \Y_\rho$. Now
    $\phi_1 \in B^\rho_d$ gives $W_2, d \models \phi_1$. Since $v^{W_2}_d =
    v_2$ and $\propmods{\phi_1} = \{v_1\}$, this means $v_1 = v_2$. But $v_1$
    and $v_2$ were assumed to be distinct: contradiction.

    \todo{Show that any strict subset is satisfiable}
\end{proof}

\Cref{kr_prop_strongcondsucc_conditioning_impossibilitity} highlights an important
difference between conditioning and score-based operators, and hints that
a fixed plausibility order may be too restrictive: we
need to allow the order to be responsive to new reports in order to satisfy
properties such as \strongcondsucc{}.

To further this point, two of the postulates involved in the characterisation of
elementary conditioning operators -- \duprem{} and \incvac{} -- are already
enough on their own to imply a somewhat questionable property when combined
with \strongcondsucc{}.

\begin{axiom}[\decisiveness{}]
    If $\tuple{i, c, \phi} \in \sigma$ then either $\phi \in B^\sigma_c$ or
    $\neg\E_i\phi \in B^\sigma_c$.
\end{axiom}

This property says that each report is either accepted, or the reporting source
is distrusted. Thus, no room is left for an operator to abstain on a particular
report. It can be easily seen that \decisiveness{} fails for \varbasedcond{},
\partbasedcond{} and \scorebasedop{} by considering the simple sequence $\sigma
= (\tuple{i, c, p}, \tuple{j, c, \neg{p}})$, and indeed we argue this is the
intuitively ``correct'' behaviour. The tension between \strongcondsucc{} and
\duprem{} -- which says that duplicate reports can be removed without effecting
beliefs -- is already evident from this example when one considers $\sigma
\concat \tuple{i, c, p}$. Formally, we have the following.

\begin{proposition}\leavevmode
    \begin{enumerate}
        \item\label{kr_item_duprem_strongcondsucc_deciciveness} The basic
            postulates, \duprem{} and \strongcondsucc{} imply \decisiveness{}.
        \item The basic postulates, \incvac{} and \strongcondsucc{} imply
              \decisiveness{}.
    \end{enumerate}
\end{proposition}

\begin{proof}\leavevmode
    \begin{enumerate}
        \item Suppose $\tuple{i, c, \phi} \in \sigma$. First suppose
              $\neg(\E_i\phi \land \phi) \notin B^\sigma_c$. Then
              \strongcondsucc{} gives $\phi \in B^{\sigma \concat \tuple{i, c,
              \phi}}_c$. But since $\tuple{i, c, \phi}$ already appears in
              $\sigma$, \rearr{} and \duprem{} give $B^{\sigma \concat
              \tuple{i, c, \phi}} = B^\sigma$. Thus $\phi \in B^\sigma_c$.

              Now suppose $\neg(\E_i\phi \land \phi) \in B^\sigma_c$. We claim
              $\neg\E_i\phi \in B^\sigma_c$, i.e. $W, c \models \neg\E_i\phi$
              for all $W \in \Y_\sigma$. Indeed, take any
              $W \in \Y_\sigma$. Then $W, c \models \neg(\E_i\phi \land \phi)$,
              so either $W, c \models \neg\E_i\phi$ or $W, c \models \neg\phi$.
              In the former case we are done. In the latter case, an
              application of \soundness{} and \containment{} gives $W, c
              \models \S_i\phi$, so in fact $W, c \models \S_i\phi \land
              \neg\phi$. But $\S_i\phi \land \neg\phi \limplies \neg\E_i\phi$
              is a validity of the logic (\cref{kr_prop_validities}
              \cref{kr_item_e_and_s_implies_phi}), so $W, c \models
              \neg\E_i\phi$ also. Hence $\neg\E_i\phi \in B^\sigma_c$, and
              \decisiveness{} is shown.

          \item By \cref{kr_lemma_condcons_duprem_follow} the basic postulates
              and \incvac{} already imply \duprem{}, so we may conclude by
              \cref{kr_item_duprem_strongcondsucc_deciciveness}.
    \end{enumerate}
\end{proof}

\section{Selective Change}
\label{kr_sec_selective_change}

In the previous section we saw how a single formula $\phi$ may be accepted when
it is received as an additional report. But what can we say about propositional
beliefs when taking into account the \emph{whole sequence} $\sigma$? To
investigate this we introduce an analogue of \emph{selective revision}
\cite{ferme1999selective}, in which propositional beliefs are formed by
``selecting'' only a part of each input report (e.g., some part consistent with
the source's expertise).
%
For example, in \cref{kr_ex_conditioning_operator} we saw that when given
$\sigma = (\tuple{\ast, c, p}, \tuple{i, c, \neg{p} \land q})$, \varbasedcond{}
outputs propositional beliefs $\proppart{B^\sigma_c} = \cnprop(p \land q)$.
Intuitively, the report from $\ast$ is taken as-is, whereas the report of
$\neg{p} \land q$ from $i$ is weakened to just $q$. The resulting formulas are
combined conjunctively to form the propositional belief set.
%
We formalise this idea via \emph{selection schemes}. In what follows, write
$\sigma \rs c = \{\tuple{i, \phi} \mid \tuple{i, c, \phi} \in \sigma\}$ for the
$c$-reports in $\sigma$.

\begin{definition}
    \label{kr_def_selectivity}
    A \emph{selection scheme} is a mapping $f$ assigning to each
    $\ast$-consistent sequence $\sigma$ a function $f_\sigma: \srcs \times \C
    \times \lprop \to \lprop$ such that $f_\sigma(i, c, \phi) \in
    \cnprop(\phi)$.
    %
    An operator is \emph{selective} if there is a selection scheme $f$ such
    that for all $\ast$-consistent $\sigma$ and $c \in \C$,
    \[
        \proppart{B^\sigma_c} = \cnprop(\{f_\sigma(i, c, \phi) \mid \tuple{i,
        \phi} \in \sigma \rs c\}).
    \]
\end{definition}

Thus, an operator is selective if its propositional beliefs in case $c$ are
formed by weakening each $c$-report and taking their consequences. Note that for
$\sigma = \emptyset$ we get $\proppart{B^\sigma_c} = \cnprop(\emptyset)$, so
selectivity already rules out non-tautological prior propositional beliefs.
Also note that in the presence of \closure{}, \containment{} and \soundness{},
selectivity implies that $\proppart{B^\sigma_c} = \proppart{B^\rho_c}$, where
$\rho$ is obtained by replacing each report $\tuple{i, c, \phi}$ with
$\tuple{\ast, c, f_\sigma(i, c, \phi)}$.

Selectivity can be characterised by a natural postulate placing an upper bound
on the propositional part of $B^\sigma_c$. For any sequence $\sigma$ and case
$c$, write $\Gamma^\sigma_c = \{\phi \in \lprop \mid \exists i \in \srcs:
\tuple{i, \phi} \in \sigma \rs c\}$.

\begin{axiom}[\boundedness{}]
    If $\sigma$ is $\ast$-consistent, $\proppart{B^\sigma_c}
    \subseteq \cnprop(\Gamma^\sigma_c)$
\end{axiom}

\boundedness{} says that the propositional beliefs in case $c$ should not go
beyond the consequences of the formulas reported in case $c$. In some sense
this can be seen as an iterated version of \axiomref{Inclusion} from AGM
revision, in the case where $\proppart{B^\emptyset_c} = \cnprop(\emptyset)$. We
have the following characterisation.

\begin{theorem}
    \label{kr_thm_selectivity_characterisation}
    A model-based operator is selective if and only if it satisfies
    \boundedness{}.
\end{theorem}

\begin{proof}
    ``if'': Suppose a model-based operator satisfies \boundedness{}. Take any
    $\ast$-consistent $\sigma$. For $c \in \C$, set
    \[
        M_c = \propmods{\proppart{B^\sigma_c}}.
    \]
    By \boundedness{}, we have $M_c \supseteq \propmods{\Gamma^\sigma_c}$. Now
    set
    \[
        F_\sigma(i, c, \phi) = \propmods{\phi} \cup M_c.
    \]
    Define a selection function $f_\sigma$ by letting $f_\sigma(i, c, \phi)$ be
    any formula with $\propmods{f_\sigma(i,c, \phi)} = F_\sigma(i, c, \phi)$.
    Since $F_\sigma(i, c, \phi)$ contains the models of $\phi$, clearly
    $f_\sigma(i, c, \phi) \in \cnprop(\phi)$. Therefore $f$ is indeed a
    selection function.

    We claim that, for any $c \in \C$,
    \[
        M_c
        = \bigcap_{\tuple{i, \phi} \in \sigma \rs c}{
            F_\sigma(i, c, \phi)
        }.
    \]
    The ``$\subseteq$'' inclusion is clear since, by definition, $F_\sigma(i, c,
    \phi) \supseteq M_c$. For the ``$\supseteq$'' inclusion, suppose for
    contradiction that there is some $v \in \bigcap_{\tuple{i, \phi} \in \sigma
    \rs c}{F_\sigma(i, c, \phi)}$ with $v \notin M_c$.

    Take any $\phi \in \Gamma^\sigma_c$. Then there is $i \in \srcs$ such that
    $\tuple{i, \phi} \in \sigma \rs c$, and hence $v \in F_\sigma(i, c, \phi)$.
    But $v \notin M_c$ by assumption, so $v \in \propmods{\phi}$. This shows $v
    \in \propmods{\Gamma^\sigma_c}$. But $\propmods{\Gamma^\sigma_c} \subseteq
    M_c$ by \boundedness{}, so $v \in M_c$; contradiction.

    From this we get
    \begin{align*}
        \propmods{\proppart{B^\sigma_c}}
        &= M_c \\
        &= \bigcap_{\tuple{i, \phi} \in \sigma \rs c}{F_\sigma(i, c, \phi)} \\
        &= \bigcap_{\tuple{i, \phi} \in \sigma \rs c}{
            \propmods{f_\sigma(i, c, \phi})
        } \\
        &= \propmods{\{
            f_\sigma(i, c, \phi) \mid \tuple{i, \phi} \in \sigma \rs c
        \}}
    \end{align*}
    Since $\proppart{B^\sigma_c}$ is deductively closed (by \closure{}, which
    holds for all model-based operators), we get
    \[
        \proppart{B^\sigma_c}
        =
        \cnprop\left(\{
            f_\sigma(i, c, \phi) \mid \tuple{i, \phi} \in \sigma \rs c
        \}\right)
    \]
    as required for selectivity.

    ``only if'': Suppose a model-based operator is selective according to some
    selection scheme $f$. Take any $\ast$-consistent $\sigma$ and $c \in \C$.
    Write
    \[
        \Delta = \{
            f_\sigma(i, c, \phi)
            \mid
            \tuple{i, \phi} \in \sigma \rs c
        \}.
    \]
    so that $\proppart{B^\sigma_c} = \cnprop(\Delta)$. For $\tuple{i, \phi} \in
    \sigma \rs c$ we have $f_\sigma(i, c, \phi) \in \cnprop(\phi) \subseteq
    \cnprop(\Gamma^\sigma_c)$ from the definition of a selection scheme and the
    fact that $\phi \in \Gamma^\sigma_c$. Hence $\Delta \subseteq
    \cnprop(\Gamma^\sigma_c)$, so
    \[
        \proppart{B^\sigma_c}
        = \cnprop(\Delta)
        \subseteq \cnprop(\cnprop(\Gamma^\sigma_c))
        = \cnprop(\Gamma^\sigma_c)
    \]
    as required for \boundedness{}.
\end{proof}

The characterisation in \cref{kr_thm_selectivity_characterisation} allows us to
easily analyse when conditioning and score-based operators are selective. In
the case of conditioning operators with $K^\emptyset = \cn(\emptyset)$, we in
fact have a precise characterisation. First, some terminology: say that a world
$W$ \emph{refines $W'$ at $c$} if for all $i \in \srcs$ we have $\Pi^W_i[v^W_c]
\subseteq \Pi^{W'}_i[v^{W'}_c]$. Intuitively, this means each source is more
knowledgable in case $c$ in world $W$ than they are in $W'$. Recall that $\W_{c\ :\
v} = \{W \in \W \mid v^W_c = v\}$ denotes the set of worlds whose $c$ valuation is
$v$. We have the following.

\begin{proposition}
    \label{kr_prop_conditioning_selectivity_characterisation}
    Suppose an elementary conditioning operator satisfies the basic postulates
    and has $K^\emptyset = \cn(\emptyset)$. Then it is selective if and only if
    for all $W$, $c$, $v$ there is $W' \in \W_{c\ :\ v}$ such that $W' \le W$
    and $W$ refines $W'$ at all cases $d \ne c$.
\end{proposition}

While the condition on $\le$ in
\cref{kr_prop_conditioning_selectivity_characterisation} is somewhat technical, it
is implied by the very natural \emph{partition-equivalence} property from
\cref{kr_sec_framework}. Consequently, \varbasedcond{} and \partbasedcond{} are
selective. For the score-based operator \scorebasedop{}, one can show
\boundedness{} holds directly using a property of the disagreement scoring
function $d$ similar to the property of $\le$ above. Consequently,
\scorebasedop{} is also selective.

To prove \cref{kr_prop_conditioning_selectivity_characterisation}, we first
state some preliminary results.

\begin{lemma}
    \label{kr_lemma_local_refinement}
    Suppose $W$ refines $W'$ at $c$. Then for any $i \in \srcs$ and $\phi \in
    \lprop$,
    \[
        W, c \models \S_i\phi
        \implies
        W', c \models \S_i\phi.
    \]
\end{lemma}

\begin{proof}
    Suppose $W, c \models \S_i\phi$. Then $v^W_c \in \Pi^W_i[\phi]$, i.e.
    $\propmods{\phi} \cap \Pi^W_i[v^W_c] \ne \emptyset$. By refinement,
    $\Pi^W_i[v^W_c] \subseteq \Pi^{W'}_i[v^{W'}_c]$. Hence $\propmods{\phi}
    \cap \Pi^{W'}_i[v^{W'}_c] \ne \emptyset$, so $v^{W'}_c \in
    \Pi^{W'}_i[\phi]$. That is, $W', c \models \S_i\phi$.
\end{proof}

\begin{lemma}
    \label{kr_lemma_local_refinement_sequence}
    For any $W \in \W$ and $c \in \C$, there is a $\ast$-consistent sequence
    $\sigma$ -- containing only reports for case $c$ -- such that for all $W'
    \in \W$,
    \[
        W' \in \mods(G^\sigma_\snd)
        \iff
        W \text{ refines } W' \text{ at } c.
    \]
\end{lemma}

\begin{proof}

    For a valuation $v \in \vals$, let $\phi(v)$ be a propositional formula
    such that $\propmods{\phi(v)} = \{v\}$. Take $\sigma$ to be any enumeration
    of reports of the form
    \[
        \tuple{i, c, \phi(v)},
    \]
    where $i \in \srcs$ and $v \in \Pi^W_i[v^W_c]$. Note that such a sequence
    exists since there are only finitely many sources and valuations. Clearly
    $\sigma$ contains only $c$-reports. Since $\Pi^W_\ast$ is the unit
    partition, the only report from $\ast$ is $\tuple{\ast, c, \phi(v^W_c)}$.
    Hence $\sigma$ is $\ast$-consistent. We show the desired equivalence.

    $\implies$: Suppose $W' \in \mods(G^\sigma_\snd)$. Take any $i \in \srcs$. We
    need to show $\Pi^W_i[v^W_c] \subseteq \Pi^{W'}_i[v^{W'}_c]$. Take $v \in
    \Pi^W_i[v^W_c]$. By construction of $\sigma$, $\tuple{i, c, \phi(v)} \in
    \sigma$. Hence $W', c \models \S_i\phi(v)$, i.e. $v^{W'}_c \in
    \Pi^{W'}_i[\phi(v)] = \Pi^{W'}_i[v]$. This shows $v \in
    \Pi^{W'}_i[v^{W'}_c]$ as required.

    $\impliedby$: Suppose $W$ refines $W'$ at $c$. Take any $\tuple{i, c,
    \phi(v)} \in \sigma$. Then $v \in \Pi^W_i[v^W_c]$, so $v^W_c \in \Pi^W_i[v]
    = \Pi^W_i[\phi(v)]$. This shows $W, c \models \S_i\phi(v)$, and
    \cref{kr_lemma_local_refinement} gives $W', c \models \S_i\phi(v)$. Hence $W'
    \in \mods(G^\sigma_\snd)$.
\end{proof}

\begin{proof}[Proof of \cref{kr_prop_conditioning_selectivity_characterisation}]
    Take an elementary conditioning operator with the basic postulates
    and $K^\emptyset = \cn(\emptyset)$.

    ``if'':  Suppose the stated property holds. Since all conditioning operators
    are model-based, by \cref{kr_thm_selectivity_characterisation} it suffices to
    show \boundedness{}. To that end, let $\sigma$ be $\ast$-consistent and
    take $c \in \C$. We need $\proppart{B^\sigma_c} \subseteq
    \cnprop(\Gamma^\sigma_c)$; or equivalently, by \closure{},
    $\propmods{\proppart{B^\sigma_c}} \supseteq \propmods{\Gamma^\sigma_c}$.

    Take any $v \in \propmods{\Gamma^\sigma_c}$. Since $\sigma$ is
    $\ast$-consistent, $B^\sigma$ is consistent by \consistency{}. Hence
    $\Y_\sigma \ne \emptyset$. Take any $W \in \Y_\sigma$. By the property in
    the statement of the result, there is $W' \in \W_{c\ :\ v}$ such that $W'
    \le W$ and $W$ refines $W'$ at all cases $d \ne c$.

    We claim $W' \in \X_\sigma$. By \cref{kr_prop_prior_knowledge}, elementariness
    and \cref{kr_lemma_model_based_elementary}, we have $\X_\sigma =
    \mods(K^\sigma) = \mods(G^\sigma_\snd)$. Take any $\tuple{i, d, \phi} \in
    \sigma$. We consider cases.
    \begin{itemize}
        \item \textbf{Case 1} ($d = c$). Here $\tuple{i, \phi} \in \sigma \rs
            c$, so $\phi \in \Gamma^\sigma_c$. Hence $v \in
            \propmods{\Gamma^\sigma_c} \subseteq \propmods{\phi}$. Since $W'
            \in \W_{c\ :\ v}$, $v$ is the $c$-valuation of $W'$. Hence $W', c
            \models \phi$, and $W', c \models \S_i\phi$ follows.

        \item \textbf{Case 2} ($d \ne c$). By assumption, $W$ refines $W'$ at
              $d$. Since $W \in \Y_\sigma \subseteq \X_\sigma$, we have $W, d
              \models \S_i\phi$. By \cref{kr_lemma_local_refinement}, $W', d
              \models \S_i\phi$ also.
    \end{itemize}
    We have shown $W' \in \mods(G^\sigma_\snd) = \X_\sigma$. Now recall that $W
    \in \Y_\sigma$ -- so $W$ is $\le$-minimal in $\X_\sigma$ -- and $W' \le W$.
    Thus $W'$ is also $\le$-minimal in $\X_\sigma$, i.e. $W' \in \Y_\sigma$.
    Since $W' \in \W_{c\ :\ v}$ also, we have by
    \cref{kr_lemma_model_based_models_of_proppart} that $v \in
    \propmods{\proppart{B^\sigma_c}}$, as required.

    ``only if'': Suppose our operator is selective, i.e. satisfies
    \boundedness{}. To show the desired property holds, take any $W$, $c$ and
    $v$. Enumerate $\C \setminus \{c\}$ as $\{d_1, \ldots, d_N\}$. By
    \cref{kr_lemma_local_refinement_sequence}, for each $1 \le n \le N$ there is a
    $\ast$-consistent sequence $\sigma_n$ such that
    \[
        \mods(G^{\sigma_n}_\snd)
        =
        \{W' \in \W \mid W \text{ refines } W' \text{ at } d_n\}.
    \]
    Now, let $\phi$ and $\psi$ be formulas with $\propmods{\phi} = \{v\}$ and
    $\propmods{\psi} = \{v^W_c\}$. Let $\rho$ be the concatenation
    \[
        \rho
        =
        \sigma_1 \cdots \sigma_n \concat
        \tuple{\ast, c, \phi \lor \psi}.
    \]
    Note that $\rho$ is $\ast$-consistent, since each $\sigma_n$ is (and only
    refers to case $d_n$). We may therefore apply \boundedness{} for case $c$.
    Taking models of both sides yields
    \[
        \propmods{\proppart{B^\rho_c}}
        \supseteq \propmods{\Gamma^\rho_c}
        = \propmods{\phi \lor \psi}
        = \{v, v^W_c\}.
    \]
    In particular, $v \in \propmods{\proppart{B^\rho_c}}$. By
    \cref{kr_lemma_model_based_models_of_proppart}, there is some $W' \in \Y_\rho
    \cap \W_{c\ :\ v}$.

    We show $W'$ has the required properties. First note that since $W$ refines
    itself at each $d_n$, we have $W \in \mods(G^{\sigma_n}_\snd)$. Clearly $W,
    c \models \psi$, so $W, c \models \S_\ast(\phi \lor \psi)$ too. Thus $W \in
    \mods(G^\rho_\snd) = \X_\rho$ (using $K^\emptyset = \cn(\emptyset)$). Since
    $W' \in \Y_\rho = \min_{\le}{\X_\rho}$, we get $W' \le W$ as required.

    Next, take any case $d \ne c$. Then there is some $n$ such that $d = d_n$.
    Since $W' \in \Y_\rho \subseteq \X_\rho = \mods(G^\rho_\snd) \subseteq
    \mods(G^{\sigma_n}_\snd)$, we get that $W$ refines $W'$ at $d$. This
    completes the proof.
\end{proof}

\subsection{Case Independence}

In the definition of a selection scheme, we allow $f_\sigma(i, c, \phi)$ to
depend on the case $c$. If one views $f_\sigma(i, c, \phi)$ as a weakening of
$\phi$ which accounts for the lack of expertise of $i$, this is somewhat at
odds with other aspects of the framework, where expertise is independent of
case. For this reason it is natural to consider \emph{case independent}
selective schemes.

\begin{definition}
    \label{kr_def_case_independent_selectivity}
    A selection scheme $f$ is \emph{case independent} if $f_\sigma(i, c, \phi)
    \equiv f_\sigma(i, d, \phi)$ for all $\ast$-consistent $\sigma$ and $i \in
    \srcs$, $c, d \in \C$ and $\phi \in \lprop$.
\end{definition}

Say an operator is \emph{case-independent-selective} if it is selective
according to some case independent scheme. This stronger notion of selectivity
can again be characterised by a postulate which bounds propositional beliefs.
For any set of cases $H \subseteq \C$, sequence $\sigma$ and $c \in \C$, write
\begin{align*}
    \Gamma^{\sigma,H}_c
    = \{
        \phi \in \lprop
        \mid
        \exists i \in \srcs:
            &\tuple{i, \phi} \in \sigma \rs c \\
            &\text{ and }
            \forall d \in H : \tuple{i, \phi} \notin \sigma \rs d
    \}.
\end{align*}

\begin{axiom}[\hboundedness{}]
    For any $\ast$-consistent $\sigma$, $H \subseteq \C$ and $c \in \C$,
    \[
        \proppart{B^\sigma_c}
            \subseteq
            \cnprop\left(
                \Gamma^{\sigma,H}_c
                \cup
                \bigcup_{d \in H}{
                    \proppart{B^\sigma_d}
                }
            \right)
    \]
\end{axiom}

Note that \boundedness{} is obtained as the special case where $H =
\emptyset$. We illustrate with an example.

\begin{example}
    \label{kr_ex_hprop}
    Consider case $c$ in the following sequence:
    \[
        \sigma
        = (
            \tuple{i, c, p},
            \tuple{j, c, q},
            \tuple{j, d, q},
            \tuple{k, d, r}
        )
    \]
    \boundedness{} requires that $\proppart{B^\sigma_c} \subseteq \cnprop(\{p,
    q\})$. However, the instance of \hboundedness{} with $H = \{d\}$ makes use
    of the fact that $j$ reports $q$ in both cases $c$ and $d$, and requires
    $\proppart{B^\sigma_c} \subseteq \cnprop(\{p\} \cup
    \proppart{B^\sigma_d})$.
    %
    This also has an interesting implication for case $d$: if $\phi \in
    \proppart{B^\sigma_c}$, then $p \limplies \phi \in
    \proppart{B^\sigma_d}$. This follows since $\beta \in \cnprop(\{\alpha\}
    \cup \Gamma)$ iff $\alpha \limplies \beta \in \cnprop(\Gamma)$ for
    $\alpha, \beta \in \lprop$.
    Intuitively, this says that if $p$ (from $i$) and $q$
    (from $j$) is enough to accept $\phi$ in case $c$, then $\phi$ is
    accepted in case $d$ \emph{if $p$ is}, given that the report of $q$ from
    $j$ is repeated for $d$.

\end{example}

The characterisation is as follows.

\begin{theorem}
    \label{kr_thm_case_independent_selectivity_characterisation}
    A model-based operator is case-independent-selective if and only if it
    satisfies \hboundedness{}.
\end{theorem}

\begin{proof}

    ``only if'': Suppose a model-based operator is selective
    according to some case-independent scheme $f$. Take any $\ast$-consistent
    $\sigma$, $H \subseteq \C$ and $c \in \C$. For any case $d$, write $M_d =
    \propmods{\proppart{B^\sigma_d}}$. Note that with $c_0$ an
    arbitrary fixed case, and writing $F_\sigma(i, \phi) =
    \propmods{f_\sigma(i, c_0, \phi})$, we have by case-independent-selectivity
    that
    \[
        M_d = \bigcap_{\tuple{i, \phi} \in \sigma \rs d}{
            F_\sigma(i, \phi)
        }.
    \]
    By closure, it is sufficient for \hboundedness{} to show that
    \begin{equation}
        \label{kr_eqn_hbnd_inclusion}
        M_c
        \supseteq
        \propmods{\Gamma^{\sigma, H}_c} \cap \bigcap_{d \in H}{M_d}.
    \end{equation}
    Take any $v$ in the set on the right-hand side. To show $v \in M_c$, take
    any $\tuple{i, \phi} \in \sigma \rs c$. If $\phi \in \Gamma^{\sigma, H}_c$,
    then clearly
    \begin{align*}
        v
        &\in \propmods{\Gamma^{\sigma, H}_c} \\
        &\subseteq \propmods{\phi} \\
        &\subseteq \propmods{f_\sigma(i, c, \phi}) \\
        &= F_\sigma(i, \phi)
    \end{align*}
    (where we use $f_\sigma(i, c, \phi) \in \cnprop(\phi)$). Otherwise, $\phi
    \notin \Gamma^{\sigma, H}_c$. Since $\tuple{i, \phi} \in \sigma \rs c$,
    this means there is $d \in H$ such that $\tuple{i, \phi} \in \sigma \rs d$.
    Hence $v \in M_d$ gives $v \in F_\sigma(i, \phi)$. This shows the inclusion
    in \cref{kr_eqn_hbnd_inclusion}, and we are done.

    ``if'': Suppose a model-based operator satisfies \hboundedness{}. Let
    $\sigma$ be a $\ast$-consistent sequence. As before, write $M_c$ for
    $\propmods{\proppart{B^\sigma_c}}$. For $i \in \srcs$ and $c \in \C$,
    write
    \[
        \C(i, \phi) = \{c \in \C \mid \tuple{i, \phi} \in \sigma \rs c\},
    \]
    and set
    \[
        F_\sigma(i, \phi)
        = \propmods{\phi} \cup \bigcup_{c \in \C(i, \phi)}{M_c}.
    \]
    Define $f$ by letting $f_\sigma(i, c, \phi)$ be any propositional formula
    with $\propmods{f_\sigma(i, c, \phi}) = F_\sigma(i, \phi)$. Then $f$ is a
    case-independent selection scheme. We show our operator is selective
    according to $f$; by closure of $\proppart{B^\sigma_c}$ for each $c$, it
    suffices to show
    \[
        M_c = \bigcap_{\tuple{i, \phi} \in \sigma \rs c}{F_\sigma(i, \phi)}.
    \]
    Fix $c$. For the left-to-right inclusion, suppose $v \in M_c$. Take any
    $\tuple{i, \phi} \in \sigma \rs c$. Then $c \in \C(i, \phi)$, so
    $F_\sigma(i, \phi) \supseteq M_c$ and thus $v \in F_\sigma(i, \phi)$ as
    required.

    For the right-to-left inclusion, suppose $v$ lies in the intersection. Set
    \[
        H = \{d \in \C \mid v \in M_d\}.
    \]
    Applying \hboundedness{} and taking the models of both sides, we obtain
    \begin{equation}
        \label{kr_eqn_hbnd_application}
        M_c
        \supseteq
        \propmods{\Gamma^{\sigma, H}_c} \cap \bigcap_{d \in H}{M_d}.
    \end{equation}
    Clearly $v \in \bigcap_{d \in H}{M_d}$ by definition of $H$. Let $\phi \in
    \Gamma^{\sigma, H}_c$. Then there is $i \in \srcs$ such that
    $\tuple{i, \phi} \in \sigma \rs c$, and consequently $v \in F_\sigma(i,
    \phi)$. We claim $v \in \propmods{\phi}$. If not, by definition of
    $F_\sigma(i, \phi)$ we must have $v \in \bigcup_{d \in \C(i, \phi)}{M_d}$,
    i.e. there is $d \in \C$ such that $\tuple{i, \phi} \in \sigma \rs d$ and
    $v \in M_d$. On the one hand, $\phi \in \Gamma^{\sigma, H}_c$ implies $d
    \notin H$. On the other, $v \in M_d$ gives $d \in H$ directly by the
    definition of $H$: contradiction. This shows $v \in \propmods{\phi}$. Since
    $\phi$ was arbitrary, we have $v \in \propmods{\Gamma^{\sigma, H}_c}$. By
    \cref{kr_eqn_hbnd_application} we get $v \in M_c$, and the proof is
    complete.
\end{proof}

The question of whether our concrete operators satisfy \hboundedness{}
(equivalently, whether they are case-independent-selective) is still open.

\subsection{Expertise and Selectivity}

In the existing literature on selective belief change (e.g.
\cite{ferme1999selective,booth_trust_2018}), the selection function typically
acts as a means to separate out the part of new information on which the
reporting sources is \emph{credible}, or \emph{trusted}. For instance,
\textcite{booth_trust_2018} use a partition $\Pi$ to represent an agent's
perception of the incoming source's expertise; a report of $\phi$ is then
weakened to $\Pi[\phi]$ -- on which the source is trusted to have expertise --
before revision takes place. In our framework, an analogous connection between
the selection function and trust can be captured as follows.

\begin{definition}
    \label{kr_def_ec_scheme}
    A selection scheme $f$ is \emph{expertise-compatible} (EC) with an operator
    $\sigma \mapsto \tuple{B^\sigma, K^\sigma}$ if for all $\ast$-consistent
    $\sigma$ and $\tuple{i, c, \phi} \in \sigma$,
    \[
        \E_i{f_\sigma(i, c, \phi)} \in B^\sigma_c.
    \]
\end{definition}

That is, $i$ is trusted on the weakened report $f_\sigma(i, c, \phi)$ whenever
$i$ reports $\phi$ in case $c$ in $\sigma$.
%
Say an operator is \emph{EC-selective} if it is selective according to some
expertise-compatible scheme. While EC-selectivity may appear natural on first
glance, we argue that it can be overly restrictive when expertise is derived
from the input sequence itself. For example, consider the sequence
\[
    \sigma = (
        \tuple{i, c, p},
        \tuple{j, c, p},
        \tuple{i, d, p},
        \tuple{j, d, \neg p}
    )
\]
By \soundness{} and \closure{}, we cannot have both $\E_i{p}$ and $\E_j{p}$ in
$B^\sigma_c$. Ideas of symmetry suggest that neither can we pick one of $i$ or
$j$ over the other, so that in fact it is reasonable to have neither $\E_i{p}$
nor $\E_j{p}$ in $B^\sigma_c$. Consequently -- assuming $p$ is the only
propositional variable -- the only formulas weaker than $p$ on which $i$ and
$j$ are believed to have expertise are tautologies. Any EC scheme $f$ must
therefore have $f_\sigma(i, c, p) \equiv f_\sigma(j, c, p) \equiv \top$.
Consequently, EC-selectivity would imply $\proppart{B^\sigma_c} =
\cnprop(\top)$. This is a very conservative stance: while there is total
consensus for $p$ in case $c$, $p$ cannot be believed due to disagreement
elsewhere. This also conflicts with the ``optimistic'' attitude described in
\cref{kr_ex_hospital_ex_formalised}. According to that view we should have
$\E_i{p}
\lor \E_j{p} \in B^\sigma_c$, but this implies $p \in B^\sigma_c$ by
\soundness{}, \containment{} and \closure{}.
%
This example already shows that \varbasedcond{}, \partbasedcond{} and
\scorebasedop{} are not EC-selective.

The core issue is that
the expertise of sources is part of the operator's output and is thus
uncertain. In order for $\E_i\phi$ to be believed, $i$ needs to be trusted on
$\phi$ in \emph{every} maximally plausible world. If there are several such
worlds with different assessments of expertise -- e.g. if $W_1$ trusts $i$ but
not $j$, and vice versa in $W_2$ -- then EC-selectivity requires reports to be
significantly weakened before expertise can be believed in \emph{all} worlds.

For model-based operators satisfying \soundness{}, this phenomenon can be
formalised: a report of $\phi$ from source $i$ is expanded by the
\emph{join}\footnotemark{} of the partitions $\Pi^W_i$, for $W \in \Y_\sigma$.
As the above example shows, this join may be strictly coarser than any of the
individual partitions $\Pi^W_i$. Formally, for a set of worlds $S \subseteq
\W$, write $\Pi^S_i = \bigjoin_{W \in S}{\Pi^W_i}$ for the join of the
$i$-partitions of worlds in $S$. We first need a preliminary result.

\footnotetext{
    The join of a set of partitions is its least upper bound with respect to
    the refinement order.
}

\begin{lemma}
    \label{kr_lemma_expertise_join}
    For any model-based operator, $\E_i\phi \in B^\sigma_c$ iff
    $\Pi^{\Y_\sigma}_i[\phi] = \propmods{\phi}$.
\end{lemma}

\begin{proof}
    ``if'': Suppose $\Pi^{\Y_\sigma}_i[\phi] = \propmods{\phi}$. Take $W \in
    \Y_\sigma$. Then since $\Pi^W_i$ refines $\Pi^{\Y_\sigma}_i$, we have
    $\Pi^W_i[\phi] \subseteq \Pi^{\Y_\sigma}_i[\phi] = \propmods{\phi}$. Since
    $\propmods{\phi} \subseteq \Pi^W_i[\phi]$ always holds, we have
    $\Pi^W_i[\phi] = \propmods{\phi}$ and thus $W, c \models \E_i\phi$. Since
    $W \in \Y_\sigma$ was arbitrary, this shows $\E_i\phi \in B^\sigma_c$.

    ``only if'': Suppose $\E_i\phi \in B^\sigma_c$. We need to show
    $\Pi^{\Y_\sigma}_i[\phi] \subseteq \propmods{\phi}$. Take $v \in
    \Pi^{\Y_\sigma}_i[\phi]$. Let $R^W_i$ be the equivalence relation
    corresponding to the partition $\Pi^W_i$. Then the relation $R$
    corresponding to the join $\Pi^{\Y_\sigma}_i$ is the smallest equivalence
    relation containing each of the $R^W_i$, which is given explicitly by the
    transitive closure $R = \left(\bigcup_{W \in
    \Y_\sigma}{R^W_i}\right)^\trcls$.

    Now, from $v \in \Pi^{\Y_\sigma}_i[\phi]$ there is some $u \in
    \propmods{\phi}$ such that $vRu$. By definition of the transitive closure,
    there are $x_0, \ldots, x_n \in \vals$ such that $v = x_0$, $u = x_n$, and
    for each $0 \le k < n$, $(x_k, x_{k + 1}) \in \bigcup_{W \in
    \Y_\sigma}{R^W_i}$. That is, there are $W_0, \ldots, W_{n - 1} \in
    \Y_\sigma$ such that $(x_k, x_{k + 1}) \in R^{W_k}_i$.
    %
    We will show that each $x_k$ lies in $\propmods{\phi}$ by backwards
    induction. For $k = n$, we have $x_n = u \in \propmods{\phi}$ by
    assumption. If $x_{k + 1} \in \propmods{\phi}$, then $(x_k, x_{k + 1}) \in
    R^{W_k}_i$ gives $x_k \in \Pi^{W_k}_i[x_{k + 1}] \subseteq
    \Pi^{W_k}_i[\phi]$. By assumption, $\E_i\phi \in B^\sigma_c$. Since $W_k
    \in \Y_\sigma$, this means $W_k, c \models \E_i\phi$ and $\Pi^{W_k}_i[\phi]
    = \propmods{\phi}$.  Hence $x_k \in \propmods{\phi}$ as desired.
    %
    This shows $v = x_0 \in \propmods{\phi}$, and we are done.  \end{proof}

\begin{proposition}
    \label{kr_prop_ec_selective_propmodels}
    If a model-based operator is EC-selective and satisfies \soundness{}, then
    \[
        \propmods{\proppart{B^\sigma_c}}
        =
        \bigcap_{\tuple{i, \phi} \in \sigma \rs c}{
            \Pi^{\Y_\sigma}_i[\phi]
        }
    \]
    for all $\ast$-consistent $\sigma$ and $c \in \C$.
\end{proposition}

\begin{proof}
    Let $\sigma$ be an $\ast$-consistent sequence and take $c \in \C$.

    $\subseteq$: Let $v \in \propmods{\proppart{B^\sigma_c}}$. By
    \cref{kr_lemma_model_based_models_of_proppart}, there is $W \in \Y_\sigma$
    such that $v = v^W_c$. Take $\tuple{i, \phi} \in \sigma \rs c$. By
    \soundness{} (and \containment{}, which holds for all model-based
    operators) we have $\S_i\phi \in B^\sigma_c$, so $W, c \models \S_i\phi$.
    Consequently $v = v^W_c \in \Pi^W_i[\phi] \subseteq
    \Pi^{\Y_\sigma}_i[\phi]$, where we use the fact that $\Pi^W_i$ refines
    $\Pi^{\Y_\sigma}_i$ in the last step.

    $\supseteq$: Let $v \in \bigcap_{\tuple{i, \phi} \in \sigma \rs
    c}{\Pi^{\Y_\sigma}_i[\phi]}$. By EC-selectivity there is some
    expertise-compatible selection scheme $f$. Write $F_\sigma(i, c, \phi) =
    \propmods{f_\sigma(i, c, \phi)}$. Then we have
    \begin{equation}
        \label{kr_eqn_ec_selective_propmodels_intersection}
        \propmods{\proppart{B^\sigma_c}}
        = \bigcap_{\tuple{i, \phi} \in \sigma \rs c}{
            F_\sigma(i, c, \phi)
        }
    \end{equation}
    and $\E_i{f_\sigma(i, c, \phi)} \in B^\sigma_c$ for each $\tuple{i, \phi}
    \in \sigma \rs c$. By \cref{kr_lemma_expertise_join},
    $\Pi^{\Y_\sigma}_i[F_\sigma(i, c, \phi)] = F_\sigma(i, c, \phi)$. We show
    $v \in \propmods{\proppart{B^\sigma_c}}$ using
    \cref{kr_eqn_ec_selective_propmodels_intersection}. Take $\tuple{i, \phi}
    \in \sigma \rs c$. By definition of a selection scheme we have $f_\sigma(i,
    c, \phi) \in \cnprop(\phi)$, so $\propmods{\phi} \subseteq F_\sigma(i, c,
    \phi)$. Since $v \in \Pi^{\Y_\sigma}_i[\phi]$ by assumption, we get
    \[
        v
        \in \Pi^{\Y_\sigma}_i[\phi]
        \subseteq \Pi^{\Y_\sigma}_i[F_\sigma(i, c, \phi)]
        = F_\sigma(i, c, \phi)
    \]
    as required.
\end{proof}

Note that \cref{kr_prop_ec_selective_propmodels} immediately implies
selectivity with respect to any scheme $f$ such that $\propmods{f_\sigma(i, c,
\phi)} = \Pi^{\Y_\sigma}_i[\phi]$. Since the right-hand side does not depend on
the case $c$, we get the following corollary.

\begin{corollary}
    \label{kr_cor_ec_selective_ci}
    If a model-based operator is EC-selective and satisfies \soundness{}, then
    it is case-independent-selective.
\end{corollary}

\cref{kr_prop_ec_selective_propmodels} also shows that propositional beliefs
in case $c$ are determined only by the reports in $\sigma \rs c$ together with
the expertise part of $B^\sigma$, via the partitions $\Pi^W_i$ for $W \in
\Y_\sigma$. This property can be expressed syntactically as follows, where
for a collection $G$ we write $\exppart{G}$ for the sub-collection of formulas
of the form $\E_i\phi$.

\begin{axiom}[\determination{}]
    For any $\ast$-consistent $\sigma$ and $c \in \C$,
    $
        \proppart{B^\sigma_c}
        =
        \proppart{\cn_c(K^\sigma \sqcup \exppart{B^\sigma})}
    $.
\end{axiom}

In other words, \determination{} says that propositional beliefs may
be fully recovered by taking (the $c$-consequences of) the knowledge set
$K^\sigma$ together with just the expertise formulas in $B^\sigma$.
%
Surprisingly, \determination{} in fact characterises EC-selectivity,
under additional mild assumptions. In what follows, recall that $G^\sigma_\snd$
denotes the collection with $(G^\sigma_\snd)_c = \{\S_i\phi \mid \tuple{i,
\phi} \in \sigma \rs c\}$.

\begin{theorem}
    \label{kr_thm_ec_selectivity_characterisation}
    A model-based operator satisfying \consistency{} and $K^\sigma =
    \cn(G^\sigma_\snd)$ for all $\sigma$ is EC-selective if and only
    if it satisfies \determination{}.
\end{theorem}

\begin{proof}
    Take any model-based operator satisfying \consistency{} and which has
    $K^\sigma = \cn(G^\sigma_\snd)$ for all $\sigma$. Note that the latter
    property implies \soundness{}.

    ``if'': Suppose \determination{} holds. We claim that for any
    $\ast$-consistent $\sigma$ and $c \in \C$,
    \begin{equation}
        \label{kr_eqn_ec_selectivity_characterisation_suff_conf}
        \propmods{\proppart{B^\sigma_c}}
        =
        \bigcap_{\tuple{i, \phi} \in \sigma \rs c}{
            \Pi^{\Y_\sigma}_i[\phi]
        }.
    \end{equation}
    This implies selectivity upon letting $f_\sigma(i, c, \phi)$ be any
    formula with models $\Pi^{\Y_\sigma}_i[\phi]$. Furthermore it implies
    EC-selectivity by \cref{kr_lemma_expertise_join}.

    The left-to-right inclusion of
    \cref{kr_eqn_ec_selectivity_characterisation_suff_conf} follows by an
    argument identical to that of \cref{kr_prop_ec_selective_propmodels} using
    \soundness{}. It suffices to show the right-to-left inclusion. Take $v \in
    \bigcap_{\tuple{i, \phi} \in \sigma \rs c}{ \Pi^{\Y_\sigma}_i[\phi] }$. By
    \consistency{}, there is some $W_0 \in \Y_\sigma$. Consider $W$ obtained
    from $W_0$ by setting its $c$-valuation to $v$, and by setting the
    partition of source $i$ to $\Pi^{\Y_\sigma}_i$:
    \[
        v^W_d = \begin{cases}
            v^{W_0},& d \ne c \\
            v,& d = c
        \end{cases},
    \]
    \[
        \Pi^W_i = \Pi^{\Y_\sigma}_i.
    \]
    Note that since $W_0 \in \Y_\sigma$, $\Pi^{W_0}_i$ refines $\Pi^W_i$ for
    each $i$. We aim to show $W \in \mods(K^\sigma \sqcup \exppart{B^\sigma})$.
    Recall that, by assumption, $K^\sigma = \cn(G^\sigma_\snd)$. It therefore
    suffices to show that $W \in \mods(G^\sigma_\snd) \cap
    \mods(\exppart{B^\sigma})$. First take $\tuple{i, d, \phi} \in \sigma$.
    By \soundness{} and \containment{} we have $W_0, d \models \S_i\phi$, i.e.
    $v^{W_0}_d \in \Pi^{W_0}_i[\phi]$. If $d \ne c$ then
    \[
        v^W_d = v^{W_0}_d \in \Pi^{W_0}_i[\phi] \subseteq \Pi^W_i[\phi],
    \]
    where we use the fact that $\Pi^{W_0}_i$ refines $\Pi^W_i$ in the last
    step. Thus $W, d \models \S_i\phi$ as required. If instead $d = c$, then
    by our assumption on $v$,
    \[
        v^W_c = v \in \Pi^{\Y_\sigma}_i[\phi] = \Pi^W_i[\phi]
    \]
    so that $W, c \models \S_i\phi$ as required. This shows $W \in
    \mods(G^\sigma_\snd)$. For $W \in \mods(\exppart{B^\sigma})$, take any
    $\E_i\phi \in B^\sigma_c$ (note that by \closure{} $\exppart{B^\sigma}$
    contains the same formulas in each case, so we may choose $c$ without loss
    of generality). Then by \cref{kr_lemma_expertise_join},
    $\Pi^{\Y_\sigma}_i[\phi] = \|\phi\|$. By construction of $W$ we evidently
    have $W, c \models \E_i\phi$.

    This shows $W \in \mods(K^\sigma \sqcup \exppart{B^\sigma})$. Finally, to
    show $v \in \propmods{\proppart{B^\sigma_c}}$, take any $\psi \in
    \proppart{B^\sigma_c}$. By \determination{}, $\psi \in \cn_c(K^\sigma
    \sqcup \exppart{B^\sigma})$. Thus $W, c \models \psi$. But by construction
    the $c$-valuation in $W$ is $v$, so $v \in \propmods{\psi}$ and we are
    done.

    ``only if'': Suppose the operator is EC-selective according to some scheme
    $f$. To show \determination{}, take any $\ast$-consistent $\sigma$ and $c
    \in \C$. By \containment{} we have $K^\sigma \sqsubseteq B^\sigma$, and
    clearly $\exppart{B^\sigma} \sqsubseteq B^\sigma$. Consequently $\K^\sigma
    \sqcup \exppart{B^\sigma} \sqsubseteq B^\sigma$; by monotonicity of $\cn$
    and \closure{} we get $\cn(K^\sigma \sqcup \exppart{B^\sigma}) \sqsubseteq
    B^\sigma$. This in turn implies $\proppart{B^\sigma_c} \supseteq
    \proppart{\cn_c(K^\sigma \sqcup \exppart{B^\sigma})}$.

    For the reverse inclusion, it is sufficient by \closure{} to show
    \begin{equation}
        \label{kr_eqn_ec_sel_implies_det}
        \propmods{\proppart{\cn_c(K^\sigma \sqcup \exppart{B^\sigma})}}
        \subseteq
        \propmods{\proppart{B^\sigma_c}}.
    \end{equation}
    So, take $v$ in the set on the left-hand side. By an argument identical to
    the proof of \cref{kr_lemma_model_based_models_of_proppart}, there is some
    $W \in \mods(K^\sigma \sqcup \exppart{B^\sigma})$ such that $v = v^W_c$.
    Since \soundness{} holds by the assumption that $K^\sigma =
    \cn(G^\sigma_\snd)$, EC-selectivity and
    \cref{kr_prop_ec_selective_propmodels} give
    $\propmods{\proppart{B^\sigma_c}} = \bigcap_{\tuple{i, \phi} \in \sigma \rs
    c}{ \Pi^{\Y_\sigma}_i[\phi] }$.

    Take $\tuple{i, \phi} \in \sigma \rs c$. Let $\psi$ be any propositional
    formula with $\propmods{\psi} = \Pi^{\Y_\sigma}_i[\phi]$. Then $\E_i\psi
    \in B^\sigma_c$, so $W \in \mods(\exppart{B^\sigma})$ gives $W, c \models
    \E_i\psi$. Now, \soundness{} and $W \in \mods(K^\sigma)$ also gives $W, c
    \models \S_i\phi$, i.e. $v = v^W_c \in \Pi^W_i[\phi]$. Since
    $\propmods{\phi} \subseteq \Pi^{\Y_\sigma}_i[\phi] = \propmods{\psi}$, we
    get
    \[
        v
        \in \Pi^W_i[\phi]
        \subseteq \Pi^W_i[\psi]
        = \propmods{\psi}
        = \Pi^{\Y_\sigma}_i[\phi].
    \]
    This shows \cref{kr_eqn_ec_sel_implies_det} and completes the proof.
\end{proof}

Note that if the basic postulates are given, the condition $K^\sigma =
\cn(G^\sigma_\snd)$ in \cref{kr_thm_ec_selectivity_characterisation} is
equivalent to $K^\emptyset = \cn(\emptyset)$ by \cref{kr_prop_prior_knowledge}.
In particular, \cref{kr_thm_ec_selectivity_characterisation} applies to our
concrete operators \varbasedcond{}, \partbasedcond{} and \scorebasedop{}. Since
we have already seen these operators are \emph{not} EC-selective, we also have
that they each fail \determination{}.

The potential problem with EC-selectivity, as expressed by \determination{}, is
that it only permits belief formation on the basis of soundness statements
together with firmly believed expertise statements in $\exppart{B^\sigma}$. A
natural weaker notion of expertise-compatible selectivity requires not that $i$
is believed to have expertise on $f_\sigma(i, c, \phi)$, but merely that such
expertise is \emph{consistent} with $B^\sigma$.

\begin{definition}
    A selection scheme $f$ is \emph{weakly expertise-compatible} with an
    operator $\sigma \mapsto \tuple{B^\sigma, K^\sigma}$ if for all
    $\ast$-consistent $\sigma$ and $\tuple{i, c, \phi} \in \sigma$,
    \[
        \neg\E_i{f_\sigma(i, c, \phi)} \notin B^\sigma_c.
    \]
\end{definition}

Mirroring earlier terminology, say an operator is \emph{weakly EC-selective} if
it is selective according to some weakly expertise-compatible scheme. Weak
EC-selectivitity overcomes the issues of EC-selectivity highlighted above on
the sequence
\[
    \sigma = ( \tuple{i, c, p}, \tuple{j, c, p}, \tuple{i, d, p},
    \tuple{j, d, \neg p}).
\]
For example, each of our example operators
\varbasedcond{}, \partbasedcond{} and \scorebasedop{} are weakly EC-selective
for this particular $\sigma$ according to the selection
\begin{align*}
    f_\sigma(i, c, p) &= f_\sigma(j, c, p) = p \\
    f_\sigma(i, d, p) &= f_\sigma(j, d, \neg{p}) = \true.
\end{align*}
However, this selection is \emph{not} case independent. Questions around the
interaction between weak EC-selectivity and case independence, as well as the
whether the example operators are weakly EC-selective and/or
case-independent-selective in general, are left for future work.

\section{Related Work}
\label{kr_sec_relatedwork}

In this section we discuss related work.

\paragraph{Belief Merging.}

In the framework of \textcite{konieczny2002merging}, a merging operator
$\Delta$ maps a multiset of propositional formulas $\Psi =
\{\phi_1,\ldots,\phi_n\}$ and an integrity constraint $\mu$ to a formula
$\Delta_\mu(\Psi)$. Here $\phi_i$ represents the input from source $i$, the
integrity constraint $\mu$ represents sure information which must be respected
-- akin to reports from $\ast$ in our framework -- and $\Delta_\mu(\Psi)$
represents the merged result. Various operators and postulates have been
proposed in the literature; see \cite{Konieczny_2011} for a review.

Merging can be seen as the special case of our framework, if we impose an upper
bound $N$ on the size of the multisets considered as inputs. Indeed,
instantiating our framework with $\srcs = \{1, \ldots, N, \ast\}$ and a single
case $\C = \{c\}$, we can interpret a multiset $\Psi$ and integrity constraint
$\mu$ as a sequence $\sigma_{\Psi, \mu}$, where
\[
    \sigma_{\Psi, \mu}
    = (
        \tuple{\ast, c, \mu},
        \tuple{1, c, \phi_1},
        \ldots
        \tuple{n, c, \phi_n}
    ).
\]
That is, $\ast$ reports the integrity constraint and each source $i$ reports
$\phi_i$.\footnotemark{}
%
In this way, any operator gives rise to a merging operator -- up to logical
equivalence -- by setting
\begin{equation}
    \label{kr_eqn_belief_merging_translation}
    \propmods{\Delta_\mu(\Psi)}
    =
    \propmods{\proppart{B^{\sigma_{\Psi,\mu}}_c}}.
\end{equation}

\footnotetext{
    Since multisets are not ordered, to ensure $\sigma_{\Psi, \mu}$ is
    well-defined we order the reports $\tuple{i, c, \phi_i}$ according to some
    arbitrary but fixed total order on $\lprop$.
}

In fact, for our specific operators \varbasedcond{}, \partbasedcond{} and
\scorebasedop{}, the corresponding merging operators $\vbcmergingop$,
$\pbcmergingop$ and $\exmmergingop$ coincide with well-known \emph{model-based}
merging operators.

\begin{definition}[\textcite{konieczny2002merging,Konieczny_2011}]
    Let $d: \vals \times \vals \to \R_{\ge 0}$ be a function such that
    $d(u, v) = d(v, u)$ and $d(u, v) = 0$ iff $u = v$.\footnotemark{} The
    merging operator $\Delta^{d, {\Sigma}}$ is defined (up to logical
    equivalence) by
    \[
        \propmods{\Delta^{d, {\Sigma}}_\mu(\Psi)}
        =
        \argmin_{v \in \propmods{\mu}}{
            \sum_{i=1}^{n}{
                \min_{u \in \propmods{\phi_i}}{
                    d(u, v)
                }
            }
        }.
    \]
\end{definition}

\footnotetext{
    Such $d$ are called \emph{distance functions} in the merging literature,
    although the triangle inequality $d(u, v) \le d(u, w) + d(w, v)$ is not
    required to hold.
}

That is, $\Delta^{d, \Sigma}_\mu(\Psi)$ selects the models of the integrity
constraint $\mu$ which minimise the sum of the distances to each formula
$\phi_i$, with the distance between $v$ and $\phi_i$ interpreted as the minimal
distance between $v$ and some $\phi_i$ model.

Typical distances $d$ include the \emph{Hamming distance} $d_H$, where $d_H(u,
v)$ is the number of propositional variables on which $u$ and $v$ differ, and
the \emph{drastic distance} $d_D$, where $d_D(u, v)$ is 0 if $u = v$ and 1
otherwise. Our operators give rise to model-based merging operators
corresponding to these distances.

\begin{restatable}{theorem}{restatekrmergingthm}
    \label{kr_thm_merging_operators}
        $\vbcmergingop \equiv \Delta^{d_H, \Sigma}$,
        $\pbcmergingop \equiv \Delta^{d_D, \Sigma}$ and
        $\exmmergingop \equiv \Delta^{d_D, \Sigma}$.
\end{restatable}

The proof can be found in \cref{kr_app_sec_thm_merging_operators}. It follows
that $\vbcmergingop$, $\pbcmergingop$ and $\exmmergingop$ satisfy the
\emph{IC postulates} \icpostulate{0} -- \icpostulate{8} of
\textcite{konieczny2002merging}.

Also note that, perhaps surprisingly, \partbasedcond{} and \scorebasedop{}
result in the same merging operator. In some sense this highlights the
restrictiveness of purely propositional merging, given that the two operators
differ substantially in our general setting (e.g. on \strongcondsucc{} and in
\cref{kr_ex_score_based}).\footnotemark{}

\footnotetext{
    Note that it is \emph{not} the case that \partbasedcond{} and
    \scorebasedop{} output the same propositional beliefs given any input
    $\sigma$; \cref{kr_thm_merging_operators} only shows this to be the case
    for $\sigma$ of the form $\sigma_{\Psi, \mu}$.
}

Indeed, we go beyond propositional merging by considering multiple cases and
explicitly modelling expertise (and trust, via beliefs about expertise). While
it may be possible to model expertise \emph{implicitly} in belief merging (for
example, say $i$ is not trusted on $\psi$ if $\Delta_\mu(\Psi) \not\vdash \psi$
when $\phi_i \vdash \psi$), bringing expertise to the object level allows us to
express more complex beliefs about expertise, such as $\E_{\drX}{p} \lor \E_{\drY}{p}$ in
\cref{kr_ex_hospital_ex_formalised}. It also facilitates postulates which refer
directly to expertise, such as the weakenings of \axiomref{Success} in
\cref{kr_sec_one_step_postulates}. Moreover, such beliefs about expertise
cannot be recovered from propositional beliefs alone, as demonstrated by the
fact that \partbasedcond{} and \scorebasedop{} differ in general but coincide
as merging operators.

However, while more general on a technical level, our problem is more
specialised than merging, since we focus specifically on conflicting
information due to lack of expertise. Belief merging may be applied more
broadly to other types of \emph{information fusion}, e.g. subjective beliefs or
goals~\cite{gregoire_fusion_2006}, where notions of objective expertise do not
apply. While our framework \emph{could} be applied in these settings, our
postulates and operators may no longer be desirable.

Furthermore, there are further postulates in the belief merging literature
which cannot be expressed in our framework due to the fixed-source assumption.
For example, consider the \emph{majority} postulate:

\begin{axiom}[\majoritymerging{}]
    $\exists n \in \Nat:\ \Delta_\mu(\Psi_1 \sqcup \Psi_2^n) \propentails
    \Delta_\mu(\Psi_2)$
\end{axiom}

Clearly \majoritymerging{} requires one to consider multisets of unbounded
size. A variable-domain approach to our belief change problem -- such as the
framework for truth discovery in \cref{chapter_td}, where the set of sources
was given as part of the input -- would allow such postulates to be expressed.

\paragraph{Trust and belief revision.}

\textcite{yasser_21,yasser_journal_2021} study the joint revision of belief and trust in the style
of belief revision theory. As with our framework, they consider reports from
multiple sources, and set out several postulates to govern the interaction
between trust in sources and belief in formulas.
%
Unlike our work, however, they take a more general view of trust -- not based
on any fixed semantic notion such as expertise -- wherein sources are assigned
degrees of trust on each topic. In this way, their view of trustworthiness is
closer to that of truth discovery in \cref{chapter_td}.

Formally, they consider totally ordered sets $\mathcal{D}_b$ and
$\mathcal{D}_t$ of \emph{belief and trust degrees}, respectively, and a finite
collection of \emph{topics} $\mathcal{O}$, where each $T \in \mathcal{O}$ is a
set of formulas such that $\bigcup\mathcal{O} = \mathcal{L}$.\footnotemark{} An
\emph{information state} $\mathcal{K}$ then consists of
%
\begin{inlinelist}
    \item a set of reports of the form $\tuple{i, \phi}$;
    \item a partial function assigning degrees of belief in formulas of
        $\mathcal{L}$; and
    \item a partial function assigning degrees of trust in a source-topic pair.
\end{inlinelist}

\footnotetext{
    Note that we have changed the notation compared to the original paper.
}
%
A revision operator $\ltimes$ then takes an information state
$\mathcal{K}$ and a new report $\tuple{i, \phi}$ and produces a new information
state $\mathcal{K} \ltimes \tuple{i, \phi}$. The authors introduce natural
notions of \emph{entrenchment}, whereby a formula $\phi$ may be more entrenched
in a state $\mathcal{K}$ over $\mathcal{K'}$, and similarly where a source $i$
may be more trusted on a topic $T$ in $\mathcal{K}$ over $\mathcal{K'}$. Such
entrenchment notions are used to state postulates to constrain $\ltimes$.

This framework generalises ours along several dimensions. Primarily, it allows
an operator to consider graded belief and trust via the degree sets
$\mathcal{D}_b$ and $\mathcal{D}_t$. In our framework we have, roughly
speaking, only three degrees:
%
\begin{inlinelist}
    \item 1, if $\phi \in B^\sigma_c$;
    \item -1, if $\neg\phi \in B^\sigma_c$; and
    \item 0, if $\phi \notin B^\sigma_c$ and $\neg\phi \notin B^\sigma_c$.
\end{inlinelist}
Likewise, we have three degrees for trust by considering membership of
$\E_i\phi$ and $\neg\E_i\phi$.

Secondly, their notion of trust is vastly more general than ours. This can be
seen as a benefit, since the framework can be applied in more settings, or
as a drawback, since the generality restricts the extent to which the authors
can state postulates which are reasonable in the general case. Indeed, part of
our motivation to study trust via expertise was to be able to introduce more
specific -- and, hopefully, more interesting -- postulates and operators.

Our notion of separate cases -- representing different instantiations of some
propositional domain -- can also be represented via an encoding trick. Namely,
one can consider the propositional language $\lprop^\C$ formed from
propositional variables of the form $p_c$, for $p \in \propvars$ and $c \in
\C$, read as ``$p$ holds in case $c$''. A report $\tuple{i, c, \phi}$ in our
framework becomes $\tuple{i, \phi_c}$, where $\phi_c$ is obtained from $\phi
\in \lprop$ by replacing each variable $p$ appearing in $\phi$ with $p_c$. To
ensure that trust is fixed across cases, one can choose as topics $T_\phi =
\{\phi_c \mid c \in \C\}$.

Ultimately, our work is complementary and addresses the problem of trust and
belief revision from a different angle. Future work could investigate further
links and differences between the two frameworks; for example by comparing
postulates.

\paragraph{Building trust.} In recent work, \textcite{hunter_building_21}
investigates how trust in a source may be determined from its record on past
reports. Our work shares a common ancestor via \cite{booth_trust_2018}, in
which trust is rooted in expertise and the ability of sources to distinguish
between states. In \cite{booth_trust_2018}, a single partition was used to
represent expertise. Here, we consider possibly several partitions $\Pi^W_i$,
for $W \in \Y_\sigma$.

\textcite{hunter_building_21} considers a richer representation of
distinguishability, in the form of a \emph{pseudo-ultrametric} $d$ on states
for each source; that is, a function $d: \vals \times \vals \to \R_{\ge 0}$
such that
\begin{inlinelist}
    \item $d(v, v) = 0$;
    \item $d(u, v) = d(v, u)$; and
    \item\label{kr_item_ultrametric_inequality}
        $d(u, v) \le \max\{d(u, w), d(w, v)\}$.
\end{inlinelist}
Here $d(u, v)$ represents the degree to which the source is trusted to be able
to distinguish states $u$ and $v$. Due to the so-called \emph{ultrametric
inequality} \cref{kr_item_ultrametric_inequality}, the set of balls of radius
$r$ -- i.e. $\{\{u \mid d(u, v) \le n\} \mid v \in \vals\}$ -- forms a
partition of $\vals$. Consequently, any threshold value $r$ gives rise to a
partition. In this sense the pseudo-ultrametric representation generalises
partitions.

More importantly, \textcite{hunter_building_21} considers how to iteratively
\emph{revise} a pseudo-ultrametric from a so-called \emph{report history}. Such
a history consists of reports of the form $\tuple{\phi, j}$, representing the
fact that the source had previously reported $\phi$ and was either correct (if
$j = 1$) or incorrect (if $j = 0$). An algorithm is given to update a
pseudo-ultrametric $d$ given a history, and thereby modify the revision agent's
perception of the source's expertise on the basis of their past performance.

Report histories can be modelled in our framework by reports from $\ast$. For
example, a negative example $\tuple{\phi, 0}$ from source $i$ can be
represented by the sequence $(\tuple{i, c, \phi}, \tuple{\ast, c, \neg\phi})$.
However, we take a different view on what to \emph{do} with such histories: our
operators select possible partitions representing the source's expertise,
whereas the algorithm of \textcite{hunter_building_21} updates a
pseudo-ultrametric representation. Investigating ways in which the two
approaches can be combined is an interesting direction for future work.

\section{Conclusion}
\label{kr_sec_conclusion}

\paragraph{Summary.} In this chapter we studied a belief change problem --
extending the classical AGM framework -- in which
beliefs about the state of the world in multiple cases, as well as expertise of
multiple sources, must be inferred from a sequence of reports. This allowed us
to take a fresh look at the
interaction between trust (seen as \emph{belief in expertise}) and belief.
By inferring the expertise of the sources from the reports,
we have generalised some earlier approaches to non-prioritised
revision which assume expertise (or reliability, credibility, priority
etc) is known up-front (e.g.
\cite{ferme1999selective,hansson_2001,booth_trust_2018,delgrande2006iterated}).
We went on to propose some concrete belief change operators, and explored their
properties through examples, postulates, and a notion of selective revision.

We saw that conditioning operators satisfy some desirable
properties, and our concrete instances make useful inferences that go
beyond \weakop{}. However, we have examples in which intuitively plausible
inferences are blocked, and conditioning is largely incompatible with
\strongcondsucc{}. Score-based operators, and in particular \scorebasedop{},
were introduced as a possible way around these limitations. \todo{Possible
downsides of \scorebasedop{}; e.g. failure of \duprem{}.}

\paragraph{Limitations and future work.}
There are many possibilities for future work.
%
Firstly, we have a representation result only for conditioning operators. A
characterisation of score-based operators -- either the class in general or the
specific operator \scorebasedop{} -- remains to be found. This would help to
further clarify the differences between conditioning and score-based operators.
%
We have also not considered any computational issues. Determining the
complexity of calculating the results of our example operators, and the
complexity for conditioning and score-based operators more broadly, is left to
future work.
%
Secondly, there is scope for deeper postulate-based analysis. For example,
there should be postulates governing how beliefs change in case $c$ in response
to reports in case $d$. We could also consider more postulates relating trust
and belief, and compare these postulates with those of \textcite{yasser_21}.
Moreover, there are many weaker version of \axiomref{Success} which have
been considered in the literature (e.g. in
\cite{ferme1999selective,hansson_2001,booth_trust_2018}); we should compare
these against our \condsucc{} and \strongcondsucc{} in future work.

Finally, as mentioned above, our framework only deals with three levels of
trust on a proposition: we can believe $\E_i\phi$, believe $\neg \E_i\phi$, or
neither. Future work could investigate how to extend our semantics to talk
about \emph{graded expertise}, and thereby permit more fine-grained
degrees of trust \cite{hunter_building_21,yasser_21,delgrande2006iterated}.

\paragraph{Outlook.}

Broadly speaking, this chapter addressed \emph{normative} properties of
operators, i.e. properties which ``reasonable'' operators should satisfy. We
did not consider any notion of \emph{truthfulness}, i.e. whether or not the
belief set $B^\sigma_c$ is true in the ``actual'' world. Thus, while the
operators introduced here may be \emph{rational} -- in that they satisfy the
postulates -- we cannot say whether they are \emph{truth-tracking}.

The following chapter addresses this gap, by combining our belief change
framework with learning-theoretic notions from formal learning theory. In doing
so we study truth-tracking; e.g. by investigating the extent to which the truth
can be found with non-expert sources and determining which operators are able
to track the truth.
