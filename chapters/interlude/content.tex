\chapter*{Interlude}
\addcontentsline{toc}{chapter}{Interlude}

In \cref{chapter_td,chapter_bipartite_tournaments} we studied how to rank
sources by trustworthiness, broadly speaking, in the context of truth discovery
and bipartite tournament ranking. However, we had no formal semantics to define
the \emph{meaning} of trustworthiness, and indeed this meaning varies between
truth discovery operators and ranking methods. This flexibility was useful for
our social-choice-style analysis, where rankings are commonly used in this
manner.

In the remainder of the thesis we take a stricter view on trustworthiness,
positioning it in relation to \emph{expertise} in a logic-based framework.
Informally, we trust a source on a topic $X$ if we \emph{believe they are an
expert} on matters relating to $X$. Once a suitable notion of expertise is
introduced, the truth discovery problem can be modelled in a similar way to
\emph{belief revision}~\cite{alchourron1985logic} and \emph{belief
merging}~\cite{konieczny2002merging} by considering how to form \emph{beliefs}
on the basis of input reports. Specifically, by considering beliefs both about
the state of the world and the expertise of the sources, we have analogues of
both the source and claim rankings from truth discovery.

Beyond it's relation to trustworthiness, expertise is also a topic of interest
in its own right. The logical properties of expertise -- and its relation to
the truthfulness of information -- are explored in detail in the next chapter
using the tools of modal logic. This framework is used as the basis for a
belief change problem in \cref{chapter_belief_change}, which can be thought of
as a logical version of truth discovery. Finally, \cref{chapter_truthtracking}
shifts the focus away from normative properties of aggregation methods -- as
expressed by axioms in \cref{chapter_td} and postulates in
\cref{chapter_belief_change} -- towards epistemic properties, by considering
\emph{truth-tracking}, i.e. how one can find the truth given reports from
non-expert sources.
