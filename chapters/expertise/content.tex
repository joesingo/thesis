\chapter{Expertise and Information}
\label{chapter_expertise}

In order to properly assess incoming information, it is important to consider
the expertise of the reporting source. We should generally believe statements
within the domain of expertise of the source, but ignore (or otherwise
discount) statements about which the source has no expertise. This applies even
when dealing with honest sources: a well-meaning but non-expert source may make
false claims due to lack of expertise on the relevant facts.
%
The situation may be further complicated if a source comments on multiple
topics at once: we must \emph{filter out} the parts of the statement within
their domain of expertise.

Problems associated with expertise have been exacerbated recently by the
COVID-19 pandemic, in which false information from non-experts has been shared
widely on social media~\cite{Llewellynm1160,dijck2020}. There have also been
high-profile instances of experts going beyond their area of expertise to
comment on issues of public health~\cite{xaudiera2020ibuprofen}, highlighting
the importance of \emph{domain-specific} notions of expertise. Identifying
experts is also an important task for \emph{liquid
democracy}~\cite{blum2016liquid}, in which voters may delegate their votes to
expertise on a given policy issue.

Expertise has been well-studied, with perspectives from behavioural and
cognitive science \cite{chi2014nature,ericsson2010expertise}, sociology
\cite{collins2008rethinking}, and philosophy
\cite{kilov2021brittleness,whyte2010trust,goldman2018expertise}, among other
fields. In this work we study the \emph{logical} content of expertise, and its
relation to truthfulness of information.

Specifically, we develop a \emph{modal logic} framework to model
\emph{expertise} and \emph{soundness of information}. Intuitively, a source has
expertise on $\phi$ if they are able to correctly refute $\phi$ in any
situation where it is false.\footnotemark{} Thus, our notion of expertise
\emph{does not depend on the ``actual'' state of affairs}, but only on the
source's epistemic state.

\footnotetext{
    Note that we could instead consider the dual case: expertise means being
    able to \emph{verify} when a proposition is true.
}

It is \emph{sound} for a source to report $\phi$ if $\phi$ is true \emph{up to
lack of expertise}: if $\phi$ is logically weakened to a proposition $\psi$ on
which the source has expertise, then $\psi$ must be true. That is, the
consequences of $\phi$ on which the source has expertise are true.
%
This formalises the idea of ``filtering out'' parts of a statement within a
source's expertise. For example, suppose $\phi = p \land q$, and the source has
expertise on $p$ but not $q$. Supposing $p$ is true but $q$ is false, $\phi$ is
false. However, if we discard information by ignoring $q$ (on which the source
has no expertise), we obtain the weaker formula $p$, on which the source
\emph{does} have expertise, and which is true. If this holds for all possible
ways to weaken $p \land q$ (this is the case, for instance, if the source does
not have expertise on any statement strictly stronger than $p$), then $p \land
q$ is \emph{false} but \emph{sound} for the source to report.
%
In terms of refutation, $\phi$ is sound if the source cannot refute $\neg\phi$.
That is, either $\phi$ is in fact true, or the source does not possess
sufficient expertise to rule out $\phi$.

This informal picture of expertise already suggests a close connection between
expertise, soundness and \emph{knowledge}. Indeed, we will see that, under
certain conditions, expertise can be equivalently interpreted in terms of
\emph{S4 or S5 knowledge}, familiar from epistemic logic.

Beyond the individual expertise of a single source, one can also consider the
\emph{collective expertise} of a group. For example, a committee may consist of
several experts across different domains, so that by working together the group
achieves expertise beyond any of its individual members. Indeed, such pooling
of expertise becomes necessary in cases where it is infeasible for an
individual to be a specialist in all relevant sub-areas. As a concrete example,
consider the \emph{Rogers Commission
report}\footnote{\url{https://en.wikipedia.org/wiki/Rogers_Commission_Report}}
into the 1986 Challenger disaster, whose members included politicians, military
generals, physicists, astronauts and rocket scientists. Beyond extending the
expertise of its constituents, the breadth of expertise among the commission
allowed it to collectively assess issues at the \emph{intersection} of its
members' specialities.

Towards defining collective expertise we will again turn to (multi-agent)
epistemic logic, borrowing from the well-known notions of \emph{distributed}
and \emph{common knowledge}~\cite{fagin2003reasoning}. Just as individual
expertise (and soundness) can be expressed in terms of knowledge, we will see
that collective expertise can be expressed in terms of collective knowledge.

While the picture of expertise painted so far has been static, it is also
natural to consider the \emph{dynamics} of expertise. For example, how does
expertise change over time as sources interact with the world and gain new
knowledge? What are the effects of \emph{announcements}, particularly when
sources are non-experts? We study the logic of such events via dynamic
operators in the style of dynamic epistemic
logic~\cite{van_Ditmarsch_2008}, and particularly \emph{dynamic evidence
logics}~\cite{van2011dynamic,vanbenthem2014106}.

\paragraph{Contributions.}

On the conceptual side, we develop a modal logic
framework to reason about the expertise of a source and soundness of
information. We also study collective expertise among multiple sources, and
consider how expertise may evolve via learning and announcements. Importantly,
both singular and collective expertise are shown to connected in a precise
sense to standard notions from epistemic logic. This formalises the conceptual
link between expertise and \emph{knowledge}.
%
On the technical side we obtain a sound and complete axiomatisation, and
axiomatise several sub-classes of models with additional axioms.

This chapter is a substantial extension of
\textcite{singleton2021logic}.\footnote{The extended version is under review at
the time of writing~\cite{singleton2022expetise}.} Several of the main proofs
have been formalised with the Lean theorem
prover.\footnote{\url{https://github.com/joesingo/expertise-and-information}}.

\section{Expertise and Soundness}
\label{exp_sec_expertise_and_soundness}

Before the formal definitions we give an example to illustrate the notions of
\emph{expertise} and \emph{soundness}, which are central to the framework.

\begin{example}
    \label{exp_ex_economist_motivation}

    Consider an economist reporting on the possible impact of a novel virus
    which has recently been detected. The virus may or may not be highly
    infectious ($i$) and go on to cause a high death toll ($d$), and there may
    or may not be economic prosperity in the near future ($p$). The economist
    reports that despite the virus, the economy will prosper and there will not
    be mass deaths ($p \land \neg d$). Assume the economist is an expert on
    matters relating to the economy ($\E p$, $\E\neg p$), but not on matters of
    public health ($\neg\E d$, $\neg\E\neg d$). For the sake of the example,
    suppose the virus will in fact cause a high death toll, but the economy
    will nonetheless prosper. Then while the report of $p \land \neg d$ is
    false, it is true if one \emph{ignores the parts on which the economist has
    no expertise} (namely, $\neg d$); in doing so we obtain $p$, which is true.
    The report therefore carries \emph{some} true information, even though it
    is false. We say $p \land \neg d$ is \emph{sound} for the economist in this
    case.

\end{example}

\paragraph{Syntax.}

Let $\Prop$ be a countable set of atomic propositions.
%
To start with, we consider a single information source. Our language $\cL$
includes modal operators to express expertise and soundness statements for this
source, and is defined by the following grammar:
\[
\phi ::=
 p \mid
 \phi \land \phi \mid
 \neg\phi \mid
 \E\phi \mid
 \S\phi \mid
 \A\phi
\]
for $p \in \Prop$. We read $\E\phi$ as ``the source has expertise on
$\phi$, and $\S\phi$ has ``$\phi$ is sound for the source to
report''. We include the universal modality $\A$
for technical convenience; $\A\phi$ is read as ``$\phi$ holds in all
states''~\cite{goranko_1992}.  Other logical connectives ($\lor$, $\limplies$,
$\liff$) and constants ($\top$, $\bot$) are introduced as
abbreviations.

\paragraph{Semantics.}

On the semantic side, we use the notion of an \emph{expertise model}.

\begin{definition}
    \label{exp_def_expertise_model}

    An \emph{expertise model} (hereafter, just \emph{model})
    is a triple $M = (X, P, V)$, where $X$ is a set
    of states, $P \subseteq 2^X$ is a collection of subsets of $X$, and $V:
    \Prop \to 2^X$ is a valuation function. An \emph{expertise frame} is a pair
    $F = (X, P)$.  The class of all models is denoted by $\M$.

\end{definition}

The sets in $P$ are termed \emph{expertise sets}, and represent the
propositions on which the source has expertise. Given the earlier informal
description of expertise as refutation, we interpret $A \in P$ as saying
that whenever the ``actual'' state is outside $A$, the source knows so.

For an expertise model $M = (X, P, V)$, the satisfaction relation between
states $x \in X$ and formulas $\phi \in \cL$ is defined recursively
as follows:
\[
    \begin{array}{lll}
     M, x &\models p &\iff x \in V(p) \\
     M, x &\models \phi \land \psi &\iff M, x \models \phi \text{ and } M, x
         \models \psi \\
     M, x &\models \neg\phi &\iff M, x \not\models \phi \\
     M, x &\models \E\phi &\iff \|\phi\|_M \in P \\
     M, x &\models \S\phi &\iff \forall A \in P: \|\phi\|_M \subseteq A \implies
         x \in A \\
     M, x &\models \A\phi &\iff \forall y \in X: M, y \models \phi
    \end{array}
\]
where $\|\phi\|_M = \{x \in X \mid M, x \models \phi\}$ is the truth set
of $\phi$. For an expertise frame $F = (X, P)$, write $F
\models \phi$ iff $M, x \models \phi$ for all models $M$ based on
$F$ and all $x \in X$. Write $M \models \phi$ iff
$M, x \models \phi$ for all $x \in X$, and $\models \phi$ iff
$M \models \phi$ for all models $M$; we say $\phi$ is \emph{valid}
in this case. Write $\phi \equiv \psi$ iff $\phi \liff \psi$ is
valid. For a set $\Gamma \subseteq \cL$, write $\Gamma \models
\phi$ iff for all models $M$ and states $x$, if $M, x \models
\psi$ for all $\psi \in \Gamma$ then $M, x \models \phi$.

The clauses for atomic propositions and propositional
connectives are standard. For expertise formulas, we have that $\E\phi$
holds exactly when the set of states where $\phi$ is true is an element
of $P$. Expertise is thus a special case of the \emph{neighbourhood semantics}
\cite{Scott1970,montague1970universal,pacuit2017neighborhood}, where each point $x \in X$ has the same
set of neighbourhoods. The clause for soundness reflects the intuition that
$\phi$ is sound exactly when all logically weaker formulas on which the
source has expertise must be true: if $A \in P$ (i.e. the source has
expertise on $A$) and $A$ contains all $\phi$ states, then
$x \in A$. In terms of refutation, $\S\phi$ holds iff there is no
expertise set $A$, false at the actual state $x$, which allows the
source to rule out $\phi$.

Our truth conditions for expertise and soundness also have topological
interpretations, if one views $P$ as the collection of closed sets of a
topology on $X$:\footnote{For this to be the case, $P$ must be closed under
intersections and finite unions, and contain both the empty set and $X$ itself.
We will turn to these closure properties in \cref{exp_sec_closure_properties}.}
$\E\phi$ holds iff $\|\phi\|_M$ is closed, and $\S\phi$ holds at $x$ iff $x$
lies in the \emph{closure} of $\|\phi\|_M$. Our semantics for soundness is
therefore dual to the \emph{interior semantics} for modal logic, where
$\modalnec\phi$ is true at $x$ iff $x$ lies in the interior of $\|\phi\|$. We
can also view the closure operation as \emph{expanding} the set $\|\phi\|_M$
along the lines of the source's expertise; $\phi$ is sound if the ``actual''
state $x$ is included in this expansion.
%
Finally, the clause for the universal modality $\A$ states that $\A\phi$ holds
iff $\phi$ holds at all states $y \in X$.

\footnotetext{
}

\def\w{1}
\def\h{0.5}
\newcommand{\examplemodel}{
    \tikzset{mynode/.style={color=black}}
    \node[mynode] (a) at (0, 0) {\large $ipd$};
    \node[mynode] (b) at (\w, 0) {\large $pd$};
    \node[mynode] (c) at (0, \h) {\large $ip$};
    \node[mynode] (d) at (\w, \h) {\large $p$};
    \node[mynode] (e) at (0, 2*\h) {\large $id$};
    \node[mynode] (f) at (\w, 2*\h) {\large $d$};
    \node[mynode] (g) at (0, 3*\h) {\large $i$};
    \node[mynode] (h) at (\w, 3*\h) {\large $\emptyset$};
}

\begin{example}
    \label{exp_ex_economist_formalisation}

    To formalise \cref{exp_ex_economist_motivation}, consider the model $M = (X, P,
    V)$ shown in \cref{exp_fig_economist_example}, where $X =
    2^{\{i,p,d\}}$, $P = \{\{ipd,pd,ip,p\},
    \{id,d,i,\emptyset\}\}$ (indicated by the solid rectangles; sets in $X$ are
    written as strings for brevity), and $V(q) =
    \{S \mid q \in S\}$. Then we have $M \models \E p$ but $M \not\models \E
    d$. The economist's report of $p \land \neg d$ is represented by the dashed
    region. We see that while $M, ipd \not\models p \land \neg d$, all
    expertise sets containing the dashed region also contain $ipd$, so $M, ipd
    \models \S(p \land \neg d)$. That is, the economist's report is false but
    sound if the ``actual'' state of the world were $ipd$. This act of
    ``expanding'' $\|p \land \neg d\|$ until we reach an expertise set
    corresponds to ignoring the parts of the report on which the economist has
    no expertise, as in \cref{exp_ex_economist_motivation}.

\end{example}

\begin{figure}
    \centering
    \begin{tikzpicture}[scale=2]
        \def\px{0.4}
        \def\py{0.2}
        \def\s{0.95}

        % economist's expertise sets
        \tikzset{expertiseset/.style={
            rounded corners,draw=red,line width=0.4mm
        }}
        \draw[expertiseset]
            (-\px, -\py*\s) -- (\w + \px, -\py*\s) -- (\w + \px, \h + \py*\s)
            -- (-\px, \h + \py*\s) -- cycle;
        \draw[expertiseset]
            (-\px, 2*\h -\py*\s) -- (\w + \px, 2*\h -\py*\s) -- (\w + \px, 3*\h + \py*\s)
            -- (-\px, 3*\h + \py*\s) -- cycle;
        \node[color=red,anchor=east] at (-\px, 0.5*\h) {\large $\|p\|$};
        \node[color=red,anchor=east] at (-\px, 2.5*\h) {\large $\|\neg p\|$};

        % e ∧ ¬d
        \def\r{0.7}
        \draw[rounded corners,draw=blue,thick,line width=0.4mm]
            (-\px*\r, \h + -\py*\s*\r) -- (\w + \px*\r, \h + -\py*\s*\r) --
            (\w + \px*\r, \h + \py*\s*\r) -- (-\px*\r, \h + \py*\s*\r) -- cycle;
        \node[color=blue,anchor=west] at (\w + \px, \h) {\large $\|p \land \neg d\|$};

        \examplemodel{}

    \end{tikzpicture}
    \caption[
        An example expertise model.
    ]{
        Expertise model from \cref{exp_ex_economist_formalisation}, which
        formalises \cref{exp_ex_economist_motivation}.
    }
    \label{exp_fig_economist_example}
\end{figure}

We further illustrate the semantics by listing some valid formulas.

\begin{proposition}
\label{exp_prop_validities}

    The following formulas are valid:

    \begin{enumerate}
        \item\label{exp_item_truths_sound} $\phi \limplies \S\phi$

        \item\label{exp_item_e_global} $\E\phi \liff \A\E\phi$

        \item\label{exp_item_weakening} $\A(\phi \limplies \psi) \limplies (\S\phi
        \land \E\psi \limplies \psi)$

        \item\label{exp_item_e_and_s} $\E\phi \limplies \A(\S\phi \limplies \phi)$

    \end{enumerate}
\end{proposition}

\begin{proof}

    Let $M = (X, P, V)$ be a model and $x \in X$. \cref{exp_item_truths_sound} and
    \cref{exp_item_e_global} are clear. For \cref{exp_item_weakening}, suppose $M, x
    \models \A(\phi \limplies \psi)$. Then $\|\phi\|_M \subseteq \|\psi\|_M$.
    Further, suppose $M, x \models \S\phi \land \E\psi$. Then $\|\phi\|_M
    \subseteq \|\psi\|_M \in P$; taking $A = \|\psi\|_M$ in the definition of
    the semantics for $\S$, we get by $M, x \models \S\phi$ that $x \in
    \|\psi\|_M$, i.e. $M, x \models \psi$. Finally, \cref{exp_item_e_and_s}
    follows from \cref{exp_item_e_global} and \cref{exp_item_weakening} by taking
    $\psi = \phi$.
\end{proof}

Here \cref{exp_item_truths_sound} says that all truths are sound.
\cref{exp_item_e_global} says that expertise is global. \cref{exp_item_weakening} says
that if the source has expertise on $\psi$, and $\psi$ is logically weaker than
some sound formula $\phi$, then $\psi$ is in fact true. This formalises the
idea that if $\phi$ is true \emph{up to lack of expertise}, then weakening
$\phi$ until expertise holds (i.e. discarding parts of $\phi$ on which the
source does not have expertise) results in something true. \cref{exp_item_e_and_s}
says that if the source has expertise on $\phi$, then whenever $\phi$ is sound
it is also true.

\section{Closure Properties}
\label{exp_sec_closure_properties}

So far we have not imposed any constraints on the collection of expertise sets
$P$. But given our interpretation of $P$, it may be natural to
require that $P$ is closed under certain set-theoretic operations. Say
a frame $F = (X, P)$ is

\begin{itemize}

    \item \emph{closed under intersections} if $\{A_i\}_{i \in I} \subseteq P$
          implies $\bigcap_{i \in I}{A_i} \in P$

    \item \emph{closed under unions} if $\{A_i\}_{i \in I} \subseteq P$ implies
          $\bigcup_{i \in I}{A_i} \in P$

    \item \emph{closed under finite unions} if $A, B \in P$ implies $A \cup B
          \in P$

    \item \emph{closed under complements} if $A \in P$ implies $X \setminus A
          \in P$

\end{itemize}

In the first two cases we allow the empty collection $\emptyset \subseteq P$,
and employ the nullary intersection convention $\bigcap \emptyset = X$.
Consequently, closure under intersections implies $X \in P$, and closure under
unions implies $\emptyset \in P$.

Say a model has any of the above properties if the underlying frame does. Write
$\Mint$, $\Munions$, $\Munions$, $\Mfunions$ and $\Mcompl$ for the classes
of models closed under intersections, unions, finite unions and complements
respectively.

What are the intuitive interpretations of these closure conditions? Consider
again our interpretation of $A \in P$: whenever the actual state is not in $A$,
the source knows so. With this in mind, closure under intersections is a
natural property: if $x \notin \bigcap_{i \in I}{A_i}$ then there is some $i
\in I$ such that $x \notin A_i$; the source can then use this to refute $A_i$
and therefore know that the actual state $x$ does not lie in the intersection
$\bigcap_{i \in I}{A_i}$. A similar argument can be made for finite unions: if
$x \notin A \cup B$ then the source can use $x \notin A$ and $x \notin B$ to
refute both $A$ and $B$. Closure under \emph{arbitrary} unions is less clear
cut; determining that $x \notin \bigcup_{i \in I}{A_i}$ requires the source to
refute (potentially) infinitely many propositions $A_i$.  This is more
demanding from a computational and cognitive perspective, and we therefore view
closure under (arbitrary) unions as an optional property which may or may not
be appropriate depending on the situation one wishes to model. Finally,
closure under complements removes the distinction between refutation and
\emph{verification}: if the agent can refute $A$ whenever $A$ is false, they
can also verify $A$ whenever $A$ is true. We view this as another optional
property, which is appropriate in situations where \emph{symmetric} expertise
is desirable (i.e. when expertise on $\phi$ and $\neg\phi$ should be considered
equivalent).

Several of these properties can be formally captured in our language at the
level of frames.

\begin{proposition}
\label{exp_prop_frame_conditions}

    Let $F = (X, P)$ be a non-empty frame. Then

    \begin{enumerate}

        \item\label{exp_item_frame_condition_intersections} $F$ is closed under
            intersections iff $F \models \A(\S\phi \limplies \phi) \limplies
            \E\phi$ for all $\phi \in \cL$

        \item\label{exp_item_frame_condition_finunions} $F$ is closed under finite
            unions iff $F \models \E\phi \land \E\psi \limplies \E(\phi \lor
            \psi)$ for all $\phi, \psi \in \cL$

        \item\label{exp_item_frame_condition_compl} $F$ is closed under complements
            iff $F \models \E\phi \liff \E\neg\phi$ for all $\phi \in \cL$

    \end{enumerate}
\end{proposition}

\begin{proof}

    We prove only the first claim; the others are straightforward.

    ``if'': We show the contrapositive. Suppose $F$ is not closed under
    intersections. Then there is a collection $\{A_i\}_{i \in I} \subseteq P$
    such that $B := \bigcap_{i \in I}A_i \notin P$. Let $p$ be an arbitrary
    atomic proposition, and define a valuation $V$ by $V(p) = B$ and $V(q) =
    \emptyset$ for $q \ne p$. Let $M = (X, P, V)$ be the corresponding model.
    Since $X$ is assumed to be non-empty, we may take some $x \in X$.

    We claim that $M, x \models \A(\S p \limplies p)$ but $M, x
    \not\models \E p$. Clearly $M, x \not\models \E p$ since
    $\|p\|_M = B \notin P$. For $M, x \models \A(\S p \limplies
    p)$, suppose $y \in X$ and $M, y \models \S p$. Let $j
    \in I$. Then $A_j \in P$, and
    \[
        \|p\|_M
        = B
        = \bigcap_{i \in I}{A_i}
        \subseteq A_j
    \]
    so by $M, y \models \S p$ we have $y \in A_j$. Hence $y
    \in \bigcap_{j \in I}A_j = B = \|p\|_M$, so $M, y \models p$. This
    shows that any $y \in X$ has $M, y \models \S p \limplies p$,
    and thus $M, x \models \A(\S p \limplies p)$. Hence $F
    \not\models \A(\S p \limplies p) \limplies \E p$.

    ``only if'': Suppose $F$ is closed under intersections. Let $M$
    be a model based on $F$ and take $x \in X$. Let $\phi
    \in \cL$. Suppose $M, x \models \A(\S\phi \limplies \phi)$. Then
    $\|\S\phi\|_M \subseteq \|\phi\|_M$. But since $\models \phi
    \limplies \S\phi$, we have $\|\phi\|_M \subseteq \|\S\phi\|_M$ too.
    Hence $\|\phi\|_M = \|\S\phi\|_M$, i.e.
    \[
        \|\phi\|_M
        = \|\S\phi\|_M
        = \bigcap\{A \in P \mid \|\phi\|_M \subseteq A\}
        \in P
    \]
    where we use the fact that $P$ is closed under intersections in the
    final step. Hence $\|\phi\|_M \in P$, so $M, x \models
    \E\phi$.
\end{proof}

The question of whether closure under (arbitrary) unions can be expressed in
the language is still open. By \cref{exp_prop_frame_conditions}
\cref{exp_item_frame_condition_intersections} and \cref{exp_prop_validities}
\cref{exp_item_e_and_s}, the language fragment $\cLSA$ containing only the $\S$
and $\A$ modalities is equally expressive as the full language $\cL$ with
respect to $\Mint$, since $\E\phi$ is equivalent to $\A(\S\phi \limplies \phi)$
in such models. In general $\cLSA$ is strictly less expressive, since $\cLSA$
cannot distinguish between a model and its closure under intersections.

\begin{lemma}
\label{exp_lemma_lsa_intersections}

    Let $M = (X, P, V)$ be a model, and $M' = (X, P', V)$ its closure under
    intersections, where $A \in P'$ iff $A = \bigcap_{i \in I}{A_i}$ for some
    $\{A_i\}_{i \in I} \subseteq P$. Then for all $\phi \in \cLSA$ and $x \in
    X$, we have $M, x \models \phi$ iff $M', x \models \phi$.

\end{lemma}

\begin{proof}

    By induction on $\cLSA$ formulas. The cases for atomic
    propositions, propositional connectives and $\A$ are straightforward. We
    treat only the case for $\S$.
    %
    The ``if'' direction is clear using
    the induction hypothesis and the fact that $P \subseteq P'$. Suppose $M, x
    \models \S\phi$. Take $A = \bigcap_{i \in I}{A_i} \in P'$, where each $A_i$
    is in $P$, such that $\|\phi\|_{M'} \subseteq A$. By the induction
    hypothesis, $\|\phi\|_M \subseteq A$. For any $i \in I$, $\|\phi\|_M
    \subseteq A \subseteq A_i$ and $M, x \models \S\phi$ gives $x \in A_i$.
    Hence $x \in \bigcap_{i \in I}{A_i} = A$. This shows $M', x \models
    \S\phi$.
\end{proof}

It follows that $\cLSA$ is strictly less expressive than $\cL$.\footnote{
    Indeed, consider $M = (X, P, V)$, where $X = \{1, 2, 3\}$, $P = \{\{1, 2\}, \{2,
    3\}\}$ and $V(p) = \{1, 2\}$, $V(q) = \{2, 3\}$ for some fixed $p, q \in
    \Prop$. Let $M'$ be as in \cref{exp_lemma_lsa_intersections}. Then $M', 1
    \models \E(p \land q)$ and $M, 1 \not\models \E(p \land q)$, but $M$ and
    $M'$ agree on $\cLSA$ formulas. Hence $\E(p \land q)$ is not equivalent to
    any $\cLSA$ formula.
}
%
To round off the discussion of closure properties, we note that within the
class of frames closed under intersections, closure under finite unions is also
captured by the well-known \textbf{K} axiom -- $\modalnec(\phi \limplies \psi)
\limplies (\modalnec\phi \limplies \modalnec\psi)$ -- for the dual soundness operator
$\hat{\S}\phi := \neg\S\neg\phi$:

\begin{proposition}
\label{exp_prop_finite_unions_frame_condition}

    Suppose $F = (X, P)$ is non-empty and closed under intersections. Then $F$
    is closed under finite unions if and only if $F \models \hat{\S}(\phi
    \limplies \psi) \limplies (\hat{\S}\phi \limplies \hat{\S}\psi)$ for all
    $\phi, \psi \in \cL$.

\begin{proof}

    ``if'': We show the contrapositive. Suppose $F$ is closed under
    intersections but not finite unions, so that there are $B_1, B_2
    \in P$ with $B_1 \cup B_2 \notin P$. Set
    \[
    C = \bigcap\{A \in P \mid B_1 \cup B_2 \subseteq A\}\]
    By closure under intersections, $C \in P$. Clearly $B_1 \cup
    B_2 \subseteq C$. Since $C \in P$ but $B_1 \cup B_2 \notin
    P$, $B_1 \cup B_2 \subset C$. Hence there is $x \in C
    \setminus (B_1 \cup B_2)$.

    Now pick distinct atomic propositions $p$ and $q$, and let $V$ be any
    valuation with $V(p) = B_1 \cup B_2$ and $V(q) = B_1$. Let $M = (X, P, V)$
    be the corresponding model. We make three claims:

    \begin{itemize}

        \item $M, x \models \S p$: Take $A \in P$ such that $\|p\|_M \subseteq
              A$. Then $B_1 \cup B_2 \subseteq A$, so $C \subseteq A$. Since $x
              \in C$, we have $x \in A$ as required.

        \item $M, x \not\models \S q$: This is clear since $B_1 \in P$,
              $\|q\|_M \subseteq B_1$, but $x \notin B_1$.

        \item $M, x \not\models \S (p \land \neg q)$: Note that $\|p \land \neg
              q\|_M = V(p) \setminus V(q) = B_2 \setminus B_1$. Therefore we
              have $B_2 \in P$ and $\|p \land \neg q\|_M \subseteq B_2$, but $x
              \notin B_2$.

    \end{itemize}

    Now set $\phi = \neg q$ and $\psi = \neg p$. We have
    \[
       \hat{\S}(\phi \limplies \psi)
       = \neg\S\neg(\phi \limplies \psi)
       \equiv \neg\S(\phi \land \neg\psi)
       \equiv \neg\S(p \land \neg q)
    \]
    \[
       \hat{\S}\phi \limplies \hat{\S}\psi
       = \neg\S\neg\phi \limplies \neg\S\neg\psi
       \equiv \neg\S q \limplies \neg\S p
       \equiv \S p \limplies \S q
    \]
    From the claims above we see that $M, x \models \hat{\S}(\phi
    \limplies \psi)$ but $M, x \not\models \hat{\S}\phi \limplies
    \hat{\S}\psi$. Since $M$ is a model based on $F$, we are
    done.

    ``only if'': Suppose $F$ is closed under intersections and finite
    unions. Let $M$ be a model based on $F$ and $x$ a state
    in $M$. Suppose $M, x \models \hat\S(\phi \limplies \psi)$
    and $M, x \models \hat\S\phi$. Then $M, x \not\models
    \S\neg(\phi \limplies \psi)$ and $M, x \not\models \S\neg\phi$.
    Hence there is $A \in P$ such that $\|\neg(\phi \limplies
    \psi)\|_M \subseteq A$ but $x \notin A$, and $B \in P$ such
    that $\|\neg\phi\|_M \subseteq B$ but $x \notin B$. Note
    \[
        \|\neg\psi\|_M
        \subseteq \|\phi \land \neg\psi\|_M \cup \|\neg\phi\|_M
        = \|\neg(\phi \limplies \psi)\|_M \cup \|\neg\phi\|_M
        \subseteq A \cup B.
    \]
    Since $x \notin A \cup B$ and $A \cup B \in P$ by closure
    under finite unions, this shows $M, x \not\models \S\neg\psi$, i.e.
    $M, x \models \hat\S\psi$. This completes the proof of $F
    \models \hat{\S}(\phi \limplies \psi) \limplies (\hat{\S}\phi \limplies
    \hat{\S}\psi)$.
\end{proof}
\end{proposition}

\section{Connection with Epistemic Logic}
\label{exp_sec_connection_with_ep_logic}

In this section we explore the connection between our logic and \emph{epistemic
logic}, for certain classes of expertise models. In particular, we show a
one-to-one mapping between classes of expertise models and \emph{S4 and S5
relational models}, and a translation from $\cL$ to the modal language with
knowledge operator $\K$ which allows expertise and soundness to be expressed in
terms of \emph{knowledge}.

First, we introduce the syntax and (relational) semantics of epistemic logic.
Let $\cLKA$ be the language formed from $\Prop$ with modal operators $\K$ and
$\A$. We read $\K\phi$ as \emph{the source knows} $\phi$.

\begin{definition}
\label{exp_def_relational_models}

    A \emph{relational model} is a triple $M^* = (X, R, V)$, where $X$ is a set
    of states, $R \subseteq X \times X$ is a binary relation on $X$, and $V:
    \Prop \to 2^X$ is a valuation function. The class of all relational models
    is denoted by $\Mrel$.

\end{definition}

The satisfaction relation for $\cLKA$ is defined recursively: the clauses
for atomic propositions, propositional connectives and $\A$ are the same
as for expertise models, and
\[
    M^*, x \models \K\phi
    \iff
    \forall y \in X: xRy \implies M^*, y \models \phi.
\]
As is standard, $R$ is interpreted as an \emph{epistemic accessibility relation}:
$xRy$ means that the source considers $y$ possible if the ``actual''
state of the world is $x$. We will be interested in the logics of S4 and
S5, which are axiomatised by \textbf{KT4} and \textbf{KT5}, respectively:

\begin{itemize}
    \item \textbf{K}: $\K(\phi \limplies \psi) \limplies (\K\phi \limplies
          \K\psi)$

    \item \textbf{T}: $\K\phi \limplies \phi$

    \item \textbf{4}: $\K\phi \limplies \K\K\phi$

    \item \textbf{5}: $\neg\K\phi \limplies \K\neg\K\phi$

\end{itemize}

\textbf{T} says that all knowledge is true, \textbf{4} expresses \emph{positive
introspection} of knowledge, and \textbf{5} expresses \emph{negative
introspection}. One can show that \textbf{K}, \textbf{T} and \textbf{5}
together prove \textbf{4}~\cite[p. 51]{zach2019}, and some authors write
\textbf{KT45} instead of \textbf{KT5}. S5 is therefore \emph{stronger} than S4,
in the sense that any formula provable in S4 is also provable in S5. In fact,
S5 has been criticised as \emph{too} strong in the philosophical literature,
since the negative introspection \textbf{5} is a rather idealised property of
knowledge. For example, it is certainly reasonable to expect that \textbf{5}
may fail for agents who are not perfectly rational (e.g. humans).

It is well known that S4 is sound and complete for the class of relational
models where $R$ is reflexive and transitive, and that S5 is sound and complete
for the class of relational models where $R$ is an equivalence relation.
Accordingly, we write $\Msfour$ for the class of all $M^*$ where $R$ is
reflexive and transitive, and $\Msfive$ for $M^*$ where $R$ is an equivalence
relation.

Our first result connecting expertise and knowledge is on the semantic side: we
show there is a bijection between expertise models closed under intersections
and unions and S4 models. Moreover, there is a close connection between the
collection of expertise sets $P$ and the corresponding relation $R$.
Since expertise models closed under intersections and unions
are \emph{Alexandrov topological spaces} (where $P$ is the set of closed sets),
this is essentially a reformulation of a known result linking relational
semantics over S4 frames and topological interior semantics over Alexandrov
spaces~\cite{van2007modal,ozgun_evidence}.\footnotemark{} To be self-contained,
we prove it for our setting here. First, we show how to map a collection of
sets $P$ to a binary relation.

\footnotetext{
    In fact, the interior semantics has an intrinsic epistemic interpretation
    (without appeal to any link with relational semantics) if one views
    open sets as \emph{evidence}~\cite[pp. 24]{ozgun_evidence}.
}

\begin{definition}
\label{exp_def_rp}

    For a set $X$ and $P \subseteq 2^X$, let $R_P$ be the binary relation on
    $X$ given by
    \[
        x{R_P}y \iff \forall A \in P: (y \in A \implies x \in A).
    \]
\end{definition}

In the case where $P$ is the collection of closed sets of a topology on $X$,
$R_P$ is the \emph{specialisation preorder}.
%
\cref{exp_fig_rp_example} shows an example of $R_P$ for $X$ and $P$ from
\cref{exp_ex_economist_formalisation}.
%
In what follows, say a set $A \subseteq X$ is \emph{downwards closed} with
respect to a relation $R$ if $xRy$ and $y \in A$ implies $x \in A$.

\begin{figure}[h]
    \centering
    \begin{subfigure}{.4\textwidth}
        \centering
        \begin{tikzpicture}[scale=2.2]
            \examplemodel{}
            % economist's accessibility relation
            \tikzset{symmacc/.style={<->,thick,draw=red}}
            \def\b{15}
            \draw[symmacc] (a) edge[bend right=\b] (b);
            \draw[symmacc] (a) -- (c);
            \draw[symmacc] (a) -- (d);
            \draw[symmacc] (b) -- (c);
            \draw[symmacc] (b) -- (d);
            \draw[symmacc] (c) edge[bend left=\b] (d);

            \draw[symmacc] (e) edge[bend right=\b] (f);
            \draw[symmacc] (e) -- (g);
            \draw[symmacc] (e) -- (h);
            \draw[symmacc] (f) -- (g);
            \draw[symmacc] (f) -- (h);
            \draw[symmacc] (g) edge[bend left=\b] (h);
        \end{tikzpicture}
    \end{subfigure}
    \begin{subfigure}{.4\textwidth}
        \centering
        \begin{tikzpicture}[scale=2.2]
            \examplemodel{}
            % non-symmetric accessibility relation
            \tikzset{acc/.style={->,thick,draw=blue}}
            \def\b{15}
            \draw[acc] (c) -- (a);
            \draw[acc] (e) edge[bend right=30] (a);
            \draw[acc] (e) -- (c);

            \draw[acc] (h) -- (g);
            \draw[acc] (h) -- (f);

            \draw[acc] (b) -- (d);
            \draw[acc] (d) -- (b);
        \end{tikzpicture}
    \end{subfigure}
    \caption[
        Illustration of the binary relation associated with an expetise frame.
    ]{
        Left: the relation $R_P$ corresponding to $X$ and $P$ from
        \cref{exp_ex_economist_formalisation} (with reflexive edges omitted). Note
        that $R_P$ is an equivalence relation, with equivalence classes $\|p\|$
        and $\|\neg p\|$. Right: an example of a non-symmetric relation $R_P$,
        corresponding to $P = \{
            \emptyset,
            X,
            \{id,ip,ipd\},
            \{id,ip\},
            \{id\},
            \{i,\emptyset\},
            \{\emptyset,d\},
            \{p,pd\}
        \}$.
    }
    \label{exp_fig_rp_example}
\end{figure}

\begin{lemma}
\label{exp_lemma_same_dc_sets_implies_equality}
    Let $X$ be a set and $R, S$ reflexive and transitive relations on $X$. Then
    if $R$ and $S$ share the same downwards closed sets, $R = S$.
\end{lemma}
\begin{proof}
    Suppose $xRy$. Set $A = \{z \in X \mid zSy\}$. By transitivity of $S$, $A$
    is downwards closed wrt $S$.  By assumption, $A$ must also be downwards
    closed wrt $R$. By reflexivity of $S$, $y \in A$. Hence $xRy$ implies $x
    \in A$, i.e. $xSy$. This shows $R \subseteq S$, and the reverse inclusion
    holds by a symmetrical argument. Hence $R = S$.
\end{proof}

\begin{lemma}
\label{exp_lemma_p_to_rp_mapping}
    Let $X$ be a set.

    \begin{enumerate}
        \item\label{exp_item_rp_ref_and_tr} For any $P \subseteq 2^X$, $R_P$ is
            reflexive and transitive.

        \item\label{exp_item_rp_dc_property} If $P \subseteq 2^X$ is closed under
             unions and intersections, then for all $A \subseteq X$:
            \[
                A \in P \iff A \text{ is downwards closed wrt } R_P.
            \]

        \item\label{exp_item_rp_surjectivity} If $R$ is a reflexive and transitive
            relation on $X$, there is $P \subseteq 2^X$ closed under unions and
            intersections such that $R_P = R$.
    \end{enumerate}
\end{lemma}

\begin{proof}\leavevmode
    \begin{enumerate}
        % \item Reflexivity is clear. For transitivity, suppose $x{R_P}y$ and
        %       $y{R_P}z$. Take $A \in P$ such that $z \in A$. Then $y{R_P}z$
        %       gives $y \in A$. But then $x{R_P}y$ gives $x \in A$. Hence
        %       $x{R_P}z$.
        \item Straightforward by the definition of $R_P$.

        \item Suppose $P$ is closed under unions and intersections and let $A
              \subseteq X$.  First suppose $A \in P$. Then $A$ is downwards
              closed with respect to $R_P$: if $y \in A$ and $x{R_P}y$ then, by
              definition of $R_P$, we have $x \in A$.

              Next suppose $A$ is downwards closed with respect to $R_P$. We
              claim \[ A = \bigcup_{y \in A}\bigcap\{B \in P \mid y \in B\} \]
              Since $P$ is closed under intersections and unions, this will
              show $A \in P$. The left-to-right inclusion is clear, since any
              $y \in A$ lies in the term of the union corresponding to $y$. For
              the right-to-left inclusion, take any $x$ in the set on the RHS.
              Then there is $y \in A$ such that $x \in \bigcap\{B \in P \mid y
              \in B\}$. But this is just a rephrasing of $x{R_P}y$. Since $A$
              is downwards closed, we get $x \in A$ as required.

        \item Take any reflexive and transitive relation $R$. Set
              \[
                  P
                  =
                  \{A \subseteq X \mid A \text{ is downwards closed wrt } R\}.
                \]
              It is easily seen that $P$ is closed under unions and
              intersections. We need to show that $R_P = R$.
              %
              By \cref{exp_item_rp_ref_and_tr}, $R_P$ is reflexive and
              transitive.  By \cref{exp_lemma_same_dc_sets_implies_equality}, it is
              sufficient to show that $R_P$ and $R$ share the same downwards
              closed sets.  Indeed, for any $A \subseteq X$ we get by
              \cref{exp_item_rp_dc_property} and the definition of $P$ that
              \[
              \begin{aligned}
                  A \text{ is downwards closed wrt } R_P
                  &\iff A \in P \\
                  &\iff A \text{ is downwards closed wrt } R.
              \end{aligned}
              \]
              Hence $R = R_P$.
    \end{enumerate}
\end{proof}

We can now state the correspondence between expertise models and S4 relational
models.

\begin{theorem}
\label{exp_thm_s4_semantic_link}

    The mapping $f: \Mint \cap \Munions \to \Msfour$ given by $(X, P, V)
    \mapsto (X, R_P, V)$ is bijective.
\end{theorem}

\begin{proof}

    \cref{exp_lemma_p_to_rp_mapping} \cref{exp_item_rp_ref_and_tr} shows that $f$ is
    well-defined, i.e. that $f(M)$ does indeed lie in $\Msfour$ for any
    expertise model $M$. Injectivity follows from \cref{exp_lemma_p_to_rp_mapping}
    \cref{exp_item_rp_dc_property}, since $P$ is fully determined by $R_P$ for
    expertise models closed under unions and intersections. Finally,
    \cref{exp_lemma_p_to_rp_mapping} \cref{exp_item_rp_surjectivity} shows that $f$
    is surjective.
\end{proof}

If we consider closure under complements together with intersections, an
analogous result holds with S5 taking the place of S4.

\begin{theorem}
\label{exp_thm_s5_semantic_link}

    The mapping $g: \Mint \cap \Mcompl \to \Msfive$ given by $(X, P, V)
    \mapsto (X, R_P, V)$ is bijective.
\end{theorem}

\begin{proof}
    First, note that $\Mint \cap \Mcompl \subseteq \Mint \cap
    \Munions$, since any union of sets in $P$ can be written as a
    complement of intersection of complements of sets in $P$. Therefore
    $g$ is simply the restriction of $f$ from
    \cref{exp_thm_s4_semantic_link} to $\Mint \cap \Mcompl$.

    To show $g$ is well-defined, we need to show that $R_P$ is an equivalence
    relation whenever $P$ is closed under intersections and complements.
    Reflexivity and transitivity were already shown in
    \cref{exp_lemma_p_to_rp_mapping} \cref{exp_item_rp_ref_and_tr}. We show $R_P$ is
    symmetric.  Suppose $x{R_P}y$. Let $A \in P$ such that $x \in A$.  Write $B
    = X \setminus A$. Then since $P$ is closed under complements, $B \in P$.
    Since $x{R_P}y$ and $x \notin B$, we cannot have $y \in B$. Thus $y \notin
    B = X \setminus A$, i.e. $y \in A$. This shows $y{R_P}x$. Hence $R_P$ is an
    equivalence relation.

    Injectivity of $g$ is inherited from injectivity of $f$ from
    \cref{exp_thm_s4_semantic_link}. For surjectivity, it suffices to show that
    $f^{-1}(M^*)$ is closed under complements when $M^* = (X, R, V) \in
    \Msfive$. Recall, from \cref{exp_lemma_p_to_rp_mapping}
    \cref{exp_item_rp_surjectivity}, that $f^{-1}(M^*) = (X, P, V)$, where $A
    \in P$ iff $A$ is downwards closed with respect to $R$. Suppose $A \in P$,
    i.e. $A$ is downwards closed. To show $X \setminus A$ is downwards closed,
    suppose $y \in X \setminus A$ and $xRy$.  By symmetry of $R$, $yRx$. If $x
    \in A$, then downwards closure of $A$ would give $y \in A$, but this is
    false. Hence $x \notin A$, i.e. $x \in X \setminus A$. Thus $X \setminus A$
    is downwards closed, so $P$ is closed under complements. This completes the
    proof.
\end{proof}

The mappings between expertise models and relational models also preserve the
truth value of formulas, via the following translation $t: \cL \to
\cLKA$, which expresses expertise and soundness in terms of knowledge:
\[
\begin{array}{lll}
 &t(p) &= p \\
 &t(\phi \land \psi) &= t(\phi) \land t(\psi) \\
 &t(\neg\phi) &= \neg t(\phi) \\
 &t(\E\phi) &= \A(\neg t(\phi) \limplies \K\neg t(\phi)) \\
 &t(\S\phi) &= \neg\K\neg t(\phi) \\
 &t(\A\phi) &= \A t(\phi).
\end{array}
\]
The only interesting cases are for $\E\phi$ and $\S\phi$. The
translation of $\E\phi$ corresponds directly to the intuition of
expertise as refutation: in all possible scenarios, if $\phi$ is false
the source knows so. The translation of $\S\phi$ says that soundness is
just the dual of knowledge: $\phi$ is sound if the source does not \emph{know}
that $\phi$ is false. Of particular interest is the case where $\phi$ lies in
the purely propositional language $\cLzero$, i.e. does not contain any
modalities. Such formulas describe the ``ontic'' facts of the world, and
do not refer to the expertise of the source. Since $t$ leaves atomic
propositions unchanged and preserves the structure of conjunctions and
negations, we have $t(\phi) = \phi$. Consequently $t(\E\phi) = \A(\neg\phi
\limplies \K\neg\phi)$ and $t(\S\phi) = \neg\K\neg\phi$.

\begin{theorem}
\label{exp_thm_s4s5_translation}

    Let $f: \Mint \cap \Munions \to \Msfour$ be the bijection from
    \cref{exp_thm_s4_semantic_link}. Then for all $M = (X, P, V) \in \Mint
    \cap \Munions$, $x \in X$ and $\phi \in \cL$:
    \begin{equation}
        \label{exp_eqn_s4s5_1}
        M, x \models \phi \iff f(M), x \models t(\phi)
    \end{equation}
    Moreover, if $g: \Mint \cap \Mcompl \to \Msfive$ is the bijection from
    \cref{exp_thm_s5_semantic_link}, then for all $M = (X, P, V) \in
    \Mint \cap \Mcompl$:
    \begin{equation}
        \label{exp_eqn_s4s5_2}
        M, x \models \phi \iff g(M), x \models t(\phi)
    \end{equation}
\end{theorem}

\begin{proof}

    Note that since $g$ is defined as the restriction of $f$ to $\Mint \cap
    \Mcompl$, \cref{exp_eqn_s4s5_2} follows from \cref{exp_eqn_s4s5_1}. We show
    \cref{exp_eqn_s4s5_1} only.
    %
    Let $M = (X, P, V) \in \Mint \cap \Munions$. Write $f(M) =
    (X, R, V)$. From the definition of $f$ and
    \cref{exp_lemma_p_to_rp_mapping} \cref{exp_item_rp_dc_property}, we have
    \[
        A \in P \iff A \text{ is downwards closed wrt R} \qquad(*)
    \]
    We show \cref{exp_eqn_s4s5_1} by induction. The only non-trivial cases are
    $\E$ and $\S$ formulas.

    \begin{itemize}
        \item $\E$: Suppose $M, x \models \E\phi$. Then $\|\phi\|_M \in P$. By
              the induction hypothesis and $(*)$, this means
              $\|t(\phi)\|_{f(M)}$ is downwards closed with respect to $R$. Now
              take $y \in X$ such that $f(M), y \models \neg t(\phi)$. Then $y
              \notin \|t(\phi)\|_{f(M)}$. Since this set is downwards closed,
              it cannot contain any $R$-successor of $y.$ Hence $f(M), y
              \models \K\neg t(\phi)$. This shows that $f(M), x \models \A(\neg
              t(\phi) \limplies \K\neg t(\phi))$, i.e. $f(M), x \models
              t(\E\phi)$.

              Now suppose $f(M), x \models t(\E\phi)$, i.e. $f(M), x \models
              \A(\neg t(\phi) \limplies \K\neg t(\phi))$. We show $\|\phi\|_M$
              is downwards closed. Suppose $yRz$ and $z \in \|\phi\|_M$. By the
              induction hypothesis, $f(M), z \not\models \neg t(\phi)$. Hence
              $f(M), y \not\models \K\neg t(\phi)$. Since $\neg t(\phi)
              \limplies \K\neg t(\phi)$ holds everywhere in $f(M)$, this means
              $f(M), y \models t(\phi)$; by the induction hypothesis again we
              get $M, y \models \phi$ and thus $y \in \|\phi\|_M$. This shows
              that $\|\phi\|_M$ is downwards closed, and by $(*)$ we have
              $\|\phi\|_M \in P$.  Hence $M, x \models \E\phi$.

        \item $\S$: We show both directions by contraposition. Suppose $M, x
              \not\models \S\phi$. Then there is $A \in P$ such that
              $\|\phi\|_M \subseteq A$ and $x \notin A$. Since $A$ is downwards
              closed (by $(*)$), this means $xRy$ implies $y \notin A$ and
              hence $y \notin \|\phi\|_M$, for any $y \in X$. By the induction
              hypothesis, we get that $xRy$ implies $f(M), y \models \neg
              t(\phi)$, i.e.  $f(M), x \models \K\neg t(\phi)$. Hence $f(M), x
              \not\models t(\S\phi)$.

              Finally, suppose $f(M), x \not\models t(\S\phi)$, i.e.  $f(M), x
              \models \K\neg t(\phi)$. Let $A$ be the $R$-downwards closure of
              $\|\phi\|_M$, i.e.
              \[
                  A = \{y \in X \mid \exists z \in \|\phi\|_M: yRz\}
              \]
              Then $\|\phi\|_M \subseteq A$ by reflexivity of $R$, and $A$ is
              downwards closed by transitivity.  Hence $A \in P$.  But $x
              \notin A$, since for all $z$ with $xRz$ we have $f(M), z \models
              \neg t(\phi)$, so $z \notin \|t(\phi)\|_{f(M)} = \|\phi\|_M$.
              Hence $M, x \not\models \S\phi$.
    \end{itemize}
\end{proof}

Taken together, the results of this section show that, when considering
expertise models closed under intersections and unions, $P$ \emph{uniquely determines}
an epistemic accessibility relation such that expertise and soundness have
precise interpretations in terms of S4 knowledge. If we also impose closure
under complements, the notion of knowledge is strengthened to S5. Moreover,
every S4 and S5 model arises from some expertise model in this way.

These results also reflect back on the closure properties of expertise models.
For example, since S5 is such a strong notion of knowledge -- with the negative
introspection axiom \textbf{5} not generally considered plasuible for modelling
human knowledge, for instance -- we may conclude that closure of expertise
under complements is a similarly strong assumption. Put differently, we may use
the properties of knowledge corresponding to closure conditions to asses the
desirability of the closure conditions themselves, making use of the extensive
literature from the epistemic logic side in the process.

\section{Axiomatisation}
\label{exp_sec_axiomatisation}

In this section we give sound and complete logics with respect to various
classes of expertise models. We start with the class of all expertise
models $\M$, and show how adding more axioms captures the closure conditions of
\cref{exp_sec_closure_properties}.

\paragraph{The General Case.}

Let $\sL$ be the extension of propositional logic generated by the axioms and
inference rules shown in \cref{exp_tab_axioms_general_case}.\footnotemark{} Note that we treat
$\A$ as a ``box'' and $\S$ as a ``diamond'' modality. Some of the axioms were
already seen in \cref{exp_prop_validities}; new ones include ``replacement of
equivalents'' for expertise $\reE$, \textbf{4} for $\S$ $\foursoundness$,
and $\weakeningS$, which says that if $\psi$ is logically weaker than $\phi$
then the same holds for $\S\psi$ and $\S\phi$. First, $\sL$ is sound, i.e. all
formulas in $\sL$ are valid.

\footnotetext{
    Formally, $\sL$ is the smallest set of formulas which
    \begin{inlinelist}
        \item contains all substitution instances of propositional tautologies
              (this means we include tautologies involving modalities, e.g.
              $\E{p} \lor \neg\E{p}$);
        \item contains all formulas taking the form of the axioms shown in
              \cref{exp_tab_axioms_general_case}; and
          \item is closed under the inference rules shown in
              \cref{exp_tab_axioms_general_case}.
    \end{inlinelist}
}

\begin{table}
    \centering
    \caption{Axioms and inference rules for $\sL$.}
    \begin{tabular}{lr}
        \toprule
         $\E\phi \liff \A\E\phi$
             & $\EA$ \\
         $\A(\phi \liff \psi) \limplies (\E\phi \liff \E\psi)$
             & $\reE$ \\
         $\A(\phi \limplies \psi) \limplies (\S\phi \land \E\psi \limplies \psi)$
             & $\weakeningE$ \\
         \midrule
         $\phi \limplies \S\phi$
             & $\Tsoundness$ \\
         $\S\S\phi \limplies \S\phi$
             & $\foursoundness$ \\
         $\A(\phi \limplies \psi) \limplies (\S\phi \limplies \S\psi)$
             & $\weakeningS$ \\
         \midrule
         $\A(\phi \limplies \psi) \limplies (\A\phi \limplies \A\psi)$
             & $\Kuniv$ \\
         $\A\phi \limplies \phi$
             & $\Tuniv$ \\
         $\neg\A\phi \limplies \A\neg\A\phi$
             & $\fiveuniv$ \\
         \midrule
         From $\phi$ infer $\A\phi$
             & $\necuniv$ \\
         From $\phi \limplies \psi$ and $\phi$ infer $\psi$
             & $\modpon$ \\
        \bottomrule
    \end{tabular}
    \label{exp_tab_axioms_general_case}
\end{table}

\begin{theorem}
\label{exp_thm_soundness_m}
    $\sL$ is sound with respect to $\M$.
\end{theorem}

\begin{proof}
    The inference rules clearly preserve validity. All axioms were either shown to be
    valid in \cref{exp_prop_validities} or are straightforward to see, with the
    possible exception of $\foursoundness$ which we will show explicitly. Let
    $M = (X, P, V)$ be an expertise model and $x \in X$. Suppose $M, x \models
    \S\S\phi$. We need to show $M, x \models \S\phi$. Take $A \in P$ such that
    $\|\phi\|_M \subseteq A$. Now for any $y \in X$, if $M, y \models \S\phi$
    then clearly $y \in A$. Hence $\|\S\phi\|_M \subseteq A$. But then $M, x
    \models \S\S\phi$ gives $x \in A$. Hence $M, x \models \S\phi$.
\end{proof}

For completeness we use a variation of the standard canonical model
method~\cite[\sectionsymbol{4.2}]{blackburn2002modal}.
In taking this approach, one constructs a model whose states are maximally
$\sL$-consistent sets of formulas, and aims to prove the \emph{truth lemma}:
that a set $\Gamma$ satisfies $\phi$ in the canonical model if and only if
$\phi \in \Gamma$. However, the truth lemma poses some difficulties for our
semantics. Roughly speaking, we find there is an obvious choice of $P$ to
ensure the truth lemma for $\E\phi$ formulas, but that this may be too small
for $\S\phi$ to be refuted when $\S\phi \notin \Gamma$ (recall that $M, x
\not\models \S\phi$ iff \emph{there exists} some $A \in P$ such that
$\|\phi\|_M \subseteq A$ and $x \notin A$). We therefore ``enlargen'' the set of
states so we can add new expertise sets $A$ -- without affecting the truth
value of expertise formulas -- to obtain the truth lemma for soundness
formulas.

First, some standard notation and terminology. Write $\thm \phi$ iff
$\phi \in \sL$. For $\Gamma \subseteq \cL$ and $\phi \in \cL$, write $\Gamma
\thm \phi$ iff there are $\psi_0, \ldots, \psi_n \in \Gamma$, $n \ge 0$, such
that $\thm (\psi_0 \land \cdots \land \psi_n) \limplies \phi$.  Say $\Gamma$
is \emph{inconsistent} if $\Gamma \entails \bot$, and
\emph{consistent} otherwise. $\Gamma$ is \emph{maximally
consistent} iff $\Gamma$ is consistent and $\Gamma \subset \Delta$ implies that
$\Delta$ is inconsistent. We recall some standard facts about maximally
consistent sets.

\begin{lemma}
\label{exp_lemma_mcs_facts}
    Let $\Gamma$ be a maximally consistent set and $\phi, \psi \in \cL$. Then
    \begin{enumerate}
        \item\label{exp_item_mcs_mem_entail} $\phi \in \Gamma$ iff $\Gamma \entails
            \phi$

        \item\label{exp_item_mcs_modpon} If $\phi \limplies \psi \in \Gamma$ and
            $\phi \in \Gamma$, then $\psi \in \Gamma$

        \item\label{exp_item_mcs_negations} $\neg\phi \in \Gamma$ iff $\phi \notin
            \Gamma$

        \item\label{exp_item_mcs_conjunctions} $\phi \land \psi \in \Gamma$ iff
            $\phi \in \Gamma$ and $\psi \in \Gamma$
    \end{enumerate}
\end{lemma}

\begin{proof}\leavevmode
    \begin{enumerate}

        \item First suppose $\phi \in \Gamma$. Since $\phi \limplies \phi$ is
              an instance of the propositional tautology $p \limplies p$, we
              have $\thm \phi \limplies \phi$. Since $\phi \in \Gamma$, this
              gives $\Gamma \entails \phi$.

              Now suppose $\Gamma \entails \phi$. Set $\Delta = \Gamma \cup
              \{\phi\}$. We claim $\Delta$ is consistent. If not, there are
              $\psi_0,\ldots,\psi_n \in \Delta$ such that $\thm (\psi_0 \land
              \cdots \land \psi_n) \limplies \bot$. Since $\Gamma$ is
              consistent, at least one of the $\psi_i$ must be equal to $\phi$.
              Without loss of generality, $\psi_0 = \phi$ and $\psi_j \in
              \Gamma$ for $j > 0$. Hence, by propositional logic and $\modpon$,
              $\thm (\psi_1 \land \cdots \land \psi_n) \limplies \neg \phi$.
              Thus $\Gamma \entails \neg\phi$.  But since $\Gamma \entails
              \phi$ also, it follows that $\Gamma \entails \bot$, and thus
              $\Gamma$ is inconsistent: contradiction. So $\Delta$ must be
              consistent after all. Clearly $\Gamma \subseteq \Delta$, and by
              maximal consistency of $\Gamma$, $\Gamma \not\subset \Delta$.
              Hence $\Delta = \Gamma$, so $\phi \in \Gamma$ as required.

        \item By propositional logic we have $\thm ((\phi \limplies \psi) \land
              \phi) \limplies \psi$. Hence $\Gamma \entails \psi$; by
              \cref{exp_item_mcs_mem_entail} we get $\psi \in \Gamma$.

        \item If $\neg\phi \in \Gamma$ then clearly $\phi \notin \Gamma$, since
              otherwise $\Gamma$ would be inconsistent. If $\phi \notin \Gamma$
              then $\Gamma \not\entails \phi$ by \cref{exp_item_mcs_mem_entail}.
              Set $\Delta = \Gamma \cup \{\neg\phi\}$. Then $\Delta$ is
              consistent (one can show that assuming $\Delta$ is inconsistent
              leads to $\Gamma \entails \phi$; a contradiction). Again, since
              $\Gamma \subseteq \Delta$ and $\Gamma$ is maximally consistent,
              we must in fact have $\Gamma = \Delta$, so $\neg\phi \in \Gamma$.

        \item If $\phi \land \psi \in \Gamma$ then both $\Gamma \entails \phi$
              and $\Gamma \entails \psi$, so $\phi, \psi \in \Gamma$ by
              \cref{exp_item_mcs_mem_entail}.  Conversely, if $\phi, \psi \in
              \Gamma$ then $\Gamma \entails \phi \land \psi$, so $\phi \land
              \psi \in \Gamma$ by \cref{exp_item_mcs_mem_entail} again.

    \end{enumerate}
\end{proof}

\begin{lemma}[Lindenbaum's Lemma]
\label{exp_lemma_lindenbaum}
    If $\Gamma \subseteq \cL$ is consistent there is a maximally consistent set
    $\Delta$ such that $\Gamma \subseteq \Delta$.
\end{lemma}

Let $X_{\sL}$ denote the set of maximally consistent sets. Define a relation
$R$ by
\[
    \Gamma R \Delta
    \iff
    \forall \phi \in \cL:
     \A\phi \in \Gamma \implies \phi \in \Delta
\]
The $\Tuniv$ and $\fiveuniv$ axioms for $\A$ show that $R$ is an equivalence
relation; this is part of the standard proof that S5 is complete for
equivalence relations.

\begin{lemma}
\label{exp_lemma_r_equiv_reln}
    $R$ is an equivalence relation.
\end{lemma}

\begin{proof}

    We first show that $R$ is reflexive and has the \emph{Euclidean property}
    ($xRy$ and $xRz$ implies $yRz$). For reflexivity, let $\Gamma \in X_{\sL}$.
    Suppose $\A\phi \in \Gamma$. By $\Tuniv$ and closure of maximally
    consistent sets under modus ponens, $\phi \in \Gamma$. Hence $\Gamma R
    \Gamma$.

    For the Euclidean property, suppose $\Gamma R \Delta$ and $\Gamma R
    \Lambda$. We show $\Delta R \Lambda$ by contraposition. Suppose $\phi
    \notin \Lambda$.  Since $\Gamma R \Lambda$, this means $\A\phi \notin
    \Gamma$. Hence $\neg\A\phi \in \Gamma$, and by $\fiveuniv$ we get
    $\A\neg\A\phi \in \Gamma$. Now $\Gamma R \Delta$ gives $\neg\A\phi \in
    \Delta$, so $\A\phi \notin \Delta$.

    To conclude we need to show $R$ is symmetric and transitive.  For symmetry,
    suppose $\Gamma R \Delta$. By reflexivity, $\Gamma R \Gamma$. The Euclidean
    property therefore gives $\Delta R \Gamma$. For transitivity, suppose
    $\Gamma R \Delta$ and $\Delta R \Lambda$. By symmetry, $\Delta R \Gamma$.
    The Euclidean property again gives $\Gamma R \Lambda$.
\end{proof}

For $\phi \in \cL$, let $|\phi| = \{\Gamma \in X_{\sL} \mid \phi \in \Gamma\}$
be the \emph{proof set} of $\phi$. For $\Sigma \in X_{\sL}$, let $X_\Sigma$ be
the equivalence class of $\Sigma$ in $R$, and write $|\phi|_\Sigma = |\phi|
\cap X_\Sigma$.
%
Using what is essentially the standard proof of the truth lemma for the modal
logic \textbf{K} with respect to relational semantics, $\Kuniv$ yields the
following.

\begin{lemma}
\label{exp_lemma_xsigma_properties}
    Let $\Sigma \in X_{\sL}$. Then
    \begin{enumerate}
        \item\label{exp_item_xsigma_mem} For any $\phi \in \cL$, $\A\phi \in
            \Sigma$ iff $|\phi|_\Sigma = X_\Sigma$

        \item\label{exp_item_xsigma_imp} For any $\phi, \psi \in \cL$, $\A(\phi
            \limplies \psi) \in \Sigma$ iff $|\phi|_\Sigma \subseteq
            |\psi|_\Sigma$

        \item\label{exp_item_xsigma_iff} For any $\phi, \psi \in \cL$, $\A(\phi
            \liff \psi) \in \Sigma$ iff $|\phi|_\Sigma = |\psi|_\Sigma$
    \end{enumerate}
\end{lemma}

\begin{proof}\leavevmode
\begin{enumerate}
      \item For the left-to-right direction, suppose $\A\phi \in \Sigma$. Let
      $\Gamma \in X_\Sigma$. Then $\Sigma R \Gamma$, so clearly $\phi
      \in \Gamma$. Hence $|\phi|_\Sigma = X_\Sigma$.
      %
      For the other direction we show the contrapositive. Suppose
      $\A\phi \notin \Sigma$. Set
      \[
        \Gamma_0 = \{\psi \mid \A\psi \in \Gamma\} \cup \{\neg\phi\}.
      \]
      We claim $\Gamma_0$ is consistent. If not, without loss of
      generality there are $\psi_0, \ldots, \psi_n \in \Gamma_0$ such
      that $\A\psi_i \in \Sigma$ for each $i$, and
      $
        \thm \psi_0 \land \cdots \land \psi_n \limplies \phi.
      $
      By propositional logic, we get
      $
          \thm \psi_0 \limplies \cdots \limplies \psi_n \limplies
          \phi
      $
      (where the implication arrows associate to the right) and by
      $\necuniv$,
      $
          \thm \A(\psi_0 \limplies \cdots \limplies \psi_n \limplies
          \phi).
      $
      Since $\Kuniv$ together with $\modpon$ says that $\A$ distributes
      over implications, repeated applications gives
      $
        \thm \A\psi_0 \limplies \cdots \limplies \A\psi_n \limplies
        \A\phi
      $
      and propositional logic again gives
      $
          \thm \A\psi_0 \land \cdots \land \A\psi_n \limplies \A\phi.
      $
      But recall that $\A\psi_i \in \Sigma$. Hence $\Sigma \entails
      \A\phi$. Since $\Sigma$ is maximally consistent, this means
      $\A\phi \in \Sigma$: contradiction.

      So $\Gamma_0$ is consistent. By Lindenbaum's lemma
      (\cref{exp_lemma_lindenbaum}), there is a maximally consistent set
      $\Gamma \supseteq \Gamma_0$. Clearly $\Sigma R \Gamma$, since if
      $\A\psi \in \Sigma$ then $\psi \in \Gamma_0 \subseteq \Gamma$.
      Moreover, $\neg\phi \in \Gamma_0 \subseteq \Gamma$, so by
      consistency $\phi \notin \Gamma$. Hence $\Gamma \in X_\Sigma
      \setminus |\phi|_\Sigma$, and we are done.

    \item Note that by \cref{exp_item_xsigma_mem} we have
        \[
        \begin{aligned}
            \A(\phi \limplies \psi) \in \Sigma
            &\iff |\phi \limplies \psi|_\Sigma = X_\Sigma \\
            &\iff \forall \Gamma \in X_\Sigma: \phi \limplies \psi \in
                \Gamma
        \end{aligned}
        \]
        Suppose $\A(\phi \limplies \psi) \in \Sigma$. Take $\Gamma \in
        |\phi|_\Sigma$. Then we have $\phi, \phi \limplies \psi \in
        \Gamma$, so $\psi \in \Gamma$. This shows $|\phi|_\Sigma
        \subseteq |\psi|_\Sigma$.
        %
        Conversely, suppose $|\phi|_\Sigma \subseteq |\psi|_\Sigma$.
        Take $\Gamma \in X_\Sigma$. If $\phi \notin \Gamma$ then
        $\neg\phi \in \Gamma$, so $\neg\phi \lor \psi \in \Gamma$ and
        thus $\phi \limplies \psi \in \Gamma$. If $\phi \in \Gamma$
        then $\Gamma \in |\phi|_\Sigma \subseteq |\psi|_\Sigma$, so
        $\psi \in \Gamma$. Thus $\phi \limplies \psi \in \Gamma$ in
        this case too. Hence $\A(\phi \limplies \psi) \in \Sigma$.

   \item First note that $\A(\alpha \land \beta)
       \in \Sigma$ iff both $\A\alpha \in \Sigma$ and $\A\beta \in \Sigma$.
       This can be shown using $\Kuniv$, $\modpon$ and instances of the
       propositional tautologies $(p \land q) \limplies p$ (for the
       left-to-right implication) and $p \limplies q \limplies (p \land q))$
       (for the right-to-left implication).
        %
        Recalling that $\phi \liff \psi$ is an abbreviation for $(\phi
        \limplies \psi) \land (\psi \limplies \phi)$, we get
        \[
        \begin{aligned}
            \A(\phi \liff \psi) \in \Sigma
            &\iff \A(\phi \limplies \psi) \in \Sigma \text{ and }
                \A(\psi \limplies \phi) \in \Sigma \\
            &\iff |\phi|_\Sigma \subseteq |\psi|_\Sigma \text { and }
                |\psi|_\Sigma \subseteq |\phi|_\Sigma \\
            & \iff |\phi|_\Sigma = |\psi|_\Sigma
        \end{aligned}
        \]
        as required.
\end{enumerate}
\end{proof}

\begin{corollary}
\label{exp_cor_xsigma_agree_on_ae}
    Let $\Sigma \in X_{\sL}$. For $\Gamma, \Delta \in X_\Sigma$ and $\phi \in
    \cL$, $\A\phi \in \Gamma$ iff $\A\phi \in \Delta$ and $\E\phi \in \Gamma$
    iff $\E\phi \in \Delta$.
\end{corollary}

\begin{proof}
    For the first point, note that if $\A\phi \in \Gamma$ then
    \cref{exp_lemma_xsigma_properties} gives $|\phi|_\Gamma = X_\Gamma$. But since
    $\Gamma$ and $\Delta$ are in the same equivalence class of $R$,
    $|\phi|_\Gamma = |\phi|_\Delta$ and $X_\Gamma = X_\Delta$.  Hence
    $|\phi|_\Delta = X_\Delta$, so $\A\phi \in \Delta$ by
    \cref{exp_lemma_xsigma_properties}. The converse holds by symmetry.

    For the second point, if $\E\phi \in \Gamma$ then $\A\E\phi \in \Gamma$ by
    $\EA$. Since $\Gamma R \Delta$, we get $\E\phi \in \Delta$. Again, the
    converse holds by symmetry.
\end{proof}

We are ready to define the ``canonical'' model (for each $\Sigma$). Set
$\widehat{X}_\Sigma = X_\Sigma \times \R$. This is the step described
informally above: we enlargen $X_\Sigma$ by considering uncountably many copies
of each point (any uncountable set would do in place of $\R$). The valuation is
straightforward: set $\widehat{V}_\Sigma(p) = |p|_\Sigma \times \R$. For the
expertise component of the model, say $A \subseteq \widehat{X}_\Sigma$ is
\emph{S-closed} iff for all $\phi \in \cL$:
\[
    |\phi|_\Sigma \times \R \subseteq A
    \implies |\S\phi|_\Sigma \times \R \subseteq A.
\]
Set $\widehat{P}_\Sigma = \widehat{P}_\Sigma^0 \cup \widehat{P}_\Sigma^1$,
where
\[
\begin{aligned}
 \widehat{P}_\Sigma^0 &= \{|\phi|_\Sigma \times \R \mid \E\phi \in \Sigma\}, \\
 \widehat{P}_\Sigma^1 &= \{
     A \subseteq \widehat{X}_\Sigma
     \mid
     A \text{ is S-closed and }
     \forall \phi \in \cL: A \ne |\phi|_\Sigma \times \R
 \}.
\end{aligned}
\]
We have a version of the truth lemma for the model $\widehat{M}_\Sigma =
(\widehat{X}_\Sigma, \widehat{P}_\Sigma, \widehat{V}_\Sigma)$.

\begin{lemma}
\label{exp_lemma_truth_lemma}
    For any $(\Gamma, t) \in \widehat{X}_\Sigma$ and $\phi \in \cL$,
    \[
        \widehat{M}_\Sigma, (\Gamma, t) \models \phi
        \iff
        \phi \in \Gamma,
    \]
    i.e. $\|\phi\|_{\widehat{M}_\Sigma} = |\phi|_\Sigma \times \R$.
\end{lemma}

\begin{proof}
    By induction. The cases for atomic propositions and
    the propositional connectives are straightforward by the definition of
    $\widehat{V}_\Sigma$ and properties of
    maximally consistent sets. The case for the universal modality $\A$ is also
    straightforward by \cref{exp_lemma_xsigma_properties,exp_cor_xsigma_agree_on_ae}.
    We treat the cases of $\E$ and $\S$
    formulas.

    \begin{itemize}
        \item $\E$: First suppose $\E\phi \in \Gamma$. By
    \cref{exp_cor_xsigma_agree_on_ae}, $\E\phi \in \Sigma$. Hence $|\phi|_\Sigma
    \times \R \in \widehat{P}_\Sigma^0$. By the induction hypothesis,
    $\|\phi\|_{\widehat{M}_\Sigma} \in \widehat{P}_\Sigma^0$. Hence
    $\widehat{M}_\Sigma, (\Gamma, t) \models \E\phi$.

    Now suppose $\widehat{M}_\Sigma, (\Gamma, t) \models \E\phi$. Then
    $\|\phi\|_{\widehat{M}_\Sigma} \in \widehat{P}_\Sigma$. By the induction
    hypothesis, $\|\phi\|_{\widehat{M}_\Sigma} = |\phi|_\Sigma \times \R$.
    Hence $|\phi|_\Sigma \times \R \in \widehat{P}_\Sigma$.
    Since $\widehat{P}_\Sigma^1$ does not contain any sets of this form, we
    must have $|\phi|_\Sigma \times \R \in \widehat{P}_\Sigma^0$. Therefore
    there is some $\psi$ such that $\E\psi \in \Sigma$ and $|\phi|_\Sigma
    \times \R = |\psi|_\Sigma \times \R$.  It follows that $|\phi|_\Sigma =
    |\psi|_\Sigma$, and \cref{exp_lemma_xsigma_properties} then gives $\A(\phi
    \liff \psi) \in \Sigma$. By \cref{exp_cor_xsigma_agree_on_ae}, we have $\E\psi
    \in \Gamma$ and $\A(\phi \liff \psi) \in \Gamma$ too. By $\reE$ we get
    $\E\phi \in \Gamma$ as required.

    \item $\S$: First suppose $\S\phi \in \Gamma$. Take $A \in \widehat{P}_\Sigma$
    such that $\|\phi\|_{\widehat{M}_\Sigma} \subseteq A$. By the induction
    hypothesis, $|\phi|_\Sigma \times \R \subseteq A$. There are two cases:
    either $A \in \widehat{P}_\Sigma^0$ or $A \in \widehat{P}_\Sigma^1$.

    If $A \in \widehat{P}_\Sigma^0$, there is $\psi$ such that $A =
    |\psi|_\Sigma \times \R$ and $\E\psi \in \Sigma$.  Since $|\phi|_\Sigma
    \times \R \subseteq A$, we have $|\phi|_\Sigma \subseteq |\psi|_\Sigma$.
    By \cref{exp_lemma_xsigma_properties}, $\A(\phi \limplies \psi) \in \Sigma$. By
    \cref{exp_cor_xsigma_agree_on_ae} we have $\E\psi, \A(\phi \limplies \psi) \in
    \Gamma$ too. Applying $\weakeningE$ gives $\S\phi \land \E\psi \limplies
    \psi \in \Gamma$; since $\S\phi, \E\psi \in \Gamma$ we have $\S\phi \land
    \E\psi \in \Gamma$ and thus $\psi \in \Gamma$. This means $(\Gamma, t) \in
    |\psi|_\Sigma \times \R = A$, as required.

    If $A \in \widehat{P}_\Sigma^1$, $A$ is S-closed by definition.  Hence
    $|\S\phi|_\Sigma \times \R \subseteq A$. Since $\S\phi \in \Gamma$ we get
    $(\Gamma, t) \in A$ as required.
    %
    In either case we have $(\Gamma, t) \in A$. This shows $\widehat{M}_\Sigma,
    (\Gamma, t) \models \S\phi$.

    For the other direction we show the contrapositive. Take any $(\Gamma, t)
    \in \widehat{X}_\Sigma$ and suppose $\S\phi \notin \Gamma$. We show that
    $\widehat{M}_\Sigma, (\Gamma, t) \not\models \S\phi$, i.e. there is $A \in
    \widehat{P}_\Sigma$ such that $\|\phi\|_{\widehat{M}_\Sigma} \subseteq A$
    but $(\Gamma, t) \notin A$.
    %
    First, set
    \[
        \mathcal{U} = \{
        |\psi|_\Sigma \times \R
        \mid
        \psi \in \cL \text{ and }
        |\psi|_\Sigma \times \R \not\subseteq |\S\phi|_\Sigma \times \R
        \}.
    \]
    Since $\cL$ is countable, $\mathcal{U}$ is at most countable.
    Hence we may write $\mathcal{U} = \{U_n\}_{n \in N}$ for some
    index set $N \subseteq \N$. Since $U_n \not\subseteq
    |\S\phi|_\Sigma \times \R$, we may choose some $(\Delta_n, t_n)
    \in U_n \setminus (|\S\phi|_\Sigma \times \R)$ for each $n$. Now
    write
    \[
      \mathcal{D} = \{(\Delta_n, t_n)\}_{n \in N} \cup \{(\Gamma,
      t)\}.
    \]
    Since $N$ is at most countable, so is $\mathcal{D}$.  Since
    $\R$ is uncountable, there is some $s \in \R$ such that $(\Gamma,
    s) \notin \mathcal{D}$.\footnote{If not, then $s \mapsto
    (\Gamma, s)$ is an injective mapping $\R \to \mathcal{D}$, which
    would imply $\R$ is countable.} We necessarily have $s \ne t$. We
    are ready to define $A$: set
    \[
      A = (|\S\phi|_\Sigma \times \R) \cup \{(\Gamma, s)\}.
    \]
    Note that $(\Gamma, t) \notin A$ since $\S\phi \notin \Gamma$ and
    $s \ne t$.
    %
    Next we show $\|\phi\|_{\widehat{M}_\Sigma} \subseteq A$. By the
    induction hypothesis, this is equivalent to $|\phi|_\Sigma \times
    \R \subseteq A$. By $\Tsoundness$ and $\necuniv$, we have
    $\A(\phi \limplies \S\phi) \in \Sigma$, and consequently
    $|\phi|_\Sigma \subseteq |\S\phi|_\Sigma$ by
    \cref{exp_lemma_xsigma_properties}. Hence $|\phi|_\Sigma \times \R
    \subseteq |\S\phi|_\Sigma \times \R \subseteq A$ as required.

    It only remains to show that $A \in \widehat{P}_\Sigma$. We claim
    that $A \in \widehat{P}_\Sigma^1$. First, $A$ is S-closed.
    Indeed, suppose $|\psi|_\Sigma \times \R \subseteq A$. We claim
    that, in fact, $|\psi|_\Sigma \times \R \subseteq |\S\phi|_\Sigma
    \times \R$. If not, then by definition of $\mathcal{U}$ there is
    $n \in N$ such that $|\psi|_\Sigma \times \R = U_n$.  Hence $U_n
    \subseteq A$. This means $(\Delta_n, t_n) \in A$. But
    $(\Delta_n, t_n) \notin |\S\phi|_\Sigma \times \R$, so we must
    have $(\Delta_n, t_n) = (\Gamma, s)$. But then $(\Gamma, s) \in
    \mathcal{D}$: contradiction. So we do indeed have $|\psi|_\Sigma
    \times \R \subseteq |\S\phi|_\Sigma \times \R$, and thus
    $|\psi|_\Sigma \subseteq |\S\phi|_\Sigma$. By
    \cref{exp_lemma_xsigma_properties}, $\A(\psi \limplies \S\phi) \in
    \Sigma$.

    Now, take any $(\Lambda, u) \in |\S\psi|_\Sigma \times \R$.
    Since $\Lambda \in X_\Sigma$, \cref{exp_cor_xsigma_agree_on_ae} gives
    $\A(\psi \limplies \S\phi) \in \Lambda$. By $\weakeningS$,
    $\S\psi \limplies \S\S\phi \in \Lambda$. Since $\Lambda \in
    |\S\psi|_\Sigma$, we get $\S\S\phi \in \Lambda$. But then
    $\foursoundness$ gives $\S\phi \in \Lambda$. That is, $(\Lambda,
    u) \in |\S\phi|_\Sigma \times \R \subseteq A$. This shows
    $|\S\psi|_\Sigma \times \R \subseteq A$, so $A$ is
    S-closed.

    Finally, we show that for all $\psi \in \cL$, $A \ne
    |\psi|_\Sigma \times \R$.  For contradiction, suppose there is
    $\psi$ with $A = |\psi|_\Sigma \times \R$. Then since $(\Gamma,
    s) \in A$, we have $\Gamma \in |\psi|_\Sigma$. But then $(\Gamma,
    t) \in |\psi|_\Sigma \times \R = A$: contradiction.

    This completes the proof that $A \in \widehat{P}_\Sigma^1$.  Thus
    $\widehat{M}_\Sigma, (\Gamma, t) \not\models \S\phi$, and we are done.
    \end{itemize}
\end{proof}

We are finally in a position to show completness. In fact, we have \emph{strong
completeness}: for all sets $\Gamma \subseteq \cL$ and $\phi \in \cL$, if
$\Gamma \models \phi$ then $\Gamma \entails \phi$.

\begin{theorem}
\label{exp_thm_strong_completeness}
    $\sL$ is strongly complete with respect to $\M$.
\end{theorem}

\begin{proof}
We show the contrapositive. Suppose $\Gamma \not\entails \phi$.
Then $\Gamma \cup \{\neg\phi\}$ is consistent. By Lindenbaum's
Lemma, there is a maximally consistent set $\Sigma \supseteq \Gamma
\cup \{\neg\phi\}$. Consider the model $\widehat{M}_\Sigma$. For
any $\psi \in \Gamma$ we have $\psi \in \Sigma$, so
\cref{exp_lemma_truth_lemma} (with $t = 0$, say) gives
$\widehat{M}_\Sigma, (\Sigma, 0) \models \psi$. Also,
$\neg\phi \in \Gamma \subseteq \Sigma$ gives
$\widehat{M}_\Sigma, (\Sigma, 0) \models \neg\phi$, so
$\widehat{M}_\Sigma, (\Sigma, 0) \not\models \phi$. This shows that
$\Gamma \not\models \phi$, and we are done.
\end{proof}

From soundness, which was shown in \cref{exp_thm_soundness_m}, one can easily
show that $\Gamma \entails \phi$ implies $\Gamma \models \phi$. Together with
strong completeness, we get $\Gamma \models \phi$ if and only if $\Gamma
\entails \phi$. That is, semantic entailment with respect to the class of all
models $\M$ is exactly the same notion as syntactic entailment using the axioms
and inference rules of $\sL$.

\paragraph{Extensions of the Base Logic.}

We now extend $\sL$ to obtain axiomatisations of sub-classes of $\M$
corresponding to closure conditions.

To start, consider closure under intersections. It was shown in
\cref{exp_prop_frame_conditions} that the formula $\A(\S\phi \limplies
\phi) \limplies \E\phi$ characterises frames closed under intersections. It is
perhaps no surprise that adding this as an axiom results in a sound and
complete axiomatisation of $\Mint$. Formally, let $\sLint$ be the
extension of $\sL$ with the following axiom
\[
    \A(\S\phi \limplies \phi) \limplies \E\phi \quad \redE,
\]
so-named since together with $\E\phi \limplies \A(\S\phi \limplies \phi)$
-- which is derivable in $\sL$ -- it allows expertise to be reduced
to soundness. That is, expertise on $\phi$ is equivalent to the statement that,
in all situations, $\phi$ is only true up to lack of expertise if it is in fact
true.

\begin{theorem}
\label{exp_thm_mint_axiomatisation}
$\sLint$ is sound and strongly complete with respect to
$\Mint$.
\end{theorem}

\begin{proof}
    For soundness, we only need to check that $\redE$ is sound for $\Mint$. But
    this follows from \cref{exp_prop_frame_conditions}
    \cref{exp_item_frame_condition_intersections}.

    For completeness, we adopt a roughly
    similar approach to the general case. Let consistency, maximal consistency and
    other standard notions and notation be defined as before, but now for $\sLint$
    instead of $\sL$. Let $X_{\sLint}$ be the set of maximally $\sLint$-consistent
    sets. Define the relation $R$ on $X_{\sLint}$ in exactly the same way. Since
    $\sLint$ extends $\sL$, $R$ is again an equivalence relation, and we have the
    analogues of \cref{exp_lemma_xsigma_properties,exp_cor_xsigma_agree_on_ae}.

    This time, however, the construction of the canonical model for a given
    $\Sigma \in X_{\sLint}$ is much more straightforward. The set of
    states is simply $X_\Sigma$, i.e. the equivalence class of
    $\Sigma$ in $R$. Overriding earlier terminology, say $A
    \subseteq X_\Sigma$ is \emph{S-closed} iff $|\phi|_\Sigma \subseteq A$
    implies $|\S\phi|_\Sigma \subseteq A$ for all $\phi \in \cL$.
    Then set
    \[
        P_\Sigma = \{A \subseteq X_\Sigma \mid A \text{ is S-closed}\}.
    \]
    Finally, set $V_\Sigma(p) = |p|_\Sigma$, and write $M_\Sigma
    = (X_\Sigma, P_\Sigma, V_\Sigma)$.

    First, we have $M_\Sigma \in \Mint$, i.e. intersections of S-closed sets are
    S-closed.  Indeed, suppose $\{A_i\}_{i \in I}$ is a collection of S-closed
    sets, and suppose $|\phi|_\Sigma \subseteq \bigcap_{i \in I}{A_i}$. Then
    $|\phi|_\Sigma \subseteq A_i$ for each $i$, so S-closure gives $|\S\phi|_\Sigma
    \subseteq A_i$.  Hence $|\S\phi|_\Sigma \subseteq \bigcap_{i \in I}{A_i}$.

    Importantly, we have the truth lemma for $M_\Sigma$: for all $\Gamma \in
        X_\Sigma$ and $\phi \in \cL$,
    \[
        M_\Sigma, \Gamma \models \phi \iff \phi \in \Gamma,
    \]
    i.e. $\|\phi\|_{M_\Sigma} = |\phi|_\Sigma$.

    As usual, the proof is by induction on formulas. The case for atomic
    propositions follows from the definition of $V_\Sigma$, the cases for
    conjunctions and negations hold by properties of maximally consistent sets,
    and the case for $\A\phi$ holds by an argument identical to the one used in
    the general case (\cref{exp_lemma_truth_lemma}). The only interesting cases are
    therefore for $\E\phi$ and $\S\phi$ formulas:

    \begin{itemize}
    \item $\E$: First suppose $\E\phi \in \Gamma$. We claim $|\phi|_\Sigma$ is
    S-closed. This will give $\|\phi\|_{M_\Sigma} \in P_\Sigma$ by the
    induction hypothesis and definition of $P_\Sigma$, and therefore $M_\Sigma,
    \Gamma \models \E\phi$.

    So, suppose $|\psi|_\Sigma \subseteq |\phi|_\Sigma$. Then
    $\A(\psi \limplies \phi) \in \Sigma$. Let $\Delta \in
    |\S\psi|_\Sigma$. Since $\Delta, \Gamma, \Sigma \in X_\Sigma$,
    we have $\E\phi \in \Delta$ and $\A(\psi \limplies \phi) \in
    \Delta$ too. By $\weakeningE$, $\S\psi \land \E\phi \limplies
    \phi \in \Delta$. But $\S\psi \in \Delta$, so $\S\psi \land
    \E\phi \in \Delta$ and thus $\phi \in \Delta$, i.e.  $\Delta
    \in |\phi|_\Sigma$. This shows $|\S\psi|_\Sigma \subseteq
    |\phi|_\Sigma$, so $|\phi|_\Sigma$ is S-closed as required.

    Now suppose $M_\Sigma, \Gamma \models \E\phi$. Then, by the
    induction hypothesis, $|\phi|_\Sigma$ is S-closed. Since
    $|\phi|_\Sigma \subseteq |\phi|_\Sigma$ clearly holds, we get
    $|\S\phi|_\Sigma \subseteq |\phi|_\Sigma$. This implies
    $\A(\S\phi \limplies \phi) \in \Sigma$, and $\redE$ gives
    $\E\phi \in \Sigma$. Since $\Gamma \in X_\Sigma$, we get
    $\E\phi \in \Gamma$ as required.

    \item $\S$: Suppose $\S\phi \in \Gamma$. Take any $A \in P_\Sigma$
    such that $\|\phi\|_{M_\Sigma} \subseteq A$. By the induction
    hypothesis, $|\phi|_\Sigma \subseteq A$. By S-closure of $A$,
    $|\S\phi|_\Sigma \subseteq A$. Hence $\Gamma \in
    |\S\phi|_\Sigma \subseteq A$. This shows $M_\Sigma, \Gamma
    \models \S\phi$.

    For the other direction we show the contrapositive. Suppose
    $\S\phi \notin \Gamma$. First, we claim $|\S\phi|_\Sigma$ is
    S-closed. Indeed, suppose $|\psi|_\Sigma \subseteq
    |\S\phi|_\Sigma$. Then $\A(\psi \limplies \S\phi) \in \Sigma$.
    Take any $\Delta \in |\S\psi|_\Sigma$. Since $\Delta \in
    X_\Sigma$, $\A(\psi \limplies \S\phi) \in \Delta$ also. By
    $\weakeningS$, $\S\psi \limplies \S\S\phi \in \Delta$.  Now
    $\S\psi \in \Delta$ implies $\S\S\phi \in \Delta$, and
    $\foursoundness$ gives $\S\phi \in \Delta$, i.e.  $\Delta \in
    |\S\phi|_\Sigma$. This shows $|\S\psi|_\Sigma \subseteq
    |\S\phi|_\Sigma$, and thus $|\S\phi|_\Sigma$ is S-closed.

    Hence $|\S\phi|_\Sigma$ is a set in $P_\Sigma$ not containing
    $\Gamma$. Moreover, $\|\phi\|_{M_\Sigma} \subseteq
    |\S\phi|_\Sigma$ by the induction hypothesis and $\Tsoundness$.
    Hence $M_\Sigma, \Gamma \not\models \S\phi$.
    \end{itemize}

    Strong completeness now follows. If $\Gamma \not\entails_{\sLint} \phi$, then
    $\Gamma \cup \{\neg\phi\}$ is consistent, so by Lindenbaum's Lemma
    there is $\Sigma \in X_{\sLint}$ with $\Sigma \supseteq \Gamma \cup
    \{\neg\phi\}$.  Considering the model $M_\Sigma \in \Mint$, we have
    $M_\Sigma, \Sigma \models \Gamma$ and $M_\Sigma, \Sigma \not\models
    \phi$ by the truth lemma. Hence $\Gamma \not\models_{\Mint} \phi$.
\end{proof}

Now we add finite unions to the mix. It was shown in
\cref{exp_prop_finite_unions_frame_condition} that within class $\Mint$, the
\textbf{K} axiom for the dual operator $\hat\S\phi = \neg\S\neg\phi$ characterises closure under
finite unions. Note that any frame $(X, P)$ closed under intersections and
finite unions is a topological space,\footnote{By the convention that the empty
intersection is the whole space $X$ and the empty union is $\emptyset$, we have
$X, \emptyset \in P$ too.} where $P$ is the set of \emph{closed} sets. Write
$\Mtop = \Mint \cap \Mfunions$ for the class of models over such frames. We
obtain an axiomatisation of $\Mtop$ by adding \textbf{K} for $\hat\S$ and a
bridge axiom linking $\hat\S$ and $\A$:
\[
    \begin{array}{lr}
        \hat{\S}(\phi \limplies \psi) \limplies (\hat{\S}\phi \limplies \hat{\S}\psi)
            & \Ksoundness \\
        \A\phi \limplies \hat\S\phi
    & \inc
    \end{array}
\]
Let $\sLtop$ be the extension
of $\sLint$ by $\Ksoundness$ and $\inc$. Note that $\sLtop$ contains the
\textbf{KT4} axioms for $\hat\S$ (recalling that $\Tsoundness$ and
$\foursoundness$ are the ``diamond'' versions of \textbf{T} and \textbf{4}).
Since \textbf{KT4} together with the bridge axiom $\inc$ is complete for the
class of relational models $\Msfour$, we can exploit
\cref{exp_thm_s4s5_translation} to obtain completeness of $\sLtop$ with respect to
$\Mint \cap \Munions$. Since this class is included in $\Mtop$, we also get
completeness with respect to $\Mtop$.\footnote{Note that \textbf{KT4} is also
complete for topological spaces with respect to the interior
semantics~\cite{van2007modal}.}

\begin{theorem}
\label{exp_thm_mtop_axiomatisation}
    $\sLtop$ is sound and strongly complete with respect to $\Mtop$.
\end{theorem}

\begin{proof}
    Soundness of $\Ksoundness$ for $\Mtop$ follows from
    \cref{exp_prop_finite_unions_frame_condition}. For $\inc$, suppose $M = (X, P,
    V) \in \Mtop$, $x \in X$ and $M, x \models \A\phi$. Then $\|\phi\|_M = X$,
    so $\|\neg\phi\|_M = \emptyset$. By the convention that the empty set is
    the empty union $\bigcup\emptyset$ (which is a finite union), we have
    $\emptyset \in P$. Taking $A = \emptyset$ in the definition of the
    semantics for $\S$, we have $\|\neg\phi\|_M \subseteq A$ but clearly $x
    \notin A$. Hence $M, x \not\models \S\neg\phi$, so $M, x \models
    \hat\S\phi$.


For completeness, we go via relational
semantics using the translation $t: \cL \to \cLKA$ and
\cref{exp_thm_s4s5_translation}. First, let $\sLsfoura$ be the logic of $\cLKA$
formulas formed by the axioms and inference rules shown in
\cref{exp_tab_axioms_sfoura}. It is well known that $\sLsfoura$ is strongly
complete with respect to $\Msfour$~\cite[Theorem 7.2]{blackburn2002modal}.

\begin{table}[h]
    \centering
    \caption{Axioms and inference rules for $\sLsfoura$.}
    \begin{tabular}{lr}
        \toprule
       $\K(\phi \limplies \psi) \limplies (\K\phi \limplies \K\psi)$
           & $\Kk$ \\
       $\K\phi \limplies \phi$
           & $\Tk$ \\
       $\K\phi \limplies \K\K\phi$
           & $\fourk$ \\
       \midrule
       $\A(\phi \limplies \psi) \limplies (\A\phi \limplies \A\psi)$
           & $\Kuniv$ \\
       $\A\phi \limplies \phi$
           & $\Tuniv$ \\
       $\neg\A\phi \limplies \A\neg\A\phi$
           & $\fiveuniv$ \\
       \midrule
       $\A\phi \limplies \K\phi$
           & $\inck$ \\
       \midrule
       From $\phi$ infer $\A\phi$
           & $\necuniv$ \\
       From $\phi \limplies \psi$ and $\phi$ infer $\psi$
           & $\modpon$ \\
        \bottomrule
    \end{tabular}
    \label{exp_tab_axioms_sfoura}
\end{table}

Now, define a translation $u: \cLKA \to \cL$ as follows:
\[
\begin{array}{lll}
 &u(p) &= p \\
 &u(\phi \land \psi) &= u(\phi) \land u(\psi) \\
 &u(\neg\phi) &= \neg u(\phi) \\
 &u(\K\phi) &= \neg\S\neg u(\phi) \\
 &u(\A\phi) &= \A u(\phi).
\end{array}\]
Recall the translation $t: \cL \to \cLKA$ from
\cref{exp_sec_connection_with_ep_logic}. While $u$ is not the inverse of $t$ (for
instance, there is no $\psi \in \cLKA$ with $u(\psi) = \E p$), for any $\phi
\in \cL$ we have that $\phi$ is $\sLtop$-provably equivalent to $u(t(\phi))$.

\begin{claim}
\label{exp_claim_ut_equivalence}
Let $\phi \in \cL$. Then $\entails_{\sLtop}
\phi \liff u(t(\phi))$.
\end{claim}

\begin{claimproof}

By induction on $\cL$ formulas. The cases of atomic
propositions and propositional connectives are straightforward. For
the other cases, first note that the ``replacement of equivalents''
rule is derivable in $\sL$ (and thus in $\sLtop$) for
$\S$, $\E$ and $\A$:
\[
    \text{From } \phi \liff \psi \text{ infer } \bigcirc\phi \liff
    \bigcirc\psi \quad (\bigcirc \in \{\S, \E, \A\}).
\]
For $\S$ this follows from $\necuniv$ and $\weakeningS$; for $\E$ from
$\necuniv$ and $\reE$, and for $\A$ from $\necuniv$ and $\Kuniv$. Now for the
inductive step, suppose $\entails_{\sLtop} \phi \liff u(t(\phi))$.

\begin{itemize}
    \item $\S$: Note that
        \[
        u(t(\S\phi))
= u(\neg\K\neg t(\phi))
= \neg\neg\S\neg\neg u(t(\phi)).\]
        By the inductive hypothesis, propositional logic and replacement
of equivalents, $\entails_{\sLtop} \S\phi \liff
u(t(\S\phi))$.

        \item $\E$: We have
        \[
        \begin{aligned}
   u(t(\E\phi))
   &= u(\A(\neg t(\phi) \limplies \K\neg t(\phi))) \\
   &= \A u(\neg t(\phi) \limplies \K\neg t(\phi)) \\
   &= \A (u(\neg t(\phi)) \limplies u(\K\neg t(\phi))) \\
   &= \A (\neg u(t(\phi)) \limplies \neg\S\neg u(\neg t(\phi))) \\
   &= \A (\neg u(t(\phi)) \limplies \neg\S\neg \neg u(t(\phi))).
\end{aligned}\]
        Taking the contrapositive of the implication, and using
replacement of equivalents together with the inductive hypothesis,
we get
        \[
        \entails_{\sLtop} u(t(\E\phi))
    \liff
    \A(
       \S\phi \limplies \phi
    ).\]
        But we have already seen that $\entails_{\sLint} \E\phi
\liff \A(\S\phi \limplies \phi)$; since $\sLtop$ extends
$\sLint$, we get $\entails_{\sLtop} \E\phi \liff
u(t(\E\phi))$.

        \item $\A$: This case is straightforward by the inductive
hypothesis and replacement of equivalents, since
$u(t(\A\phi)) = \A u(t(\phi)$.

        \end{itemize}
\end{claimproof}

Next we show that if $\phi \in \cLKA$ is a theorem of
$\sLsfoura$, then $u(\phi)$ is a theorem of $\sLtop$.

\begin{claim}
\label{exp_claim_u_thm}

Let $\phi \in \cLKA$. Then
$\entails_{\sLsfoura} \phi$ implies $\entails_{\sLtop}
u(\phi)$.

\end{claim}
    \begin{claimproof}
    By induction on the length of $\sLsfoura$ proofs. The base
case consists of showing that if $\phi$ is an instance of an
$\sLsfoura$ axiom or a substitution instance of a
propositional tautology, then $\entails_{\sLtop} u(\phi)$.
The case for instances of tautologies is straightforward, since
$u$ does not affect the structure of a propositional formula.
We take the axioms of $\sLsfoura$ in turn.

    \begin{itemize}
    \item $\Kk$: We have
        \[
        \begin{aligned}
   &u(\K(\phi \limplies \psi) \limplies (\K\phi \limplies \K\psi))
    \\
   &\qquad
   = \neg\S\neg (u(\phi) \limplies u(\psi))
        \limplies (\neg\S\neg u(\phi) \limplies \neg\S\neg u(\psi))
       \\
   &\qquad
   = \hat\S (u(\phi) \limplies u(\psi)) \limplies
       (\hat\S u(\phi) \limplies \hat\S u(\psi))
\end{aligned}\]
        which is an instance of $\Ksoundness$.

        \item $\Tk$: We have
        \[
        u(\K\phi \limplies \phi)
= \neg\S\neg u(\phi) \limplies u(\phi)\]
        Taking the contrapositive, this is $\sLtop$-provably
equivalent to $\neg u(\phi) \limplies \S\neg u(\phi)$, which
is an instance of $\Tsoundness$.

        \item $\fourk$: We have
        \[
        u(\K\phi \limplies \K\K\phi)
= \neg\S\neg u(\phi) \limplies \neg\S\neg\neg\S\neg u(\phi)\]
        This is provably equivalent to $\S\S\neg u(\phi) \limplies
\S\neg u(\phi)$, which is an instance of $\foursoundness$.

        \item $\Kuniv$: We have
        \[
        u(\A(\phi \limplies \psi) \limplies (\A\phi \limplies \A\psi))
=
\A (u(\phi) \limplies u(\psi)) \limplies (\A u(\phi) \limplies \A u(\psi))\]
        which is an instance of $\Kuniv$ in $\sLtop$.

        \item $\Tuniv$: We have
        \[
        u(\A\phi \limplies \phi)
=
\A u(\phi) \limplies u(\phi)\]
        which is an instance of $\Tuniv$ in $\sLtop$.

        \item $\fiveuniv$: We have
        \[
        u(\neg\A\phi \limplies \A\neg\A\phi)
=
\neg\A u(\phi) \limplies \A \neg\A u(\phi)\]
        which is an instance of $\fiveuniv$ in $\sLtop$.

        \item $\inck$: We have
        \[
        u(\A\phi \limplies \K\phi)
=
\A u(\phi) \limplies \neg\S\neg u(\phi)
=
\A u(\phi) \limplies \hat\S u(\phi)\]
        which is an instance of $\inc$.

        \end{itemize}
    For the inductive step, we show that for each inference rule
$\frac{\psi_1,\ldots,\psi_n}{\phi}$, if
$\entails_{\sLtop}u(\psi_i)$ for each $i$ then
$\entails_{\sLtop}u(\phi)$.

    \begin{itemize}
    \item $\necuniv$: If $\entails_{\sLtop} u(\phi)$, then from
$\necuniv$ in $\sLtop$ we get
$\entails_{\sLtop} \A u(\phi)$. But $\A u(\phi) =
u(\A\phi)$, so we are done.

        \item $\modpon$: Similarly, this clear from $\modpon$ for
$\sLtop$ and the fact that $u(\phi \limplies \psi) =
u(\phi) \limplies u(\psi)$.
        \end{itemize}
    \end{claimproof}

\cref{exp_claim_ut_equivalence,exp_claim_u_thm} easily imply the following.

\begin{claim}
\label{exp_claim_t_thm}

Let $\phi \in \cL$. Then
$\entails_{\sLsfoura} t(\phi)$ implies $\entails_{\sLtop}
\phi$.

\end{claim}

    \begin{claimproof}
    Suppose $\entails_{\sLsfoura} t(\phi)$. By \cref{exp_claim_u_thm},
$\entails_{\sLtop} u(t(\phi))$. By \cref{exp_claim_ut_equivalence},
$\entails_{\sLtop} \phi \liff u(t(\phi))$. By
$\modpon$, $\entails_{\sLtop} \phi$.
    \end{claimproof}

We can now show strong completeness. Suppose $\Gamma \subseteq \cL$, $\phi \in
\cL$ and $\Gamma \models_{\Mtop} \phi$. We claim $t(\Gamma) \models_{\Msfour}
t(\phi)$. Indeed, if $M^* \in \Msfour$ and $x$ is a state in $M^*$ with $M^*, x
\models t(\psi)$ for all $\psi \in \Gamma$, then with $f$ as in
\cref{exp_thm_s4s5_translation} we have $f^{-1}(M^*), x \models \psi$ for all $\psi
\in \Gamma$. Since $f^{-1}(M^*) \in \Mint \cap \Munions \subseteq \Mtop$,
$\Gamma \models_{\Mtop} \phi$ gives $f^{-1}(M^*), x \models \phi$, and thus
$M^*, x \models t(\phi)$.

By (strong) completeness of $\sLsfoura$ for $\Msfour$, we get $t(\Gamma)
\entails_{\sLsfoura} t(\phi)$. That is, there are $\psi_0, \ldots, \psi_n \in
\Gamma$ such that $\entails_{\sLsfoura} t(\psi_0) \land \cdots \land t(\psi_n)
\limplies t(\phi)$. Since $t$ passes over conjunctions and implications, this
means $\entails_{\sLsfoura} t(\psi_0 \land \cdots \land \psi_n \limplies
\phi)$. By \cref{exp_claim_t_thm}, $\entails_{\sLtop} \psi_0 \land \cdots \land
\psi_n \limplies \phi$. Hence $\Gamma \entails_{\sLtop} \phi$, and we are done.
\end{proof}

Just as the connection between S4 and $\Mint \cap \Munions$ allowed us
to obtain a complete axiomatisation of $\Mtop$, we can axiomatise $\Mint \cap
\Mcompl$ by considering S5. Let $\sLintcompl$ be the extension of $\sLtop$
with the \textbf{5} axiom for $\hat\S$, which we present in the ``diamond''
form:
\[
    \S\neg\S\phi \limplies \neg\S\phi
    \quad
    \fivesoundness
\]

\begin{theorem}
\label{exp_thm_mintcompl_axiomatisation}
$\sLintcompl$ is sound and strongly complete with respect to
$\Mint \cap \Mcompl$.
\end{theorem}

\begin{proof}
    For soundness, we need to check that $\fivesoundness$ is valid on
    $\Mint \cap \Mcompl$. Let $M = (X, P, V)$ be closed under intersections and
    complements, and suppose $M, x \models \S\neg\S\phi$. Note that
    $\|\S\phi\|_M = \bigcap\{A \in P \mid \|\phi\|_M \subseteq A\}$ is an
    intersection from $P$, so $\|\S\phi\|_M \in P$. By closure under
    complements, $\|\neg\S\phi\|_M \in P$ too. Hence $M, x \models \S\neg\S\phi
    \land \E\neg\S\phi$. By \cref{exp_prop_validities} \cref{exp_item_e_and_s},
    we get $M, x \models \neg\S\phi$.

    The completeness proof goes in exactly the same way as
    \cref{exp_thm_mtop_axiomatisation}. Letting $\sLsfivea$ be the extension of
    $\sLsfoura$ with the $\fivek$ axiom $\neg\K\phi \limplies \K\neg\K\phi$, it
    can be shown that $\sLsfivea$ is strongly complete with respect to
    $\Msfive$.  With $u$ as in the proof of \cref{exp_thm_mtop_axiomatisation}, we
    have that $\entails_{\sLsfivea} \phi$ implies $\entails_{\sLintcompl}
    u(\phi)$, for $\phi \in \cLKA$ (the only new part to check there is that
    $u(\neg\K\phi \limplies \K\neg\K\phi)$ is a theorem of $\sLintcompl$, but
    this follows from $\fivesoundness$). The remainder of the proof goes
    through as before, this time appealing to the bijection $g: \Mint \cap
    \Mcompl \to \Msfive$.
\end{proof}

\section{The Multi-source Case}
\label{exp_sec_multisource}

So far we have been able to model the expertise of only a single
source. In this section we generalise the setting to handle \emph{multiple}
sources. This allows us to consider not only the expertise of different
sources individually, but also notions of \emph{collective expertise}. For
example, how may sources \emph{combine} their expertise? Is there a suitable
notion of
\emph{common expertise}? To answer these questions we take inspiration from the
well-studied notions of \emph{distributed knowledge} and \emph{common
knowledge} from epistemic logic \cite{fagin2003reasoning}, and
establish connections between collective expertise and collective knowledge.

\subsection{Collective Knowledge}

Let $\J$ be a finite, non-empty set of sources. Turning briefly to epistemic
logic interpreted under relational semantics, we recount several notions of
collective knowledge. First, a \emph{multi-source relational model} is a triple
$M^* = (X, \{R_j\}_{j \in \J}, V)$, where $R_j$ is a binary relation on $X$ for
each $j$. Consider the following knowledge operators \cite{fagin2003reasoning}:

\begin{itemize}
\item $\K_j\phi$ (individual knowledge): for $j \in J$ and a formula
$\phi$, set
    \[
    M^*, x \models \K_j\phi
\iff
\forall y \in X: x{R_j}y \implies M^*, y \models \phi.\]
    This is the straightforward adaptation of knowledge in the single-source case
to the multi-source setting.

    \item $\Kdist_J\phi$ (distributed knowledge): for $J \subseteq \J$
non-empty, set
    \[
    M^*, x \models \Kdist_J\phi
\iff
\forall y \in X: (x, y) \in \bigcap_{j \in J}{R_j}
    \implies M^*, y \models \phi.\]
    That is, knowledge of $\phi$ is distributed among the sources in
$J$ if, by combining their accessibility relations $R_j$, all
states possible at $x$ satisfy $\phi$. Here the $R_j$ are
combined by taking their intersection: a state $y$ is possible
according to the group at $x$ iff \emph{every} source in $J$ considers
$y$ possible at $x$.

    \item $\Kshared_J\phi$ (shared knowledge):\footnotemark{} for $J \subseteq
      \J$ non-empty, set
    \[
    M^*, x \models \Kshared_J\phi
\iff
\forall j \in J: M^*, x \models \K_j\phi.\]
    That is, a group $J$ have shared knowledge of $\phi$ exactly when
each agent in $J$ knows $\phi$. Thus we have
$\Kshared_J\phi \equiv \bigland_{j \in J}\K_j\phi$.

    \footnotetext{In \textcite{fagin2003reasoning}, shared knowledge is denoted
        $\mathsf{E}_J\phi$ for ``everybody knows $\phi$''. We opt to use the
        term ``shared'' knowledge to avoid conflict with our notation for
        expertise.}

    \item $\Kcommon_J\phi$ (common knowledge): write $\K_J^1\phi$ for
$\Kshared_J\phi$, and for $n \in \N$ write $\K_J^{n +
1}\phi$ for $\Kshared_J\K_J^n\phi$. Then
    \[
    M^*, x \models \Kcommon_J\phi
\iff
\forall n \in \N: M^*, x \models \K_J^n\phi.\]
    Here $\K_J^1\phi$ says that everyone in $J$ knows $\phi$,
$\K_J^2\phi$ says that everybody in $J$ knows that everybody in
$J$ knows $\phi$, and so on. There is common knowledge of
$\phi$ among $J$ if this nesting of ``everybody knows'' holds for
any order $n$.

    \end{itemize}

In what follows we write $\cLKAJ$ for the language formed from $\Prop$ with
knowledge operators $\K_j$, $\Kdist_J$, $\Kshared_J$ and $\Kcommon_J$, for $j
\in \J$ and $J \subseteq \J$ non-empty, and the universal modality $\A$.

\subsection{Collective Expertise}
\label{exp_sec_collective_expertise}

Returning to expertise semantics, define a \emph{multi-source expertise model}
as a triple $M = (X, \{P_j\}_{j \in \J}, V)$, where $P_j \subseteq 2^X$ is the
collection of expertise sets for source $j$. Say $M$ is closed under
intersections, unions, complements etc. if each $P_j$ is. Since the connection
between expertise and S4 knowledge (\cref{exp_thm_s4s5_translation}) holds for
expertise models closed under unions and intersections, we restrict
attention to this class of (multi-source) models in this section.

The counterpart of individual knowledge -- individual expertise -- is
straightforward: we may simply introduce expertise and soundness operators
$\E_j$ and $\S_j$ for each source $j \in \J$, and interpret
$\E_j\phi$ and $\S_j\phi$ as in the single-source case using
$P_j$. For notions of collective expertise and soundness, we define new
collections $P_J$ by combining the $P_j$ in an appropriate way.

\paragraph{Distributed Expertise.}

For distributed expertise, the intuition is clear: the sources in a group
$J$ should combine their expertise collections $P_j$ to form a
larger collection $\Pdist_J$. A first candidate for $\Pdist_J$
would therefore be $\bigcup_{j \in J}{P_j}$. However, since we assume
each $P_j$ is closed under unions and intersections, we suppose that each
source $j$ has the cognitive or computational capacity to combine expertise
sets $A \in P_j$ by taking unions or intersections. We argue that the
same should be possible for the group $J$ as a whole, and therefore let
$\Pdist_J$ be the closure of $\bigcup_{j \in J}{P_j}$ under unions
and intersections:
\[
\Pdist_J
= \bigcap\left\{
 P' \supseteq \bigcup_{j \in J}{P_j}
 \mid
 P' \text{ is closed under unions and intersections}
\right\}.\]
Note that $\Pdist_J$ is closed under unions and intersections, and
$P_j \subseteq \Pdist_J$ for all $j \in J$ (in fact,
$\Pdist_J$ is the smallest set with these properties). While
$\Pdist_J$ depends on the model $M$, we suppress this from the
notation.

$\Pdist_J$ also has a topological interpretation. As in
\cref{exp_sec_connection_with_ep_logic}, each $P_j$ gives rise to an Alexandrov
topology $\tau_j$ (where $P_j$ are the closed sets) if it is closed under
unions and intersections. By the aforementioned properties, $\tau^\dist_J$
corresponds to the coarsest Alexandrov topology finer than each $\tau_j$. On
the other hand, since the join (in the lattice of topologies on $X$) of
finitely many Alexandrov topologies is again Alexandrov~\cite[Theorems 2.4,
2.5]{steiner1966lattice}, it follows that $\tau^\dist_J$ is equal to the join
$\bigvee_{j \in J}{\tau_j}$.

Now, recall from \cref{exp_thm_s4s5_translation} that our semantics for
expertise and soundness is connected to relational semantics via the mapping
$P \mapsto R_P$ (\cref{exp_def_rp}). The following result shows that
$\Pdist_J$ corresponds to distributed knowledge under this mapping. For
ease of notation, write $\Rdist_J$ for $R_{\Pdist_J}$ and
$R_j$ for $R_{P_j}$.

\begin{proposition}
\label{exp_prop_rpdist}
For any multi-source expertise model $M$ and $J \subseteq \J$ non-empty,
\[
\Rdist_J = \bigcap_{j \in J}{R_j}.\]
\end{proposition}

\begin{proof}
    ``$\subseteq$'': Suppose $x{\Rdist_J}y$. Let $j \in J$.  We need to show
    $x{R_j}y$. Take any $A \in P_j$ such that $y \in A$. Then $A \in \Pdist_J$,
    so $x{\Rdist_J}y$ gives $x \in A$. Hence $x{R_j}y$.

    ``$\supseteq$'': Suppose $(x, y) \in \bigcap_{j \in J}{R_j}$, i.e. $x{R_j}y$
    for all $j \in J$. Set
    \[
        P' = \{A \in \Pdist_J \mid y \in A \implies x \in A\}
        \subseteq \Pdist_J.
    \]
    Then $P' \supseteq \bigcup_{j \in J}{P_j}$, since if $j \in J$ and $A \in
    P_j$ then $A \in \Pdist_J$ and $y \in A$ implies $x \in A$ by $x{R_j}y$.
    We claim $P'$ is closed under intersections. Suppose $\{A_i\}_{i \in I}
    \subseteq P'$ and write $A = \bigcap_{i \in I}{A_i}$. Since $P' \subseteq
    \Pdist_J$ and $\Pdist_J$ is closed under intersections, $A \in \Pdist_J$.
    Suppose $y \in A$.  Then $y \in A_i$ for each $i$, so $x \in A_i$ by the
    defining property of $P'$. Hence $x \in \bigcap_{i \in I}{A_i} = A$. This
    shows $A \in P'$ as desired. A similar argument shows that $P'$ is also
    closed under unions.

    We see from the definition of $\Pdist_J$ that $\Pdist_J \subseteq P'$, so
    in fact $P' = \Pdist_J$. It now follows that $x{\Rdist_J}y$: for any $A \in
    \Pdist_J$ with $y \in A$ we have $A \in P'$, so $x \in A$ also.
\end{proof}

\paragraph{Common Expertise.}

Common expertise admits a straightforward definition: simply take the expertise
sets in common with all $P_j$:
\[
\Pcommon_J = \bigcap_{j \in J}{P_j}.\]
If each $P_j$ is closed under unions and intersections, then so too is
$\Pcommon_J$.

At first this may appear \emph{too} straightforward. The form of the definition
is closer to \emph{shared} knowledge than to common knowledge. But in fact,
shared knowledge has \emph{no} expertise counterpart which admits the type of
connection established in \cref{exp_thm_s4s5_translation}. Indeed, shared knowledge
may fail positive introspection (axiom \textbf{4}: $\K\phi \limplies
\K\K\phi$), but we have seen that the knowledge derived from expertise and
soundness satisfies S4 (when the collection of expertise sets is closed under
unions and intersections).

However, this problem is only apparent in the translation of $\S\phi$ as
$\neg\K\neg\phi$. For our translation of $\E\phi$ as $\A(\neg\phi \limplies
\K\neg\phi)$, the universal quantification via $\A$ dissolves the differences
between shared and common knowledge.

\begin{proposition}
\label{exp_prop_shared_common_collapse}
Let $\phi \in \cLKAJ$ and let $J \subseteq \J$ be non-empty.
Then
\[
\A(\neg\phi \limplies \Kcommon_J\neg\phi)
\equiv
\A(\neg\phi \limplies \Kshared_J\neg\phi).\]
\end{proposition}

\begin{proof}

Let $M^* = (X, \{R_j\}_{j \in \J}, V)$ be a multi-source relational
model. Since $\Kcommon_J\psi \limplies \Kshared_J\psi$ is valid for
any $\psi$, the left-to-right implication of the above equivalence
is straightforward.

For the right-to-left implication, suppose $M^*, x \models
\A(\neg\phi \limplies \Kshared_J\neg\phi)$. We show by induction that
$M^*, x \models \A(\neg\phi \limplies \K_J^n\neg\phi)$ for all
$n \in \N$, from which the result follows.

The base case $n = 1$ is given, since $\K_J^1\neg\phi =
\Kshared_J\neg\phi$. For the inductive step, suppose $M^*, x
\models \A(\neg\phi \limplies \K_J^n\neg\phi)$. Take $y \in X$ such
that $M^*, y \models \neg\phi$. Let $j \in J$. Take $z
\in X$ such that $y{R_j}z$. From the initial assumption we have
$M^*, y \models \Kshared_J\neg\phi$, so $M^*, y \models
\K_j\neg\phi$ and thus $M^*, z \models \neg\phi$. By the inductive
hypothesis, $M^*, z \models \K_J^n\neg\phi$. This shows that
$M^*, y \models \K_j\K_J^n\neg\phi$ for all $j \in J$, and
thus $M^*, y \models K^{n+1}_J\neg\phi$. Hence $M^*, x
\models \A(\neg\phi \limplies \K^{n+1}_J\neg\phi)$ as required.
\end{proof}

\cref{exp_prop_shared_common_collapse} shows that when interpreting collective
expertise on $\phi$ as collective refutation of $\phi$ whenever
$\phi$ is false, there is no difference between using common knowledge
and just shared knowledge.

We now confirm that $\Pcommon_J$ does indeed correspond to common
knowledge. First we recall a well-known result from \textcite{fagin2003reasoning}.
In what follows, write $R^\trcls = \bigcup_{n \in \N}{R^n}$ for the transitive
closure of $R$.

\begin{lemma}[\textcite{fagin2003reasoning}, Lemma 2.2.1]
\label{exp_lemma_common_knowledge_transitive_closure}
Let $M^* = (X, \{R_j\}_{j \in \J}, V)$ be a multi-source relational
model and $J \subseteq \J$ non-empty. Write $R' =
\left(\bigcup_{j \in J}{R_j}\right)^\trcls$. Then for all $x \in X$
and $\phi \in \cLKAJ$:
\[
M^*, x \models \Kcommon_J\phi
\iff
\forall y \in X: x{R'}y \implies M^*, y \models \phi.\]
\end{lemma}

By \cref{exp_lemma_common_knowledge_transitive_closure}, common knowledge has an
interpretation in terms of the usual relational semantics for knowledge, where
we use the transitive closure of the union of the accessibility relations of
the sources in $J$.
%
Writing $\Rcommon_J$ for $R_{\Pcommon_J}$, we have the following.

\begin{proposition}
\label{exp_prop_rcommon}
Let $M$ be a multi-source model closed under unions and intersections.
Then for $J \subseteq \J$ non-empty,
$
\Rcommon_J = \left(
  \bigcup\nolimits_{j \in J}{R_j}
\right)^{\trcls}.$
\end{proposition}
\begin{proof}

    Write $R' = (\bigcup_{j \in J}{R_j})^{\trcls}$. Note that $\Rcommon_J$ is
    reflexive and transitive by \cref{exp_lemma_p_to_rp_mapping}
    \cref{exp_item_rp_ref_and_tr}. $R'$ is transitive by its definition as a
    transitive closure, and reflexive since each $R_j$ is (and $J \ne
    \emptyset$).
    %
    It is therefore sufficient by
\cref{exp_lemma_same_dc_sets_implies_equality} to show that any set
is downwards closed wrt $\Rcommon_J$ iff it is
downwards closed wrt $R'$. Since each $P_j$ is closed under
unions and intersections, so too is $\Pcommon_J$. Using
\cref{exp_lemma_p_to_rp_mapping} \cref{exp_item_rp_dc_property}, we have
\[
\begin{aligned}
   A \text{ downwards closed wrt } \Rcommon_J
   &\iff A \in \Pcommon_J \\
   &\iff \forall j \in J: A \in P_j \\
   &\iff \forall j \in J: A \text{ downwards closed wrt } R_j \\
   &\iff A \text{ downwards closed wrt } \bigcup\nolimits_{j \in J}{R_j} \\
   &\iff A \text{ downwards closed wrt } R'
\end{aligned}\]
where the last step uses the fact that $A$ is downwards closed with respect to
some relation if and only if it is downwards closed with respect to the
transitive closure. This completes the proof.
\end{proof}

\paragraph{Collective semantics.}

We now formally define the syntax and semantics of collective expertise. Let
$\cLJ$ be the language defined by the following grammar:
\[
\phi ::=
p \mid
\phi \land \phi \mid
\neg\phi \mid
\E_j\phi \mid \S_j\phi \mid
\E_J^g\phi \mid \S_J^g\phi \mid
\A\phi\]
for $p \in \Prop$, $j \in \J$, $g \in \{\dist, \common\}$ and
$J \subseteq \J$ non-empty. For a multi-source expertise model $M =
(X, \{P_j\}_{j \in \J}, V)$, define the satisfaction relation as before for
atomic propositions, propositional connectives and
$\A$, and set
\[
\begin{array}{lllr}
 M, x &\models \E_j\phi &\iff \|\phi\|_M \in P_j \\
 M, x &\models \E_J^g\phi &\iff \|\phi\|_M \in P_J^g
     & (g \in \{\dist, \common\}) \\
 M, x &\models \S_j\phi &\iff \forall A \in P_J: \|\phi\|_M \subseteq A
     \implies x \in A \\
 M, x &\models \S_J^g\phi &\iff \forall A \in P_J^g: \|\phi\|_M \subseteq A
     \implies x \in A
     & (g \in \{\dist, \common\})
\end{array}\]
Note that expertise and soundness are interpreted as before, but with respect
to different collections $P$. Consequently, the interactions
shown in \cref{exp_prop_validities} still hold for individual and collective
notions of expertise and soundness.

\begin{example}
\label{exp_ex_mutlisource}

    Extending \cref{exp_ex_economist_motivation,exp_ex_economist_formalisation},
    consider $\J = \{\econ, \dr, \analyst\}$, where $\econ$ is the economist,
    $\dr$ is a doctor with expertise on $i$ only, and $\analyst$ has access to
    aggregate data distinguishing three levels of virus activity: minimal
    ($\neg i \land \neg d$), high ($(i \lor d) \land \neg(i \land d)$) and very
    high ($i \land d$). This can be modelled by a multi-source model $M$ with
    $X$, $V$ and $P_\econ$ as in \cref{exp_ex_economist_formalisation}, and $P_\dr
    = \{\emptyset, X, \{ipd, ip, id, i\}, \{pd, p, d, \emptyset\}\}$,
    $P_\analyst$ is the closure under unions of $\{\emptyset, X, \{ipd, id\},
    \{ip, pd, i, d\}, \{p, \emptyset\}\}$.

    Note that neither $\dr$ nor $\analyst$ have expertise on $d$ individually.
    However, if $\dr$ can communicate whether or not $i$ holds, this gives
    $\analyst$ enough information to disambiguate the ``high activity'' case
    and therefore determine the truth value of $d$. Indeed, we have $\|d\| = \|i \land d\| \cup
    (\|i \lor d\| \setminus \|i \land d\| \cap \|\neg i\|)$, which is formed by
    unions and intersections from $P_\dr \cup P_\analyst$, and thus $\|d\| \in
    \Pdist_{\{\dr, \analyst\}}$. Hence $M \models \E_{\{\dr, \analyst\}}^\dist
    d$.
    %
    Similarly, $\dr$ and $\analyst$ have distributed expertise on $\neg d$.
    Bringing back $\econ$, the grand coalition $\J$ has distributed expertise
    on the original report $p \land \neg d$ from
    \cref{exp_ex_economist_motivation}. Consequently, the report is no longer sound
    at ``actual'' state $idp$: all sources together have sufficient expertise to
    know it is false.

\end{example}

The following validities express properties
specific to collective expertise.
% In what follows, we write $\MJ$ for
% the class of multi-source models, and adopt similar notation for sub-classes.

\begin{proposition}
\label{exp_prop_collective_validities}
The following formulas are valid.

\begin{enumerate}
    \item For $j \in J$, $\E_j\phi \limplies \E_J^\dist\phi$

    \item $\E_J^\common\phi \liff \bigland_{j \in J}{\E_j\phi}$

    \item\label{exp_item_common_exp_fp} $\S_J^\common\phi \liff \biglor_{j \in
        J}\S_j\S_J^\common\phi$

    \item $\E_{\{j\}}^\dist\phi \liff \E_j\phi$ is valid on $\MintJ \cap
          \MunionsJ$

    \end{enumerate}
\end{proposition}
\begin{proof}
    We prove only \cref{exp_item_common_exp_fp}; the others are straightforward.
    The right implication is valid since $\psi \limplies \S_j\psi$ is, with
    $\psi$ set to $\S_J^\common\phi$ and $j \in J$ arbitrary (recall $J$ is
    non-empty).
    %
    For the left implication, suppose there is $j \in J$ with $M, x \models
    \S_j\S_J^\common\phi$. Then $x \in \bigcap\{A \in P_j \mid
    \|\S_J^\common\phi\|_M \subseteq A\}$. Now take $B \in P_J^\common$ such
    that $\|\phi\|_M \subseteq B$. Note that if $y \in \|\S_J^\common\phi\|$
    then $y \in B$ by the definition of the semantics for $\S_J^\common$, so
    $\|\S_J^\common\phi\|_M \subseteq B$. Since $B \in P_J^\common \subseteq
    P_j$, we get $x \in B$.  This shows $M, x \models \S_J^\common\phi$.
\end{proof}

Validity \cref{exp_item_common_exp_fp} comes from the
\emph{fixed-point axiom} for common knowledge: $\Kcommon_J\phi \liff
\Kshared_J(\phi \land \Kcommon_J\phi)$. Our version says
$\S_J^\common\phi$ is a fixed-point of the function $\theta \mapsto \biglor_{j
\in J}{\S_j\theta}$. In words, $\phi$ is true up to lack of \emph{common}
expertise iff there is some source for whom $\S_J^\common\phi$ is true up to
their lack of (individual) expertise.

As promised, there is a tight link between our notions of collective
expertise and knowledge. Define a translation $t: \cLJ \to
\cLKAJ$ inductively by
\[
\begin{array}{lllr}
 &t(\E_j\phi) &= \A(\neg t(\phi) \limplies \K_j\neg t(\phi)) \\
 &t(\E_J^g\phi) &= \A(\neg t(\phi) \limplies \K_J^g\neg t(\phi))
     &(g \in \{\dist, \common\}) \\
 &t(\S_j\phi) &= \neg\K_j\neg t(\phi) \\
 &t(\S_J^g\phi) &= \neg\K_J^g\neg t(\phi)
     &(g \in \{\dist, \common\}) \\
\end{array}\]
where the other cases are as for $t$ in \cref{exp_sec_connection_with_ep_logic}.
This is essentially the same translation as before, but with the various
types of expertise and soundness matched with their knowledge counterparts. We
have an analogue of \cref{exp_thm_s4s5_translation}.

\begin{theorem}
\label{exp_thm_collective_s4s5_translation}
The mapping $f: \MintJ \cap \MunionsJ \to \MsfourJ$ given by
$(X, \{P_j\}_{j \in \J}, V) \mapsto (X, \{R_{P_j}\}_{j \in \J}, V)$ is
bijective, and for $x \in X$ and $\phi \in \cLJ$:
\[
M, x \models \phi \iff f(M), x \models t(\phi).\]
Moreover, the restriction of this map to $\MintJ \cap \McomplJ$ is a
bijection into $\MsfiveJ$.
\end{theorem}

\begin{proof}
That the map is bijective follows easily from
\cref{exp_thm_s4_semantic_link,exp_thm_s5_semantic_link}. For
the stated property we proceed by induction on $\cLJ$ formulas. As in
\cref{exp_thm_s4s5_translation}, the cases for atomic propositions,
propositional connectives and $\A$ are straightforward.
%
For expertise and soundness, the argument in the proof of
\cref{exp_thm_s4s5_translation} showed that $\E\phi$ and
$\S\phi$ interpreted via some collection $P$ is
equivalent to $t(\E\phi)$ and $t(\S\phi)$ interpreted wrt
relational semantics via $R_P$.
%
It is therefore sufficient to show that for each notion of individual and
collective expertise interpreted in $M$ via $P$, its
corresponding notion of individual or collective knowledge (used in the
translation $t$) is interpreted in $f(M)$ via $R_P$.
This is self-evident for individual expertise. For distributive expertise
this was shown in \cref{exp_prop_rpdist}. For common expertise this was
shown in \cref{exp_lemma_common_knowledge_transitive_closure,exp_prop_rcommon}.
\end{proof}

\cref{exp_thm_collective_s4s5_translation} can be used to adapt any sound and
complete axiomatisation for $\MsfourJ$ (resp., $\MsfiveJ$) over
the language $\cLKAJ$ to obtain an axiomatisation for $\MintJ \cap
\MunionsJ$ (resp., $\MintJ \cap \McomplJ$) over $\cLJ$, in the
same way as we did earlier when adapting S4 and S5 in
\cref{exp_thm_mtop_axiomatisation,exp_thm_mintcompl_axiomatisation}.

\section{Dynamic Extension}
\label{exp_sec_dynamic_extension}

So far our picture has been entirely static. We cannot speak of expertise
changing over time, nor of the information in a model changing via
announcements from sources. To remedy this, we extend the framework with two
\emph{dynamic} operators: one to account for \emph{increases in expertise} --
e.g. after a process of learning or acquisition of new evidence -- and one to
model \emph{sound announcements}. For simplicity, we return to the
single-source case.

\subsection{Expertise Increase}

As a source interacts with the world over time, they may learn to make more
distinctions between possible states of the world, and thereby increase their
expertise. Leaving the particulars of the learning mechanism unspecified, we
study only the end result: the source's expertise collection $P$ is
expanded to include a new set $A$.

However, this may not be so simple as setting $P' = P \cup \{A\}$ in light of
the closure properties that may be imposed $P$. As remarked in
\cref{exp_sec_closure_properties}, closure conditions correspond to assumptions
about the source's cognitive or computational capabilities. It seems natural
that if the source has the ability to combine sets in $P$ by taking
intersections, for example, then they should also be able to do after the
learning, i.e.  $P'$ should also be closed under intersections. Thus, the new
collection $P'$ should inherit any closure properties from $P$, while extending
$P \cup \{A\}$. In principle, we could therefore consider an expertise increase
operation for \emph{each} combination of closure properties.

For concreteness we will not do this, and will instead focus on the class
$\Mint$ of models closed under intersections. Conceptually, this is a minimal
requirement, since we argued in section \cref{exp_sec_closure_properties} that
closure under intersections is a natural property. There are also technical
advantages: we will later show that closure under intersections allows us to
find reduction axioms which allow the formulas involving expertise increase to
be equivalently expressed in the static language.

\begin{definition}
\label{exp_def_expinc_model}
Given an expertise model $M = (X, P, V)$ and a formula $\phi$,
define the model $M^{\expinc\phi} = (X, P^{\expinc\phi}, V)$ by
setting
\[
P^{\expinc\phi}
=
\left\{\bigcap \mathcal{A} \mid \mathcal{A} \subseteq P \cup \{\|\phi\|_M\}\right\}.
\]
That is, $P^{\expinc\phi}$ is obtained by adding
$\|\phi\|_M$ to $P$ and closing under intersections.

\end{definition}

Syntactically, we introduce formulas of the form $[\expinc\phi]\psi$,
which are to be read as ``$\psi$ holds after the source gains expertise on
$\phi$''. The truth condition for $[\expinc\phi]\psi$ in a model
$M$ is defined in terms of $M^{\expinc\phi}$:
\[
M, x \models [\expinc\phi]\psi
\iff
M^{\expinc\phi}, x \models \psi.\]
If $\cLzero$ denotes the propositional language built from $\Prop$,
then $[\expinc\alpha]\E\alpha$ is valid for all $\alpha \in
\cLzero$. That is, expertise increase is successful for any propositional
formula. However, this is not the case for general formulas $\phi \in
\cL$. This comes from the fact that expertise is represented \emph{semantically} via
sets of states. The operator $[\expinc\phi]$ represents the source
obtaining expertise on the set of $\phi$ states, where $\phi$ is
interpreted \emph{before the increase took place}. If $\phi$ refers to
expertise (with $\E$ or $\S$) then the meaning of $\phi$ may
change after the increase. For example, consider the model $M = (X, P,
V)$ with
\[
\begin{aligned}
    &X = \{1, 2, 3, 4\} \\
    &P = \{\emptyset, X, \{1, 3\}\} \\
    &V(p) = \{1\} \\
    &V(q) = \{2, 3\}
\end{aligned}\]
Then, with $\phi = p \lor (q \land \neg\S p)$ we have $M, 1 \not\models
[\expinc\phi]\E\phi$.\footnotemark{} This counterexample is reminiscent of
\emph{Moore sentences} as formalised in Dynamic Epistemic Logic; e.g. an agent
cannot know $p \land \neg\K p$ (``$p$ is true but I do not know it'') after this
is truthfully announced~\cite{baltag2008qualitative}.

\footnotetext{In detail, we have $\|\phi\|_M = \{1, 2\}$, so
$P^{\expinc\phi} = \{\emptyset, X, \{1, 3\}, \{1, 2\}, \{1\}\}$. Then
$\|\phi\|_{M^{\expinc\phi}} = \{1, 2, 3\} \notin P^{\expinc\phi}$, so
$M^{\expinc\phi}, 1 \not\models \E\phi$.}

Next we give reduction axioms to express any formula involving
$[\expinc\phi]$ by an equivalent formula in the static language
$\cL$.

\begin{proposition}
\label{exp_prop_expinc_reduction_axioms}
The following formulas are valid on $\M$:
\[
\begin{aligned}
[\expinc\phi]p &\quad\liff\quad [\expinc\phi]p \\
[\expinc\phi](\psi \land \theta)
     &\quad\liff\quad [\expinc\phi]\psi \land [\expinc\phi]\theta \\
[\expinc\phi]\neg\psi &\quad\liff\quad \neg[\expinc\phi]\psi \\
[\expinc\phi]\A\psi &\quad\liff\quad \A[\expinc\phi]\psi \\
[\expinc\phi]\S\psi
    &\quad\liff\quad
         \S[\expinc\phi]\psi
         \land
         (
             \A([\expinc\phi]\psi \limplies \phi)
             \limplies
             \phi
         )
    \\
[\expinc\phi]\E\psi
    &\quad\liff\quad
     \A(
         (
             \S[\expinc\phi]\psi
             \land
             (
                 \A([\expinc\phi]\psi \limplies \phi)
                 \limplies
                 \phi
             )
         )
         \limplies
         [\expinc\phi]\psi
     )
\end{aligned}\]
\end{proposition}
\begin{proof}
The cases for atomic propositions, propositional connectives and
$\A$ are straightforward. We show the reduction axiom for
$\S$. Let $M = (X, P, V)$ be a model and $x \in X$.

``$\limplies$'': Suppose $M, x \models [\expinc\phi]\S\psi$.
Then $M^{\expinc\phi}, x \models \S\psi$. Hence
\[
x \in
\bigcap\{
   A \in P^{\expinc\phi}
   \mid
   \|\psi\|_{M^{\expinc\phi}} \subseteq A
\}
\quad (*)\]
Note that $\|\psi\|_{M^{\expinc\phi}} = \|[\expinc\phi]\psi\|_M$.
Now take $A \in P$ such that $\|[\expinc\phi]\psi\|_M
\subseteq A$. Since $P \subseteq P^{\expinc\phi}$, we get $x
\in A$ from $(*)$. Hence $M, x \models \S[\expinc\phi]\psi$.

Now suppose $M, x \models \A([\expinc\phi]\psi \limplies \phi)$.
Then $\|[\expinc\phi]\psi\|_M \subseteq \|\phi\|_M$, so
$\|\psi\|_{M^{\expinc\phi}} \subseteq \|\phi\|_M$. Since
$\|\phi\|_M \in P^{\expinc\phi}$, we get $x \in \|\phi\|_M$
from $(*)$, i.e. $M, x \models \phi$ as required.

``$\limpliedby$'': Suppose $M, x \models \S[\expinc\phi]\psi$
and $M, x \models \A([\expinc\phi]\psi \limplies \phi) \limplies
\phi$. Take $A \in P^{\expinc\phi}$ such that
$\|\psi\|_{M^{\expinc\phi}} \subseteq A$. Then
$\|[\expinc\phi]\psi\|_M \subseteq A$. By definition of
$P^{\expinc\phi}$, there is a collection $\mathcal{A}
\subseteq P \cup \{\|\phi\|_M\}$ such that $A =
\bigcap\mathcal{A}$. Let $B \in \mathcal{A}$. If $B \in P$,
then $\|[\expinc\phi]\psi\|_M \subseteq A \subseteq B$ and
$M, x \models \S[\expinc\phi]\psi$ give $x \in B$. Otherwise,
$B = \|\phi\|_M$. Hence $\|[\expinc\phi]\psi\|_M \subseteq
\|\phi\|_M$, so $M, x \models \A([\expinc\phi]\psi \limplies
\phi)$. By the second assumption, we get $M, x \models \phi$, i.e.
$x \in \|\phi\|_M = B$. We have now shown that $x \in
\bigcap\mathcal{A} = A$, and thus $M^{\expinc\phi}, x \models
\S\psi$ and $M, x \models [\expinc\phi]\S\psi$.

For the reduction axiom for $\E$, note that since
$M^{\expinc\phi} \in \Mint$ we have $M^{\expinc\phi}, x
\models E\psi$ iff $M^{\expinc\phi}, x \models \A(\S\psi \limplies
\psi)$. Using the reduction axioms for $\A$ and $\S$ (and the
reduction axiom for the implication, derived from those for $\neg$
and $\land$), we obtain the desired equivalence.
\end{proof}

Note that only the reduction axiom for $[\expinc\phi]\E\psi$ requires
$M^{\expinc\phi}$ to be closed under intersections.

\subsection{Sound Announcements}

In logics of public announcement \cite{plaza2007logics,van_Ditmarsch_2008},
the dynamic operator $[!\phi]$ represents a \emph{public} and \emph{truthful}
announcement of $\phi$; the formula $[!\phi]\psi$ is read as ``after
$\phi$ is announced, $\psi$ holds''. Such an announcement changes
the information available in a model: after the announcement, all
$\neg\phi$ states are eliminated.

Since the premise of our work is to deal with non-expert sources, the
truthfulness requirement is too strong for an announcement operator in our
setting. Instead, we consider \emph{sound announcements}: the source may announce
$\phi$ whenever $\phi$ is sound at the current state. That is, the
source may announce any (possibly false) statement which is true up to their
lack of expertise.

Such an announcement is denoted syntactically by $[\sndann\phi]$. As with
the expertise increase operator, we define a model update operation
$M \mapsto M^{\sndann\phi}$.

It is clear how one should define new set of
states: since the announcement tells us $\phi$ is sound, we eliminate
unsound states by setting $X^{\sndann\phi} = \|\S\phi\|_M$. The valuation
is also straightforward, since announcements should not change the meaning of
atomic propositions.

What about the new expertise collection $P^{\sndann\phi}$? If we restrict
attention to models closed under intersections, as we did for expertise
increase, then a natural choice is to simply restrict each $A \in P$ to
$X^{\sndann\phi}$ by intersection. Since $X^{\sndann\phi} =
\|\S\phi\|_M = \bigcap\{B \in P \mid \|\phi\|_M \subseteq B\}$, by the closure
property we will have $P^{\sndann\phi} \subseteq P$, so that
announcements do not increase expertise. This assumption will also permit us to
find reduction axioms later on.

\begin{definition}
\label{exp_def_sndann_model}
Let $M = (X, P, V)$ be an expertise model. For a formula $\phi$,
define the model $M^{\sndann\phi} = (X^{\sndann\phi}, P^{\sndann\phi},
V^{\sndann\phi})$ by setting
\[
\begin{aligned}
  X^{\sndann\phi} &= \|\S\phi\|_M \\
  P^{\sndann\phi} &= \{A \cap X^{\sndann\phi} \mid A \in P\} \\
  V^{\sndann\phi} &= V(p) \cap X^{\sndann\phi}
\end{aligned}\]
\end{definition}

Semantically, the truth condition for $[\sndann\phi]\psi$ is as follows.
\[
M, x \models [\sndann\phi]\psi
    \iff
(M, x \models \S\phi \implies M^{\sndann\phi}, x \models \psi).
\]
Here we have the precondition that $\S\phi$ is true: if $\phi$ is unsound,
$[\sndann\phi]\psi$ is true for \emph{any} $\psi$. Note that a sound
announcement of $\phi$ can also be seen as a public (\emph{truthful})
announcement of $\S\phi$.

\begin{example}
\label{exp_ex_dynamic}

    The report of the economist in \cref{exp_ex_economist_motivation} can be
    modelled as $[\sndann(p \land \neg d)]$. Note that, with $M$ as in
    \cref{exp_ex_economist_formalisation}, $\|\S(p \land \neg d)\|_M = \|p\|_M$.
    The updated model $M^{\sndann(p \land \neg d)}$ therefore consists only of
    the bottom half of $M$ as shown in \cref{exp_fig_economist_example}. We see
    that $M, idp \models [\sndann(p \land \neg d)]d$ -- showing that even
    propositional announcements can ``fail'' due to lack of expertise -- and $M
    \models [\sndann(p \land \neg d)]\A p$ -- showing that the parts of the
    report on which the source does have expertise are \emph{always} true
    after their announcement.

\end{example}

As with the expertise increase operator, sound announcements remain sound for
purely propositional formulas $\alpha \in \cLzero$:
$[\sndann\alpha]\S\alpha$ is valid on $\Mint$. This is
not true for general formulas $\phi \in \cL$ which may refer to expertise
itself. For example, in the model $M = (X, P, V) \in \Mint$ given by
$X = \{1,2,3,4\}$, $P = \{\emptyset, X, \{1\}, \{2\}, \{1, 2,
3\}\}$, $V(p) = \{1, 2\}$ and $V(q) = \{2, 4\}$, with $\phi =
p \land \neg\E q$ we have $M, 1 \not\models [\sndann\phi]\S\phi$.

The following reduction axioms allow formulas involving announcements to be
expressed in the static language.

\begin{proposition}
\label{exp_prop_sndann_reduction_axioms}
The following formulas are valid on $\M$:
\[
\begin{aligned}
  [\sndann\phi]p &\quad\liff\quad
      \S\phi \limplies p \\
  [\sndann\phi](\psi \land \theta) &\quad\liff\quad
      [\sndann\phi]\psi \land [\sndann\phi]\theta \\
  [\sndann\phi]\neg\psi &\quad\liff\quad
      \S\phi \limplies \neg[\sndann\phi]\psi \\
  [\sndann\phi]\A\psi &\quad\liff\quad
      \S\phi \limplies \A[\sndann\phi]\psi \\
  [\sndann\phi]\S\psi &\quad\liff\quad
      \S\phi \limplies \S(\S\phi \land [\sndann\phi]\psi)
\end{aligned}\]
and the following is valid on $\Mint$:
\[
[\sndann\phi]\E\psi \quad\liff\quad
    \S\phi \limplies \E(\S\phi \land [\sndann\phi]\psi)\]
\end{proposition}
\begin{proof}
The cases of atomic propositions, propositional connectives and the
universal modality $\A$ are straightforward.

For the reduction axiom for $\S$, first note that
$\|\psi\|_{M^{\sndann\phi}} = \|\S\phi \land
[\sndann\phi]\psi\|_M$. We need to show that $M, x \models
[\sndann\phi]\S\psi$ iff $M, x \models \S\phi \limplies \S(\S\phi
\land [\sndann\phi]\psi)$. If $M, x \not\models \S\phi$ this is
clear. Otherwise $x \in \|\S\phi\|_M$, and we have
\[
\begin{aligned}
   M, x \models [\sndann\phi]\S\psi
   &\iff M^{\sndann\phi}, x \models \S\psi \\
   &\iff \forall B \in P^{\sndann\phi}: \|\psi\|_{M^{\sndann\phi}}
   \subseteq B \implies x \in B \\
   &\iff \forall A \in P: \|\S\phi \land
       [\sndann\phi]\psi\|_M \subseteq A \cap
       \|\S\phi\|_M \implies x \in A \cap \|\S\phi\|_M \\
   &\iff \forall A \in P: \|\S\phi \land
       [\sndann\phi]\psi\|_M \subseteq A \implies x \in A \\
   &\iff M, x \models \S(\S\phi \land [\sndann\phi]\psi)
\end{aligned}\]
and the result follows.

For the $\E$ reduction axiom, take $M \in \Mint$. Again,
suppose without loss of generality that $x \in \|\S\phi\|_M$. Then
we have
\[
\begin{aligned}
   M, x \models [\sndann\phi]\E\psi
   &\iff M^{\sndann\phi}, x \models \E\psi \\
   &\iff \|\psi\|_{M^{\sndann\phi}} \in P^{\sndann\phi} \\
   &\iff \|\S\phi \land [\sndann\phi]\psi\|_M \in P^{\sndann\phi} \\
   &\iff \|\S\phi \land [\sndann\phi]\psi\|_M \in P \\
   &\iff M, x \models \E(\S\phi \land [\sndann\phi]\psi)
\end{aligned}\]
where the forwards direction of the penultimate equivalence holds since
$P^{\sndann\phi} \subseteq P$ when $M$ is closed under
intersections, and the backwards direction holds since $\|\S\phi
\land [\sndann\phi]\psi\|_M \subseteq \|\S\phi\|_M = X^{\sndann\phi}$. It
follows that $M, x \models [\sndann\phi]\E\psi$ iff $M, x
\models \S\phi \limplies \E(\S\phi \land [\sndann\phi]\psi)$, as
required.
\end{proof}

To conclude, we note some interesting validities involving the dynamic
operators and their interaction.

\begin{proposition}
\label{exp_prop_dynamic_validities}
For any $\alpha, \beta \in \cLzero$, the following formulas are valid
on $\Mint$:

\begin{enumerate}
    \item\label{exp_item_exp_sndann} $\E\alpha \liff \A[\sndann\alpha]\alpha$

    \item\label{exp_item_weakening_sndann} $\A(\alpha \limplies \beta) \limplies
          [\expinc\beta][\sndann\alpha]\beta$

  \item\label{exp_item_weakening_sndann_eq} $[\expinc\alpha][\sndann\alpha]\alpha$

    \end{enumerate}
\end{proposition}

\begin{proof}\leavevmode
\begin{enumerate}
\item Using the reduction axioms for atomic propositions, conjunctions and
negations, one can show by induction that $[\sndann\phi]\alpha$
is equivalent to $\S\phi \limplies \alpha$. Applying this with
$\phi = \alpha$, we have that $\A[\sndann\alpha]\alpha$ is
equivalent to $\A(\S\alpha \limplies \alpha)$, which is
equivalent to $\E\alpha$ for models closed under intersections.

    \item We use the following fact, whose proof is straightforward by
induction on $\cLzero$ formulas.

    \begin{itemize}
    \item For $\alpha \in \cLzero$, $\phi \in \cL$ and any model
$M$, $\|\alpha\|_{M^{\expinc\phi}} = \|\alpha\|_M$ and
$\|\alpha\|_{M^{\sndann\phi}} = \|\alpha \land \S\phi\|_M$.

        \end{itemize}
    Now, take $M = (X, P, V) \in \Mint$, $x \in X$, and
suppose $M, x \models \A(\alpha \limplies \beta)$. Then
$\|\alpha\|_M \subseteq \|\beta\|_M$.

    We need to show $M, x \models
[\expinc\beta][\sndann\alpha]\beta$, i.e. $M^{\expinc\beta}, x
\models [\sndann\alpha]\beta$. Suppose $M^{\expinc\beta}, x
\models \S\alpha$. To show $(M^{\expinc\beta})^{\sndann\alpha},
x \models \beta$, we need
    \[
    x \in \|\beta\|_{(M^{\expinc\beta})^{\sndann\alpha}}
= \|\beta \land \S\alpha\|_{M^{\expinc\beta}}\]
    where the equality follows from the claim above. By assumption
$M^{\expinc\beta}, x \models \S\alpha$, so we only need to show
$M^{\expinc\beta}, x \models \beta$.

Since $[\expinc\beta]\E\beta$ is valid in $M$, we have $M^{\expinc\beta}, x
\models \E\beta$. From \cref{exp_prop_validities} \cref{exp_item_weakening},
        $M^{\expinc\beta}, x \models \A(\alpha \limplies \beta) \limplies
        (\S\alpha \land \E\beta \limplies \beta)$.  But from the above claim
        and $\|\alpha\|_M \subseteq \|\beta\|_M$ we have
        $\|\alpha\|_{M^{\expinc\beta}} \subseteq \|\beta\|_{M^{\expinc\beta}}$,
        i.e. $M^{\expinc\beta}, x \models \A(\alpha \limplies \beta)$. Hence
        $M^{\expinc\beta}, x \models \beta$, and we are done.

    \item Taking $\beta = \alpha$, this validity follows from
        \cref{exp_item_weakening_sndann}.

    \end{enumerate}
\end{proof}

In words, \cref{exp_item_exp_sndann} says that expertise on a propositional
formula $\alpha$ is equivalent to the guarantee that $\alpha$ is true whenever
it is soundly announced. \cref{exp_item_weakening_sndann} is essentially a
reformulation of \cref{exp_prop_validities} \cref{exp_item_weakening}; it says that
if $\beta$ is logically weaker than $\alpha$, gaining expertise on $\beta$
ensures that $\beta$ is at least true after a sound announcement of the
stronger formula $\alpha$.  \cref{exp_item_weakening_sndann_eq} is the special
case of \cref{exp_item_weakening_sndann} with $\beta = \alpha$, which says that
$\alpha$ is true following a sound announcement after the sources gains
expertise on $\alpha$.

\section{Conclusion}
\label{exp_sec_conclusion}

\paragraph{Summary.} This chapter presented a modal logic framework to
reason about the expertise of information sources and soundness of information.
We investigated both conceptual and technical issues, establishing
completeness results for various classes of expertise models. The connection with
epistemic logic showed how expertise and soundness may be given precise
interpretations in terms of knowledge; if expertise is closed under
intersections and unions this results in S4 knowledge, and closure under
complements strengthens this to S5. The framework was then extended to handle
multiple sources, permitting the study of several notions of collective
expertise. Finally, we considered dynamic operators to model evolving expertise
and sound announcements.

\paragraph{Limitations and future work.}

On a technical level, some open questions remain. For example, can frames
closed under arbitrary unions be be expressed in our language, as other closure
properties were expressed in \cref{exp_prop_frame_conditions}? Similarly, can
one axiomatise the class of models closed under arbitrary unions, without also
requiring closure under intersections? One could also consider computational
properties, such as decidability and the complexity of the satisfiability
problem.

There are also conceptual limitations and areas for future study. Firstly, our
notion of expertise is absolute: either the source is an expert on $\phi$ or
they are not. In reality things are more nuanced, and source may have varying
levels of expertise. Our assumption that expertise is independent of the actual
state of the world could also be considered too strong, since it forbids any
possibility of \emph{context-dependent} expertise. As a somewhat contrived
example, the economist in our running examples may have expertise on $p$ in
ordinary times, but not if they are suffering from the virus which affects
cognitive ability.

\paragraph{Outlook.}

Equipped with the notions of expertise and soundness from this chapter, the
following chapter poses a belief change problem -- in the style of AGM
revision~\cite{alchourron1985logic} and belief
merging~\cite{konieczny2002merging} -- in which expertise is not assumed to be
known upfront, but must be estimated from a sequence of reports. For simplicity
we dispense with some of the generality of this chapter, by
\begin{inlinelist}
    \item considering only finite models whose states are the propositional
          valuations over a fixed, finite set of variables; and
    \item assuming expertise collections are closed under both intersections
          and complements.
\end{inlinelist}
By \cref{exp_thm_s5_semantic_link}, such models are in one-to-one
correspondence with S5 relational models, so that their corresponding binary
relation $R_P$ is an equivalence relation. Equivalently, each collection $P$
closed under intersections and complements corresponds to a partition $\Pi_P$
over the set of states, whose cells are simply the equivalence classes of
$R_P$. Since one can express the semantic conditions for expertise and
soundness directly in terms of this partition, in what follows we in fact take
the partition $\Pi_P$ as primitive instead of the expertise collection $P$.
Given that the equivalence relation $R_P$ corresponding to $\Pi_P$ can be
understood as an epistemic accessibility relation (by
\cref{exp_thm_s4s5_translation}), we can interpret $\Pi_P$ as expressing an
\emph{indistinguishibility relation} over states: two states lie in the same
partition cell if the source lacks expertise to distinguish them.
