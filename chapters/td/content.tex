\chapter{Truth Discovery}
\label{chapter_td}

There is an increasing amount of data available in today's world, particularly
from the web, social media platforms and crowdsourcing systems. The openness of
such platforms makes it simple for a wide range of users to share information
quickly and easily, potentially reaching a wide international audience. It is
inevitable that amongst this abundance of data there are \emph{conflicts},
where data sources disagree on the truth regarding a particular object or
entity. For example, low-quality sources may mistakenly provide erroneous data
for topics on which they lack expertise, or malicious sources may try to
deliberately deceive.

Resolving such conflicts and determining the true facts is therefore an
important task. Truth discovery has emerged as a set of techniques to achieve
this by considering the \emph{trustworthiness} of
sources~\cite{li_survey_2016,gupta2011survey,berti2015veracity}. The general
principle is that true claims are those reported by trustworthy sources, and
trustworthy sources are those that report believable claims. Note that there is
a \emph{mutual dependence} between the trust and belief parts of the problem,
whereby highly trusted sources bestow credibility on their claims and vice
versa. Application areas include real-time traffic navigation~\cite{du2019},
drug side-effect discovery~\cite{ma2017}, crowdsourcing and social
sensing~\cite{zhang_robust_2016,wang_truth_2012,ma_faitcrowd_2015}.

For a simple example of a situation where trust can play an important role in
conflict resolution, consider the following example.

\begin{figure}
    \centering
    \begin{tikzpicture}[thick]
        \LARGE
        \networkinitwithnames{s,t,u,v}{s,t,u,v}{c,d,e,f}{+,-,+,-}
        \object{o_1}{c}{d};
        \object{o_2}{e}{f};
        \report{s}{c};
        \report{s}{e};
        \report{t}{d};
        \report{t}{f};
        \report{u}{f};
        \report{v}{f};
    \end{tikzpicture}
    \caption{
        Illustrative example of a truth discovery problem, with sources
        $s, t, u, v$ and objects $o_1, o_2$, each with associated values $+$
        and $-$.
    }
    \label{td_new_fig_intro_example_with_values}
\end{figure}

\begin{example}
\label{td_new_ex_intro_example}
    Let $o_1$ and $o_2$ represent images for which crowdsourcing workers are
    asked to label $+$ or $-$ (in the truth discovery terminology, $o_1$ and
    $o_2$ are called \emph{objects}; $+$ and $-$ are \emph{values}). An object
    paired with a value is called a \emph{claim}.
    %
    Consider workers (the data sources) $s, t, u$ and $v$ who make claims
    reports as shown in \cref{td_new_fig_intro_example_with_values}. Without
    considering trust information, the label for $o_1$ appears a tie, with both
    options $+$ and $-$ receiving one vote from sources $s$ and $t$
    respectively.

    Taking a trust-aware approach, however, we can look beyond object $o_1$ to
    consider the \emph{trustworthiness} of $s$ and $t$. Indeed, when it comes
    to $o_2$, $t$ agrees with two extra sources $u$ and $v$, whereas $s$
    disagrees with everyone. In principle there could be many extra sources
    here instead of just two, in which case the effect would be even more
    striking. We may therefore postulate that $s$ is \emph{less trustworthy}
    than $t$. Returning to $o_1$, we see that the label $-$ is supported by a
    more trustworthy source, and conclude that it should be accepted over $-$.
\end{example}

Many truth discovery algorithms have been proposed in the literature with a
wide range of techniques used, e.g. iterative heuristic-based
methods~\cite{pasternack2010,galland2010}, probabilistic models~\cite{yin2008},
maximum likelihood estimation and optimisation-based methods~\cite{li2016}, and
neural network
models~\cite{kotonya2020explainable,marshall2017neural,wang2018eann}. It is
common for such algorithms to be evaluated empirically by running them against
real-world or synthetic datasets for which the true facts are already known;
this allows \emph{accuracy} and other metrics to be calculated, and permits
comparison between algorithms (see \cite{waguih2014truth} for a systematic
empirical evaluation of this kind). This may be accompanied by some theoretical
analysis, such as calculating run-time complexity~\cite{gupta2011survey},
proving convergence of an iterative algorithm~\cite{yin_supervised_2011}, or
proving convergence to the ``true'' facts under certain assumptions on the
distribution of source
trustworthiness~\cite{xiao2016,xiao_thesis2018,ghosh_2011}.

A limitation of this kind of analysis is that the results only apply narrowly
to particular algorithms, due to the assumptions made (for instance, that
claims from sources follow a particular probability distribution). Such
assumptions can be problematic in domains where the desired truth is somewhat
``fuzzy''; for example, image classification problems and determining the
copyright status of
books.\footnote{\url{https://www.nytimes.com/2019/08/19/technology/amazon-orwell-1984.html}}

In this work we take first steps towards a more general approach, in which we
aim to study truth discovery without reference to any specific methodology or
probabilistic framework. To do so we note the similarities between truth
discovery and problems such as judgement aggregation~\cite{endriss2016ja},
voting theory~\cite{zwicker2016voting} ranking and recommendation
systems~\cite{altman2008,altman2005ranking,andersen2008,tennenholtz2004} in
which the \emph{axiomatic approach} of social choice has been successfully
applied.
%
In taking the axiomatic approach one aims to formulate \emph{axioms} that
encode intuitively desirable properties that an algorithm may possess. The
interaction between these axioms can then be studied; typical results include
\emph{impossibility results}, where it is shown that a set of axioms cannot
hold simultaneously, and \emph{characterisation results}, where it is shown
that a set of axioms are uniquely satisfied by a particular algorithm.

Such analysis brings a new \emph{normative} perspective to the truth discovery
literature. This complements empirical evaluation: in addition to seeing how
well an algorithm performs in practise on test datasets, one can check how well
it does against theoretical properties that any ``reasonable'' algorithm should
satisfy. The satisfaction (or failure) of such properties then shines new light
on the \emph{intuitive behaviour} of an algorithm, and may guide development of
new ones.

With this in mind, we develop a framework for truth discovery in
which axioms can be formulated, and go on to give impossibility results and an
axiomatic characterisation of a baseline voting algorithm. We also define the
class of \emph{recursive} truth discovery algorithms, which includes most
examples from the literature. We outline several specific examples:
\emph{Sums}~\cite{pasternack2010},
\emph{TruthFinder}~\cite{yin2008} and \emph{CRH}~\cite{li2016}, and
analyse Sums in more detail with respect to the axioms. Surprisingly, Sums
fails some crucial axioms, which leads us to introducing a modified version
with better axiomatic properties.

However, as a first step towards a social choice perspective of truth
discovery, our framework involves a number of simplifying assumptions not
commonly made in the truth discovery literature.

\begin{itemize}

    \item \textbf{Collusion.} Our axioms assume sources act independently, in
        that there is no \emph{collusion} or
        \emph{copying}~\cite{dong_truth_2009} among sources.

    \item \textbf{Object correlations.} We do not model \emph{correlations}
          between the objects of interest in the truth discovery
          problem~\cite{yang_probabilistic_2019}. For example, the
          crowdsourcing example it may be known in advance that objects $o_1$
          and $o_2$ are similar, so that their true labels are correlated; this
          cannot be expressed in our framework.

    \item \textbf{Ordinal outputs.} For the most part, the outputs of our truth
          discovery methods consist of \emph{rankings} of the sources and
          facts. Thus, we describe when a source is considered \emph{more
          trustworthy} than another, but do not assign precise numerical values
          representing trustworthiness. This breaks with tradition in the truth
          discovery literature, but is a common point of view in social choice
          theory.
\end{itemize}

While this is something of a simplification compared to the current body of
work in truth discovery, we argue that the problem is non-trivial even in this
simplified setting, and that interesting axioms can still be put forth. The
framework as set out here lays the groundwork for these assumptions to be
lifted in future work.

\paragraph{Contributions.}

The primary contribution of this chapter is a mathematical framework for truth
discovery, which allows for axiomatic analysis of truth discovery algorithms in
the style of social choice theory. We introduce several axioms -- many inspired
by similar axioms in the social choice literature -- which to date have not
been considered in relation to truth discovery. Moreover, we observe that one
particularly well-known truth discovery method fails one of our axioms, and
propose a modification to resolve this issue.

This chapter is a significantly re-worked version of previously published
work~\cite{singleton_booth_2020,singleton2022towards}.

\section{Preliminaries}
\label{td_new_sec_preliminaries}

In this section we give the basic definitions which form our formal framework.

\paragraph{Input.}

Intuitively, a truth discovery problem consists of a number of \emph{sources}
and a number of \emph{objects} of interest. Each source provides a number of
\emph{claims}, where a claim is comprised of an object and a \emph{value}.
Different sources may give conflicting claims by providing different values for
the same object. For simplicity, we only consider categorical values in this
work. Note that while this restriction is made in some approaches in the
literature~\cite{pasternack2010,yin2008,wang_truth_2012,dong_truth_2009,zhang2018},
in general truth discovery methods also handle continuous
values~\cite{li2016,xiao2016}.

To formalise this, let $\S$, $\O$ and $\V$ be infinite, disjoint sets,
representing the possible sources, objects and values. The input to the truth
discovery problem is a \emph{network}, defined as follows.

\begin{definition}
    \label{td_new_def_network}
    A \emph{truth discovery network} is a tuple $N = (S, O, D, R)$, where
    \begin{itemize}
        \item $S \subseteq \S$ is a finite set of \emph{sources}.
        \item $O \subseteq \O$ is a finite set of \emph{objects}.
        \item $D = \{D_o\}_{o \in O}$ are the \emph{domains} of the objects,
              where each $D_o \subseteq \V$ is a finite set of values. We write
              $V = \bigcup_{o \in O}{D_o}$.
        \item $R \subseteq S \times O \times V$ is a set of \emph{reports}.
    \end{itemize}
    such that
    \begin{enumerate}
        \item\label{td_new_item_val_in_domain}
            For each $(s, o, v) \in R$, we have $v \in D_o$.
        \item\label{td_new_item_sources_self_consistent}
            If $(s, o, v) \in R$ and $(s, o, v') \in R$, then $v = v'$.
    \end{enumerate}
\end{definition}

Note that while $\S$, $\O$ and $\V$ are infinite, each network is finite. The
set $R$ is the core data associated with the network: we interpret $(s, o, v)
\in R$ as source $s$ claiming that $v$ is the true value for object $o$.
Constraint \cref{td_new_item_val_in_domain} says that all claimed values are in
the domain of the relevant object. Constraint
\cref{td_new_item_sources_self_consistent} is a basic consistency
requirement: a source cannot provide distinct values for a single object. That
is, a source provides \emph{at most one value} per object.  Thus, while sources
may be in conflict with \emph{other sources}, they are not in conflict with
themselves.  While this is a simplifying assumption, we argue the truth
discovery problem is still rich enough when conflicts only arise between
distinct sources.

When a network $N$ is understood, we often write $S, O, D$ and $R$ to
implicitly refer to the components of $N$. Any decoration applied to $N$ will
also be applied to its components (e.g. $N'$ has sources $S'$, $\hat{N}$ has
sources $\hat{S}$ etc\ldots). If necessary, we write $S_N, O_N, D_N$ and $R_N$
to make the dependence on $N$ explicit.

A \emph{claim} is a pair $c = (o, v)$, where $o \in O$ and $v \in D_o$. We
write $\obj(c) = o$ in this case, and let $C$ denote the set of all possible
claims in a network $N$, i.e.
\[
    C = \{(o, v) \mid o \in O, v \in D_o\}.
\]
Note that not every claim is necessarily reported by some source. With slight
abuse of notation, we write $(s, c)$ for the report $(s, o, v)$. Then $R$ can
be viewed as a subset of $S \times C$, i.e. a relation between sources and
claims. In fact, we will take this claim-centric view in the remainder of the
chapter, with objects and values only playing a role insofar as they tell us
which claims are in conflict with one another.

\begin{figure}
    \centering
    \begin{tikzpicture}[thick]
        \LARGE
        \networkinit{s,t,u,v}{c,d,e,f}
        \object{o_1}{c}{d};
        \object{o_2}{e}{f};
        \report{s}{c};
        \report{s}{e};
        \report{t}{d};
        \report{t}{f};
        \report{u}{f};
        \report{v}{f};
    \end{tikzpicture}
    \caption{
        Claim-centric presentation of the network described in
        \cref{td_new_fig_intro_example_with_values,td_new_ex_intro_example}.
    }
    \label{td_new_fig_intro_example}
\end{figure}

\begin{example}
    \label{td_new_ex_intro_example_formalised}
    The network illustrated in
    \cref{td_new_fig_intro_example_with_values,td_new_ex_intro_example} is
    given by $S = \{s, t, u, v\}$, $O = \{o_1, o_2\}$ and $D_{o_1} = D_{o_2} =
    \{+, -\}$. Labelling the claims $c = (o_1, +)$, $d = (o_1, -)$, $e = (o_2,
    +)$ and $f = (o_2, -)$, we have
    $
        R = \{
            (s, c), (s, e),
            (t, d), (t, f),
            (u, f),
            (v, f)
        \}
    $. This ``claim-centric'' view of the network is shown in
    \cref{td_new_fig_intro_example}, where the values $+$ and $-$ are
    suppressed.
\end{example}

\cref{td_new_ex_intro_example_formalised} highlights a special case of our
framework: the ``binary'' case in which the domain of each object consists of
two values $D_o = \{+, -\}$.  In this case we can
think of each object as a propositional variable. This brings us close to the
setting studied in \emph{judgement aggregation}~\cite{endriss2016ja} and,
specifically (since sources do not necessarily provide a claim for each
object) to the setting of \emph{binary aggregation with
abstentions}~\cite{christoffbinary,dokow}. An important difference, however, is
that for simplicity we do not assume any \emph{constraints} on the possible
configurations of true claims across objects. That is, any combination of truth
values is feasible. In judgement aggregation such an assumption has the effect
of neutralising the impossibility results that arise in that domain (see
e.g.,~\cite{christoffbinary}). We shall see later that that is not the case in
our setting.

\paragraph{Notation.}

We introduce some notation to extract information about a network. For $c \in
C$ and $s \in S$, write
\begin{align*}
    \src_N(c) &= \{s \in S \mid (s, c) \in R\},\\
    \claims_N(s) &= \{c \in C \mid (s, c) \in R\}.
\end{align*}
The set of sources making a claim on object $o$ is
\[
    \src_N(o) = \bigcup\{\src_N(c) \mid c \in C, \obj(c) = o\}.
\]
The claims associated with $o$ are
\[
    \claims_N(o) = \{c \in C \mid \obj(c) = o\}.
\]
The set of claims in conflict with a given claim $c = (o, v)$, i.e.  claims for
$o$ with a value other than $v$, is denoted by
\[
    \conflict_N(c) = \{(o, v') \mid v' \in D_o \setminus \{v\}\}.
\]
The ``antisources'' of $c$ are then defined to be the sources for claims
conflicting with $c$:
\[
    \antisrc_N(c) = \bigcup\{\src_N(d) \mid d \in \conflict_N(c)\}.
\]
Note that property \cref{td_new_item_sources_self_consistent} in the definition
of a network ensures $\src_N(c) \cap \antisrc_N(c) = \emptyset$.

\paragraph{Output.}

With the input defined, we now come to the output of the truth discovery
problem.
The primary goal is to produce an assessment of the trustworthiness of the
sources, and the \emph{true values} for the objects. Approaches differ
regarding values: some truth discovery methods output only a single value for
each object~\cite{li2016,ding_finding_2016,yang_continuous_2018}, whereas
others give an assessment of the believability (or confidence, probability
etc\ldots) of \emph{each claim} $(o,
v)$~\cite{yin2008,pasternack2010,galland2010,zhi2015,zhang_robust_2016,zhang2018}.
We opt for the latter, more general, approach.

On the specific form of these assessments, we face a tension between the social
choice and truth discovery perspectives. In social choice theory, one generally
looks at \emph{rankings}: e.g. the ranking of candidates in an election result
according to a voting rule. Consequently, axioms are generally \emph{ordinal
properties}, which constrain how candidates (for example) compare
\emph{relative to each other}. In contrast, truth discovery methods universally
use \emph{numeric values}. This is more convenient for defining and using truth
discovery methods in practise, and induces a ranking by simply comparing the
numeric scores. The magnitude of the differences between scores also gives
information about \emph{confidence} in distinguishing sources and claims.

However, numeric scores are often not comparable between
different methods (for example, some methods output probabilities, whereas
others are interpreted as weights which may take negative values) and in
general may not carry any semantic meaning at all. This means that meaningful
axioms for truth discovery should not refer to specific numeric scores, but
only the ranking they introduce.

We will ultimately take a hybrid approach: our methods and examples will be
defined in terms of numeric scores, but the axioms will only refer to ordinal
properties. This approach is summarised succinctly by \textcite{altman2008},
who write of ranking systems: ``We feel that the numeric approach is more
suitable for defining and executing ranking systems, while the global ordinal
approach is more suitable for axiomatic classification.''

An \emph{operator} maps each network to score and claim scores.

\begin{definition}
    A \emph{truth discovery operator} $T$ maps each network $N$ to a function
    $T_N: S_N \cup C_N \to \R$.
\end{definition}

Intuitively, the higher the score $T_N(s)$ for a source $s \in S$, the
\emph{more trustworthy} $s$ is, according to $T$ on the basis of $N$.
Similarly, the higher $T_N(c)$ for a claim $c \in C$, the \emph{more
believable} $c$ is deemed to be. We define the source and claim rankings
associated with $T$ and $N$ by
\begin{align*}
    s \sle_N^T s' &\iff T_N(s) \le T_N(s'), \\
    c \cle_N^T c' &\iff T_N(c) \le T_N(c').
\end{align*}
Then $s \sle_N^T s'$ if $s'$ is at least as trustworthy as $s$, and similar for
$\cle_N^T$. Note that $\sle_N^T$ and $\cle_N^T$ are total
preorders.\footnotemark{} We denote
the strict parts by $\slt_N^T$ and $\clt_N^T$ respectively, and the symmetric
parts by $\seq_N^T$ and $\ceq_N^T$. We omit the sub- and super-scripts when
$N$ and $T$ are clear from context.

\footnotetext{
    A total preorder is a transitive and complete binary relation.
}

Given that our axioms will only refer to the rankings produced by operators,
two operators yielding exactly the same rankings -- possibly with different
scores -- appear the same with respect to axiomatic analysis. We say operators
$T$ and $T'$ are \emph{ranking equivalent}, denoted $T \rankequiv T'$, if for
all networks $N$ we have ${\sle_N^T} = {\sle_N^{T'}}$ and ${\cle_N^T} =
{\cle_N^{T'}}$.

In \cref{td_new_sec_example_operators} we will introduce operators defined as
the limit of an iterative procedure. To allow for possible non-convergence we
also consider \emph{partial operators}, which assign a mapping $T_N: S \cup C
\to \R$ for only a subset of networks.

\section{Example Operators}
\label{td_new_sec_example_operators}

In this section we capture several example operators from the literature in our
framework: a baseline \emph{voting} method and its generalisation to
\emph{weighted} voting, \emph{Sums}~\cite{pasternack2010},
\emph{TruthFinder}~\cite{yin2008} and \emph{CRH}~\cite{li2016}. As is the case
with many methods in the literature, the latter three methods operate
iteratively: starting with an initial estimate, scores are repeatedly updated
according to some procedure until convergence. Typically the update procedure
is recursive, with source scores being updated on the basis of the current
claims scores, and vice versa.  To simplify the definition and analysis of such
methods, we will introduce the class of \emph{recursive operators}.

\subsection{Voting}
\label{td_new_sec_voting}

It is common in the literature to evaluate truth discovery methods against a
non-trust-aware method, such as a simple voting procedure.\footnotemark{} Here
we consider each source to ``vote'' for their claims, and claims are ranked
according to the number of votes received, i.e. by $|\src_N(c)|$. While this
ignores the trust aspect of truth discovery entirely, this method will be
useful for us as an axiomatic baseline. For example, axioms which aim to
address the trust aspect should not hold for voting, and an axiom referring to
the ranking of claims may be too strong if it does hold for voting.

\footnotetext{
    This is often called \emph{majority voting} in the truth discovery
    literature (e.g.~\cite{li_survey_2016,xiao_22,li2016}), but using the
    terminology of social choice theory it is better described as
    \emph{plurality voting}.
}

\begin{definition}
    $\voting$ is the operator defined by
    \begin{align*}
        \voting_N(s) &= 1,\\
        \voting_N(c) &= |\src_N(c)|.
    \end{align*}
\end{definition}

Applying $\voting{}$ to the network in \cref{td_new_fig_intro_example}, we have
that all sources rank equally ($s \seq t \seq u \seq v$) and $c \ceq d \ceq e
\clt f$.

The problem with $\voting{}$ is that all reports are equally weighted. If we
have a mechanism by which sources can be weighted by trustworthiness, the idea
behind voting may still have some merit. We define \emph{weighted voting} as
follows.

\begin{definition}
    A \emph{weighting} $w$ maps each network $N$ to a function $w_N: S \to \R$.
    The associated \emph{weighted voting} operator $\wvoting{w}$ is defined by
    \begin{align*}
        \wvoting{w}_N(s) &= w_N(s), \\
        \wvoting{w}_N(c) &= \sum_{s \in \src_N(c)}{w_N(s)}.
    \end{align*}
\end{definition}

Note that $\voting$ arises via the weighting $w_N \equiv 1$. Note that a
weighting is essentially just half of a truth discovery operator, where we only
output scores for sources. This is completed to an operator $\wvoting{w}$ by
letting the score for a claim be the sum of the weights of its sources. Note
also that we allow the possibility of ``untrustworthy'' sources with $w_N(s) <
0$.  Reports from such sources \emph{decrease} the credibility of a claim.

\begin{example}
    \label{td_new_ex_weighted_voting}
    Set
    \[
        \wagree_N(s) = {
            \sum_{c \in \claims_N(s)}{
                \frac{|\src_N(c)|}{|\claims_N(s)|}
            }
        }.
    \]
    Then the weight assigned to a source $s$ is the average number of sources
    agreeing with the claims of $s$. We call the corresponding operator
    \emph{Weighted Agreement}. Taking $N$ from
    \cref{td_new_fig_intro_example}, we have $\wagree_N(s) = 1$, $\wagree_N(t) = 2$,
    $\wagree_N(u) = 3$, $\wagree_N(v) = 3$. Consequently,
    \begin{align*}
        \wvoting{\wagree}_N(c) &= \wagree_N(s) = 1,\\
        \wvoting{\wagree}_N(d) &= \wagree_N(t) = 2,\\
        \wvoting{\wagree}_N(e) &= \wagree_N(s) = 1,\\
        \wvoting{\wagree}_N(f) &= \wagree_N(t) + \wagree_N(u) + \wagree_N(v) = 8,
    \end{align*}
    yielding the rankings $s \slt t \slt u \seq v$ and $c \ceq e \clt d \clt
    f$. Note that claim $d$ fares better here than with $\voting$ due to its
    association with source $t$, who is more trustworthy than $s$.
\end{example}

As we will see in \cref{td_new_sec_satisfaction_of_the_axioms}, some operators do not correspond
exactly to a weighting $w$, but give rise to the same rankings. Let us say an
operator $T$ is \emph{weightable} if there exists a weighting $w$ such that $T
\rankequiv \wvoting{w}$. Given that weighted voting expresses a clear
relationship between source and claim scores, this notion will simplify
some aspects of axiomatic analysis later.

\subsection{Recursive Operators}
\label{td_new_sec_recursive_operators}

To capture the mutual dependence between trust in sources and belief in claims,
truth discovery methods generally involve recursive
computation~\cite{pasternack2010,yin2008,yang_probabilistic_2019,du2019,zhang2018,li2016,galland2010,zhi2015}.
Claim scores are updated on the basis of currently estimated source scores,
before claim scores are updated on the basis of the new sources scores. If this
process converges, the limiting scores should be a fixed-point of the update
procedure, reflecting the desired mutual dependence. To formalise this idea, we
define recursive operators.

\begin{definition}
    \label{td_new_def_recursive_scheme}
    A \emph{recursive scheme} is a tuple $(\D, T^0, U)$, where
    \begin{itemize}
        \item $\D$ is a set of operators.
        \item $T^0 \in \D$ is the \emph{initial operator}.
        \item $U: \D \to \D$ is the \emph{update function}.
    \end{itemize}
    A recursive scheme \emph{converges} to an operator $T^*$ if for all
    networks $N$ and all $z \in S \cup C$, $\lim_{n \to \infty}{U^n(T_0)_N(z)}
    = T^*_N(z)$. In this case $T^*$ is said to be the \emph{limit} of the
    scheme.
\end{definition}

The main component of interest here is the update function $U$, which describes
how the scores of one iteration are transformed to obtain scores for the next.
The domain of operators $\D$ is used for technical reasons; for example, some
operators need to exclude the trivial operator in which scores are identically
zero in order for $U$ to be well-defined.

Note that the limit operator $T^*$ is unique, when it exists. We can consider
any scheme to converge to a \emph{partial} operator $T^*$, defined on the
networks $N$ such that $\lim_{n \to \infty}{U^n(T_0)_N(z)}$ exists for all $z
\in S \cup C$. We now consider examples of recursive operators from the
literature.

\paragraph{Sums.}

Sums~\cite{pasternack2010} is a simple and well-known operator adapted from the
\emph{Hubs and Authorities}~\cite{kleinberg1999} algorithm for ranking web
pages. The premise is to extend the linear sum of weighted voting to both claim
and source scores: we update the score of each source as the sum of the scores
of its claims, and update the score of each claim as the sum of the scores of
its sources. To prevent scores from growing without bound, they are normalised
at each iteration by dividing by the maximum score (for sources and claims
separately).

\begin{definition}
    \label{td_new_def_sums}
    \emph{Sums} is the recursive scheme $(\D, T^0, U)$, where $\D$ is the set
    of all operators with scores in $[0, 1]$, $T^0_N \equiv 1 / 2$, and $U(T) =
    T'$, with
    \begin{align*}
        T'_N(s) &=
            \alpha
            \sum_{c \in \claims_N(s)}{
                T_N(c)
            },
        \\
        T'_N(c) &=
            \beta
            \sum_{s \in \src_N(c)}{
                T'_N(s)
            }.
    \end{align*}
    where $
        \alpha = 1 / \max_{t \in S}{
            \left|
                \sum_{c \in \claims_N(t)}{
                    T_N(c)
                }
            \right|
        }
    $ and $
        \beta = 1 / \max_{d \in C}{
            \left|
                \sum_{s \in \src_N(d)}{
                    T'_N(s)
                }
            \right|
        }
    $ are normalisation factors (which we set to 0 if the denominator is 0).
    Write $\sums$ for the associated limit operator.
\end{definition}

Taking the network $N$ from \cref{td_new_fig_intro_example}, one can show that
$\sums_N(s) = 0$, $\sums_N(t) = 1$ and $\sums_N(u) = \sums_N(v) = \sqrt{2} / 2
\approx 0.7071$, giving a source ranking $s \slt u \seq v \slt t$.  For claims,
we have $\sums_N(c) = \sums_N(e) = 0$, $\sums_N(d) = \sqrt{2} - 1 \approx
0.4142$ and $T_N(f) = 1$, giving a claim ranking $c \ceq e \clt d \clt f$. Note
that the claim ranking is identical to that of
\cref{td_new_ex_weighted_voting}. For sources, we see that $t$ moves strictly
upwards in the ranking compared to \cref{td_new_ex_weighted_voting}.
Intuitively, this is because source $t$ claims a superset of the claims of $u$
and $v$, so receives more weight from its claims at each iteration.

\paragraph{TruthFinder.} TruthFinder~\cite{yin2008} is a pseudo-probabilistic
method, and was defined in the first work to introduce (and coin the phrase)
truth discovery. It is formulated in a setting more general than ours: the
authors suppose claims may \emph{support} each other, as well as conflict, and
that support of conflict may occur to varying degrees. Formally, each pair of
claims $c, c'$ has an ``implication'' value $\operatorname{imp}(c \to c') \in
[-1, 1]$, where a negative value implies confidence in $c$ should decrease
confidence in $c'$, and a positive value implies confidence in $c$ should
\emph{increase} confidence in $c'$. In contrast, our framework assumes claims
for the same object are mutually exclusive, so that all implications are
negative. To express TruthFinder in our framework, we take
$\operatorname{imp}(c \to c')$ to be $-\lambda$ if $c$ and $c'$ have the same
object and $0$ otherwise, for some fixed parameter $0 \le \lambda \le 1$.

\begin{definition}
    Given parameters $\rho, \gamma \in (0, 1)$ and $\lambda \in [0, 1]$,
    \emph{TruthFinder} is the recursive scheme $(\D, T^0, U)$, where $\D$ is
    the set of operators with $0 < T_N(s) < 1$ for all $N$ and $s \in S$ with
    $\claims_N(s) \ne \emptyset$, $T^0 \equiv 0.9$, and $U(T) = T'$, with
    \begin{align}
        \label{td_new_eqn_truth_finder_claim_update}
        T'_N(c) &= \left[
            1 +
            \frac{
                \prod_{s \in \src_N(c)}{(1 - T_N(s))^\gamma}
            }{
                \prod_{t \in \antisrc_N(c)}{(1 - T_N(t))^{\gamma\rho\lambda}}
            }
        \right]^{-1}, \\
        T'_N(s) &= \sum_{c \in \claims_N(s)}{
            \frac{T'_N(c)}{|\claims_N(s)|}
        }.
    \end{align}
    We write $\truthfinder$ for the associated limit operator.
\end{definition}

We refer the reader to the original TruthFinder paper~\cite{yin2008} for the
interpretation of $\rho$ and $\gamma$. As described above, $\lambda$ controls
the amount to which conflicting claims play a role in the evaluation of a given
claim. Of special interest is the case $\lambda = 0$, in which the denominator
in \cref{td_new_eqn_truth_finder_claim_update} is $1$. Note that in
\cref{td_new_eqn_truth_finder_claim_update} we have unfolded the definitions
of~\cite{yin2008} in order to obtain a single expression of $T'_N(c)$ in terms
of the $T_N(s)$, at the expense of interpretability.

Let us return again to the network in \cref{td_new_fig_intro_example}. We take
parameters $\rho = 0.5$ and $\gamma = 0.3$ (as per the experimental setup of
\textcite{yin2008}) and $\lambda = 0.5$. Assuming that TruthFinder does indeed
converge on this network -- as it appears to do empirically -- we have
$\truthfinder_N(s) \approx 0.5067$, $\truthfinder_n(t) \approx 0.6590$ and
$\truthfinder_N(u) = \truthfinder_N(v) = 0.7510$, which gives the ranking $s
\slt t \slt u \seq v$ on the sources. We have $\truthfinder_N(c) \approx
0.5328$, $\truthfinder_N(d) \approx 0.5670$, $\truthfinder_N(e) \approx 0.4807$
and $\truthfinder_N(f) \approx 0.7510$, which gives the ranking $e \clt c \clt
d \clt f$ on the claims. Note that the source ranking coincides with that of
\cref{td_new_ex_weighted_voting}, and the claim ranking refines that of
\cref{td_new_ex_weighted_voting} and Sums by ranking $e$ \emph{strictly} worse
than $c$. Intuitively, this occurs because $e$ has more sources reporting the
conflicting claim (namely, $f$) than $c$ does. If we instead take $\lambda =
0$, so that sources for conflicting claims are not considered, then the ranking
reverts to $c \ceq e \clt d \clt f$ (and the source ranking remains the same).

\paragraph{CRH.} Standing for ``Conflict Resolution on Heterogeneous Data'', CRH
is an optimisation-based framework for truth discovery~\cite{li2016}. It is
again set in a more general setting, in which a metric $d_o$ is available to
measure the distance between values in $D_o$, for each object $o$. The
optimisation problem jointly chooses weights for each source and a value for
each object, such that the weighted sum of $d_o$-distances from each source's
claim on $o$ is minimised.

To express CRH in our framework we use the ``probabilistic'' encoding of
categorical variables as described in \cite[\sectionsymbol{2.4.1}]{li2016},
where each categorical value is represented as a one-hot vector, and the source
weight regularisation from \cite[Eq. (4)]{li2016}. We make a minor
modification, however, by adding a small quantity $\epsilon$ to $\alpha_s$ and
$T'_N(s)$ defined below; this ensures the logarithm in $T'_N(s)$ and the
division in $T'_N(c)$ is well-defined and simplifies analysis of CRH later on.

\begin{definition}
    \label{td_new_def_crh}
    Given $\epsilon > 0$, \emph{CRH-$\epsilon$} is the recursive scheme $(\D,
    T^0, U)$, where $\D$ is the set of operators with $0 \le T_N(c) \le 1$ for
    all $N$ and $c \in C$,
    \[
        T^0_N(s) = 0, \qquad
        T^0_N(c) = \frac{|\src_N(c)|}{|S|}.
    \]
    and $U(T) = T'$, where
    \begin{align*}
        T'_N(s) &= \epsilon - \log\left(\frac{\alpha_s}{\sum_{t \in S}{\alpha_t}}\right), \\
        T'_N(c) &=
            \frac{\sum_{s \in \src_N(c)}{T'_N(s)}}{\sum_{t \in S}{T'_N(t)}},
    \end{align*}
    with
    \[
        \alpha_s = \epsilon + \sum_{c \in \claims_N(s)}{
            \sum_{d \in \claims_N(\obj(c))}{
                (T_N(d) - \indicator{d = c})^2
            }
        }.
    \]
    The limit operator is denoted by $\crh{\epsilon}$.\footnote{In the
    degenerate case $S = \emptyset$, we set $T_N \equiv 0$.}
\end{definition}

Note that in the case where each source provides a report on \emph{all} objects
-- which is the setting in which CRH was originally introduced -- we have
$\sum_{c \in \claims_N(o)}{T'_N(c)} = 1$. Consequently, $T'_N$ gives rise to a
probability distribution over claims for each object $o$. The term of the sum
in $\alpha_s$ corresponding to $c$ is the squared Euclidean distance between
this distribution and the distribution put forward by source $s$, which places
all the probability mass in their report $c$.

In the network from \cref{td_new_fig_intro_example} with $\epsilon = 10^{-5}$,
we have $\crh{\epsilon}_N(s) \approx 0.2577$, $\crh{\epsilon}_N(t) \approx
1.4827$ and $\crh{\epsilon}_N(u) = \crh{\epsilon}_N(v) \approx 9.3567$, giving
the source ranking $s \slt t \slt u \seq v$.  Note that this is the same
ranking on sources as $\truthfinder$ gives. For claims, we have
$\crh{\epsilon}_N(c) = \crh{\epsilon}_N(e) \approx 0.0126$,
$\crh{\epsilon}_N(d) \approx 0.0725$ and $\crh{\epsilon}_N(f) \approx 0.9874$,
giving the ranking $c \ceq e \clt d \clt f$; this is the same as $\sums$.

\cref{td_new_tab_example_outputs} summaries the source and claim rankings for
each example operator on the network $N$ from \cref{td_new_fig_intro_example}.

\begin{table}
    \centering
	\caption{Output rankings of the example operators on the network from
    \cref{td_new_fig_intro_example}.}
	\begin{tabular}{lcc}
    \toprule
    Voting             & $s \seq t \seq u \seq v$ & $c \ceq d \ceq e \clt f$ \\
    Weighted Agreement & $s \slt t \slt u \seq v$ & $c \ceq e \clt d \clt f$ \\
    Sums               & $s \slt u \seq v \slt t$ & $c \ceq e \clt d \clt f$ \\
    TruthFinder        & $s \slt t \slt u \seq v$ & $e \clt c \clt d \clt f$ \\
    TruthFinder ($\lambda = 0$) & $s \slt t \slt u \seq v$ & $c \ceq e \clt d \clt f$ \\
    CRH-$\epsilon$     & $s \slt t \slt u \seq v$ & $c \ceq e \clt d \clt f$ \\
    \bottomrule
	\end{tabular}
    \label{td_new_tab_example_outputs}
\end{table}

\section{The Axioms}
\label{td_new_sec_axioms}

Having laid out the formal framework, we now introduce axioms for truth
discovery. Such axioms are formal properties an operator may satisfy, which
encode intuitively desirable behaviour. Many of our axioms are adaptations of
axioms for various problem in social choice theory (e.g. from
voting~\cite{zwicker2016voting} and ranking systems~\cite{altman2008}), in
which the axiomatic method has seen great success. We also consider standard
social choice axioms which are \emph{not} desirable for truth discovery, to
highlight the differences with classical problems such as voting. We will later
revisit the example operators of the previous section to see to what extent our
axioms hold in practise.

\subsection{Coherence}
\label{td_new_sec_coherence}

The guiding principle of truth discovery is that claims backed by trustworthy
sources should be believed, and sources making believable claims are
trustworthy. All truth discovery methods aim to implement this principle to
some extent, and the examples of \cref{td_new_sec_example_operators} illustrate
several different approaches.

We aim to formulate this principle axiomatically as a \emph{coherency} property
relating the source ranking $\sle$ and the claim ranking $\cle$: sources making
higher $\cle$-ranked claims should rank highly in $\sle$, and vice versa. To do
so we adapt the idea behind the \emph{Transitivity} axiom of
\textcite{altman2008} for ranking systems.

Now, a difficulty arises when considering how to compare the claims of two
sources. For a simple example, suppose sources have either \emph{low},
\emph{medium} or \emph{high} trustworthiness. How should we rank a claim $c$
with one \emph{medium} sources versus a claim $d$ with a \emph{low} and a
\emph{high} source? In some situations we may want to prioritise the number of
claims, so that $d$ is preferred. In others we may want to avoid trusting
\emph{low} sources as much as possible, so that $c$ is preferred. The third
option of ranking $c$ and $d$ equally believable is also reasonable.

To avoid these ambiguous cases, we focus on scenarios where there is an
``obvious'' ordering between two sets of claims (or sources). For example,
consider the network depicted in \cref{td_new_fig_coherence_intro}. Suppose an
operator gives a source ranking $s \slt u \slt t \slt v$. Note that claims $c$
and $d$ have the same number of sources. Moreover, we can pair up these sources
one-to-one such that the source for $c$ is less trustworthy than the
corresponding source for $d$: we have $s \slt u$ and $t \slt v$. On aggregate,
we may reasonably say that $\src_N(c)$ is less trustworthy (with respect to
$\sle$) than $\src_N(d)$. We should therefore have $c \clt d$; any operator
violating this has failed to realise the dependence between source
trustworthiness and claim believability. Similarly, this reasoning can be
applied to the set of claims from two sources.

\begin{figure}
    \centering
    \begin{tikzpicture}[thick]
        \LARGE
        \networkinit{s,t,u,v}{c,d}
        \object{o}{c}{d};
        \report{s}{c};
        \report{t}{c};
        \report{u}{d};
        \report{v}{d};
    \end{tikzpicture}

    \caption{
        A network illustrating \claimcoherence{}.
    }
    \label{td_new_fig_coherence_intro}
\end{figure}

This will form the basis of our first set of axioms. First, we formalise the
above idea of a one-to-one correspondence respecting a ranking.

\begin{definition}
    If $\le$ is a relation on a set $X$ and $A, B \subseteq X$, then $A$
    \emph{precedes $B$ pairwise} with respect to $\le$ if
    \begin{equation}
        \label{td_new_eqn_pwo}
        \exists f: A \to B \text{ bijective s.t. }
        \forall x \in A:\ x \le f(x).
    \end{equation}
    Say $A$ \emph{strictly precedes $B$} if $A$ precedes $B$ but $B$ does not
    precede $A$.
\end{definition}

If $f$ satisfies the condition in \cref{td_new_eqn_pwo}, we say $f$
\emph{witnesses} the fact that $A$ precedes $B$, and write
$\witprec{f}{A}{B}{\le}$.  Note that if $\le$ is a preorder on $X$, the
``precedes pairwise'' relation is a preorder on $2^X$.  Indeed, it is reflexive
(by considering the identity map $A \to A$, for each $A \subseteq X$) and
transitive (if $\witprec{f}{A}{B}{\le}$ and $\witprec{g}{B}{C}{\le}$, then
$\witprec{g \circ f}{A}{C}{\le}$). The strict pairwise order associated has a
natural interpretation, as we now prove: there must exist some $x$ in
\cref{td_new_eqn_pwo} for which the comparison is strict.

\begin{proposition}
    \label{td_new_prop_pwo_strict_part}
    Suppose $X$ is finite and $\le$ is a total preorder on $X$. Then $A$
    strictly precedes $B$ pairwise with respect to $\le$ if and only if there
    is $\witprec{f}{A}{B}{\le}$ such that there is some $x_0 \in A$ with $x_0 <
    f(x_0)$.
\end{proposition}

We need a  preliminary lemma.

\begin{lemma}
    \label{td_new_lemma_pwo_strict_helper}
    Suppose $\le$ is a total preorder on a finite set $X$ and $f: X \to X$ is
    an injective mapping such that $x \le f(x)$ for all $x \in X$. Then $x
    \approx f(x)$ for all $x$.
\end{lemma}

\begin{proof}
    Take $x \in X$. Consider the sequence of iterates $(f^n(x))_{n \ge 1}$.
    Since this is an infinite sequence taking values in a finite set, there
    must be some point at which the sequence repeats, i.e. there are $n, k \ge
    1$ such that $f^n(x) = f^{n + k}(x)$. Then $f(f^{n - 1}(x)) = f(f^{n + k -
    1}(x))$, so injectivity gives $f^{n - 1}(x) = f^{n + k - 1}(x)$. Repeating
    this argument, we find $x = f^0(x) = f^k(x)$. By hypothesis, $f(x) \le
    f^k(x)$, i.e. $f(x) \le x$. Since $x \le f(x)$ also, this gives $x \approx
    f(x)$ as required.
\end{proof}

\begin{proof}[Proof of \cref{td_new_prop_pwo_strict_part}]
    ``if'': Clearly $A$ precedes $B$. Suppose for contradiction that this is not
    strict. Then there is some $\witprec{g}{B}{A}{\le}$. Note that $g \circ f$
    is a bijection $A \to A$, and for all $x \in X$ we have $x \le f(x) \le
    g(f(x))$. By \cref{td_new_lemma_pwo_strict_helper}, $x \approx g(f(x))$. In
    particular, we have $f(x_0) \le g(f(x_0)) \approx x_0$, but this
    contradicts $x_0 < f(x_0)$.

    ``only if'': Suppose $A$ strictly precedes $B$. Then there is some
    $\witprec{f}{A}{B}{\le}$. Note that $f^{-1}$ is a bijection $B \to A$.
    Since $B$ does not precede $A$, there must be some $y_0 \in B$ such that
    $y_0 \not\le f^{-1}(y_0)$. By totality of $\le$, we get $f^{-1}(y_0) <
    y_0$. Taking $x_0 = f^{-1}(y_0)$, we are done.
\end{proof}

We are now ready to state our first two axioms.

\begin{axiomlist}
\begin{axiom}[\claimcoherence{}]
    If $\src_N(c)$ strictly precedes $\src_N(c')$ pairwise with respect to
    $\sle_N^T$, then $c \clt_N^T c'$.
\end{axiom}
\begin{axiom}[\sourcecoherence{}]
    If $\claims_N(s)$ strictly precedes $\claims_N(s')$ pairwise with respect to
    $\cle_N^T$, then $s \slt_N^T s'$.
\end{axiom}
\end{axiomlist}

In words, \claimcoherence{} says that whenever we can pair up the sources for
$c$ and $c'$ so that each source for $c$ is less trustworthy
than the corresponding source for $c'$ (and \emph{strictly} less, for at least
one pair of sources), then $c$ is strictly less believable than $c'$. Likewise,
\sourcecoherence{} says that if the claims of $s$ and $s'$ can be paired up
with the claims for $s$ less believable than the claims for $s'$, then $s$ is
strictly less trustworthy than $s'$.

\begin{example}
    \label{td_new_ex_coherence_ilustration}
    Consider the network $N$ from \cref{td_new_fig_intro_example} again, and
    consider Sums. Recall that $\sums$ gives the source ranking $s \slt u \seq
    v \slt t$, and claim ranking $c \ceq e \clt d \clt f$.

    Note that $\src_N(c) = \{s\}$ and $\src_N(d) = \{t\}$. Since $s \slt t$, we
    have that $\{s\}$ strictly precedes $\{t\}$ with respect to $\sle$.
    \claimcoherence{} therefore requires that $c \clt d$. Indeed, this does
    hold.

    For \sourcecoherence{}, note that $\claims_N(s) = \{c, e\}$ and
    $\claims_N(t) = \{d, f\}$. Since $c \clt d$ and $e \clt f$, we see that
    $\claims_N(s)$ strictly precedes $\claims_N(t)$ with respect to $\cle$.
    Accordingly, \sourcecoherence{} requires $s \slt t$, which does hold.

    So, $\sums$ satisfies both coherence properties for this specific network.
    We will analyse $\sums$ and the other examples more generally in
    \cref{td_new_sec_satisfaction_of_the_axioms}.
\end{example}

The reader may wonder why we only consider the \emph{strict} pairwise relation
in \claimcoherence{} (and \sourcecoherence{}). An alternative axiom might
require that $c \cle c'$ whenever $\src_N(s)$ precedes $\src_N(s')$ with
respect to $\sle$ (not necessarily strictly). However, this property implies
that $c \ceq c'$ whenever $\src_N(c) = \src_N(c')$. We have already seen an
example operator where this does not hold: TruthFinder ranks $e \clt c$ in the
network $N$ from \cref{td_new_fig_intro_example}, but $\src_N(c) = \src_N(e) =
\{s\}$. Intuitively, $c$ and $e$ are ``tied'' when it come to the quality of
their own sources, but there are fewer sources \emph{disagreeing} with $c$ (the
``antisources'') than $e$. Stating our coherence properties in the strict form
permits an operator to consider antisources in cases where there is no clear
comparison on the basis of sources alone.

Having said this, an operator with \claimcoherence{} is limited in the extent
to which it can take antisources into account. We formulate an antisource
version of coherence in \cref{td_new_sec_conflicting_claims}, and show that
it is incompatible with \claimcoherence{} when taken with some other basic
axioms.

\subsection{Symmetry}

A standard class of axioms in social choice theory express \emph{symmetry
properties}. In voting, for example, symmetry with respect to voters says that
a voting rule should not care about the ``names'' of the voters: if voters $i$
and $j$ swap their ballots, the election result remains the same (this is
called \emph{anonymity} in the literature). Similarly, symmetry with respect to
candidates says that if we re-label candidates, the outcome remains the same up
to re-labelling (this is called \emph{neutrality}). In general, symmetry
requires that the output of some process depends only on \emph{structural}
features of the input, not the specific ``names'' of the entities involved.

For truth discovery, we can consider symmetry with respect to sources, objects
and claims. The central concept is an \emph{isomorphism} between networks.

\begin{definition}
    An \emph{isomorphism} between networks $N$ and $N'$ is mapping $F: S \cup O
    \cup C \to S' \cup O' \cup C'$ such that
    \begin{enumerate}
        \item $\restrict{F}{S}$, $\restrict{F}{O}$ and $\restrict{F}{C}$ are
              bijections $S \to S'$, $O \to O'$ and $C \to C'$, respectively.
        \item For all $s \in S$ and $c \in C$: $(s, c) \in R$ iff $(F(s), F(c))
              \in R'$.
        \item For all $c \in C$, $\obj(F(c)) = F(\obj(c))$.
    \end{enumerate}
\end{definition}

That is, $F$ is a one-to-one correspondence between the sources, objects and
claims of $N$ and their $N'$ counterparts, which respects the structure of the
network. One can easily check that we also have $F(\src_N(c)) =
\src_{N'}(F(c))$ and $F(\claims_N(s)) = \claims_{N'}(F(s))$.
%
The symmetry axiom says an operator should not distinguish isomorphic networks.

\begin{axiom}[\symmetry{}]
    If $F$ is an isomorphism between $N$ and $N'$, then
    $s \sle_N^T s'$ iff $F(s) \sle_{N'}^T F(s')$ and $c \cle_N^T c'$ iff $F(c)
    \cle_{N'}^T F(c')$.
\end{axiom}

\begin{figure}
    \centering
    \begin{tikzpicture}[thick]
        \LARGE
        \networkinit{s,t,u,v}{c,d,e,f}
        \object{o_1}{c}{d};
        \object{o_2}{e}{f};
        \report{s}{c};
        \report{t}{c};
        \report{t}{e};
        \report{u}{d};
        \report{u}{f};
        \report{v}{c};
    \end{tikzpicture}
    \caption{
        A network isomorphic to the one shown in \cref{td_new_fig_intro_example}.
    }
    \label{td_new_fig_symmetry_example}
\end{figure}

We illustrate \symmetry{} with an example.

\begin{example}
    Consider the network $N$ from \cref{td_new_fig_intro_example} and $N'$ from
    \cref{td_new_fig_symmetry_example}, where we take the sources, objects and
    domains to be the same in both networks. Then $N$ and $N'$ are isomorphic
    via the mapping $F$ expressed in cycle notation as $(suv)(cf)(de)(o_1o_2)$.
    For example, $s$ plays the same role in $N$ as $u$ in $N'$, $c$ plays the
    same role in $N$ as $f$ in $N'$, the role of objects $o_1$ and $o_2$ are
    swapped, etc. \symmetry{} requires that the source and claim rankings in
    $N'$ are already determined by the rankings of $N$.  For example, if the
    source ranking in $N$ is $s \slt_N u \seq_N v \slt_N t$, we must have $u
    \slt_{N'} v \seq_{N'} s \slt_{N'} t$.
\end{example}

An \emph{automorphism} is an isomorphism $F$ from a network $N$ to itself. For
example, $F$ which swaps $u$ and $v$ in $N$ from
\cref{td_new_fig_intro_example} is an automorphism, since $u$ and $v$ play
exactly the same role in $N$. \symmetry{} implies that $u \seq v$, and in fact
this holds more generally.

\begin{proposition}
    \label{td_new_prop_automorphism}
    If $F$ is an automorphism on $N$ and $T$ satisfies \symmetry{}, then $s
    \seq_N^T F(s)$ and $c \ceq_N^T F(c)$, for all $s \in S$ and $c \in C$.
\end{proposition}

\begin{proof}
    We show $s \seq_N^T F(s)$ for all sources $s$; the result for claims is
    similar.
    %
    Take $s \in S$. Since $S$ is finite and $F$ restricts to a bijection $S \to
    S$, an argument identical to the one in the proof of
    \cref{td_new_lemma_pwo_strict_helper} shows there is some $k \ge 1$ such
    that $s = F^k(s)$.

    First suppose $s \sle_N^T F(s)$. By \symmetry{} we may apply $F$ to both
    sides; doing so repeatedly yields $F^n(s) \sle_N^T F^{n+1}(s)$ for all $n
    \ge 1$. By transitivity of $\sle_N^T$, we get $F(s) \sle_N^T F^n(s)$.
    Taking $n = k$ gives $F(s) \sle_N^T F^k(s) = s$, so $s \seq_N^T F(s)$.

    Now suppose $F(s) \sle_N^T s$. By an identical argument, $F^n(s) \sle_N^T
    F(s)$ for all $n \ge 1$; taking $n = k$ gives $s \sle_N^T F(s)$, so $s
    \seq_N^T F(s)$ again.

    Since $\sle_N^T$ is total these cases are exhaustive, and we are done.
\end{proof}

\cref{td_new_prop_automorphism} is useful for showing certain sources and
claims must rank equally. For example, take the network $N$ from
\cref{td_new_fig_coherence_intro}. Intuitively this network displays internal
symmetry within the sources for each claim and between the claims themselves.
Indeed, the functions $F = (st)(uv)$ and $G = (su)(tv)(cd)$ are automorphisms.
By \cref{td_new_prop_automorphism}, any operator $T$ satisfying \symmetry{}
must output flat rankings $s \seq t \seq u \seq v$ and $c \ceq d$.

\subsection{Monotonicity}

Given that voting is not a viable truth discovery method, the believability of
a claim $c$ should not increase monotonically with $|\src_N(c)|$. Moreover, it
should not increase with the \emph{set} of sources $\src_N(c)$, ordered by set
inclusion: $\src_N(c) \subseteq \src_N(d)$ should not in general imply $c \cle
d$. Indeed, consider an adversarial source $t$ deliberately making false
claims, and suppose $\src_N(c) = \{s\}$ and $\src_N(d) = \{s, t\}$. Then
$\src_N(c) \subseteq \src_N(d)$, but the extra support from $t$ should actually
\emph{decrease} the believability of $d$ -- since $t$ only provides false
claims -- not increase it.

Nevertheless, there is a sense in which -- all else being equal -- a claim with
more sources is more believable. The above examples show that some subtlety is
needed in formulating this as a general principle, and that trust should be
taken into account in doing so.

In this section we consider monotonicity properties of two kinds: monotonicity
\emph{within} a network, and monotonicity \emph{between} networks as more
reports are added. We start with the latter by adapting the idea of
\emph{positive responsiveness} from social choice theory.

\paragraph{Responsiveness.}

In the context of voting, positive responsiveness requires that if
a voter switches their vote from candidate $B$ to a winning candidate $A$, then
$A$ becomes the unique winner~\cite{zwicker2016voting}.
%
A naive version of positive responsiveness for truth discovery says that if we
change a network $N$ by adding a new report $(s, c)$ -- possibly removing
reports from $s$ conflicting with $c$ -- then $c$ should move strictly up in
the claim ranking. Clearly this neglects to consider the trustworthiness of
$s$, and is thus an undesirable property (e.g. consider $s$ adversarial as
described above). Our first monotonicity axiom weakens this naive property by
only considering ``fresh'' sources $s$ not providing any reports in the original
network $N$. Intuitively, we have no reason to believe such sources are
untrustworthy, and they should therefore have a positive effect when making a
claim.
%
In what follows, when $\claims_N(s) = \emptyset$ we write $\addclaim{N}{s}{c}$
for the network $(S, O, D, R \cup \{(s, c)\})$.

\begin{axiom}[\freshposresp{}]
    Suppose $\claims_N(s) = \emptyset$. Then for all $c \in C$ and $d \in C
    \setminus \{c\}$, $d \cle_N^T c$ implies $d \clt_{\addclaim{N}{s}{c}}^T c$.
\end{axiom}

That is, if $c$ was already at least as believable as $d$, then a fresh report
makes $c$ \emph{strictly} more believable in the new network.\footnotemark{}
%
What about the effects of a fresh report for $c$ on source trustworthiness?
According to the mutual dependence between the source and claim rankings --
captured in a static network via the coherence properties -- sources already
claiming $c$ should become more trusted, whereas those claiming a conflicting
claim $d$ should become less trusted.

\footnotetext{
    Note that $N$ and $\addclaim{N}{s}{c}$ share the same set of objects $O$
    and domains $D$, so the set of possible claims in both networks are the
    same. Consequently we are justified in treating $c$ and $d$ as claims in
    both networks.
}

\begin{axiom}[\sourceposresp{}]
    Suppose $s \in \antisrc_N(c)$, $t \in \src_N(c)$, and $\claims_N(u) =
    \emptyset$. Then $s \sle_N^T t$ implies $s \slt_{\addclaim{N}{u}{c}}^T t$.
\end{axiom}

\begin{figure}
    \centering
    \begin{tikzpicture}[thick]
        \LARGE
        \networkinit{s,t,u,v}{c,d,e,f}
        \object{o_1}{c}{d};
        \object{o_2}{e}{f};
        \report{s}{c};
        \report{s}{e};
        \report{t}{d};
        \report{t}{f};
        \report[dashed]{u}{f};
        \report[densely dashed]{v}{f};
    \end{tikzpicture}
    \caption{
        Networks $N_0$ (solid edges only), $N_1 = \addclaim{N_0}{u}{f}$ and
        $N_2 = \addclaim{N_1}{v}{f}$ illustrating \freshposresp{} and
        \sourceposresp{}.
    }
    \label{td_new_fig_sourceposresp_example}
\end{figure}

Note that \sourceposresp{} does not say anything about the ranking of the fresh
source $u$. We consider another example.

\begin{example}
    \label{td_new_ex_posresps}
    \cref{td_new_fig_sourceposresp_example} illustrates \freshposresp{} and
    \sourceposresp{}. Let $N_0$ denote the network including only the solid
    edges, $N_1 = \addclaim{N_0}{u}{f}$, and $N_2 = \addclaim{N_1}{v}{f}$. Note
    that $N_2$ is our running example network from
    \cref{td_new_fig_intro_example}.
    %
    Assuming \symmetry{}, everything is tied in $N_0$: we have $s \seq_{N_0} t$
    and $c \ceq_{N_0} d \ceq_{N_0} \ceq_{N_0} e \ceq_{N_0} f$. Since $N_1$ is
    the result of adding the report $(u, f)$ and $u$ makes no claims in $N_0$,
    \freshposresp{} gives $e \clt_{N_1} f$. Since $s \in \src_{N_0}(e)
    \subseteq \antisrc_{N_0}(f)$ and $t \in \src_{N_0}(f)$, \sourceposresp{}
    gives $s \slt_{N_1} t$. Going from $N_1$ to $N_2$ we can repeat exactly the
    same arguments to find $e \clt_{N_2} f$ and $s \slt_{N_2} t$.

    Bringing \claimcoherence{} in too, $s \slt_{N_2} t$ gives $c \clt_{N_2} d$.
    Thus, \claimcoherence{}, \symmetry{}, \freshposresp{} and \sourceposresp{}
    are enough to capture our intuitions about this network as described in the
    introduction.
\end{example}

In the special case where a network contains reports only for a single object,
the responsiveness properties and \symmetry{} actually force an operator to
rank claims by voting, and to rank sources by the vote count of their claims.
Note that each source provides at most one report in this case, by condition
\cref{td_new_item_sources_self_consistent} in the definition of a network.
Consequently there is little structure in such networks, as we cannot look at
how sources interact over multiple objects to determine trustworthiness. We
therefore argue that voting is reasonable behaviour in this special case.

\begin{proposition}
    \label{td_new_prop_symm_fpr_single_object_voting}
    Suppose there is $o \in O$ such that $\src_N(o') = \emptyset$ for all $o
    \ne o'$. Then
    \begin{enumerate}
        \item\label{td_new_item_prop_symm_fpr_single_object_voting_claims}
        If $T$ satisfies \symmetry{} and \freshposresp{}, then for all $c, d
            \in \claims_N(o)$:
        \[
            c \cle_N^T d \iff |\src_N(c)| \le |\src_N(d)|.
        \]
        \item\label{td_new_item_prop_symm_fpr_single_object_voting_sources}
        If $T$ satisfies \symmetry{} and \sourceposresp{}, then for all $s, t
            \in S$ with $\claims_N(s), \claims_N(t) \ne \emptyset$,
        \[
            s \sle_N^T t
            \iff
            |\src_N(c_s)| \le |\src_N(c_t)|,
        \]
        where $c_s$ and $c_t$ are the unique claims reported by $s$ and $t$
        respectively.
    \end{enumerate}
\end{proposition}

While \cref{td_new_prop_symm_fpr_single_object_voting} only addresses a
somewhat trivial case, it will turn out to be useful in characterising voting
behaviour more generally in
\cref{td_new_sec_independence,td_new_sec_a_characterisation_of_voting}. It can
be seen as one of the many generalisations of \emph{May's
Theorem}~\cite{may1952set}, which characterises the majority voting rule in
two-candidate elections.
%
To prove it, we need a preliminary result.

\begin{lemma}
    \label{td_new_lemma_symm_equal_sources_one_claim}
    Suppose $|\src_N(c)| = |\src_N(d)|$, $\obj(c) = \obj(d)$, and for all $s
    \in \src_N(c) \cup \src_N(d)$, $|\claims_N(s)| = 1$. Then for any operator
    $T$ satisfying \symmetry{}, $c \ceq_N^T d$.
\end{lemma}

\begin{proof}
    Without loss of generality, assume $c \ne d$. Since $\obj(c) = \obj(d)$, we
    have $c \in \conflict_N(d)$ and thus $\src_N(c) \cap \src_N(d) =
    \emptyset$. Since $|\src_N(c)| = |\src_N(d)|$ there exists a bijection
    $\hat{\phi}: \src_N(c) \to \src_N(d)$. We extend this to a bijection $\phi:
    S \to S$ by
    \[
        \phi(s) = \begin{cases}
            \hat{\phi}(s),& s \in \src_N(c) \\
            \hat{\phi}^{-1}(s),& s \in \src_N(d) \\
            s,& \text{ otherwise}.
        \end{cases}
    \]
    Now let $F: S \cup C \cup O \to S \cup C \cup O$ be defined by
    $\restrict{F}{S} = \phi$, $\restrict{F}{C} = (cd)$ and $\restrict{f}{O} =
    \identity$. That is, $F$ permutes sources according to $\phi$, swaps claims
    $c$ and $d$, and leaves objects as they are. Since $F(c) = d$, to show $c
    \ceq_N^T d$ it is sufficient by \cref{td_new_prop_automorphism} to show
    that $F$ is an automorphism on $N$.

    It is easily seen that the restrictions of $F$ to $S$, $C$ and $O$
    respectively, are bijective. Moreover, we have $\obj(F(e)) = F(\obj(e))$
    for all claims $e$ since $F(o) = o$ and $\obj(c) = \obj(d)$. It remains to
    show that $(s, e) \in R$ iff $(F(s), F(e)) \in R$.

    For the left-to-right direction, suppose $(s, e) \in R$. First suppose $s
    \in \src_N(c)$. Then $F(s) = \hat{\phi}(s) \in \src_N(d)$, so $(F(s), d)
    \in R$. By assumption we have $|\claims_N(s)| = 1$, so in fact $c$ is the
    unique claim reported by $s$. Thus $e = c$. Consequently
    \[
        (F(s), F(e)) = (F(s), d) \in R
    \]
    as required. The case for $s \in \src_N(d)$ follows by a near-identical
    argument. Finally, if $s \notin \src_N(c) \cup \src_N(d)$ then $F(s) = s$
    and $e \notin \{c, d\}$, so $F(e) = e$. Thus $(F(s), F(e)) = (s, e) \in R$.

    For the right-to-left direction, suppose $(F(s), F(e)) \in R$. Applying the
    argument above we have $(F^2(s), F^2(e)) \in R$ also. But note that $F
    = F^{-1}$, so $F^2 = \identity$. Hence $(s, e) \in R$, as required.
    %
    This completes the proof.
\end{proof}

\begin{proof}[Proof of \cref{td_new_prop_symm_fpr_single_object_voting}]
    We prove \cref{td_new_item_prop_symm_fpr_single_object_voting_claims} only,
    since \cref{td_new_item_prop_symm_fpr_single_object_voting_sources} can be
    shown using essentially the same argument with \sourceposresp{} taking the
    place of \freshposresp{}.

    Suppose $T$ satisfies \symmetry{} and \freshposresp{}, and take $N$ as
    stated in \cref{td_new_prop_symm_fpr_single_object_voting}. It is
    sufficient to show that, for all $c, d \in \claims_N(o)$,
    \begin{align}
        \label{td_new_eqn_src_le_implies_cle}
        |\src_N(c)| \le |\src_N(d)| &\implies c \cle_N^T d \\
        \label{td_new_eqn_src_lt_implies_clt}
        |\src_N(c)| < |\src_N(d)| &\implies c \clt_N^T d.
    \end{align}
    First we show \cref{td_new_eqn_src_le_implies_cle}. Suppose $|\src_N(c)|
    \le |\src_N(d)|$. Assume without loss of generality that $c \ne d$. Write
    $k = |\src_N(d)| - |\src_N(c)| \ge 0$. Let $X = \{s_1, \ldots, s_k\}$ be an
    arbitrary subset of $\src_N(d)$ of size $k$. Let $N_0$ denote the network
    in which all claims from sources in $X$ are removed. Note that since $N$
    does not contain reports for objects other than $o$, by the consistency
    property \cref{td_new_item_sources_self_consistent} in
    \cref{td_new_def_network} we have that sources in $X$ \emph{only} report
    $d$.  We construct networks $N_1, \ldots, N_k$ in which these claims are
    added back in: for $0 \le i
    \le k - 1$, set
    \[
        N_{i + 1} = \addclaim{N_i}{s_{i + 1}}{d}.
    \]
    Then $N_k$ is just the original network $N$. Note that $\claims_{N_i}(s_j)
    = \emptyset$ for $j > i$. Next we show by induction that for all $0 \le i
    \le k$,
    \begin{equation}
        \label{td_new_eqn_inductive_property}
        c \cle_{N_i}^T d, \text{ and if } i > 0 \text{ then }
        c \clt_{N_i}^T d.
    \end{equation}

    For the base case $i = 0$, note that since only reports for $d$ were
    removed in constructing $N_0$, we have $\src_{N_0}(c) = \src_N(c)$.
    Consequently,
    \[
        |\src_{N_0}(d)|
        = |\src_N(d) \setminus X|
        = |\src_N(d)| - k
        = |\src_N(c)|
        = |\src_{N_0}(c)|.
    \]
    Note also that $\obj(c) = \obj(d)$ -- since by assumption $c, d \in
    \claims_N(o)$ -- and for $s \in \src_{N_0}(c) \cup \src_{N_0}(d)$ we have
    $|\claims_{N_0}(s)| = 1$ since $N_0$ also only contains reports for $o$.
    The hypothesis of \cref{td_new_lemma_symm_equal_sources_one_claim} are
    satisfied, so we have $c \ceq_{N_0}^T d$. In particular, $c \cle_{N_0}^T d$
    as required.

    Now for the inductive step, suppose \cref{td_new_eqn_inductive_property}
    holds for $i$. Since $\claims_{N_i}(s_{i + 1}) = \emptyset$,
    \freshposresp{} and the inductive hypothesis give $c \clt_{N_{i + 1}}^T d$,
    as required.

    Finally, \cref{td_new_eqn_src_le_implies_cle} follows by taking $i = k$ in
    \cref{td_new_eqn_inductive_property}, recalling that $N = N_k$. Moreover,
    \cref{td_new_eqn_src_lt_implies_clt} follows by exactly the same argument,
    noting that when $|\src_N(c)| < |\src_N(d)|$ we have $k > 0$, so $c
    \clt_{N_k}^T d$ by \cref{td_new_eqn_inductive_property} again.
\end{proof}

\paragraph{Trust-based monotonicity.}

Suppose $\src_N(d) = \src_N(c) \cup \{s\}$. The relative ranking of $c$ and
$d$ depends on the marginal effect of $s$: if $s$ is ``trustworthy'' then $d$
gains credibility from the extra support of $s$, whereas if $s$ is
``untrustworthy'' this extra support has the opposite effect. Our next axiom
requires that such marginal effects are compatible with the source
trustworthiness ranking. First, some terminology is required.

\begin{definition}
    Given a network $N$, a source $s \in S$ is \emph{marginally trustworthy}
    with respect to an operator $T$ if there exist claims $c, d \in C$ such
    that $s \notin \src_N(c)$, $\src_N(d) = \src_N(c) \cup \{s\}$ and $c
    \cle_N^T d$.
    %
    Similarly, $s$ is \emph{marginally untrustworthy} if there are $c, d  \in
    C$ such that $s \notin \src_N(c)$, $\src_N(d) = \src_N(c) \cup \{s\}$ and
    $d \cle_N^T c$.
\end{definition}

These properties express something about the trustworthiness of sources
via the \emph{claim} ranking $\cle_N^T$, akin to how \sourcecoherence{} looks
at trustworthiness via the claims reported by a source. Note that it is
possible for a source to be both marginally trustworthy and untrustworthy.
Naturally, marginally untrustworthy sources should rank lower than marginally
trustworthy ones.

\begin{axiom}[\marginaltrustworthiness{}]
    If $s$ is marginally untrustworthy and $t$ is marginally trustworthy, then
    $s \sle_N^T t$.
\end{axiom}

Equipped with a notion of marginal trustworthiness, we can also state a
trust-aware monotonicity axiom for claims.

\begin{axiom}[\trustbasedmon{}]
    Suppose $\src_N(d) = \src_N(c) \cup Z$, where $\src_N(c) \cap Z =
    \emptyset$. Then
    \begin{enumerate}
        \item If each $s \in Z$ is marginally trustworthy, $c \cle_N^T d$.
        \item If each $s \in Z$ is marginally untrustworthy, $d \cle_N^T c$.
    \end{enumerate}
\end{axiom}

Informally, \trustbasedmon{} says that if each $s \in Z$ has a positive (or at
least, not negative) impact on some claim in $N$, as measured by $\cle_N^T$,
then the sources in $Z$ acting collectively should also have a positive impact.
Also note that in the case $Z = \{s\}$, \trustbasedmon{} implies that the
marginal impact of $s$ is consistent across the network.

\subsection{Independence}
\label{td_new_sec_independence}

Another common class of axioms in social choice theory are \emph{independence}
axioms, which require that some aspect of the output is independent of
``irrelevant'' parts of the input. The original example is Arrow's
\emph{Independence of Irrelevant Alternatives} (IIA) in voting
theory~\cite{arrow1952}, which says, roughly speaking, that the ranking of
candidates $A$ and $B$ should depend only on the individual rankings of $A$ and
$B$, not on any ``irrelevant'' alternative $C$. It has been adapted to several
settings in which the axiomatic method has been applied. Perhaps closest to our
setting is judgement aggregation, where independence requires the collective
acceptance of a report $\phi$ does not depend on how the individuals accept or
reject some other report $\psi$~\cite{endriss2016ja}.

A version of IIA can be easily stated in our framework: the ranking of claims
$c$ and $d$ should depend only on the sources reporting $c$ and $d$, not on the
sources for other claims. However, this axiom is clearly \emph{undesirable} for
truth discovery. Indeed, consider again the network $N$ from
\cref{td_new_fig_intro_example}. As we have argued informally, claim $c$ is
intuitively weaker than $d$ because how of their respective sources interact
with other claims in the network. Nevertheless, we state this axiom as a point
of comparison with classical social choice problems such as voting.

\begin{axiom}[\classicalindependence{}]
    Suppose $C_N = C_{N'}$. Then $\src_N(c) = \src_{N'}(c)$ and $\src_N(d) =
    \src_{N'}(d)$ implies $c \cle_N^T d$ iff $c \cle_{N'}^T d$.
\end{axiom}

That is, if $c$ and $d$ have the same sources in $N$ and $N'$, they have the
same relative ranking in both networks. The undesirability of
\classicalindependence{} can be formalised axiomatically: together with our
earlier axioms, it implies voting-like behaviour within the claims for each
object.\footnotemark{} Note that for the special case of binary networks,
similar results have been shown in the literature on binary aggregation with
abstentions~\cite{christoffbinary}.

\footnotetext{
    We give a further axiom which implies voting behaviour for claims of
    \emph{different} objects -- and leads to a complete characterisation of
    voting -- in \cref{td_new_sec_a_characterisation_of_voting}.
}

\begin{proposition}
    \label{td_new_prop_symm_fpr_ci_voting}
    Suppose $T$ satisfies \symmetry{}, \freshposresp{} and
    \classicalindependence{}. Then for all $o \in O$ and $c, d \in
    \claims_N(o)$,
    \[
        c \cle_N^T d \iff |\src_N(c)| \le |\src_N(d)|.
    \]
\end{proposition}

\begin{proof}
    Take $c, d \in \claims_N(o)$. Let the network $N'$ have the same sources,
    objects and domains as $N$, but with reports $R' = R \cap (S \times \{c,
    d\})$. That is, $N'$ discards all reports for claims other than $c$ and
    $d$. Then we have $\src_{N'}(c) = \src_N(c)$, $\src_{N'}(d) = \src_N(d)$,
    and $\src_{N'}(e) = \emptyset$ for all $e \notin \{c, d\}$. By
    \classicalindependence{}, $c \cle_N^T d$ iff $c \cle_{N'}^T d$.

    Now, note that since $c, d \in \claims_N(o)$, for $o' \ne o$ and $e \in
    \claims_N(o')$ we have $e \notin \{c, d\}$, so $\src_N(e) = \emptyset$.
    Hence $\src_N(o') = \emptyset$ for such $o'$. Since $T$ satisfies
    \symmetry{} and \freshposresp{}, we may apply
    \cref{td_new_prop_symm_fpr_single_object_voting}
    \cref{td_new_item_prop_symm_fpr_single_object_voting_claims} to find $c
    \cle_{N'}^T d$ iff $|\src_{N'}(c)| \le |\src_{N'}(d)|$. But $|\src_{N'}(c)|
    = |\src_N(c)|$, and likewise for $d$. Consequently
    \[
        c \cle_N^T d
        \iff c \cle_{N'}^T d
        \iff |\src_{N'}(c)| \le |\src_{N'}(d)|
        \iff |\src_{N}(c)| \le |\src_{N}(d)|
    \]
    as desired.
\end{proof}

While this result appears similar to
\cref{td_new_prop_symm_fpr_single_object_voting}, the crucial difference is
that we no longer restrict to the case sources only report on a single object,
where voting is justified. This is the (overly strong) role
\classicalindependence{} plays: it allows the complexity of a multi-object
network to be reduced to a single-object network, where the ranking
trivialises.

Recalling from \cref{td_new_ex_posresps} that \claimcoherence{}, \symmetry{},
\freshposresp{} and \sourceposresp{} are enough to ensure $c \clt d$ in our
running example network from \cref{td_new_fig_intro_example} (whereas
per-object voting gives $c \ceq d$), we obtain an impossibility result with
\classicalindependence{}. In fact we obtain \emph{two} impossibility results,
since \sourceposresp{} can also be replaced with \sourcecoherence{}.

\begin{theorem}
    \label{td_new_thm_classical_indep_impossibility}
    Suppose and operator satisfies \symmetry{}, \claimcoherence{} and
    \freshposresp{}. Then the following axioms cannot hold simultaneously.
    \begin{enumerate}
        \item \label{td_new_item_classical_indep_impossibility_spr}
              \sourceposresp{} and \classicalindependence{}.
        \item \sourcecoherence{} and \classicalindependence{}.
    \end{enumerate}
\end{theorem}

\begin{proof}\leavevmode
    \begin{enumerate}
        \item The impossibility of these axioms holding together follows from
              \cref{td_new_ex_posresps} and
              \cref{td_new_prop_symm_fpr_ci_voting}, as described above.

          \item Let $N$ be as shown in \cref{td_new_fig_intro_example}. Suppose
                some operator $T$ satisfies the stated axioms. From
                \cref{td_new_prop_symm_fpr_ci_voting} we get $c \ceq_N^T d$ and
                $e \clt_N^T f$. Considering sources $s$ and $t$,
                \sourcecoherence{} gives $s \slt_N^T t$. But now
                \claimcoherence{} gives $c \clt_N^T d$: contradiction.
    \end{enumerate}
\end{proof}

\begin{figure}
    \centering
    \begin{tikzpicture}[thick]
        \LARGE
        \networkinit{s,t,u,v}{c,d,e,f}
        \object{o_1}{c}{d};
        \object{o_2}{e}{f};
        \report{s}{c};
        \report{t}{c};
        \report{u}{e};
        \report{v}{f};
    \end{tikzpicture}
    \caption{
        A network illustrating \disjointindependence{}.
    }
    \label{td_new_fig_disjointindep_example}
\end{figure}

By only looking at a claim's sources, \classicalindependence{} ignores the
indirect interaction with other sources and claims in the network. Our next
axiom accounts for such interactions by considering networks with
\emph{disjoint sub-networks}, such as the one shown in
\cref{td_new_fig_disjointindep_example}. Intuitively, while the sources and
claims within a sub-network may interact in complex ways, the fact that the
sub-networks have no sources or objects in common means there is no interaction
\emph{between} them. Accordingly, the ranking for one should not depend on the
other. We formalise this by considering unions of \emph{disjoint
networks}.\footnotemark{}

\footnotetext{
    Note that it is possible to define the disjoint union of an arbitrary
    collection of (not necessarily disjoint) networks in a manner similar to
    the disjoint union of a collection of sets $\bigsqcup_{i \in I}{X_i}$, but
    we do not need this generality here.
}

\begin{definition}
    Networks $N$ and $N'$ are \emph{disjoint} if $S \cap S' = \emptyset$ and $O
    \cap O' = \emptyset$. For $N, N'$ disjoint, their \emph{union} is the
    network $N \sqcup N' = (S \cup S', O \cup O', \hat{D}, R \cup R')$, where
    $\hat{D}_o = D_o$ for $o \in O$, and $\hat{D}_o = D'_o$ for $o \in O'$.
\end{definition}

Note that if $N$ and $N'$ are disjoint, it follows that $C \cap C' = \emptyset$
also.
%
The following axiom says that the ranking of sources and claims is unaffected
by the addition of a disjoint network.

\begin{axiom}[\disjointindependence{}]
    If $N$ and $N'$ are disjoint, $s, t \in S$, and $c, d \in C$, then
    $s \sle_N^T t$ iff $s \sle_{N \sqcup N'}^T t$ and $c \cle_N^T d$ iff $c
    \cle_{N \sqcup N'}^T d$.
\end{axiom}

\todo{Explain graph-theoretic interpretation in terms of connected components.}

\subsection{Conflicting claims}
\label{td_new_sec_conflicting_claims}

Our axioms so far have not made use of the conflict relation between claims.
Intuitively, distinct claims $c, c'$ for the same object $o$ cannot both be
true, so belief in $c$ should come at the expense of belief in $c'$. Similarly,
if the antisources of $c$ -- that is, the sources who report claims conflicting
with $c$ -- are seen as less trustworthy than the antisources of $c'$, then the
attack on $c$ is less damaging than that of $c'$, so $c$ should be more
believable than $c'$. Note that these are again coherence principles, which
constrain how the claim ranking $\cle$ coheres with both the source ranking
$\sle$ and the conflict relation. We formulate them as axioms.

\begin{axiomlist}
\begin{axiom}[\conflictcoherence{}]
    If $\conflict_N(c)$ strictly precedes $\conflict_N(c')$ pairwise with
    respect to $\cle_N^T$, then $c' \clt_N^T c$.
\end{axiom}
\begin{axiom}[\anticoherence{}]
    If $\antisrc_N(c)$ strictly precedes $\antisrc_N(c')$ pairwise with respect
    to $\sle_N^T$, then $c' \clt_N^T c$.
\end{axiom}
\end{axiomlist}

While both \conflictcoherence{} and \anticoherence{} appear reasonable in
isolation, there is an inherent tension between them and our earlier coherence
axioms. Together with symmetry and responsiveness axioms, we have an
impossibility result.

\begin{theorem}
    \label{td_new_thm_conflict_impossibilities}
    Suppose an operator satisfies \symmetry{} and \claimcoherence{}. Then the
    following axioms cannot hold simultaneously.
    \begin{enumerate}
        \item \freshposresp{}, \sourcecoherence{} and \conflictcoherence{},
        \item \sourceposresp{} and \conflictcoherence{}.
        \item \sourceposresp{} and \anticoherence{}.
    \end{enumerate}
\end{theorem}

\begin{figure}
    \centering
    \begin{tikzpicture}[thick]
        \LARGE
        \networkinit{s,t,u,s',t'}{c,d,e,f,c',d'}
        \object{o}{c}{d}
        \object{p}{e}{f}
        \object{o'}{c'}{d'}
        \report{s}{c}
        \report{s}{e}
        \report{t}{d}
        \report{t}{e}
        \report[dashed]{u}{f}
        \report{s'}{f}
        \report{s'}{c'}
        \report{t'}{f}
        \report{t'}{d'}
    \end{tikzpicture}
    \caption{
        Network used to illustrate the impossibility results of
        \cref{td_new_thm_conflict_impossibilities}.
    }
    \label{td_new_fig_conflict_impossibilities}
\end{figure}

\begin{proof}
    Suppose $T$ satisfies \symmetry{} and \claimcoherence{}.
    Throughout the proof, let $N_0$ denote the network shown in
    \cref{td_new_fig_conflict_impossibilities} excluding the dashed edge, and
    let $N_1 = \addclaim{N}{u}{f}$ denote the network including the dashed
    edge. We first note some consequences of the axioms in both networks. In
    $N_0$, the mapping $(s\ s')(t\ t')(c\ c')(d\ d')(o\ o')(e\ f)$ is an
    automorphism, so we have $s \seq_{N_0}^T s'$ and $e \ceq_{N_0}^T f$. Note
    that $\src_{N_0}(u) = \emptyset$, $s \in \antisrc_{N_0}(f)$ and $s' \in
    \src_{N_0}(f)$. If $T$ additionally satisfies \freshposresp{}, we get $e
    \clt_{N_1}^T f$. If $T$ instead satisfies \sourceposresp{}, we get $s
    \slt_{N_1}^T s'$.
    %
    Considering $N_1$ alone, the mapping $(s\ t)(s'\ t')(c\ d)(c'\ d')$ is an
    automorphism, so \symmetry{} gives $c \ceq_{N_1}^T d$ and $c' \ceq_{N_1}^T
    d'$.

    \begin{enumerate}

        \item Suppose $T$ also satisfies \freshposresp{}, \sourcecoherence{}
              and \conflictcoherence{}. First we claim $c \ceq_{N_1}^T c'$.
              Indeed, suppose not. If $c' \clt_{N_1}^T c$, we may note that
              $\conflict_{N_1}(d) = \{c\}$ and $\conflict_{N_1}(d') = \{c'\}$,
              and apply \conflictcoherence{} to get $d \clt_{N_1}^T d'$. But by
              \symmetry{} as above, we have $c \ceq_{N_1}^T d$ and $c'
              \ceq_{N_1}^T d'$. Consequently $c \ceq_{N_1}^T d \clt_{N_1}^T d'
              \ceq_{N_1}^T c'$, i.e. $c \clt_{N_1}^T c'$. Clearly this
              contradicts $c' \clt_{N_1}^T c$. If $c \clt_{N_1}^T c'$ we obtain
              a contradiction by an identical argument. Hence $c \ceq_{N_1}^T
              c'$.

              Now, by \freshposresp{} and \symmetry{} as noted above, we have
              $e \clt_{N_1}^T f$. \sourcecoherence{} for $s$ and $s'$ therefore
              gives $s \slt_{N_1}^T s'$. But considering $c$ and $c'$,
              \claimcoherence{} gives $c \clt_{N_1}^T c'$. This contradicts $c
              \ceq_{N_1}^T c'$, and we are done.

        \item Suppose $T$ additionally satisfies \sourceposresp{} and
              \conflictcoherence{}. By the same argument as above,
              \conflictcoherence{} and \symmetry{} together dictate that $c
              \ceq_{N_1}^T c'$. But by \symmetry{} and \sourceposresp{}, we
              have $s \slt_{N_1}^T s'$. \claimcoherence{} then implies $c
              \clt_{N_1}^T c'$: contradiction.

        \item Suppose $T$ additionally satisfies \sourceposresp{} and
              \anticoherence{}. Again, $s \slt_{N_1}^T s'$. \claimcoherence{}
              implies $c \clt_{N_1}^T c'$. Since $\antisrc_{N_1}(d) = \{s\}$
              and $\antisrc_{N_1}(d') = \{s'\}$, \anticoherence{} gives $d'
              \clt_{N_1}^T d$. But recall that, by \symmetry{}, $c \ceq_{N_1}^T
              d$ and $c' \ceq_{N_1}^T d'$. Hence $c \clt_{N_1}^T c'
              \ceq_{N_1}^T d' \clt_{N_1}^T d \ceq_{N_1}^T c$, i.e. $c
              \clt_{N_1}^T c$: contradiction.
    \end{enumerate}
\end{proof}

Note that all four coherence axioms can be satisfied at the same time, e.g. by
the trivial operator which outputs constant scores $T_N(s) = T_N(c) = 0$. Of
course, this operator violates both \freshposresp{} and \sourceposresp{}.

\subsection{Axiomatic Characterisation of Voting}
\label{td_new_sec_a_characterisation_of_voting}

Recall from \cref{td_new_prop_symm_fpr_ci_voting} that \symmetry{},
\freshposresp{} and \classicalindependence{} force an operator to rank claims
for the object simply by their number of sources, as in voting from
\cref{td_new_sec_voting}. In this section we give two further axioms which
force this ranking even for claims across different objects, and thus
characterise $\voting{}$ completely. Like \classicalindependence{}, these
axioms are \emph{not} desirable properties, and are introduced only to capture
the behaviour of voting. The first axiom simply says that the source ranking is
flat.

\begin{axiom}[\flatsources{}]
    For all $s, s' \in S$, $s \seq_N^T s'$.
\end{axiom}

The second axiom says that objects play no role: it is only the relation
between sources and claims which affects the rankings. That is, we can ignore
the conflict relation between claims. To define the axiom we introduce a notion
of ``reducing'' the objects of a network.

\begin{definition}
    A network $N'$ is an \emph{object reduction} of $N$ via $f: C_N \to C_{N'}$
    if
    \begin{enumerate}
        \item $S' = S$.
        \item $f$ is a bijection $C_N \to C_{N'}$ such that $(s, c) \in R$
              iff $(s, f(c)) \in R'$.
        \item For al $o \in O'$, $|D'_o| = 1$.
    \end{enumerate}
\end{definition}

\begin{figure}
    \centering
        \begin{subfigure}{.4\textwidth}
            \centering
            \begin{tikzpicture}[thick]
                \LARGE
                \networkinit{s,t,u}{c,d,e,f}
                \object{o}{c}{d}
                \object{p}{e}{f}
                \report{s}{c}
                \report{s}{e}
                \report{t}{d}
                \report{u}{f}
            \end{tikzpicture}
            \caption{$N$}
        \end{subfigure}
        \begin{subfigure}{.4\textwidth}
            \centering
            \begin{tikzpicture}[thick]
                \LARGE
                \networkinit{s,t,u}{c_1,c_2,c_3,c_4}
                \objectsingleclaim{o_1}{c_1}
                \objectsingleclaim{o_2}{c_2}
                \objectsingleclaim{o_3}{c_3}
                \objectsingleclaim{o_4}{c_4}
                \report{s}{c_1}
                \report{s}{c_3}
                \report{t}{c_2}
                \report{u}{c_4}
            \end{tikzpicture}
            \caption{$N'$}
        \end{subfigure}
    \caption{
        Illustration of an object reduction of a network.
    }
    \label{td_new_fig_object_reduction}
\end{figure}

Note that every network $N$ has an object reduction since the set of
possible objects $\O$ is infinite; we may take $O'$ to be any subset of $\O$ of
size $|C_N|$, take $D'_o = \{v\}$ for some fixed $v \in \V$, and set $R'$
accordingly.
%
\cref{td_new_fig_object_reduction} shows an example of an object reduction.
Note that the network $N'$ has only a single claim for each object, and the
structure of the reports -- i.e. the edges shown in
\cref{td_new_fig_object_reduction} -- is the same in $N$ and $N'$. Going from
$N$ to $N'$ loses information about which claims conflict with one another, and
our axioms in \cref{td_new_sec_conflicting_claims} explicitly require that this
information \emph{does} affects the rankings. Voting does not use this
information, however, which leads to the following axiom.

\begin{axiom}[\objectirrelevance]
    If $N'$ is an object reduction of $N$ via $f$, then $c \cle_N^T d$ iff
    $f(c) \cle_N^T f(d)$.
\end{axiom}

Note that \objectirrelevance{} is similar in form to \symmetry{}, but rather
than requiring rankings are invariant under isomorphisms -- which
preserve the relevant structure of a network -- it requires rankings are
invariant under object reductions.

We can now characterise voting, up to ranking equivalence.

\begin{figure}
    \centering
    \begin{subfigure}{.4\textwidth}
        \centering
        \begin{tikzpicture}[thick,scale=0.7]
            \Large
            \networkinit{s,t,u,v,w}{c,c',d,d',d''}
            \object{o_1}{c}{c'}
            \object{o_2}{d}{d''}
            \report{s}{c}
            \report{t}{c}
            \report{t}{d}
            \report{u}{c'}
            \report{u}{d}
            \report{v}{d}
            \report{w}{d''}
        \end{tikzpicture}
        \caption{$N$}
    \end{subfigure}
    \begin{subfigure}{.4\textwidth}
        \centering
        \begin{tikzpicture}[thick,scale=0.7]
            \Large
            \networkinit{s,t,u,v,w}{c,c',d,d',d''}
            \object{o_1}{c}{c'}
            \object{o_2}{d}{d''}
            \report{s}{c}
            \report{t}{c}
            \report{t}{d}
            \report{u}{d}
            \report{v}{d}
        \end{tikzpicture}
        \caption{$N'$}
    \end{subfigure}
    \\
    \begin{subfigure}{.9\textwidth}
        \centering
        \begin{tikzpicture}[thick,scale=0.7]
            \Large
            \networkinit{s,t,u,v,w}{c,c',d,d',d''}
            \objectsingleclaim{o_1}{c}
            \objectsingleclaim{o_2}{c'}
            \objectsingleclaim{o_3}{d}
            \objectsingleclaim{o_4}{d'}
            \objectsingleclaim{o_5}{d''}
            \report{s}{c}
            \report{t}{c}
            \report{t}{d}
            \report{u}{d}
            \report[dashed]{v}{d}
        \end{tikzpicture}
        \caption{
            $N''$ (all edges) and $N_0$ (excluding dashed edge)
        }
    \end{subfigure}
    \caption{
        Illustration of the proof of \cref{td_new_thm_voting_characterisation}.
        In $N'$, reports for claims other than $c$ and $d$ are removed. $N''$
        is an object reduction of $N'$. The dashed edge shows the reports added
        when \freshposresp{} is applied.
    }
    \label{td_new_fig_voting_characterisation_example}
\end{figure}

\begin{theorem}
    \label{td_new_thm_voting_characterisation}
    An operator $T$ satisfies \symmetry{}, \freshposresp{},
    \classicalindependence{}, \flatsources{} and \objectirrelevance{} if and
    only if $T \rankequiv \voting$.
\end{theorem}

\begin{proof}[Proof (sketch)]
    The ``if'' direction is straightforward. For the
    ``only if'' direction, take an operator $T$ with the stated axioms.
    \flatsources{} immediately implies ${\sle_N^T} = {\sle_N^{\voting}}$ for
    all networks $N$. For the claim rankings, we take a similar approach to the
    proof of \cref{td_new_prop_symm_fpr_ci_voting} and only sketch the argument
    here. An illustration of the proof is shown in
    \cref{td_new_fig_voting_characterisation_example}.

    Take any network $N$ and claims $c, d$. We first remove all reports for
    other claims to produce $N'$; this preserves rankings by
    \classicalindependence{}. Taking $N''$ to be any object reduction of $N'$,
    we ensure $c$ and $d$ are the only claims for their respective
    objects,\footnotemark{} and rankings are again preserved by
    \objectirrelevance{}. As before, it suffices to show that $|\src_N(c)| \le
    |\src_N(d)|$ implies $c \cle_N^T d$ and $|\src_N(c)| < |\src_N(d)|$ implies
    $c \clt_N^T d$, since $c$ and $d$ are arbitrary.
    %
    \footnotetext{
        Strictly speaking, we should define an object reduction $f$ between
        $N'$ and $N''$, and refer to $f(c)$ and $f(d)$ in $N''$ instead of $c$
        and $d$. For simplicity we identify $c$ with $f(c)$ and $d$ with $f(d)$
        in this proof sketch.
    }

    Write $k = |\src_N(d)| - |\src_N(c)| \ge 0$. Choosing $k$ sources from
    $\src_N(d) \setminus \src_N(c)$, let $N_0$ be the network obtained from
    $N''$ in which reports for $d$ from these sources are removed. Note that
    such sources \emph{only} report $d$, since reports for other claims were
    removed in the construction of $N'$. Then $|\src_{N_0}(c)| =
    |\src_{N_0}(d)|$. The fact that $|D''_{\obj(c)}| = |D''_{\obj(d)}| = 1$
    ensures we are able to choose an automorphism on $N_0$ which swaps $c$ and
    $d$ (and swaps $\src_{N_0}(c) \setminus \src_{N_0}(d)$ with $\src_{N_0}(d)
    \setminus \src_{N_0}(c)$). By \symmetry{}, $c \ceq_{N_0}^T d$.

    If $k = 0$ then $N_0 = N''$, and we are done. Otherwise, by repeated
    applications of \freshposresp{} we may add the removed reports back in to
    $N_0$ to get $c \clt_{N''}^T d$. Since claim rankings are the same in $N''$
    as in $N$, this completes the proof.
\end{proof}

\section{Satisfaction of the Axioms}
\label{td_new_sec_satisfaction_of_the_axioms}

In the previous section we introduced several axioms for truth discovery. We
now turn back to some of the example operators from
\cref{td_new_sec_example_operators}, to assess which axioms hold for each
operator. For simplicity we skip TruthFinder and CRH, which due to their
somewhat complicated form are not straightforward to analyse. The results are
summarised in \cref{td_new_tab_axiom_satisfaction}.

\begin{table}
    \centering
    \caption{Axiom satisfaction for the example operators.}
    \def\yes{\checkmark}
    \def\no{\sffamily{X}}
    \def\notsure{\sffamily{?}}
    \begin{tabular}{lccccc}
        &
        Voting             &
        WeightedAgg        &
        Sums               &
        USums              \\
        % CRH-$\epsilon$     &
        % TruthFinder        &
        % TruthFinder ($\lambda = 0$) \\

        \toprule

        \claimcoherence{} &
            \yes{} &
            \yes{} &
            \yes{} &
            \yes{} &
            \\
        \sourcecoherence{} &
            \no{} &
            \no{} &
            \yes{} &
            \yes{} &
            \\
        \symmetry{} &
            \yes{} &
            \yes{} &
            \yes{} &
            \yes{} &
            \\
        \freshposresp{} &
            \yes{} &
            \yes{} &
            \no{} &
            \notsure{} &
            \\
        \sourceposresp{} &
            \no{} &
            \yes{} &
            \no{} &
            \notsure{} &
            \\
        \marginaltrustworthiness{} &
            \yes{} &
            \yes{} &
            \yes{} &
            \yes{} &
            \\
        \trustbasedmon{} &
            \yes{} &
            \yes{} &
            \yes{} &
            \yes{} &
            \\
        \classicalindependence{} &
            \yes{} &
            \no{} &
            \no{} &
            \no{} &
            \\
        \disjointindependence{} &
            \yes{} &
            \yes{} &
            \no{} &
            \no{} &
            \\
        \conflictcoherence{} &
            \no{} &
            \no{} &
            \no{} &
            \no{} &
            \\
        \anticoherence{} &
            \no{} &
            \no{} &
            \no{} &
            \no{} &
            \\

        \bottomrule

	\end{tabular}
    \label{td_new_tab_axiom_satisfaction}
\end{table}

\paragraph{Weighted Voting.}

First we consider weighted voting. The following axioms hold for \emph{any}
choice of weighting $w$.

\begin{lemma}
    \label{td_new_lemma_wvoting_always_hold}
    Let $w$ be a weighting. Then $\wvoting{w}$ satisfies \claimcoherence{},
    \marginaltrustworthiness{} and \trustbasedmon{}.
\end{lemma}

\begin{proof}
    \claimcoherence{} follows easily using the definition of weighted
    voting and \cref{td_new_prop_pwo_strict_part}.

    One can easily show that if $s$ is marginally trustworthy with respect to
    $\wvoting{w}$ then $w_N(s) \ge 0$, and if $s$ is marginally untrustworthy
    with respect to $\wvoting{w}$ then $w_N(s) \ge 0$, and
    \marginaltrustworthiness{} follows.

    Finally, for \trustbasedmon{} suppose $\src_N(d) = \src_N(c) \cup Z$, where
    $\src_N(c) \cap Z = \emptyset$. Then $\wvoting{w}_N(d) = \wvoting{w}_N(c) +
    \sum_{s \in Z}{w_N(s)}$. If each $s \in Z$ is marginally trustworthy then
    each $w_N(s)$ is non-negative, and so too is the sum. Hence
    $\wvoting{w}_N(d) \ge \wvoting{w}(c)$, so $c \cle_N^{\wvoting{w}} d$. If
    each $s \in Z$ is marginally untrustworthy then each $w_N(s)$ is
    non-positive, and similarly we get $d \cle_N^{\wvoting{w}} c$ as required.
\end{proof}

\begin{corollary}
    \label{td_new_cor_weightable_axioms}
    Any weightable operator satisfies \claimcoherence{},
    \marginaltrustworthiness{} and \trustbasedmon{}.
\end{corollary}

\begin{proof}
    This follows directly from \cref{td_new_lemma_wvoting_always_hold} since
    each axiom only refers to ordinal properties of operators.
\end{proof}

Voting arises via the uniform weighting $w_N \equiv 1$. We have the following.

\begin{theorem}
    Voting satisfies \claimcoherence{}, \symmetry{}, \freshposresp{},
    \marginaltrustworthiness{}, \trustbasedmon{}, \classicalindependence{} and
    \disjointindependence{}. It does not satisfy \sourcecoherence{},
    \sourceposresp{}, \conflictcoherence{} or \anticoherence{}.
\end{theorem}

The proof is for the most part straightforward, and is omitted for brevity. For
the particular choice of $w$ for Weighted Agreement from
\cref{td_new_ex_weighted_voting}, we have the following.

\begin{theorem}
    Weighted Agreement satisfies \claimcoherence{}, \symmetry{},
    \freshposresp{}, \sourceposresp{}, \marginaltrustworthiness{},
    \trustbasedmon{} and \disjointindependence{}. It does not satisfy
    \sourcecoherence{}, \classicalindependence{}, \conflictcoherence{} or
    \anticoherence{}.
\end{theorem}

\begin{proof}
    For brevity, let $w$ denote $\wagree$ and $T$ denote $\wvoting{\wagree}$.
    %
    \claimcoherence{}, \marginaltrustworthiness{} and \trustbasedmon{} follow
    from \cref{td_new_lemma_wvoting_always_hold}.

    For \symmetry{}, suppose $F$ is an isomorphism between networks $N$ and
    $N'$. From the definition of an isomorphism we have $(s, c) \in R$ iff
    $(F(s), F(c)) \in R'$. Consequently $\src_N(c) = \{F^{-1}(s') \mid s' \in
    \src_{N'}(F(c))\}$ and $\claims_N(s) = \{F^{-1}(c') \mid c' \in
    \claims_{N'}(F(s))\}$. From this one can show $w_N(s) = w_{N'}(F(s))$,
    which then implies $T_N(s) = T_{N'}(F(s))$ and $T_N(c) = T_{N'}(F(c))$.
    \symmetry{} now follows.

    For \freshposresp{} and \sourceposresp{}, we use the following auxiliary
    result.

    \begin{claim}
        \label{td_new_claim_wagree_new_report}
        Suppose $\claims_N(u) = \emptyset$ and let $c$ be a claim. Then for all
        $s \ne u$ with $\claims_N(s) \ne \emptyset$,
        \[
            w_{\addclaim{N}{u}{c}}(s) = w_N(s) +
            \frac{\indicator{c \in \claims_N(s)}}{|\claims_N(s)|}.
        \]
    \end{claim}
    \begin{claimproof}
        First, note that for any claim $d$,
        \[
            |\src_{\addclaim{N}{u}{c}}(d)| = |\src_N(d)| + \indicator{c = d},
        \]
        and since $s \ne u$ we have $\claims_{\addclaim{N}{u}{c}}(s) =
        \claims_N(s)$. Consequently
        \begin{align*}
            w_{\addclaim{N}{u}{c}}(s)
            &= \sum_{d \in \claims_{\addclaim{N}{u}{c}}(s)}{
                \frac{|\src_{\addclaim{N}{u}{c}}(d)|}{|\claims_{\addclaim{N}{u}{c}}(s)|}
            } \\
            &= \sum_{d \in \claims_N(s)}{
                \frac{|\src_N(d)| + \indicator{c = d}}{|\claims_N(s)|}
            } \\
            &=
            \underbrace{
                \sum_{d \in \claims_N(s)}{
                    \frac{|\src_N(d)|}{|\claims_N(s)|}
                }
            }_{= w_N(s)}
            + \sum_{d \in \claims_N(s)}{
                \underbrace{
                    \frac{\indicator{c = d}}{|\claims_N(s)|}
                }_{= 0 \text{ unless } c = d}
            } \\
            &= w_N(s) + \frac{\indicator{c \in \claims_N(s)}}{|\claims_N(s)|}.
        \end{align*}
    \end{claimproof}

    Now, for \freshposresp, suppose $\claims_N(u) = \emptyset$, $c \ne d$ and
    $d \cle_N^T c$. We need to show $d \clt_{\addclaim{N}{u}{c}}^T c$. Indeed,
    using \cref{td_new_claim_wagree_new_report} we have
    \begin{align*}
        &T_{\addclaim{N}{u}{c}}(c) - T_{\addclaim{N}{u}{c}}(d)
        = w_{\addclaim{N}{u}{c}}(u)
        + \sum_{s \in \src_N(c)}{
            w_{\addclaim{N}{u}{c}}(s)
        }
        - \sum_{s \in \src_N(d)}{
            w_{\addclaim{N}{u}{c}}(s)
        } \\
        &= |\src_N(c)| + 1
        + \sum_{s \in \src_N(c)}{
            \left(
                w_N(s)
                + \frac{1}{|\claims_N(s)|}
            \right)
        }
        - \sum_{s \in \src_N(d)}{
            \left(
                w_N(s)
                + \frac{\indicator{c \in \claims_N(s)}}{|\claims_N(s)|}
            \right)
        } \\
        &= |\src_N(c)| + 1
        + T_N(c)
        + \sum_{s \in \src_N(c)}{
            \frac{1}{|\claims_N(s)|}
        }
        - T_N(d)
        - \sum_{s \in \src_N(c) \cap \src_N(d)}{
            \frac{1}{|\claims_N(s)|}
        } \\
        &= |\src_N(c)| + 1
        + \underbrace{T_N(c) - T_N(d)}_{\ge 0}
        + \sum_{s \in \src_N(c) \setminus \src_N(d)}{
            \frac{1}{|\claims_N(s)|}
        } \\
        &\ge 1 > 0.
    \end{align*}
    This shows $T_{\addclaim{N}{u}{c}}{c} > T_{\addclaim{N}{u}{c}}(d)$, and
    thus $d \clt_{\addclaim{N}{u}{c}}^T c$ as required.

    For \sourceposresp{}, suppose $s \in \antisrc_N(c)$, $t \in \src_N(c)$,
    $\claims_N(u) = \emptyset$ and $s \sle_N^T t$. Then
    \begin{align*}
        T_{\addclaim{N}{u}{c}}(t) &- T_{\addclaim{N}{u}{c}}(s)
        = w_{\addclaim{N}{u}{c}}(t) - w_{\addclaim{N}{u}{c}}(s) \\
        &= \underbrace{w_N(t) - w_N(s)}_{\ge 0}
        + \frac{\indicator{c \in \claims_N(t)}}{|\claims_N(t)|}
        - \underbrace{
            \frac{\indicator{c \in \claims_N(s)}}{|\claims_N(s)|}
        }_{= 0} \\
        &\ge \frac{1}{|\claims_N(t)|} \\
        &> 0
    \end{align*}
    where we use the fact that $s \in \antisrc_N(c)$ means $c \notin
    \claims_N(s)$. Hence $s \slt_{\addclaim{N}{u}{c}}^T t$.

    Finally, \disjointindependence{} follows easily by noting that for disjoint
    networks $N$, $N'$ and $s \in S_N$, $c \in C_N$, we have $\claims_{N \sqcup
    N'}(s) = \claims_N(s)$ and $\src_{N \sqcup N'}(c) = \src_N(c)$.

    \begin{figure}
        \centering
            \begin{tikzpicture}[thick]
                \LARGE
                \networkinit{t_1,s,s',t_2,t_3,t_4}{c,c',d}
                \object{o}{c}{c'}
                \objectsingleclaim{p}{d}
                \report{t_1}{c}
                \report{s}{c}
                \report{s'}{c'}
                \report{t_2}{c'}
                \report{t_2}{d}
                \report{t_3}{d}
                \report{t_4}{d}
            \end{tikzpicture}
        \caption{
            Counterexample for \sourcecoherence{} for Weighted Agreement.
        }
        \label{td_new_fig_weighted_agreement_sourcecoh_counterex}
    \end{figure}

    To see that \sourcecoherence{} does not hold, let $N$ be the network shown
    in \cref{td_new_fig_weighted_agreement_sourcecoh_counterex}. One can easily
    check that $c \clt_N^T c'$ yet $s \seq_N^T s'$.

    \classicalindependence{} cannot hold by the impossibility result
    \cref{td_new_thm_classical_indep_impossibility}
    \cref{td_new_item_classical_indep_impossibility_spr}, since \symmetry{},
    \claimcoherence{}, \freshposresp{} and \sourceposresp{} have already been
    shown to hold. Similarly, the failure of \conflictcoherence{} and
    \anticoherence{} follow from \cref{td_new_thm_conflict_impossibilities}.
\end{proof}

\paragraph{Sums.}

To simplify axiomatic analysis of Sums, we first show that $\sums$ is a fixed
point of the update function $U$ for Sums. In what follows, let $(\D, T^0, U)$
denote the recursive scheme corresponding to Sums from \cref{td_new_def_sums}.
Recall that $\sums$ is defined as the limit of this recursive scheme. For
simplicity we assume $\sums$ converges on all input networks.\footnotemark{} We
also write $T^n = U^n(T^0)$ for the $n$-th step of the iteration of Sums.

\footnotetext{
    While \textcite{pasternack2010} do not consider convergence, Sums is an
    adaptation of the \emph{Hubs and Authorities} algorithm, for which
    \textcite{kleinberg1999} proves convergence: phrased in our terminology, he
    shows that the vector of source scores converge to a unit eigenvector of
    the matrix $MM^\text{T}$ corresponding to the largest eigenvector (in
    absolute value), where $M$ is the $|S| \times |C|$ matrix defined by
    $M_{sc} = \indicator{s \in \src_N(c)}$. Similarly, claim scores converge to
    a unit eigenvector of $M^{\text{T}}M$.
}

The following lemma helps to deal with the normalisation factors used in the
update function for Sums.

\begin{lemma}
    \label{td_new_lemma_max_lemma}
    Let $(x^i_n)_{n \in \Nat}$ be convergence sequences in $\R$, for $1
    \le i \le k$. Then
    \[
        \lim_{n \to \infty}{\max_i{|x^i_n|}}
        = \max_i{|\lim_{n \to \infty}{x^i_n}|}.
    \]
\end{lemma}

\begin{proof}
    Let $\epsilon > 0$. Write $y^i = \lim_{n \to \infty}{x^i_n}$. For each $i$,
    hence $|x^i_n| \to |y^i|$ -- since the absolute value function $\|\cdot\|$
    is continuous -- and so there is $n_i \in \Nat$ such that $||x^i_n| -
    |y^i|| < \epsilon$ for all $n \ge n_i$. Take $m = \max_i{n_i}$. Let $n \ge
    m$. For any $i$, we have
    \[
        |y^i| - \epsilon < |x^i_n| < |y^i| + \epsilon.
    \]
    Thus
    \[
        |x^i_n|
        < |y^i| + \epsilon
        \le \max_j|y^j| + \epsilon.
    \]
    Since the maximum is achieved for some $i$, we get
    \begin{equation}
        \label{td_new_eqn_max_lemma_1}
        \max_i|x^i_n| < \max_j|y^j| + \epsilon.
    \end{equation}

    Now, take $j$ such that $\max_i|y^i| = |y^j|$. Then
    \begin{equation}
        \label{td_new_eqn_max_lemma_2}
        \max_i|x^i_n|
        \ge |x^j_n|
        > |y^j| - \epsilon
        = \max_i|y^i| - \epsilon.
    \end{equation}
    Combining \cref{td_new_eqn_max_lemma_1} and \cref{td_new_eqn_max_lemma_2},
    we get
    \[
        |\max_i|x^i_n| - \max_i|y^i|| < \epsilon
    \]
    as required.
\end{proof}

\begin{lemma}
    \label{td_new_lemma_sums_fixedpoint}
    $\sums \in \D$, and $U(\sums) = \sums$.
\end{lemma}

\begin{proof}
Note
    that $T_N^n(z) \in [0, 1]$ for all $n$ and $z \in S \cup C$. Consequently
    $T_N^*(z) = \lim_{n \to \infty}{T^n_N(z)} \in [0, 1]$, since $[0, 1]$ is
    closed. Hence $\sums \in \D$.

    Take any network $N$. If $N$ contains no reports -- i.e. $R = \emptyset$,
    then $T^n \equiv 0$ for all $n > 1$. Hence $\sums_N \equiv 0$ and
    $U(\sums)_N = \sums_N$. Now suppose $N$ contains at least one report $(s_0,
    c_0)$. It is easily checked that in this case $T_N^n(s_0), T_N^n(c_0) > 0$
    for all $n$.  Consequently the maximums in the definition of $\alpha$ and
    $\beta$ in \cref{td_new_def_sums} are non-zero. For any $s \in S$, we
    therefore have
    \begin{align}
        \sums_N(s)
        &= \lim_{n \to \infty}{T^n_N(s)} \nonumber \\
        &= \lim_{n \to \infty}{T^{n+1}_N(s)} \nonumber \\
        &= \lim_{n \to \infty}{
            \frac{
                \sum_{c \in \claims_N(s)}{
                    T^n_N(c)
                }
            }{
                \max_{t \in S}{\left|
                    \sum_{c \in \claims_N(t)}{
                        T^n_N(c)
                    }
                    \right|
                }
            }
        }
        \label{td_new_eqn_sums_max_limit}
    \end{align}
    We need to show that the denominator in \cref{td_new_eqn_sums_max_limit}
    converges to a non-zero limit. By the normalisation step for claim scores,
    for each $n > 1$ there is a claim $c_n$ with $|T^n_N(c_n)| = 1$. Since
    there are only finitely many claims, this implies we cannot have
    $\sums_N(c) = 0$ for all $c$, so there is some $c_1$ with $\sums_N(c_1) >
    0$. Furthermore, $\src_n(c_1) \ne \emptyset$ (otherwise one can easily
    show $\sums_N(c_1) = 0$). Likewise, there is some $s_1$ such that
    $\sums_N(s_1) > 0$. Now using the fact that $T^n_N(c) \to \sums_N(c)$ for
    each $c$ and taking the limit of the sum, \cref{td_new_lemma_max_lemma}
    gives
    \[
        \lim_{n \to \infty}{
            \max_{t \in S}{\left|
                \sum_{c \in \claims_N(t)}{
                    T^n_N(c)
                }
                \right|
            }
        }
        =
        \max_{t \in S}{\left|
            \sum_{c \in \claims_N(t)}{
                \sums_N(c)
            }
            \right|
        }
        \ge \sums_N(c_1)
        > 0.
    \]
    Splitting the limit across the quotient in
    \cref{td_new_eqn_sums_max_limit}, we find
    \begin{align*}
        \sums_N(s)
        &=
        \frac{
            \lim_{n \to \infty}{
                \sum_{c \in \claims_N(s)}{
                    T^n_N(c)
                }
            }
        }{
            \lim_{n \to \infty}{
                \max_{t \in S}{\left|
                    \sum_{c \in \claims_N(t)}{
                        T^n_N(c)
                    }
                    \right|
                }
            }
        } \\
        &=
        \frac{
            \sum_{c \in \claims_N(s)}{
                \sums_N(c)
            }
        }{
            \max_{t \in S}{\left|
                \sum_{c \in \claims_N(t)}{
                    \sums_N(c)
                }
                \right|
            }
        } \\
        &= U(\sums)_N(s)
    \end{align*}
    as required. One can show $\sums_N(c) = U(\sums)_N(c)$ for any claim $c$ by
    a near-identical argument, and thus $U(\sums)_N = \sums_N$. Since $N$ was
    arbitrary this shows $U(\sums) = \sums$, and the proof is complete.
\end{proof}

\begin{corollary}
    \label{td_new_cor_sums_weightable}
    $\sums$ is weightable.
\end{corollary}

\begin{proof}
    We define a weighting $w$ as follows. If $N$ contains no reports, set $w_N
    \equiv 0$. Otherwise, set
    \begin{equation}
        w_N(s) =
        \frac{
            \sums_N(s)
        }
        {
            \max_{c \in C}{\left|
                \sum_{t \in \src_N(c)}{
                    \sums_N(t)
                }
            \right|}
        }.
        \label{td_new_eqn_sums_weightable}
    \end{equation}
    We need to show $\sums \rankequiv \wvoting{w}$, i.e. that $\sums$ and
    $\wvoting{w}$ give the same rankings on all networks $N$. If $N$ contains
    no reports then both $\sums_N$ and $\wvoting{w}$ are zero, and therefore
    output the same rankings. Suppose $N$ contains at least one report. Since
    we just divide by a constant in \cref{td_new_eqn_sums_weightable}, $s
    \sle_N^{\sums} s'$ iff $s \sle_N^{\wvoting{w}} s'$ for all sources $s$ and
    $s'$. Using the fact that $\sums = U(\sums)$ from
    \cref{td_new_lemma_sums_fixedpoint}, it is easily seen that $\sums_N(c) =
    \sum_{s \in \src_N(c)}{w_N(s)} = \wvoting{w}_N(c)$. Hence $\sums_N$ and
    $\wvoting{w}_N$ give exactly the same scores for claims, and in particular
    the rankings also coincide.
\end{proof}

We come to the axioms satisfied by Sums. While it satisfies both
\claimcoherence{} and \sourcecoherence{}, it is notable that Sums fails both
monotonicity properties and \disjointindependence{}. In some sense these
problems are caused by the normalisation step, where source and claim scores
are divided by their respective maximums. We present a modified version of Sums
without these deficiencies in \cref{td_new_sec_modifying_sums}.

\begin{theorem}
    Sums satisfies \claimcoherence{}, \sourcecoherence{}, \symmetry{},
    \marginaltrustworthiness{}, \trustbasedmon{}. It does not satisfy
    \freshposresp{}, \sourceposresp{}, \classicalindependence{},
    \disjointindependence{}, \conflictcoherence{} or \anticoherence{}.
\end{theorem}

\begin{proof}
    \claimcoherence{}, \marginaltrustworthiness{} and \trustbasedmon{} follow
    directly from
    \cref{td_new_cor_weightable_axioms,td_new_cor_sums_weightable}. For
    \sourcecoherence{}, let $N$ be a network and suppose $\claims_N(s)$ strictly
    precedes $\claims_N(s')$ with respect to $\cle_N^{\sums}$. Then by
    \cref{td_new_prop_pwo_strict_part}, there is a bijection $f: \claims_N(s)
    \to \claims_N(s')$ such that $\sums_N(c) \le \sums_N(f(c))$ for all $c \in
    \claims_N(s)$, and there is some $c_0$ with $\sums_N(c_0) <
    \sums_N(f(c_0))$. It follows that $N$ must contain at least one report,
    since otherwise no strict inequalities hold. For any source $t$,
    \cref{td_new_lemma_sums_fixedpoint} implies
    \[
        \sums_N(t)
        = \alpha \sum_{c \in \claims_N(t)}{\sums_N(c)},
    \]
    where $\alpha = 1 / \max_{t' \in S}{|\sum_{c \in
    \claims_N(t')}{\sums_N(c)}|} > 0$ is a constant. Using the fact that
    $f$ maps bijectively from $\claims_N(s)$ to $\claims_N(s')$, we get
    \begin{align*}
        \sums_N(s) - \sums_N(s')
        &= \alpha \left(
            \sum_{c \in \claims_N(s)}{\sums_N(c)}
            -
            \sum_{c' \in \claims_N(s')}{\sums_N(c')}
        \right) \\
        &= \alpha \left(
            \sum_{c \in \claims_N(s)}{\sums_N(c)}
            -
            \sum_{c \in \claims_N(s)}{\sums_N(f(c))}
        \right) \\
        &= \alpha \sum_{c \in \claims_N(s)}{\left(
            \sums_N(c) - \sums_N(f(c))
        \right)}.
    \end{align*}
    By assumption $\sums_N(c) - \sums_N(f(c)) \le 0$ for each $c$, and the
    inequality is strict for $c = c_0$. Hence $\sums_N(s) < \sums_N(s')$, and
    $s \slt_N^{\sums} s'$ as required.

    Finally, \symmetry{} can be shown in similar way to Weighted Agreement,
    since Sums is defined only in terms of $\src_N$ and $\claims_N$.

    \begin{figure}
        \centering
            \begin{tikzpicture}[thick]
                \LARGE
                \networkinit{s,t,u,v,x_1,x_2,x_3,x_4}{c,d,e_1,e_2}
                \object{o}{c}{d}
                \objectsingleclaim{p_1}{e_1}
                \objectsingleclaim{p_2}{e_2}
                \report{s}{c}
                \report{t}{c}
                \report{u}{d}
                \report[dashed]{v}{d}
                \report{x_1}{e_1}
                \report{x_1}{e_2}
                \report{x_2}{e_1}
                \report{x_2}{e_2}
                \report{x_3}{e_1}
                \report{x_3}{e_2}
                \report{x_4}{e_1}
                \report{x_4}{e_2}
            \end{tikzpicture}
        \caption{
            Networks used as counterexamples for Sums axiom failures.
        }
        \label{td_new_fig_sums_counterex}
    \end{figure}

    For the negative axioms, we refer to networks shown in
    \cref{td_new_fig_sums_counterex}.
    %
    For \freshposresp{} and \sourceposresp{}, let $N_0$ denote the network
    without the dashed report $(v, d)$, so that $\addclaim{N_0}{v}{d}$ is the
    full network. It can be shown that the rankings are the same under Sums in
    both networks, with $s \seq t \seq u \seq v \slt x_1 \seq x_2 \seq x_3 \seq
    x_4$ and $c \ceq d \clt e_1 \ceq e_2$. This violates \freshposresp{}, since
    $c \cle_{N_0}^{\sums} d$ but $c \not\clt_{\addclaim{N_0}{v}{d}}^{\sums} d$.
    It also violates \sourceposresp{}, since $s \in \antisrc_{N_0}(d)$, $u \in
    \src_n(d)$ and $s \sle_{N_0}^{\sums} u$, but $s
    \not\slt_{\addclaim{N_0}{v}{d}}^{\sums} u$.

    For \classicalindependence{} and \disjointindependence{}, let $N_1$ and
    $N_2$ denote the upper and lower components of the network in
    \cref{td_new_fig_sums_counterex}, excluding the dashed report $(v, d)$.
    Then $N_0 = N_1 \sqcup N_2$. Hence $c \ceq_{N_1 \sqcup N_2}^{\sums} d$.
    However, it is straightforward to check that in the network $N_1$ alone we
    have $d \clt_{N_1}^{\sums} c$; this violates \disjointindependence{}.
    Taking $N_0'$ to be the network obtained from $N_0$ by removing all
    reports from $x_1,\ldots,x_4$, we have $d \clt_{N_0'}^{\sums} c$ and $c
    \ceq_{N_0}^{\sums} d$. Since $c$ and $d$ have the same sources in both
    networks, this violates \classicalindependence{}.

    Finally, for \conflictcoherence{} and \anticoherence{} we can reuse the
    network $N$ from \cref{td_new_fig_conflict_impossibilities} (including the
    dashed report). Applying Sums to this network, we have $\sums_N(s) =
    \sums_N(t) = 0$, $\sums_N(u) = \sqrt{3} - 1 \approx 0.7321$, $\sums_N(s') =
    \sums_N(t') = 1$ and $\sums_N(c) = \sums_N(d) = \sums_N(e) = 0$,
    $\sums_N(f) = 1$, $\sums_N(c') = \sums_N(d') = \frac{1}{2}(\sqrt{3} - 1)
    \approx 0.3660$, yielding rankings $s \seq t \slt u \slt s' \seq t'$ and $c
    \ceq d \ceq e \clt c' \ceq d' \clt f$. This ranking violates
    \conflictcoherence{} since $\conflict_N(c) = \{d\}$ strictly precedes
    $\conflict_N(c') = \{d'\}$ but $c' \not\clt_N^{\sums} c$. It also violates
    \anticoherence{}, since $\antisrc_N(c) = \{t\}$ strictly precedes
    $\antisrc_N(c') = \{t'\}$ but $c' \not\clt_N^{\sums} c$.
\end{proof}

The key to the counterexamples derived from \cref{td_new_fig_sums_counterex} in
the above proof lies in the lower disjoint component, which takes the form of a
\emph{connected} bipartite graph. That is, each source $x_i$ reports each claim
$e_j$ in the component. Moreover, sources elsewhere in the network claim fewer
facts than the $x_i$, and claims elsewhere are reported by fewer sources than
the $e_j$.

Since Sums assigns scores by a simple sum, this results in the scores for the
$x_i$ and $e_j$ dominating those of the other sources and claims. The
normalisation step (i.e. the factors $\alpha$ and $\beta$ in
\cref{td_new_def_sums}) then divides these scores by the (comparatively large)
maximum. As the next result shows, under certain conditions this causes scores
to decrease \emph{exponentially} and become $0$ in the limit. In particular, we
can generate pathological examples such as \cref{td_new_fig_sums_counterex}
where a whole component receives scores of 0, which leads to failure of
\disjointindependence{} and the monotonicity axioms.

\begin{proposition}
\label{td_new_prop_obliteration}
    Let $N$ be a network. Suppose there are non-empty sets $X \subseteq S$, $Y
    \subseteq C$ such that
    \begin{enumerate}
        \item $\claims_N(x) = Y$ for each $x \in X$;
        \item $\src_N(y) = X$ for each $y \in Y$;
        \item $|\claims_N(s)| \le \frac{|Y|}{2}$ for each $s \in S \setminus
              X$; and
        \item $|\src_N(c)| \le \frac{|X|}{2}$ for each $f \in C \setminus Y$.
    \end{enumerate}
    Then, with $T^n$ denoting the $n$-th step in the iteration of Sums, for all
    $n > 0$ we have
    \[ T_N^n(s) \le \frac{1}{2^{n-1}} \qquad (s \in S \setminus X), \]
    \[ T_N^n(c) \le \frac{1}{2^{n-1}} \qquad (c \in C \setminus Y), \]
    \[ T_N^n(x) = 1, \qquad (x \in X) \]
    \[ T_N^n(y) = 1. \qquad (y \in Y) \]
    In particular, $\sums_N(s) = \sums_N(c) = 0$ for all $s \in S \setminus X$
    and $c \in C \setminus Y$.
\end{proposition}

The proof follows by induction on $n$.

% \paragraph{TruthFinder.}

% As with Sums, we can greatly simplify axiomatic analysis of TruthFinder by
% first showing that the limit operator $\truthfinder$ is a fixed point of
% TruthFinder's update function $U$. We face a slight difficulty here, however,
% since $\truthfinder$ does not necessarily lie in the domain $\D$ of $U$.
% Whereas $\D$ consists of the operators $T$ with $0 < T_N(s) < 1$ whenever
% $\claims_N(s) \ne \emptyset$, it is possible for some sources to have limiting
% scores of 1. Attempting to plug $\truthfinder$ into $U$ may consequently result
% in division by zero in \cref{td_new_eqn_truth_finder_claim_update} when
% $\lambda > 0$.

% We work around this by using the conventions $c / 0 = \infty$, $c / \infty = 0$
% and $c + \infty = \infty$ for $0 < c < \infty$. In doing so, we extend the
% update function $U$ of TruthFinder to $\widehat{U}$, defined on the extended
% domain $\widehat{\D}$ of operators $T$ satisfying $0 \le T_N(s) \le 1$ whenever
% $\claims_N(s) \ne \emptyset$. Note that the recursive scheme $(\widehat{D},
% T^0, \widehat{U})$ also converges to $\truthfinder$, since $\widehat{U}$ is
% equal to $U$ on $\D$.  As before, we write $T^n = \widehat{U}^n(T^0)$ for the
% $n$-th step of the iteration of TruthFinder, and assume for simplicity that
% TruthFinder converges on all networks.

% \begin{lemma}
%     $\truthfinder \in \widehat{D}$, and $\widehat{U}(\truthfinder) = \truthfinder$.
% \end{lemma}

% \begin{proof}
%     If $\claims_N(s) \ne \emptyset$ then, since $T^n \in \D$ for all $n \in
%     \Nat$, we have $T^n_N(s) \in (0, 1)$. Hence $\truthfinder_N(s) = \lim_{n
%     \to \infty}{T^n_N(s)} \in [0, 1]$, so $\truthfinder \in \widehat{\D}$.

%     Now, take any network $N$. We show $\widehat{U}(\truthfinder)_N(z) =
%     \truthfinder_N(z)$ for all $z \in S \cup C$, by an argument similar to
%     the one used in \cref{td_new_lemma_sums_fixedpoint} for Sums. First, take
%     $c \in C$. We have
%     \begin{align}
%         \truthfinder_N(c)
%         &= \lim_{n \to \infty}{T^{n+1}_N(c)} \nonumber \\
%         &= \lim_{n \to \infty}{
%             \left[
%                 1 +
%                 \frac{
%                     \prod_{s \in \src_N(c)}{(1 - T^n_N(s))^\gamma}
%                 }{
%                     \prod_{t \in \antisrc_N(c)}{(1 - T^n_N(t))^{\gamma\rho\lambda}}
%                 }
%             \right]^{-1}
%         }
%         \label{td_new_eqn_truth_finder_fixedpoint_computation}
%     \end{align}
%     First suppose that either $\lambda = 0$ or there is no $t \in
%     \antisrc_N(c)$ with $\truthfinder_N(t) = 1$. Then the denominator in
%     \cref{td_new_eqn_truth_finder_fixedpoint_computation} has non-zero limit.
%     Since this expression for $\truthfinder_N(c)$ is a composition of
%     continuous functions, we may take the limit inside to obtain
%     \begin{align*}
%         \truthfinder_N(c)
%         &=
%         \left[
%             1 +
%             \frac{
%                 \prod_{s \in \src_N(c)}{(1 - \truthfinder_N(s))^\gamma}
%             }{
%                 \prod_{t \in \antisrc_N(c)}{(1 - \truthfinder_N(t))^{\gamma\rho\lambda}}
%             }
%         \right]^{-1} \\
%     &= U(\truthfinder)_N(c) \\
%     &= \widehat{U}(\truthfinder)_N(c).
%     \end{align*}
%     Now suppose $\lambda > 0$ and there is some $t_0 \in \antisrc_N(c)$ with
%     $\truthfinder_N(t_0) = 1$. Then $(1 - T^n_N(t_0))^{\gamma\rho\lambda} \to
%     0$ as $n \to \infty$. Hence $\prod_{t \in \antisrc_N(c)}{(1 -
%     T^n_N(t))^{\gamma\rho\lambda}} \to 0$ also, and since $\prod_{s \in
%     \src_N(c)}{(1 - T^n_N(s))^\gamma}$ has a finite and positive limit,
%     \todo{wrong: might be 0 if some $s$ has limit score $1$. We can break out
%     into another case here, but looks like things are going to get messy.}
%     \[
%         \lim_{n \to \infty}{
%             \frac{
%                 \prod_{s \in \src_N(c)}{(1 - T^n_N(s))^\gamma}
%             }{
%                 \prod_{t \in \antisrc_N(c)}{(1 - T^n_N(t))^{\gamma\rho\lambda}}
%             }
%         } = +\infty.
%     \]
%     It is easily seen that if $x_n \to \infty$ then $1 / (1 + x_n) \to 0$.
%     From \cref{td_new_eqn_truth_finder_fixedpoint_computation}, we get
%     $\truthfinder_N(c) = 0$. On the other hand,
%     \[
%         \widehat{U}(\truthfinder)_N(c)
%         =
%             \left[
%                 1 + \frac{\prod_{s \in \src_N(c)}{(1 - \truthfinder_N(s))^\gamma}}{0}
%             \right]^{-1}
%         = 0,
%     \]
%     so $\truthfinder_N(c) = \widehat{U}(\truthfinder)_N(c)$ as required.
%     \todo{Correct and finish or ditch.}
% \end{proof}

% \paragraph{CRH.}

% As with Sums, we can greatly simplify axiomatic analysis of CRH by first
% showing that the limit operator $\crh{\epsilon}$ is a fixed point of the update
% function $U$. Take $\epsilon > 0$, and let $(\D, T^0, U)$ denote the recursive
% scheme of CRH-$\epsilon$ from \cref{td_new_def_crh}. As before, we write $T^n =
% U^n(T^0)$ for the $n$-th step of the iteration, and assume for simplicity that
% CRH-$\epsilon$ converges on all networks.

% \begin{lemma}
%     \label{td_new_lemma_crh_fixedpoint}
%     $\crh{\epsilon} \in \D$, and $U(\crh{\epsilon}) = \crh{\epsilon}$.
% \end{lemma}

% \begin{proof}
%     First we show $\crh{\epsilon} \in \D$; that is, we have $0 \le
%     \crh{\epsilon}_N(c) \le 1$ for all networks $N$ and claims $c$. First, note
%     that for any operator $T$ and source $s \in S$, we have $U(T)_N(s) > 0$.
%     Consequently $U(T)_N(c) \ge 0$, and
%     \[
%         U(T)_N(c)
%         = \frac{\sum_{s \in \src_N(c)}{U(T)_N(s)}}{\sum_{t \in S}{U(T)_N(t)}}
%         \le \frac{\sum_{s \in S}{U(T)_N(s)}}{\sum_{t \in S}{U(T)_N(t)}}
%         = 1.
%     \]
%     Since $T^n = U(T^{n - 1})$ for $n > 1$, we have $T^n(c) \in [0, 1]$ for all
%     $n > 1$. Consequently $\crh{\epsilon}_N(c) = \lim_{n \to \infty}{T^n_N(c)}
%     \in [0, 1]$ also. Thus $\crh{\epsilon} \in \D$.

%     Now, take any network $N$. We aim to show $\crh{\epsilon}_N(z) =
%     U(\crh{\epsilon})_N(z)$ for all $z \in S \cup C$. First take $s \in S$.
%     For any $t \in S$, write
%     \[
%         \alpha^n_t = \epsilon + \sum_{c \in \claims_N(t)}{
%             \sum_{d \in \claims_N(\obj(c))}{
%                 (T^n_N(d) - \indicator{d = c})^2
%             }
%         }.
%     \]
%     Then $\lim_{n \to \infty}{\alpha^n_t} = \epsilon + \sum_{c \in
%     \claims_N(t)}{\sum_{d \in \claims_N(\obj(c))}{(\crh{\epsilon}_N(d) -
%     \indicator{d = c})^2}}$. We have
%     \begin{align}
%         \crh{\epsilon}_N(s)
%         &= \lim_{n \to \infty}{T^{n+1}_N(s)} \nonumber \\
%         &= \lim_{n \to \infty}{\left(
%             \epsilon
%             -
%             \log\left(
%                 \frac{\alpha^n_s}{\sum_{t \in S}{\alpha^n_t}}
%             \right)
%         \right)}.
%         \label{td_new_eqn_crh_fixpoint_limit}
%     \end{align}
%     Now, since $T^n_N(d) \in [0, 1]$ for all $n \in \Nat$, and clearly
%     $\indicator{d = c} \in [0, 1]$, we have
%     \[
%         \epsilon \le
%         \alpha^n_t
%         = \epsilon + \sum_{c \in \claims_N(t)}{
%             \sum_{d \in \claims_N(\obj(c))}{
%                 \underbrace{(T^n_N(d) - \indicator{d = c})^2}_{\le 1}
%             }
%         }
%         \le \epsilon + |C|^2.
%     \]
%     Hence
%     \[
%         \frac{\alpha^n_s}{\sum_{t \in S}{\alpha^n_t}}
%         \ge \frac{\epsilon}{\sum_{t \in S}{(\epsilon + |C|^2)}}
%         = \frac{\epsilon}{|S|(\epsilon + |C|^2)} > 0,
%     \]
%     assuming $S \ne \emptyset$. Since this lower bound is independent of $n$,
%     this implies $\lim_{n \to \infty}{\frac{\alpha^n_s}{\sum_{t \in
%     S}{\alpha^n_t}}} > 0$. By continuity of the logarithm and
%     \cref{td_new_eqn_crh_fixpoint_limit}, we get
%     \[
%         \crh{\epsilon}_N(s)
%         = \epsilon - \log\left(
%             \frac{\lim_{n \to \infty}{\alpha^n_s}}{\sum_{t \in S}{\lim_{n \to
%             \infty}{\alpha^n_t}}}
%         \right)
%         = U(\crh{\epsilon})_N(s)
%     \]
%     as required.

%     Now take any $c \in C$. From above we have $\lim_{n \to \infty}{T^n_N(t)} =
%     \crh{\epsilon}_N(t) \ge \epsilon > 0$ for each $t \in S$. By simple
%     manipulation of limits we find
%     \begin{align*}
%         \crh{\epsilon}_N(c)
%         &= \lim_{n \to \infty}{T^n_N(c)} \\
%         &= \lim_{n \to \infty}{
%             \frac{\sum_{s \in \src_N(c)}{T^n_N(s)}}{\sum_{t \in S}{T^n_N(t)}}
%         } \\
%         &= \frac{
%                 \sum_{s \in \src_N(c)}{\lim_{n \to \infty}{T^n_N(s)}}
%             }{
%                 \sum_{t \in S}{\lim_{n \to \infty}{T^n_N(t)}}
%             } \\
%         &= \frac{
%             \sum_{s \in \src_N(c)}{\crh{\epsilon}_N(s)}
%             }{
%                 \sum_{t \in S}{\crh{\epsilon}_N(t)}
%             } \\
%         &= U(\crh{\epsilon})_N(c).
%     \end{align*}
%     This completes the proof.
% \end{proof}

% \begin{corollary}
%     $\crh{\epsilon}$ is weightable.
% \end{corollary}

% \begin{proof}
%     From \cref{td_new_lemma_crh_fixedpoint} we have
%     \[
%         \crh{\epsilon}_N(c)
%         = \frac{
%             \sum_{s \in \src_N(c)}{\crh{\epsilon}_N(s)}
%         }{
%             \sum_{t \in S}{\crh{\epsilon}_N(t)}
%         }.
%     \]
%     Defining a weighting $w$ by $w_N(s) = \frac{\crh{\epsilon}_N(s)}{\sum_{t
%     \in S}{\crh{\epsilon}_N(t)}}$, it is easily seen that $\crh{\epsilon}
%     \rankequiv \wvoting{w}$.
% \end{proof}

% \todo{intro to CRH axioms}

% \begin{theorem}
%     \label{td_new_thm_crh_axioms}
%     Take $\epsilon > 0$.
% \end{theorem}

\subsection{Modifying Sums}
\label{td_new_sec_modifying_sums}

While Sums satisfies some desirable axioms, the failure of
\disjointindependence{} and the monotonicity axioms is problematic. We saw
through \cref{td_new_prop_obliteration} that this is in some sense caused by
the normalisation step, where source and claim scores are divided by the
``global'' maximum scores across the network.

A seemingly natural fix for \disjointindependence{} is to therefore use
different normalisation factors $\alpha$ and $\beta$ for each disjoint
component. However, this does not escape the negative consequences of
\cref{td_new_prop_obliteration}. Indeed, if one modifies the network(s) in
\cref{td_new_fig_sums_counterex} so that claim $e_1$ is associated with object
$o$ instead of $p_1$, the network no longer has two disjoint components.
Consequently, the ``per-component Sums'' operator gives the same result as Sums
itself, and in particular the counterexamples for \freshposresp{} and
\sourceposresp{} still hold. Perhaps even worse, one can show that
\claimcoherence{} fails for this modified operator. We consider loss of
\claimcoherence{} too high a price to pay for \disjointindependence{}.

Instead, let us take a step back and consider if normalisation is truly
necessary. On the one hand, without normalisation the source and claim scores
are unbounded and therefore do not converge. On the other, we are not
interested in the numeric scores for their own sake, but rather for the
\emph{rankings} that they induce. It may be possible that whilst the scores
diverge without normalisation, the induced rankings \emph{do} converge to a
fixed one, which we may take as the ``ordinal limit''. This is in fact the
case.  We call this new operator \emph{UnboundedSums}

\begin{definition}
    \emph{UnboundedSums} is the recursive scheme $(\D, T^0, U)$, where $\D$ is
    the set of all operators with scores in $[0, 1]$,
    \begin{align*}
        T^0_N(s) &= 1, \\
        T^0_N(c) &= |\src_N(c)|
    \end{align*}
    and $U(T) = T'$, where
    \begin{align*}
        T'_N(s) &=
            \sum_{c \in \claims_N(s)}{
                T_N(c)
            },
        \\
        T'_N(c) &=
            \sum_{s \in \src_N(c)}{
                T'_N(s)
            }.
    \end{align*}
\end{definition}

Note that the update function $U$ is almost identical to that of Sums from
\cref{td_new_def_sums}, expect that it does not include the normalisation
factors $\alpha$ and $\beta$. Also note that to simplify the proof of ordinal
convergence, we use a different initial operator $T^0$ compared to Sums. In
what follows, let $T^n$ denote the $n$-th step in the iteration of
UnboundedSums.

\begin{restatable}{theorem}{restatetdusumsthm}
    \label{td_new_thm_usums_ordinal_convergence}
    UnboundedSums is ordinally convergent in the following sense: for every
    network $N$ there is $m \in \Nat$ such that for all $n \ge m$, $s, s' \in
    S$ and $c, c' \in C$,
    \begin{align*}
        T^n_N(s) \le T^n_N(s') &\iff T^m_N(s) \le T^m_N(s'), \\
        T^n_N(c) \le T^n_N(c') &\iff T^m_N(c) \le T^m_N(c').
    \end{align*}
    That is, the rankings induced by $T^n_N$ are constant for $n \ge m$.
\end{restatable}

The proof -- which can be found in \cref{td_new_app_sec_usums} -- expresses the
update function of UnboundedSums in terms of matrix multiplication, and uses
techniques and results from linear algebra.
%
In light of \cref{td_new_thm_usums_ordinal_convergence}, we may define an
operator $\usums$ by setting $\usums_N(z) = T_N^m(z)$, where $m$ depends on $N$
and is taken sufficiently large. It follows that $U(\usums) \rankequiv \usums$.

With the normalisation problems aside, UnboundedSums provides an example of a
principled operator satisfying many of our core axioms, including
\sourcecoherence{}, \claimcoherence{} and \disjointindependence{}. In
particular, we avoid the undesirable behaviour of Sums in
\cref{td_new_fig_sums_counterex}; whereas Sums trivialises the ranking of
sources and claims in the upper component, UnboundedSums allows meaningful
comparison (e.g. we have $d \clt c$). We conjecture that UnboundedSums also
satisfies the monotonicity properties \freshposresp{} and \sourceposresp{}, but
this remains to be proven.\footnotemark{}

\footnotetext{
    We have experimentally verified that UnboundedSums satisfies all the
    specific instances of \freshposresp{} with the starting network $N$ as in
    \cref{td_new_fig_intro_example}.
}

\begin{theorem}
    UnboundedSums satisfies \sourcecoherence{}, \claimcoherence{}, \symmetry{},
    \marginaltrustworthiness{}, \trustbasedmon{} and \disjointindependence{}.
    It does not satisfy \classicalindependence{}, \conflictcoherence{} or
    \anticoherence{}.
\end{theorem}

\begin{proof}[Proof (sketch)]
    Setting $w_N(s) = U(\usums)_N(s) = \sum_{c \in \claims_n(s)}{\usums_n(c)}$,
    one can show $\usums \rankequiv \wvoting{w}$, and thus UnboundedSums is
    weightable. By \cref{td_new_cor_weightable_axioms}, \claimcoherence{},
    \marginaltrustworthiness{} and \trustbasedmon{} hold.

    For \sourcecoherence{}: if $\claims_N(s)$ strictly precedes $\claims_N(s')$
    with respect to $\cle_N^{\usums}$, then taking the same approach as in the
    proof of \sourcecoherence{} for Sums, we have
    \[
        \sum_{c \in \claims_N(s)}{\usums_N(c)}
        <
        \sum_{c' \in \claims_N(s')}{\usums_N(c')}
    \]
    and thus $U(\usums)_N(s) < U(\usums)_N(s')$. Since $\usums \rankequiv
    U(\usums)$, we get $s \slt_N^{\usums} s'$ as required.

    \symmetry{} holds by an argument similar to the one employed for Weighted
    Agreement.

    For \disjointindependence{}, suppose $N$ and $N'$ are disjoint. One can
    show (e.g. by induction) that $T_N^n(z) = T_{N \sqcup N'}^n(z)$ for all $z
    \in S \cup C$ and $n \in \Nat$. Taking $n$ sufficiently
    large, $T_N^n$ and $T_{N \sqcup N'}^n$ yield the same rankings on $S$ and
    $C$ as $\usums_N$ and $\usums_{N \sqcup N'}$ respectively. Consequently,
    for any $s, t \in S$,
    \begin{align*}
        s \sle_N^{\usums} t
        &\iff T_N^n(s) \le T_N^n(t) \\
        &\iff T_{N \sqcup N'}^n(s) \le T_{N \sqcup N'}^n(t) \\
        &\iff s \sle_{N \sqcup N'}^{\usums} t.
    \end{align*}
    Similarly, $c \cle_N^{\usums} d$ iff $c \cle_{N \sqcup N'}^{\usums} d$ for
    $c, d \in C$, and \disjointindependence{} is shown.

    To see \classicalindependence{} does not hold, consider the network $N$
    from \cref{td_new_fig_intro_example} and the network $N'$ obtained by
    removing reports from sources $u$ and $v$. Then one can show $c
    \clt_N^{\usums} d$ but $c \ceq_{N'}^{\usums} d$, so
    \classicalindependence{} fails. Finally, the same counterexamples for
    \conflictcoherence{} and \anticoherence{} for Sums provide counterexamples
    for UnboundedSums.
\end{proof}

\section{Related Work}
\label{td_new_sec_related_work}

\paragraph{Ranking systems.} \textcite{altman2008} initiated axiomatic study of
ranking systems. First we discuss their framework in relation to ours, and then
turn to their axioms. In their framework, a \emph{ranking system} $F$ maps any
(finite) directed graph $G = (V, E)$ to a total preorder $\le_G^F$ on the
vertex set $V$. In their view this is a variation of the classical social
choice setting, in which the set of voters and alternatives coincide. Nodes $v
\in V$ ``vote'' on their peers in $V$ by a form of approval
voting~\cite{laslier2010handbook}: an edge $v \to u$ is interpret as a vote for
$u$ from $v$. A ranking system then outputs a ranking of $V$, following the
general intuition that the more ``votes'' $v$ receives (i.e. the more incoming
edges), the higher $v$ should rank. As with the ranking of claims in truth
discovery, this does not necessarily mean ranking nodes simply by the
\emph{number} of votes received, since the \emph{quality} of the voters should
also be taken in account. For example, a ranking system may prioritise nodes
which receive few votes from highly ranked nodes over those with many votes
from lower ranked nodes.

Note that our truth discovery networks $N$ can be viewed as directed graphs on
the vertex set $S \cup C \cup O$; this is the presentation we have used in the
figures throughout this chapter. However, naively applying a ranking system to
such graphs directly makes little sense: sources never receive any ``votes'',
and the resulting ranking includes objects, which do not need to be ranked in
our setting. Perhaps a more sensible approach is to consider the bipartite
graph $G_N = (V_N, E_N)$ associated with a network $N$, where
\[
    V_N = S \cup C, \qquad E_N = \bigcup_{(s, c) \in R}{ \{(s, c), (c, s)\} }.
\]
That is, we
take the edges from sources to claims together with the reversal of such edges.
The edges in $G_N$ have some intuitive interpretation: a source votes for the
claims which it believes are true, and a claim votes for the sources who vouch for
it.  Any ranking system $F$ thus gives rise to a truth discovery operator,
where $s \sle_N^T t$ iff $s \le_{G_N}^F t$, and similar for claims.

However, some characteristic aspects of the truth discovery problem are lost in
this translation to ranking systems. Notably, the objects play no role at all
in $G_N$. Sources and claims are also treated symmetrically, where they perhaps
should not be. For example, a claim $c$ receiving more claims than $d$ is
beneficial for $c$, all else being equal, but a source $s$
reporting more claims than $t$ does not tell us anything about the relative
trustworthiness of $s$ and $t$.

While other choices of $G_N$ may be possible to alleviate some of these
problems, we believe the truth discovery is sufficiently specialised beyond
general graph ranking so that a bespoke modelling is required to capture its
nuances appropriately. Our framework provides this novel contribution.

In \cite{altman2008}, Altman and Tennenholtz also introduce axioms for ranking
systems. Their first set of axioms deal with the transitive effects of voting
when the alternatives are the voters themselves. As mentioned in
\cref{td_new_sec_coherence}, these axioms provided the inspiration for our
coherence axioms. The core idea is that if the predecessors of a node $v$ are
weaker than those of $u$, then $v$ should be ranked below $u$. If $v$
additionally has \emph{more} predecessors, $v$ should rank \emph{strictly}
below. Coherence applies this idea both in the direction of sources-to-claims
(\claimcoherence{}) and from claims-to-sources (\sourcecoherence{}). A notable
difference is that we only consider the case where the number of sources for
two claims (or the number of claims, for two sources) is the same. For example,
a source reporting more claims does not give it the strict boost Transitivity
would dictate. Under the mapping $N \mapsto G_N$ described above, any ranking
system satisfying Transitivity induces a truth discovery operator which
satisfies both \sourcecoherence{} and \claimcoherence{}.

The other axiom of \textcite{altman2008} is the independence axiom \emph{RIIA}
(ranked independence of irrelevant alternatives), which adapts the classical
IIA axiom from social choice theory to the ranking system setting, although in
a different manner to our independence axioms of \cref{td_new_sec_independence}. We
describe the axiom in rough terms, deferring to the paper for the technical
details. Suppose the relative ranking of $u_1$'s predecessors compared to
$u_2$'s predecessors is the same as that of $v_1$'s compared to $v_2$'s. Then
RIIA requires $u_1 \le u_2$ iff $v_1 \le v_2$. Informally, ``the relative
ranking of two agents must only depend on the pairwise comparison of the ranks
of their predecessors''~\cite{altman2008}.
%
While we do not have an analogous axiom, the idea can be adapted to truth
discovery networks. Intuitively, such an axiom would state that the ranking of
two claims depends only on comparisons between their corresponding sources (and
similar for the ranking of sources).

However, the main result of Altman and Tennenholtz is an impossibility:
Transitivity is incompatible with RIIA. Moreover, the result remains true even
when restricting to bipartite graphs, such as $G_N$ described above.
Accordingly, we can expect a similar impossibility result to hold in the truth
discovery setting between the coherence axioms and any analogue of RIIA.

\paragraph{PageRank.} PageRank~\cite{page_pagerank_1999} is a well-known
algorithm for ranking web pages based on the hyperlink structure of the web,
viewed as a directed graph. It has also been studied through the lens of social
choice and characterised
axiomatically~\cite{altman2005ranking,skibski_pagerank}.\footnotemark{}

\footnotetext{
    \textcite{skibski_pagerank} axiomatise the \emph{numerical
    scores} of PageRank, whereas \textcite{altman2005ranking} axiomatise the resulting ranking.
    Moreover, Wąs and Skibski point out that Altman and Tennenholtz in fact
    only consider a simplified version of PageRank called \emph{Katz prestige},
    defined only for strongly connected graphs.
}

\textcite{altman2005ranking} propose several \emph{invariance
axioms}, each of which requires that the ranking of pages is not affected by a
certain transformation of the web graph. For example, the axiom \emph{Self
Edge} says that adding a self loop from a page $a$ to itself does not change
the relative ranking of other pages, and results in a strictly positive boost
for $a$ (c.f. our monotonicity axioms). However, if we identify a truth
discovery network $N$ with the graph $G_N$ as described above, most of the
transformations involved do not respect the bipartite, symmetric structure of
$G_N$. That is, the transformed graph does not correspond to any $G_{N'}$, for
a network $N'$.  Consequently, the PageRank axioms have no truth discovery
counterpart in our setting. The only exception is \emph{Isomorphism}, where the
transformation in question is graph isomorphism; this axiom is analogous to our
\symmetry{} axiom. However, since PageRank is similar to the \emph{Hubs and
Authorities}~\cite{kleinberg1999} algorithm on which Sums is based -- which
also uses the hyperlink structure of the web to rank pages -- we expect there
may be additional axioms which can be expressed both for general graphs
\emph{and} truth discovery networks, satisfied by PageRank and Sums. We leave
the task of finding such axioms to future work.

\section{Conclusion}
\label{td_new_sec_conclusion}

\paragraph{Summary.}

In this chapter we formalised a mathematical framework for truth discovery.
While a number of simplifying assumptions were made compared to the mainstream
truth discovery literature, we are able to express several algorithms in the
framework. This provided the setting for the axiomatic method of social choice
to be applied. To our knowledge, this is the first such axiomatic treatment in
this context.

It was possible to adapt many axioms from social choice theory and related
areas. In particular, the \emph{Transitivity} axiom studied in the context of
ranking systems~\cite{tennenholtz2004,altman2008} took on new life in the form
of the coherence axioms, which we consider essential for truth discovery
operators. We proceeded to highlight the differences between voting theory and
truth discovery via an impossibility result involving a classical independence
axiom, and used this axiom to characterise Voting. Another impossibility result
-- \cref{td_new_thm_conflict_impossibilities} -- established the tension
between methods which rank claims on the basis of their sources, and those
which rank on the basis of antisources.

On the practical side, we analysed the existing method Sums, and found that,
surprisingly, it fails \disjointindependence{}. This is a serious issue for
Sums which has not been discussed in the literature to date, and its discovery
here highlights the benefits of the axiomatic method. To resolve this, we
suggested a modification to Sums -- which we call UnboundedSums -- for which
\disjointindependence{} \emph{is} satisfied.

\paragraph{Limitations and future work.}

While UnboundedSums resolves axiomatic problems with Sums, it may introduce
computational difficulties (since the numeric scores involved grow without
bound). Moreover, its status with respect to the monotonicity axioms remains
unknown, although we suspect that the axioms do hold. We leave further
investigation of UnboundedSums to future work.

A restriction of our analysis is that only one algorithm from the literature
was studied in detail. Further axiomatic analysis of algorithms provides a
deeper understanding of how algorithms operate on an intuitive level, but is
made difficult by the complexity of the state-of-the-art truth discovery
methods. New techniques for establishing the satisfaction (or otherwise) of
axioms would be helpful in this regard.

There is also scope for extensions to our model of truth discovery in the
framework itself. For example, we make the somewhat simplistic assumption that
there are only finitely many possible facts for sources to claim. This
effectively means we can only consider \emph{categorical values}; modelling an
object whose domain is the set of real numbers, for example, is not
straightforward in our framework.

Next, our model does not account for any associations or constraints between
objects, whereas in reality the belief in a fact for one object may strengthen
or weaken our belief in other facts for related objects. These types of
constraints or correlations have been studied both on the theoretical side
(e.g. in judgement aggregation) and practical side in truth
discovery~\cite{yang_probabilistic_2019}.

The axioms can also be further refined to relax some of the simplifying
assumptions we make regarding source attitudes; e.g. that they do not collude
or attempt to manipulate.

Finally, it may be argued that truth discovery as formulated in this chapter
risks simply to find \emph{consensus} among sources, rather than the
\emph{truth}. In particular, we do not put forward any model for possible
states of the world, nor of how sources produce their reports
(c.f.~\cite{meir_proxy_2019}). Without such ingredients one cannot make precise
what it means to find the truth. In a sense this is by design: our goal in
applying the axiomatic approach is to find general principles which should hold
for truth discovery methods under \emph{any} notion of truth-tracking.
However, truth-tracking will be addressed in the second half of the thesis --
in \cref{chapter_truthtracking} -- although using a logic-based framework in
the style of belief revision as opposed to truth discovery.

\paragraph{Outlook.}

In the following chapter we continue with a social-choice-based approach, and
tackle the problem of \emph{bipartite tournament ranking}. This is related to
truth discovery -- and we again consider ranking-based operators and axioms
governing them -- but is a separate problem in its own right.
%
But in fact, \cref{chapter_belief_change,chapter_truthtracking} are more
closely connected to truth discovery. We consider a logic-based belief change
problem, in which an operator takes as input reports from several sources of
unknown trustworthiness, and produces a conjecture concerning the true facts of
the world and the trustworthiness of the sources themselves. A major point of
difference as compared to truth discovery, however, is that we impose strict
semantics on ``trustworthiness'', rooted in \emph{expertise}. That is, a source
is considered trustworthy if they are believed to be an expert in the relevant
domain.
