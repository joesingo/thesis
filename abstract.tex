This thesis studies problems involving unreliable information. We look at how
to aggregate conflicting reports from multiple unreliable sources, how to
assess the trustworthiness and expertise of sources, and investigate the extent
to which the truth can be found with imperfect information.
%
We take a formal approach, developing mathematical frameworks in which these
problems can be formulated precisely and their properties studied. The results
are of a conceptual and technical nature, which aim to elucidate interesting
properties of the problem at the core abstract level.

In the first half we adopt the axiomatic approach of \emph{social choice
theory}. We formulate \emph{truth discovery} -- the problem of aggregating
reports to estimate true information and reliability of the sources -- as a
social choice problem. We apply the axiomatic method to investigate desirable
properties of such aggregation methods, and analyse a specific truth discovery
method from the literature.
%
We go on to study ranking methods for \emph{bipartite tournaments}. This
setting can be applied to rank sources according to their accuracy on a number
of topics, and is also of independent interest.

In the second half we take a logic-based perspective. We use modal logic to
formalise the notion of expertise, and explore connections with knowledge and
truthfulness of information. We use this as the foundation for a belief change
problem, in which reports must be aggregated to form beliefs about the true
state of the world and the expertise of the sources. We again take an axiomatic
approach -- this time in the tradition of belief revision -- where several
postulates are proposed as rationality criteria.
%
Finally, we address \emph{truth-tracking}: the problem of finding the truth
given non-expert reports. Adapting recent work combining logic with formal
learning theory, we investigate the extent to which truth-tracking is possible,
and how truth-tracking interacts with rationality.
