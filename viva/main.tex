\documentclass[10pt]{beamer}

% Packages
\usepackage{graphicx}
\usepackage{tikz}
\usepackage{bm}
\usepackage{xcolor}
\usepackage{pifont}
\usepackage{booktabs}
\usepackage{listofitems}

% tikz setup
\usetikzlibrary{calc}
\usetikzlibrary{decorations.pathreplacing}

% Metropolis options
\usetheme{metropolis}
\metroset{progressbar=frametitle,
          block=fill}

% Sets
\renewcommand{\S}{\mathcal{S}}  % Note: overrides existing command
\renewcommand{\O}{\mathcal{O}}  % Note: overrides existing command
\newcommand{\F}{\mathcal{F}}
\newcommand{\N}{\mathcal{N}}
\newcommand{\R}{\mathbb{R}}
\newcommand{\Nat}{\mathbb{N}}
\newcommand{\uS}{\mathbb{S}}
\newcommand{\uF}{\mathbb{F}}
\newcommand{\uO}{\mathbb{O}}
\newcommand{\D}{\mathcal{D}}
\newcommand{\sign}{\operatorname{sign}}
% Source and fact orderings
\newcommand{\sle}{\sqsubseteq}
\newcommand{\slt}{\sqsubset}
\newcommand{\sge}{\ge}
\newcommand{\seq}{\simeq}
\newcommand{\fle}{\preceq}
\newcommand{\flt}{\prec}
\newcommand{\fgt}{\succ}
\newcommand{\fge}{\succeq}
\newcommand{\feq}{\approx}

\newcommand{\src}{\mathsf{src}}
\newcommand{\facts}{\mathsf{facts}}
\newcommand{\obj}{\mathsf{obj}}
\newcommand{\mut}{\mathsf{mut}}
\newcommand{\orderings}{\mathcal{L}}
\newcommand{\num}{\mathcal{T}_{Num}}
\newcommand{\rec}{\mathsf{rec}}
\newcommand{\norm}{\mathsf{norm}}

\newcommand{\ord}[1]{\langle{#1}\rangle}

\renewcommand{\phi}{\varphi}
\renewcommand{\epsilon}{\varepsilon}

% Shortcuts
\newcommand{\limn}{\lim_{n \to \infty}}
\newcommand{\voting}{\emph{Voting}}
\newcommand{\sums}{\emph{Sums}}
\newcommand{\usums}{\emph{UnboundedSums}}
\newcommand{\avlog}{\emph{Average$\cdot$Log}}
\newcommand{\scvoting}{\emph{SC-Voting}}
\newcommand{\scoh}{\mathrel{\lhd}}
\newcommand{\fcoh}{\mathrel{\blacktriangleleft}}


% Metadata
\title{
    Trustworthiness and Expertise: Social Choice and Logic-based Perspectives
}
\date{}
\author{
    Joseph Singleton
    \\singletonj1@cardiff.ac.uk
}
\titlegraphic{\includegraphics[width=1cm]{images/cardifflogo}}

\begin{document}

\maketitle

\begin{frame}{Thesis Overview}
    \begin{itemize}
        \item The thesis studies problems relating to \alert{unreliable
              information} and \alert{expertise}
        \item Emphasis on applying formal methods
        \begin{itemize}
            \item social choice theory
            \item modal logic
            \item belief revision
            \item formal learning theory
        \end{itemize}
    \end{itemize}
\end{frame}

\begin{frame}{Social Choice Perspectives}
    \begin{itemize}
        \item The first half of the thesis uses the methodology of
              computational social choice theory
          \item We develop an axiomatic framework for \alert{truth discovery}
              and \alert{bipartite tournament ranking}
        \item Axiomatic analysis complements empirical work for comparing and
              developing new methods
    \end{itemize}
\end{frame}

\begin{frame}{Truth Discovery}
    \begin{itemize}
        \item \alert{Truth Discovery} has recently arisen as a branch of the
              literature on crowdsourcing
        \item Central question: given conflicting information, who should we
              trust and what should we believe?
        \item We set out new axioms for truth discovery, and analyse an
              existing method from the literature
    \end{itemize}

    % Copied from thesis with minor modifications
    \begin{figure}
        \centering
        \begin{tikzpicture}[thick,scale=0.8]
            \LARGE
            \networkinitwithnames{s,t,u,v}{s,t,u,v}{c,d,e,f}{+,-,+,-}
            \object{o_1}{c}{d};
            \object{o_2}{e}{f};
            \report{s}{c};
            \report{s}{e};
            \report{t}{d};
            \report{t}{f};
            \report{u}{f};
            \report{v}{f};
        \end{tikzpicture}
    \end{figure}

\end{frame}

\begin{frame}{Bipartite Tournament Ranking}
    \begin{itemize}
        \item ``Ground truth'' data can help with truth discovery:
        \begin{itemize}
            \item We already know something about the trustworthiness of
                  sources
            \item But this is not straightforward if objects vary in
                  \emph{difficulty}
        \end{itemize}
        \item We generalise aspects of this problem: how should players in a
              \alert{bipartite tournament} be ranked?
    \end{itemize}

    \begin{figure}
        \centering
        \begin{tikzpicture}[thick,scale=0.8]
            \tournamentinit{a_1,a_2,a_3,a_4}{b_1,b_2,b_3}
            \tournmatch{a_1}{b_1}
            \tournmatch{a_2}{b_1}
            \tournmatch{a_2}{b_2}
            \tournmatch{a_3}{b_1}
            \tournmatch{a_3}{b_3}
            \tournmatch{a_4}{b_3}
        \end{tikzpicture}
    \end{figure}

\end{frame}

\begin{frame}{Logic-based Perspectives}
    \begin{itemize}
        \item The second half of the thesis uses logic-based methods:
        \begin{itemize}
            \item \alert{Modal logic} framework to reason about expertise
            \item Multi-source \alert{belief change} problem with non-experts
            \item Investigation into \alert{truth-tracking} with non-experts
        \end{itemize}
    \end{itemize}
\end{frame}

\begin{frame}{Logic of Expertise}
    \begin{itemize}
        \item We develop a modal logic framework to reason about
              \alert{expertise}
        \item Key notion: information is \emph{sound} if it is true ``up to
              lack of expertise''
        \item We explore connections between expertise and \emph{knowledge} via
              epistemic logic
        \item This serves as the logical foundation for the following two
              chapters\ldots
    \end{itemize}

    \[
        \E\phi \limplies \A(\S\phi \limplies \phi)
    \]
\end{frame}

\begin{frame}{Multi-Source Belief Change}

    \begin{itemize}
        \item How can the methods of \alert{belief revision} be used to
              handle information from non-expert sources?
        \item How do we revise \emph{trust} in sources?
        \item We set out a belief change problem using the expertise framework
              of the previous chapter
        \item Axiomatic approach once again; inspired by AGM-style rationality
              postulates
    \end{itemize}

    \[
        \phi \in K(\sigma) \implies \neg\E\phi \in K(\sigma \concat
        \tuple{i, \neg\phi})
    \]
\end{frame}

\begin{frame}{Truth-Tracking with Non-Expert Sources}
    \begin{itemize}
        \item AGM revision focusses on rationality, not on \alert{finding the
              truth}
        \item We augment the belief change problem with notions of
              truth-tracking from formal learning theory
        \item This shows what can be learned in principle with non-experts
        \item Even with strong assumptions, there are fundamental limits
    \end{itemize}
\end{frame}

\end{document}
